%\tolerance=10000
%\documentclass[prl,twocoloumn,preprintnumbers,amssymb,pla]{revtex4}
\documentclass[prl,twocolumn,showpacs,preprintnumbers,superscriptaddress]{revtex4}
\documentclass{article}
\usepackage{graphicx}
\usepackage{color}
\usepackage{dcolumn}
%\linespread{1.7}
\usepackage{bm}
%\usepackage{eps2pdf}
\usepackage{graphics}
\usepackage{pdfpages}
\usepackage{caption}
%\usepackage{subcaption}
\usepackage[demo]{graphicx} % omit 'demo' for real document
%\usepackage{times}
\usepackage{multirow}
\usepackage{hhline}
\usepackage{subfig}
\usepackage{amsbsy}
\usepackage{amsmath}
\usepackage{amsfonts}
\usepackage{amsthm}
\usepackage{float}
\documentclass{article}
\usepackage{amsmath,systeme}

\sysalign{r,r}

% \textheight = 8.5 in
% \topmargin = 0.3 in

%\textwidth = 6.5 in
% \textheight = 8.5 in
%\oddsidemargin = 0.0 in
%\evensidemargin = 0.0 in

%\headheight = 0.0 in
%\headsep = 0.0 in
%\parskip = 0.2in
%\parindent = 0.0in

% \newcommand{\ket}[1]{\left|#1\right\rangle}
% \newcommand{\bra}[1]{\left\langle#1\right|}
\newcommand{\ket}[1]{| #1 \rangle}
\newcommand{\bra}[1]{\langle #1 |}
\newcommand{\braket}[2]{\langle #1 | #2 \rangle}
\newcommand{\ketbra}[2]{| #1 \rangle \langle #2 |}
\newcommand{\proj}[1]{| #1 \rangle \langle #1 |}
\newcommand{\al}{\alpha}
\newcommand{\be}{\beta}
\newcommand{\op}[1]{ \hat{\sigma}_{#1} }
\def\tred{\textcolor{red}}
\def\tgre{\textcolor{green}}


\theoremstyle{plain}
\newtheorem{theorem}{Theorem}

\newtheorem{lemma}[theorem]{Lemma}
\newtheorem{corollary}[theorem]{Corollary}
\newtheorem{proposition}[theorem]{Proposition}
\newtheorem{conjecture}[theorem]{Conjecture}

\theoremstyle{definition}
\newtheorem{definition}[theorem]{Definition}


\begin{document}
\begin{widetext}
\\
\\
\\

\begin{wrapfigure}
\centering
%\includegraphics[\textwidth]{TS_IISc.png}
\end{wrapfigure}
\begin{figure}[h!]
 \begin{right}
  \hfill\includegraphics[\textwidth, right]{TS_IISc.png}
 \end{right}
\end{figure}
\\
\\
\\
\noindent\textbf{1. If RDD1 = \{mango, mango, orange, apple, strawberry\} and RDD2 = \{mango, apple, banana\}, then match the following RDD transformations with their correct outputs:
}
\begin{figure}[H]
\begin{center}
    \includegraphics[width=1.0\textwidth,centering]{RDDTransform.pdf}
\end{center}
    %\caption{}
\end{figure}
\noindent A. $(i) - (c), (ii) - (d), (iii) - (a), (iv) - (b)$ 
\\
\\
\\
B. $(i) - (a), (ii) - (b), (iii) - (d), (iv) - (c)$
\\
\\
\\
C. $(i) - (d), (ii) - (b), (iii) - (c), (iv) - (a)$
\\
\\
\\
D. $(i) - (a), (ii) - (c), (iii) - (d), (iv) - (b)$
\\
\\
\\
\textbf{Answer: C}
\\
\\
\\
\\
\textbf{2. Select the False statement w.r.t. RDD persistence:}
\\
\\
\\
\noindent A. Dependent RDDs may be recomputed for each action\\
\\
\\
B. Persisting RDDs will guarantee its reuse without recompute\\
\\
\\
C. Recompute will happen if node fails or on LRU (least recently used) eviction\\
\\
\\
D. We can explicitly call persist on an RDD
\\
\\
\\
\textbf{Answer: B}
\\
\\
\\
\\
\newpage
\textbf{3. Select the correct output for ‘pink’ after the following aggregation transformation is performed on a pair RDD:}
\\
\\
\\
\\
rdd.mapValues(lambda x: (x, 1)).reduceByKey(lambda x, y: (x[0] + y[0], x[1] + y[1])) 
\begin{figure}[H]
\begin{center}
    \includegraphics[width=1.0\textwidth,centering]{key-value.pdf}
\end{center}
    %\caption{}
\end{figure}
\\
\\
\\
\noindent A. (1, 5)\\
\\
\\
B. (1, 5)\\
\\
\\
C. (1, 1)\\
\\
\\
D. (3, 2)
\\
\\
\\
\textbf{Answer: B}
\\
\\
\\
\\
\textbf{4. Select the correct sequence of steps for the combineByKey transformation operation:}  
\\
\\
\\
$(i)$ mergeCombiners: Function used to combine accumulator values for the same key from multiple partitions 
\\
\\
$(ii)$ createCombiner: Function called the first time a key is seen on each partition. Initializes the accumulator value for that key 
\\
\\
$(iii)$ mergeValue: Function called for each subsequent value for a key on a partition. Merges value with current accumulator’s value 
\\
\\
\\
A. $(i), (ii), (iii)$
\\
\\
\\
B. $(iii), (i), (ii)$
\\
\\
\\
C. $(i), (iii), (ii)$
\\
\\
\\
D. $(ii), (iii), (i)$
\\
\\
\\
\textbf{Answer: D}
\\
\\
\\
\\
\newpage
\noindent\textbf{5. Select the statement(s) that correctly describe(s) the characteristics of Lineage Graph:}
\\
\\
\\
$(i)$ Helps lazily materialize RDD 
\\
\\
$(ii)$ Does not keep track of operations used to derive an RDD
\\
\\
$(iii)$ Helps recover RDD or their partitions that are lost 
\\
\\
$(iv)$ Keeps track of operations used to derive an RDD 
\\
\\
$(v)$ Does not help recover RDD or their partitions that are lost 
\\
\\
\\
\noindent A. Only $(i)$ 
\\
\\
\\
B.  Only $(i)$, $(iii)$ and $(iv)$ 
\\
\\
\\
C. Only $(i)$, $(ii)$ and $(iv)$
\\
\\
\\
D. Only $(i)$, $(iii)$ and $(v)$
\\
\\
\\
\textbf{Answer: B}
\\
\\
\textbf{Solution:}\ 
Statement (i), (iii) and (iv) describe the characteristics of Lineage Graph. Statements (ii) and (v) are incorrect.
\\
\\
\\
\end{widetext}
\end{document}
