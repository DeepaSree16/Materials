\documentclass[11pt,twoside]{article}

\usepackage{fancyhdr}
\usepackage{float}
\usepackage[T1]{fontenc}
\usepackage{graphicx}
\usepackage[utf8]{inputenc}
\usepackage{txfonts}
\usepackage[normalem]{ulem}
\usepackage[paperheight=27.94cm,paperwidth=21.59cm,left=2.54cm,right=2.06cm,top=2.0cm,bottom=1.75cm]{geometry}


\setlength\parindent{0pt}
\renewcommand{\arraystretch}{1.3}\fancyfoot[LO]{\thepage}
\fancyfoot[LE]{\thepage}




\begin{document}

\begin{center}
{\Huge \uline{Module-6}}
\end{center}


\begin{center}
{\LARGE \uline{(Angular Integration with REST API)}}
\end{center}

\vspace{2\baselineskip}
\section*{Module Overview}

\vspace{1\baselineskip}

\textbf{CODING EXERCISE}

\vspace{1\baselineskip}
In this module, you will learn to design a dynamic Single Page Web Application using Angular4 with features mentioned below:

\begin{enumerate}
  \item{REST API Calls}
\end{enumerate}

\vspace{1\baselineskip}
\textbf{OBJECTIVES}

\vspace{1\baselineskip}
By designing the Web Application, you will be able to:

\begin{enumerate}
  \item How to retrieve data from REST API

  \item How to send data to REST API

\end{enumerate}
\vspace{1\baselineskip}
\textbf{SCRIPTING CONSTRUCTS}

\vspace{1\baselineskip}
To designing the web page, you will require the following scripting constructs:

\begin{enumerate}
  \item HttpClientModule

  \item HttpClient

  \item Observables

\end{enumerate}
\vspace{17\baselineskip}
\section*{REST API Calls using HttpClient}

In Angular 4.3 the new HttpClientModule has been introduced. This new module is available in package

@angular/common/http and a complete re-implementation of the former HttpModule. The new HttpClient service is included in HttpClientModule and can be used to initiate HTTP request and process responses within your application.

\vspace{1\baselineskip}
To be able to use the HttpClient service within your components we first need to include the HttpClientModule in the Angular application. Firstly, we need to import HttpClient module in the application's root module in file 

\vspace{1\baselineskip}
\subsection*{{\textit{\uline{app.module.ts:}}}}

\vspace{1\baselineskip}
import $\{$ BrowserModule $\}$ from '@angular/platform-browser'; 

import $\{$ NgModule $\}$ from '@angular/core'; 

import $\{$ HttpClientModule $\}$ from '@angular/common/http'; 

import $\{$ AppComponent $\}$ from './app.component'; 

\vspace{1\baselineskip}
@NgModule($\{$ declarations: [ AppComponent],

imports: [ BrowserModule, HttpClientModule],

providers: [], bootstrap: 

[AppComponent]

$\}$)

\vspace{1\baselineskip}
export class AppModule $\{$ 

\vspace{1\baselineskip}
$\}$

\vspace{2\baselineskip}
Once imported you can make use of HttpClient in your components. To make HttpClient available in the component class you need to inject it into the class constructor like you can see in the following:

\vspace{1\baselineskip}
import $\{$ Component, OnInit $\}$ from '@angular/core'; 

import $\{$ HttpClient $\}$ from '@angular/common/http';

\vspace{1\baselineskip}
@Component($\{$

selector: 'app-root',

templateUrl: './app.component.html', styleUrls: ['./app.component.css']

$\}$)

export class AppComponent $\{$ title $=$ 'app'; 

constructor(private http: HttpClient)$\{$

$\}$

$\}$

\vspace{5\baselineskip}
.

HttpClient will use the XMLHttpRequest browser API to execute HTTP request. In order to execute HTTP request of a specific type you can use the following methods which corresponds to HTTP verbs:

\begin{itemize}
  \item get
  \item post
  \item put
  \item delete
  \item patch
  \item head

\end{itemize}
\vspace{1\baselineskip}
\textbf{\uline{Using HttpClient To Request Data:-}}

\vspace{1\baselineskip}
Let's implement a simple example which uses GitHub's REST API to request user data. Insert the following code in file app.component.ts:

\vspace{1\baselineskip}
import $\{$ Component, OnInit $\}$ from '@angular/core'; 

import $\{$ HttpClient $\}$ from '@angular/common/http';

@Component($\{$ selector: 'app-root',

templateUrl: './app.component.html', styleUrls: ['./app.component.css'] 

$\}$)

export class AppComponent implements OnInit $\{$

title = 'app';

results = ”;

constructor(private http: HttpClient)$\{$

$\}$

ngOnInit(): void 

$\{$

\hspace*{25} this.http.get('https://api.github.com/users/seeschweiler').subscribe(data => $\{$

console.log(data); $\}$); $\}$

$\}$

\vspace{17\baselineskip}
.

The output which is displayed in the browser console should look like the following:
\begin{center}
  \includegraphics[width=0.8\textwidth]{picture_1.png}
\end{center}

 The results shows that it's possible to directly access the JSON response by subscribing to the Observable which is returned from the get method.
Typed Response
\vspace{1\baselineskip}

From the console output you can see that the returned JSON object has a lot of properties. If you now try to access one of those properties by using the dot notation you'll get back an error:

\vspace{1\baselineskip}
console.log(data.login);

\vspace{1\baselineskip}
The error message is saying: “Property 'login' does not exist on type 'Object”'. As data is of type Object you can not access properties directly. However we're able to cast the response Object to a type which is containing the corresponding properties. Let's define an interface type which is containing some if the properties which are part of the response:

\vspace{1\baselineskip}
interface UserResponse $\{$

\hspace*{10} login: string;

\hspace*{10} bio: string;

\hspace*{10} company: string;

$\}$

\vspace{1\baselineskip}
Next, let's make use of UserResponse to cast the return type of the get call:

\vspace{1\baselineskip}
this.http.get<UserResponse>('https://api.github.com/users/seeschweiler').subscribe(data => $\{$

console.log(”User Login: ” + data.login); console.log(”Bio: ” + data.bio); console.log(”Company: ” +

data.company);

$\}$);

\vspace{1\baselineskip}
Accessing the values by using data.login, data.bio and data.company is possible now. The output in the browser console should no deliver the following result:

\vspace{2\baselineskip}
\subsection*{Error Handling}

\vspace{1\baselineskip}
A HTTP request can fail. Because of a poor network connection or other circumstances which can not be foreseen. Therefore you should always include code which handlers an error situation.

\vspace{1\baselineskip}
Adding a second callback method to the subscribe method is the way this is done:

\vspace{1\baselineskip}
this.http.get<UserResponse>('https://api.github.com/users/seeschweiler').subscribe( data => $\{$ 

console.log(”User Login: ” + data.login); console.log(”Bio: ” + data.bio); console.log(”Company: ” +

data.company); $\}$, err => $\{$ console.log(”Error occured.”) $\}$ );

\vspace{4\baselineskip}
You can also get more specific information about the error by defining a parameter of type HttpErrorResponse for the error handler function. HttpErrorResponse needs to be imported from @angular/common/http:

\vspace{1\baselineskip}
this.http.get<UserResponse>('https://api.github.com/users/seeschweiler').subscribe( data => $\{$ 

console.log(”User Login: ” + data.login);

console.log(”Bio: ” + data.bio);

console.log(”Company: ” + data.company);

$\}$,

(err: HttpErrorResponse) => $\{$ if (err.error

instanceof Error) $\{$ console.log(”Client-side

error occured.”); $\}$ else $\{$ console.log(”Server-side error occured.”); 

$\}$

$\}$

);

\vspace{18\baselineskip}
\section*{HttpClient GET Request}

Let's Use the following fetch module of CRUD API in NodeJS.

\vspace{1\baselineskip}
\textbf{\uline{Fetch.js}}

\vspace{1\baselineskip}
var express = require('express');

var bodyParser = require('body-parser');

\vspace{1\baselineskip}
let mongodb = require("mongodb");

let talentsprint = mongodb.MongoClient;

\vspace{1\baselineskip}
let fetch = express.Router().get("/",(req,res)=>$\{$ 

\hspace*{10}talentsprint.connect("mongodb://localhost:27017/crud",(err,db)=>$\{$

\hspace*{20}if(err)$\{$

\hspace*{30}throw err;

\hspace*{20}$\}$ else$\{$

\hspace*{30}console.log('fetch called...');

\hspace*{30}db.collection("employee").find($\{$$\}$).toArray((err,array)=>$\{$

\hspace*{40}if(err)$\{$ 

\hspace*{50}throw err;

\hspace*{40}$\}$ else$\{$

\hspace*{50}if(array.length > 0)$\{$

\hspace*{60}res.send(array); //data will be sent to client (browser or Angular)

\hspace*{50}$\}$ else $\{$

\hspace*{60}res.send($\{$message:"Record Not Found..."$\}$);

\hspace*{50}\}

\hspace*{40}}

\hspace*{30}\});

\hspace*{20}\}

\hspace*{10}\});

\});

module.exports = fetch;

\vspace{16\baselineskip}
.

\textbf{\uline{Server.js}}

\vspace{1\baselineskip}
//import express module

let express = require("express");

//import body-parser module (Useful to convert clients data into json) 

let bodyparser = require("body-parser");

//create the rest object

let app = express();

//enable the ports communication

let cors = require("cors");

//set the JSON as MIME Type

app.use(bodyparser.json());

//read the json

app.use(bodyparser.urlencoded({extended:false}));

app.use(cors());

//calling apis

//use login module

app.use("/login",require("./login/login"));

//use register module 

app.use("/register",require("./register/register"));

//use register module 

app.use("/update",require("./update/update"));

//use register module

app.use("/delete",require("./delete/delete"));

//use fetch module

app.use("/fetch",require("./fetch/fetch"));

\vspace{2\baselineskip}
//assign the port no

app.listen(3000);

console.log("server listening the port no.3000");

\vspace{18\baselineskip}
.

\subsection*{Let's make a Client Call using Angular as follows: -} 

\vspace{1\baselineskip}
\subsection*{1) \uline{emp.service.ts:-}}

\vspace{1\baselineskip}
import \{ HttpClient \} from '@angular/common/http';

import \{ Injectable \} from '@angular/core';

\vspace{1\baselineskip}
@Injectable(\{ 
  
\hspace*{5}providedIn: 'root'

\})

export class EmpService \{

\vspace{1\baselineskip}
\hspace*{5}constructor(private httpClient: HttpClient) \{ 

\hspace*{5}\}

\hspace*{5}fetchDetails()\{

\hspace*{10}return this.httpClient.get('http://localhost:3000/fetch'); //Observables

\hspace*{5}\} 

\vspace{1\baselineskip}
\}

\vspace{1\baselineskip}
\subsection*{2) \uline{showemp.component.ts:-}}

\vspace{1\baselineskip}
import \{ HttpClient \} from '@angular/common/http'; 

import \{ Component, OnInit \} from '@angular/core'; 

import \{ EmpService \} from '../emp.service';

\vspace{1\baselineskip}
@Component(\{

\hspace*{5}selector: 'app-showemp',

\hspace*{5}templateUrl: './showemp.component.html',

\hspace*{5}styleUrls: ['./showemp.component.css']

\})

export class ShowempComponent implements OnInit \{

  \hspace*{5}employees:any;

\vspace{1\baselineskip}
\hspace*{5}constructor(private httpClient: HttpClient,private service: EmpService) \{

  \hspace*{10}// this.employees = [\{ empId: 1000, empName: 'ELIAS', salary: 9999.99,gender:'M',doj:'04-09-1999'\},

  \hspace*{10}// \{ empId: 2000, empName: 'PASHA', salary: 8888.88,gender:'M',doj:'12-12-2018'\},

  \hspace*{10}// \{ empId: 3000, empName: 'JOHN', salary: 7777.77,gender:'M',doj:'09-10-2004'\},

  \hspace*{10}// \{ empId: 4000, empName: 'LUCI', salary: 7887.76,gender:'F',doj:'10-11-2012'\}];

  \hspace*{5}\}

\vspace{1\baselineskip}
\hspace*{5}ngOnInit(): void \{ 
  
  \hspace*{10}this.service.fetchDetails().subscribe((result:any)=>\{

    \hspace*{15}this.employees = result; 

    \hspace*{10}\});

    \hspace*{5}\} 

\vspace{1\baselineskip}
\}

\subsection*{3) \uline{showemp.component.html:-}}

\vspace{1\baselineskip}
<head>

\hspace*{10}<!-- Latest compiled and minified CSS -->

\hspace*{10}<link rel="stylesheet" href="https://stackpath.bootstrapcdn.com/bootstrap/3.4.1/css/bootstrap.min.css"

\hspace*{20}integrity="sha384-HSMxcRTRxnN+Bdg0JdbxYKrThecOKuH5zCYotlSAcp1+c8xmyTe9GYg1l9a69psu"

crossorigin="anonymous">

</head>

\vspace{1\baselineskip}
<div class="container">

<table class="table table-bordered table-striped table-hover table-sm">

  \hspace*{10}<thead>

  \hspace*{10}<tr class="bg-info text-white"><th>ID</th><th>Name</th><th>Salary</th><th>Gender</th>

  \hspace*{20}<th>Email ID</th><th>Date Of Joining</th><th>Experience</th><th colspan=2>Actions</th></tr> 

  \hspace*{10}</thead>

  \hspace*{10}<tbody>

  \hspace*{10}<tr *ngFor="let employee of employees">

  \hspace*{20}<td>\{\{employee.\_id\}\}</td>

  \hspace*{20}<td>\{\{employee.empName |gender: employee.gender | uppercase\}\}</td> 

  \hspace*{20}<td>\{\{employee.salary | currency : 'INR'\}\}</td>

  \hspace*{20}<td>\{\{employee.gender | lowercase\}\}</td>

  \hspace*{20}<td>\{\{employee.email\}\}</td>

  \hspace*{20}<td>\{\{employee.joinDate | date :'dd-MMMM-yyyy hh:mm:ss'\}\}</td>

  \hspace*{20}<td>\{\{employee.joinDate | exp\}\}</td>

  \hspace*{20}<td><button (click)="deleteEmp(employee)" class="glyphicon glyphicon-trash"></button></td> 

  \hspace*{20}<td><button class="glyphicon glyphicon-pencil"></button></td>

  \hspace*{10}</tr> 

  \hspace*{10}</tbody>

</table>

</div>

\vspace{16\baselineskip}
\section*{HttpClient POST Request}

\vspace{1\baselineskip}
Let's Use the following register module of CRUD API in NodeJS.

\vspace{1\baselineskip}
\textbf{\uline{register.js}}

\vspace{1\baselineskip}
let mongodb = require("mongodb");

let talentsprint = mongodb.MongoClient;

\vspace{1\baselineskip}
let register = require("express").Router().post("/",(req,res)=>\{

   \hspace*{15}talentsprint.connect("mongodb://localhost:27017/crud",(err,db)=> 
   
   \hspace*{15}\{

    \hspace*{25}if(err)

    \hspace*{35}throw err;

    \hspace*{25}else\{

\vspace{1\baselineskip}
db.collection("employee").insertOne(\{"empName":req.body.empName,"salary":req.body.salary,"email":req.body.em ail,"gender":req.body.gender,"password":req.body.password,"joinDate":req.body.joinDate\},
      
\hspace*{35}(err, result)=> \{
            
  \hspace*{45}if (err) throw err;

  \hspace*{45}res.send(\{message:"1 document inserted"\});

  \hspace*{45}db.close();

  \hspace*{40}\});
          
  \hspace*{35}\}

  \hspace*{30}\});

  \hspace*{25}\});

module.exports = register;

\vspace{21\baselineskip}
.

We will be using HttpClient class post method along with RxJS Observables.

\subsection*{\uline{register.component.html:-}}

\vspace{1\baselineskip}
<input type="text" placeholder="NAME" [(ngModel)]='employee.empName'><br/>

\vspace{1\baselineskip}
<input type="text" placeholder="GENDER" [(ngModel)]='employee.gender'><br/>

\vspace{1\baselineskip}
<input type="date" placeholder="DATE OF JOING" [(ngModel)]='employee.joinDate'><br/> 

\vspace{1\baselineskip}
<input type="text" placeholder="SLARY" [(ngModel)]='employee.salary'><br/>

\vspace{1\baselineskip}
<input type="text" placeholder="EMAIL" [(ngModel)]='employee.email'><br/>

\vspace{1\baselineskip}
<input type="text" placeholder="PASSWORD" [(ngModel)]='employee.password'><br/> 

\vspace{1\baselineskip}
<input type="submit" value="REGISTER" (click)="registerEmp()">

\vspace{1\baselineskip}
\subsection*{\uline{register.component.ts:-}}

import \{ HttpClient \} from '@angular/common/http';

import \{ Component, OnInit \} from '@angular/core';

import \{ EmpService \} from '../emp.service';

\vspace{1\baselineskip}
@Component(\{

  \hspace*{5}selector: 'app-register',

  \hspace*{5}templateUrl: './register.component.html',

  \hspace*{5}styleUrls: ['./register.component.css']

\})

export class RegisterComponent implements OnInit \{

  \hspace*{5}employee: any;

  \hspace*{5}constructor(private httpClient: HttpClient,private service:EmpService) \{

    \hspace*{10}this.employee =\{empName:'',gender:'',joinDate:'',salary:'',email:'',password:''\};

    \hspace*{5}\}

\vspace{1\baselineskip}
\hspace*{5}ngOnInit(): void \{

\vspace{1\baselineskip}
\hspace*{5}\}

\vspace{1\baselineskip}
\hspace*{5}registerEmp()\{

  \hspace*{10}this.service.registerEmp(this.employee).subscribe( 
  
    \hspace*{15}(result:any)=>console.log(result));

  \hspace*{5}\}

\}

\vspace{1\baselineskip}
.

\subsection*{\uline{register.service.ts:-}}

import \{ HttpClient \} from '@angular/common/http'; 

import \{ Injectable \} from '@angular/core';

\vspace{1\baselineskip}
@Injectable(\{ 

  \hspace*{5}providedIn: 'root'

\})

export class EmpService \{

  \vspace{1\baselineskip}
  \hspace*{5}constructor(private httpClient: HttpClient) \{

  \hspace*{5}\}

\vspace{1\baselineskip}
\hspace*{5}registerEmp(employee: any)\{

  \hspace*{10}return this.httpClient.post('http://localhost:3000/register',employee);

  \hspace*{5}\} 

\}

\vspace{31\baselineskip}
\section*{HttpClient Query Parameters}

Angular 4 HttpClient Query Parameters , here in this post we will learn how to send Query Parameters value in Java Rest Api. By the help of HttpParams we can send query param value to the server.

\vspace{1\baselineskip}
Let's take create service and component class to call Rest Api with id query param.

\subsection*{\uline{student.component.html:-}}

\vspace{1\baselineskip}
<div>

<input type=”text” name=”roll” [(ngModel)]=”roll” placeholder=”Enter student roll”>

<button class=”btn btn-primary” (click)=”getStudentByRoll()”>Get Details</button> 

</div> student.component.ts

\vspace{1\baselineskip}
import \{ Component \} from '@angular/core';

import \{ Observable \} from 'rxjs/Observable';

import \{ StudentService \} from './student.service';

import \{ Student \} from '../customer/student.model'; 

@Component(\{ selector:'student', templateUrl:'./student.html' 

\})

export class StudentComponent \{ name: any;

constructor(private studentService: StudentService) \{ \}

getStudentByRoll() \{

  \hspace*{25}this.studentService.getStudent(roll).subscribe((data: any) => console.log(data), (err: any) => console.log(err),

  \hspace*{25}() => console.log(”——–Done——–”) );

  \hspace*{25}\}

\}

\vspace{2\baselineskip}
\textbf{\uline{student.service.ts:-}}

\vspace{1\baselineskip}
import \{ Injectable \} from '@angular/core';

import \{ HttpClient \} from '@angular/common/http'; 

import \{ Student \} from '../customer/student.model'; 

import \{ Observable \} from 'rxjs/Observable'; 

@Injectable()

\vspace{1\baselineskip}
export class StudentService \{

\vspace{1\baselineskip}
URL = “http://localhost:3000/info”;

public student = new Student();

\vspace{1\baselineskip}
constructor(public http: HttpClient) \{

\vspace{1\baselineskip}
\}

getStudentInfo(roll: string) 

\{

  \hspace*{25}return this.http.get(URL+”/”+roll) 

\}

\}


\subsection*{\uline{info.js:-}}

\vspace{1\baselineskip}
Here student service will be calling below Rest Api. 

//this file acting as module (info)

//login module accepts the roll from angular

//login module compares with "mongodb" database

\vspace{1\baselineskip}
let mongodb = require("mongodb");

let talentsprint = mongodb.MongoClient;

\vspace{1\baselineskip}
let info = require("express").Router().get("/:roll",(req,res)=>\{ 

    \hspace*{15}talentsprint.connect("mongodb://localhost:27017/crud",(err,db)=>\{

      \hspace*{30}if(err)\{

        \hspace*{45}throw err;

        \hspace*{30}\} 

        \hspace*{30}else\{

          \hspace*{45}db.collection("employee").findOne(\{"roll":req.params.roll \}, 

          \hspace*{45}(err, result) => \{

            \hspace*{60}if (err) \{

              \hspace*{75}throw err;

              \hspace*{60}\} 
            
              \hspace*{60}else\{

                \hspace*{75}res.send(result);

                \hspace*{60}\}

          \hspace*{45}\});

        \hspace*{30}\}

      \hspace*{15}\});

\});

module.exports = login;
\end{document}