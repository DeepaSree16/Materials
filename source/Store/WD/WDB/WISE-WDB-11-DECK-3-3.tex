

\documentclass{article}

\usepackage[utf8]{inputenc} 
\usepackage[english]{babel} 
\usepackage{amsmath}
\usepackage{amssymb}
\usepackage{txfonts}
\usepackage{mathdots}
\usepackage[classicReIm]{kpfonts}
\usepackage{graphicx}
\usepackage[margin=1.0in]{geometry}



\begin{document}
\noindent \textbf{React-redux}

\noindent \textbf{}

\noindent 
\\
React is one of the most popular JavaScript libraries~which is used for front-end development. It has made our application development easier and faster by providing the component based approach.

\noindent 
\\
As you might know, it's not the complete framework but just the view part of the MVC (Model-View-Controller) framework. So, how do you keep track of the data and handle the events in the applications developed using React? Well, this is where Redux comes as a savior and handles the data flow of the application from the backend.

\noindent \\
\textbf{Why Redux with React}

\noindent \\
As we know that React follows the component based approach, where the data flows through the components.~In-fact, the data in React always flows from parent to child components which makes it~\textit{unidirectional}. This surely keeps our data organized and helps us in controlling the application better. Because of this, the application's state is contained in specific stores and as a result, the rest of the components remain loosely coupled. This makes our application more flexible leading to increased efficiency. That's why the~communication from a parent component to a child component~is convenient.

\noindent 

\noindent 

\noindent \\
But what happens when we try to communicate from a non-parent component?

\noindent\\
A child component can never pass data back up to the parent component. React does not provide any way for direct component-to-component communication. Even though React has features to support this approach, it is considered to be a poor practice. It is prone to errors and leads to~spaghetti code. So, how can two non-parent components pass data to each other?

\noindent \\
This is where the React fails to provide a solution and Redux comes into the picture.

\noindent \\
Redux provides a ``\textbf{store}'' as a solution to this problem. A store~is a place where~you~can store~all your application state together. Now the components can ``\textit{dispatch}'' state changes to the store and not directly to the other components.~Then the components that need the updates about the state changes can ``\textit{subscribe}'' to the store.

\noindent \\
Thus, with Redux, it becomes clear where the components get their state from as well as where should they send their states to. Now the~component initiating the change does not have to worry about the list of components needing the state change and can simply dispatch~the change to the store. This is how Redux makes the~\textit{data flow}~easier.

\noindent 

\noindent \\
Install redux and react-redux in the project as below.

\noindent \textbf{}

\begin{center}
	\noindent \includegraphics*[width=6.23in, height=1.62in]{IMG-11-05}\textbf{}
\end{center}
\newpage
\noindent \textbf{Example:}

\noindent \textbf{}

\begin{center}
	\noindent \includegraphics*[width=6.24in, height=4.35in]{IMG-11-06}\textbf{}

	\noindent \includegraphics*[width=6.24in, height=4.35in]{IMG-11-07}\textbf{}
\end{center}

\noindent 

\begin{center}
	\noindent \includegraphics*[width=6.25in, height=4.65in]{IMG-11-08}
\end{center}

\noindent 

\noindent 

\noindent 

\begin{center}
	\noindent \includegraphics*[width=6.22in, height=2.53in]{IMG-11-09}
\end{center}
\end{document}