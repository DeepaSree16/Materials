
\documentclass{article}

\usepackage[utf8]{inputenc} 
\usepackage[english]{babel} 
\usepackage{amsmath}
\usepackage{amssymb}
\usepackage{txfonts}
\usepackage{mathdots}
\usepackage[classicReIm]{kpfonts}
\usepackage{graphicx}
\usepackage[margin=1.0in]{geometry}



\begin{document}

\noindent \textbf{Install React ES7 Extension in Visual Studio Code}

\noindent 

\noindent\\
 Open Visual Studio Code and Click on Extensions icon in the left pane.

\noindent 

\begin{center}
	\noindent \includegraphics*[width=6.02in, height=4.22in]{IMG-07-01}
\end{center}

\noindent 

\noindent 

\noindent 

\noindent Now under the search bar, search for react and click on install button of \textbf{``ES7 React/Redux/GraphQL/React-Native snippets''.}

\noindent 

\begin{center}
	\noindent \includegraphics*[width=6.21in, height=3.34in]{IMG-07-02}
\end{center}

\noindent 

\noindent Now \textbf{``ES7 React/Redux/GraphQL/React-Native snippets''} are installed in your visual studio code. From now you can use the shortcuts for creating the class or function or any other components. Just scroll down, you can see the short cuts for all the components in React.

\noindent 

\begin{center}
	\noindent \textbf{\includegraphics*[width=6.05in, height=3.29in]{IMG-07-03}}
\end{center}

\noindent 

\noindent 

\noindent Now let us test these shortcuts in the new class component.

\noindent Create on \textbf{Test.js} component under the \textbf{src} folder

\begin{center}
	\noindent \includegraphics*[width=6.15in, height=3.22in]{IMG-07-04}
\end{center}

\noindent 

\noindent 

\noindent 

\noindent 

\noindent 

\noindent 

\noindent 

\noindent 
\newpage
\noindent Once the file is create just type \textbf{``imprc''}, it will displayed as a suggestion from the dropdown. 

\begin{center}
	\noindent \includegraphics*[width=6.06in, height=2.50in]{IMG-07-05}
\end{center}

\noindent 

\noindent 

\noindent Now select the suggestion and press enter, it will automatically import the import statement.

\begin{center}
	\noindent \includegraphics*[width=6.24in, height=1.89in]{IMG-07-06}
\end{center}

\noindent 

\noindent 

\noindent Now for class component, just type \textbf{``rafc'',} it will give the suggestion same as above. Select it and press enter, it will automatically, create the function component.

\begin{center}
	\noindent \includegraphics*[width=6.17in, height=2.31in]{IMG-07-07}
\end{center}

\noindent 

\noindent 

\noindent 

\noindent 

\noindent 

\noindent 

\noindent 

\noindent\\
\newpage
\noindent Function component is create and it will import the related import statement automatically.

\begin{center}
	\noindent \includegraphics*[width=6.25in, height=2.87in]{IMG-07-08}
\end{center}

\noindent 

\noindent 

\noindent\\
 You can see in the above images/image, export is provided to the starting of the function component. No need to give an explicit export statement.

\noindent 

\noindent\\
 \textbf{\underbar{Note:}} Try for another related shortcuts, which is directly provided in \textbf{``ES7 React/Redux/GraphQL/React-Native snippets''.}

\noindent 

\noindent\\
 Just scroll down the document and you can find all the related short cuts.

\end{document}