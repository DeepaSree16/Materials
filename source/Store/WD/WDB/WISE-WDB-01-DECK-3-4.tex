% Options for packages loaded elsewhere

\PassOptionsToPackage{unicode}{hyperref}
\PassOptionsToPackage{hyphens}{url}
%
\documentclass[
]{article}
\usepackage{amsmath,amssymb}
\usepackage{lmodern}
\usepackage{iftex}
\usepackage{graphicx}
\usepackage[margin=1.0in]{geometry}
\ifPDFTeX
  \usepackage[T1]{fontenc}
  \usepackage[utf8]{inputenc}
  \usepackage{textcomp} % provide euro and other symbols
  
\else % if luatex or xetex
  \usepackage{unicode-math}
  \defaultfontfeatures{Scale=MatchLowercase}
  \defaultfontfeatures[\rmfamily]{Ligatures=TeX,Scale=1}
\fi
% Use upquote if available, for straight quotes in verbatim environments
\IfFileExists{upquote.sty}{\usepackage{upquote}}{}
\IfFileExists{microtype.sty}{% use microtype if available
  \usepackage[]{microtype}
  \UseMicrotypeSet[protrusion]{basicmath} % disable protrusion for tt fonts
}{}
\makeatletter
\@ifundefined{KOMAClassName}{% if non-KOMA class
  \IfFileExists{parskip.sty}{%
    \usepackage{parskip}
  }{% else
    \setlength{\parindent}{0pt}
    \setlength{\parskip}{6pt plus 2pt minus 1pt}}
}{% if KOMA class
  \KOMAoptions{parskip=half}}
\makeatother
\usepackage{xcolor}
\IfFileExists{xurl.sty}{\usepackage{xurl}}{} % add URL line breaks if available
\IfFileExists{bookmark.sty}{\usepackage{bookmark}}{\usepackage{hyperref}}
\hypersetup{
  hidelinks,
  pdfcreator={LaTeX via pandoc}}
\urlstyle{same} % disable monospaced font for URLs
\setlength{\emergencystretch}{3em} % prevent overfull lines
\providecommand{\tightlist}{%
  \setlength{\itemsep}{0pt}\setlength{\parskip}{0pt}}
\setcounter{secnumdepth}{-\maxdimen} % remove section numbering
\ifLuaTeX
  \usepackage{selnolig}  % disable illegal ligatures
\fi

\author{}
\date{}

\begin{document}

{\textbf{Installing ExpressJS}}

{\textbf{}}\strut \\

Open the NodeJS Command Prompt in Admin Mode, then go the project folder
and type the following command to start the ExpressJS installation:

\hfill    

\textbf{npm install express -\/-save}

\begin{center}
	 \includegraphics*[width=4.48in, height=2.13in]{IMG-01-01}
\end{center}


After the installation is completed a node\_module folder is created,
which contains all the libraries and the ExpressJS dependencies are
added in the package.json file.

\begin{center}
	\includegraphics*[width=4.48in, height=2.13in]{IMG-01-02}
\end{center}

In the previous NodeJS project, the code of ExpressJS is added as
follows:

\begin{center}
	\includegraphics*[width=4.48in, height=2.13in]{IMG-01-03}
\end{center}


In the above program, line-3, is an API get() method used to receive get
request from the client.

\hfill    

The app.get() function routes the HTTP GET Requests to the path which is
being specified with the specified callback functions. Basically, is it
intended for binding the middleware to your application.

\hfill    

{Syntax:}

\textbf{app.get(}path\textbf{,} callback\textbf{)}

\textbf{}\strut \\

{Parameters:}

\textbf{path}: It is the path for which the middleware function is being
called.

\textbf{callback}: They can be a middleware function or series/array of
middleware functions.

\end{document}
