% Options for packages loaded elsewhere
\PassOptionsToPackage{unicode}{hyperref}
\PassOptionsToPackage{hyphens}{url}
%
\documentclass[
]{article}
\usepackage{amsmath,amssymb}
\usepackage{lmodern}
\usepackage{iftex}
\usepackage{graphicx}
\usepackage[margin=1.0in]{geometry}
\ifPDFTeX
  \usepackage[T1]{fontenc}
  \usepackage[utf8]{inputenc}
  \usepackage{textcomp} % provide euro and other symbols
\else % if luatex or xetex
  \usepackage{unicode-math}
  \defaultfontfeatures{Scale=MatchLowercase}
  \defaultfontfeatures[\rmfamily]{Ligatures=TeX,Scale=1}
\fi
% Use upquote if available, for straight quotes in verbatim environments
\IfFileExists{upquote.sty}{\usepackage{upquote}}{}
\IfFileExists{microtype.sty}{% use microtype if available
  \usepackage[]{microtype}
  \UseMicrotypeSet[protrusion]{basicmath} % disable protrusion for tt fonts
}{}
\makeatletter
\@ifundefined{KOMAClassName}{% if non-KOMA class
  \IfFileExists{parskip.sty}{%
    \usepackage{parskip}
  }{% else
    \setlength{\parindent}{0pt}
    \setlength{\parskip}{6pt plus 2pt minus 1pt}}
}{% if KOMA class
  \KOMAoptions{parskip=half}}
\makeatother
\usepackage{xcolor}
\IfFileExists{xurl.sty}{\usepackage{xurl}}{} % add URL line breaks if available
\IfFileExists{bookmark.sty}{\usepackage{bookmark}}{\usepackage{hyperref}}
\hypersetup{
  hidelinks,
  pdfcreator={LaTeX via pandoc}}
\urlstyle{same} % disable monospaced font for URLs
\setlength{\emergencystretch}{3em} % prevent overfull lines
\providecommand{\tightlist}{%
  \setlength{\itemsep}{0pt}\setlength{\parskip}{0pt}}
\setcounter{secnumdepth}{-\maxdimen} % remove section numbering
\ifLuaTeX
  \usepackage{selnolig}  % disable illegal ligatures
\fi

\author{}
\date{}

\begin{document}

{\textbf{Creating NodeJS Project}}

{\textbf{}}\strut \\

Demo "Connection Meetup" Node app

\hfill    

\textbf{Step-1:}

Go to NodeJS Command Prompt and create a folder called ``nodejs''.

md nodejs

\hfill    

\textbf{Step-2:}

Go inside the project and create another folder app

cd nodejs

md app

\hfill    

\textbf{Step-3:}

Now open the NodeJS project folder inside the Visual Studio Code (IDE).

\hfill    

\textbf{Step-4:}

Open Terminal from Visual Studio Code as (View{~ }Terminal), then type
the following command to install package.json file.

\hfill    

npm init

\begin{center}
	\includegraphics*[width=4.48in, height=2.13in]{IMG-01-04}
\end{center}
\begin{center}
	\includegraphics*[width=4.48in, height=2.13in]{IMG-01-05}
\end{center}
\begin{center}
	\includegraphics*[width=4.48in, height=2.13in]{IMG-01-06}
\end{center}

{\textbf{Code Explanation:}}

{\textbf{}}\strut \\

\begin{itemize}
\tightlist
\item
  As a first step in the program, we need to import http using
  require()→ Line 1
\item
  This http variable contains a function called create Server. This is
  all you need to do to create an http server.
\item
  Line 2, 3, 4, 5:~This function is a call-back → We use the response
  variable that's passed in to the call-back to write the head and pass
  in the content type and we end that response with connection meetups.
  → This function returns an object that we are going to put in to our
  server variable and this object is going to have another function
  called ``listen'' →
\item
  Line 7:~the minute that we call listen that's when our server is going
  to start running as we listen to this port. You can choose any port.
\item
  By default, HTTP uses port80 but ideally you should specify a port.
\item
  You can then run your server by going to your terminal and typing:
\item
  node node/app.js
\end{itemize}

\begin{center}
	\includegraphics*[width=4.48in, height=2.13in]{IMG-01-07}
\end{center}


{\textbf{Note:}}{~}

In the above output HTML tags are not affected because content-type is
set to ``plain''.{~ }It has to be replaced with ``html'' as follows.

\begin{center}
	\includegraphics*[width=4.48in, height=2.13in]{IMG-01-08}
\end{center}


\end{document}
