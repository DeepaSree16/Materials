
\documentclass{article}

\usepackage[utf8]{inputenc} 
\usepackage[english]{babel} 
\usepackage{amsmath}
\usepackage{amssymb}
\usepackage{txfonts}
\usepackage{mathdots}
\usepackage[classicReIm]{kpfonts}
\usepackage{graphicx}
\usepackage[margin=1.0in]{geometry}



\begin{document}

\noindent \textbf{State Management in Class Component}

\noindent 

\noindent \\
In our previous session, we have discussed how to implement state management using the setState in the Class Component. Now let us discuss about the state management in the functional component.

\noindent 

\noindent \\
Create a file called \textbf{StateEx.js} in the \textbf{src} folder.

\begin{center}
	\noindent \includegraphics*[width=4.12in, height=2.45in]{IMG-09-01}
\end{center}

\noindent 

\noindent \\
Now write the following code:

\begin{center}
	\noindent \includegraphics*[width=5.92in, height=4.58in]{IMG-09-02}
\end{center}

\noindent 

\noindent 

\noindent 

\noindent 

\noindent \\
Now call the \textbf{StateEx.js} from the \textbf{App.js}

\begin{center}
	\noindent \includegraphics*[width=6.26in, height=3.40in]{IMG-09-03}
\end{center}

\noindent 

\noindent \\
Now run the application and see the output in the browser.

\begin{center}
	\noindent \includegraphics*[width=5.25in, height=3.34in]{IMG-09-04}
\end{center}

\noindent 

\noindent \\
In the above output, it is not printing the Boolean value. It is printing the name, age and marital Status but not Boolean value.\\


\noindent 

\noindent $\mathrm{<}$h1$\mathrm{>}$~$\mathrm{\{}$this.state.maritalStatus$\mathrm{\}}$~$\mathrm{<}$/h1$\mathrm{>}$

\noindent 

\noindent \\
As we have provided the Boolean value directly in the expression as shown in the above statement. Expressions doesn't support Boolean values, we need convert them into the string type using the ``\textbf{JSON.stringify()''} method, as provided in the below statement.\\


\noindent 

\noindent $\mathrm{<}$h1$\mathrm{>}$~$\mathrm{\{}$JSON.stringify(this.state.maritalStatus)$\mathrm{\}}$~$\mathrm{<}$/h1$\mathrm{>}$

\noindent 

\noindent 

\noindent \\
Modifying the StateEx.js file with the \textbf{``JSON.stringify()''} method as:

\begin{center}
	\noindent \includegraphics*[width=5.98in, height=4.07in]{IMG-09-05}
\end{center}

\noindent 

\noindent \\
You can see the output in the browser.

\begin{center}
	\noindent \includegraphics*[width=4.06in, height=2.62in]{IMG-09-06}
\end{center}

\noindent 

\noindent 

\noindent \\
As you can see, now we are getting the Boolean value.

\noindent 

\noindent 

\noindent 

\noindent 

\noindent 

\noindent \\
Now in the above example I want add address, modify the \textbf{StateEx.js}. Here the export is provided directly to the class declaration.

\begin{center}
	\noindent \includegraphics*[width=5.10in, height=4.60in]{IMG-09-07}
\end{center}

\noindent 

\noindent \\
See the output in the browser. It will display nothing, because, we cannot print the object in the react directly.

\noindent 

\begin{center}
	\noindent \includegraphics*[width=3.83in, height=2.46in]{IMG-09-08}
\end{center}

\noindent 

\noindent 

\noindent \\
In order to print the address, we need to provide the individual properties from the address object.

\noindent 

\noindent 

\noindent 

\noindent \\
Modify the \textbf{StateEx.js}, to print address values as:

\begin{center}
	\noindent \includegraphics*[width=6.02in, height=5.33in]{IMG-09-09}
\end{center}

\noindent 
\newpage
\noindent \\
See the output in the browser.

\begin{center}
	\noindent \includegraphics*[width=4.82in, height=4.43in]{IMG-09-10}
\end{center}

\noindent 

\noindent 

\noindent \\
Now you can see, we are able to print address properties (doorNo and street).


\newpage
\noindent \\
\textbf{Example2:}

\noindent \\
We will modify the above StateEx.js, to add the array of employees as:

\begin{center}
	\noindent \includegraphics*[width=5.99in, height=6.45in]{IMG-09-11}
\end{center}

\noindent 

\begin{center}
	\noindent \includegraphics*[width=5.93in, height=4.54in]{IMG-09-12}
\end{center}

\noindent 

\noindent \\
The complete StateEx.js is provided in two images/images. Please check the continuity of the code.

\noindent \\
Here in the above code in line number 32, we have added a method called \textbf{``map()''}, which is used to iterate the employees collection. Just like \textbf{``*ngFor''} in angular and \textbf{``foreach''} in java.

\noindent 

\noindent \\
As we have provided some bootstrap classes in line number 20, 21 and 23, let us add bootstrap to our application.

\noindent 

\noindent \\
First go to google and search for \textbf{``bootstrap 4 cdn links''. } And click on the first link.

\begin{center}
	\noindent \includegraphics*[width=6.26in, height=2.39in]{IMG-09-13}
\end{center}

\noindent 

\noindent 
Now copy the \textbf{``CSS''} code:

\begin{center}
	\noindent \includegraphics*[width=6.24in, height=3.36in]{IMG-09-14}
\end{center}

\noindent 

\noindent \\
Either you can select the code and do the manual copy or you can see the copy option provided at the right side top corner of the CSS code.

\noindent 

\noindent \\
Once the code is copied, paste the code in \textbf{``index.html''}, from the \textbf{public} folder of your project.

\begin{center}
	\noindent \includegraphics*[width=6.24in, height=3.26in]{IMG-09-15}
\end{center}

\noindent 

\noindent \\
If you are not able to paste the code, stop the execution of your project, once again manually copy the \textbf{``CSS''} code from the above website and try one more time.

\noindent 

\noindent \\
Once you are done with pasting the CDN link of the bootstrap, save the \textbf{index.html} file.

\noindent 

\noindent 
\textbf{App.js: }Call the \textbf{StateEx.js} from the \textbf{App.js file:}

\begin{center}
	\noindent \includegraphics*[width=5.40in, height=4.08in]{IMG-09-16}
\end{center}

\noindent 

\noindent 

\noindent 

\noindent 

\noindent 

\noindent \\
Now run the application and see the output in the browser.

\begin{center}
	\noindent \includegraphics*[width=5.37in, height=2.92in]{IMG-09-10IMG-09-17}
\end{center}

\noindent 

\noindent \\
Till now we have used bootstrap to design our output in the browser. But in react, we are having a dedicated ReactStrap, to design the UI. Let us see how to add this ReactStrap in our application.

\noindent 

\noindent \\
Stop your Application by providing the command in the terminal as: \textbf{``Ctrl + C''.}  Once the application execution is stopped, then let us install the ReactStrap from terminal of the Visual Studio code.

\noindent 

\noindent \\
\textbf{Go to google and search for ReactStrap.}

\begin{center}
	\noindent \includegraphics*[width=4.56in, height=2.34in]{IMG-09-18}
\end{center}

\noindent 

\begin{center}
	\noindent \includegraphics*[width=4.75in, height=2.53in]{IMG-09-19}
\end{center}

\noindent 

\noindent \\
Scroll down the web page you can see the commands for adding the bootstrap and reactstrap.

\begin{center}
	\noindent \includegraphics*[width=6.13in, height=1.58in]{IMG-09-20}
\end{center}

\noindent 

\noindent \\
Open the terminal from the Visual Studio code and provide the commands to install bootstrap and reactstrap. We can add both bootstrap and reactstrap in one command as:

\begin{center}
	\noindent \includegraphics*[width=6.15in, height=3.21in]{IMG-09-21}
\end{center}

\noindent 

\noindent \\
You can see installation is started for both bootstrap and reactstrap. 

\begin{center}
	\noindent \includegraphics*[width=6.20in, height=2.32in]{IMG-09-22}
\end{center}

\noindent 

\noindent 

\noindent \\
Once installation is completed. Run the application as \textbf{``npm start''.}

\noindent 

\noindent 

\noindent 

\noindent 

\noindent 
\newpage
\noindent 
Open \textbf{``index.js''} file and import the \textbf{bootstrap.min.css} file init.

\begin{center}
	\noindent \includegraphics*[width=5.40in, height=2.40in]{IMG-09-23}
\end{center}

\noindent 

\noindent \\
Now create a file called \textbf{ReactStrapCP.js} in the \textbf{src} folder and write the following code:

\begin{center}
	\noindent \includegraphics*[width=5.04in, height=3.20in]{IMG-09-24}
\end{center}

\noindent 

\noindent \\
Now call the \textbf{``ReactStrapCP.js''} from \textbf{``App.js''} file:

\begin{center}
	\noindent \includegraphics*[width=4.71in, height=3.21in]{IMG-09-25}
\end{center}

\noindent 

\noindent \\
Now run the application and you can see the output in the browser.

\begin{center}
	\noindent \includegraphics*[width=4.08in, height=2.47in]{IMG-09-26}
\end{center}

\end{document}