% Options for packages loaded elsewhere
\PassOptionsToPackage{unicode}{hyperref}
\PassOptionsToPackage{hyphens}{url}
%
\documentclass[
]{article}
\usepackage{amsmath,amssymb}
\usepackage{lmodern}
\usepackage{iftex}
\usepackage{graphicx}
\usepackage[margin=1.0in]{geometry}
\ifPDFTeX
  \usepackage[T1]{fontenc}
  \usepackage[utf8]{inputenc}
  \usepackage{textcomp} % provide euro and other symbols
\else % if luatex or xetex
  \usepackage{unicode-math}
  \defaultfontfeatures{Scale=MatchLowercase}
  \defaultfontfeatures[\rmfamily]{Ligatures=TeX,Scale=1}
\fi
% Use upquote if available, for straight quotes in verbatim environments
\IfFileExists{upquote.sty}{\usepackage{upquote}}{}
\IfFileExists{microtype.sty}{% use microtype if available
  \usepackage[]{microtype}
  \UseMicrotypeSet[protrusion]{basicmath} % disable protrusion for tt fonts
}{}
\makeatletter
\@ifundefined{KOMAClassName}{% if non-KOMA class
  \IfFileExists{parskip.sty}{%
    \usepackage{parskip}
  }{% else
    \setlength{\parindent}{0pt}
    \setlength{\parskip}{6pt plus 2pt minus 1pt}}
}{% if KOMA class
  \KOMAoptions{parskip=half}}
\makeatother
\usepackage{xcolor}
\IfFileExists{xurl.sty}{\usepackage{xurl}}{} % add URL line breaks if available
\IfFileExists{bookmark.sty}{\usepackage{bookmark}}{\usepackage{hyperref}}
\hypersetup{
  hidelinks,
  pdfcreator={LaTeX via pandoc}}
\urlstyle{same} % disable monospaced font for URLs
\setlength{\emergencystretch}{3em} % prevent overfull lines
\providecommand{\tightlist}{%
  \setlength{\itemsep}{0pt}\setlength{\parskip}{0pt}}
\setcounter{secnumdepth}{-\maxdimen} % remove section numbering
\ifLuaTeX
  \usepackage{selnolig}  % disable illegal ligatures
\fi

\author{}
\date{}

\begin{document}

{\textbf{Development CRUD API in NodeJS}}

{\textbf{}}\strut \\

2) Download the following "node modules" to implement server (node
server).

{~ ~ }=\textgreater{} express

{~ ~ }=\textgreater{} mongodb@2.2.32

{~ ~ }=\textgreater{} cors

{~ ~ }=\textgreater{} body-parser

\hfill    

{~ ~ }- "express" module helps to develop the "rest apis"

{~ ~ }- "mongodb@2.2.32" module helps to interact with "mongodb
database".

{~ ~ }- "cors" module enable the communication between "ports".

{~ ~ }- "body-parser" module helps to read the "client data".

{~}

{~ ~ }\textgreater{} npm install express mongodb@2.2.232 cors
body-parser -\/-save

\hfill    

3) Develop the "restapi" by using "NodeJS".

\hfill    


{\textbf{CRUD}}

\begin{center}
	\includegraphics*[width=4.48in, height=2.13in]{IMG-05-01}
\end{center}

\hfill    
\newpage
{\textbf{Step-4}}

{~~ }Start the mongo server

{~~ }\textgreater{} mongod

{~ ~}

\hfill    

\begin{center}
	\includegraphics*[width=4.48in, height=2.13in]{IMG-05-02}
\end{center}


\hfill    

{\textbf{server.js}}

Create a file server.js and write the following code:

//node starts the execution from server.js file

//this file acting as main server file

//this file used to collabrate the modules

//@login{~ }@register{~ }@update{~ ~ }@delete

\hfill    

//import express module

let express = require("express");

//import body-parser module

let bodyparser = require("body-parser");

//import cors module

let cors = require("cors");

//create the rest object

let app = express();

//enable the ports communication

app.use(cors());

//set the JSON as MIME Type

app.use(bodyparser.json());

//read the json

app.use(bodyparser.urlencoded(\{extended:false\}));

//use login module

app.use("/login",require("./login/login"));

//use register module

app.use("/register",require("./register/register"));

\hfill    



//use register module

app.use("/update",require("./update/update"));

//use register module

app.use("/delete",require("./delete/delete"));

\hfill    

//use fetch module

app.use("/fetch",require("./fetch/fetch"));

\hfill    

//assign the port no

app.listen(8080);

console.log("server listening the port no.8080");

\hfill    

\newpage
{\textbf{fetch.js}}

//this file acting as module (fetch)

//login module accepts the{~ }from angular

//login module fetch from "mongodb" database

\hfill    



let mongodb = require("mongodb");

let talentsprint = mongodb.MongoClient; {~}

\hfill    

let login{~ }=
require("express").Router().get("/",(req,res)=\textgreater\{

{~ ~ ~ ~
}talentsprint.connect("mongodb://localhost:27017/crud",(err,db)=\textgreater\{

{~ ~ ~ ~ ~ ~ }if(err)\{

{~ ~ ~ ~ ~ ~ ~ ~ }throw err;

{~ ~ ~ ~ ~ ~ }\}

{~ ~ ~ ~ ~ ~ }else\{

{~ ~ ~ ~ ~ ~ ~ ~
}db.collection("employee").find(\{\}).toArray((err,array)=\textgreater\{

{~ ~ ~ ~ ~ ~ ~ ~ ~ ~ }if(err) throw err;

{~ ~ ~ ~ ~ ~ ~ ~ ~ ~ }else\{

{~ ~ ~ ~ ~ ~ ~ ~ ~ ~ ~ ~ }if(array.length\textgreater0)\{

{~ ~ ~ ~ ~ ~ ~ ~ ~ ~ ~ ~ ~ ~ }res.send(array); {~ ~ ~ ~ ~ ~ ~ ~ ~ ~}

{~ ~ ~ ~ ~ ~ ~ ~ ~ ~ ~ ~ }\} else \{

{~ ~ ~ ~ ~ ~ ~ ~ ~ ~ ~ ~ ~ ~ }res.send(\{message:"Record Not
Found..."\});

{~ ~ ~ ~ ~ ~ ~ ~ ~ ~ ~ ~ }\}

{~ ~ ~ ~ ~ ~ ~ ~ ~ ~ }\}

{~ ~ ~ ~ ~ ~ ~ ~ }\});

{~ ~ ~ ~ ~ ~ }\}

{~ ~ ~ ~ }\});

\});

module.exports = login;

\hfill    

\newpage
{\textbf{login.js}}

{\textbf{}}\strut \\

//this file acting as module (login)

//login module accepts the loginId\&password from angular

//login module compares with "mongodb" database

\hfill    

let mongodb = require("mongodb");

let talentsprint = mongodb.MongoClient; {~}

\hfill    

let login{~ }=
require("express").Router().get("/:loginId/:password",(req,res)=\textgreater\{

{~ ~ ~ ~
}talentsprint.connect("mongodb://localhost:27017/crud",(err,db)=\textgreater\{

{~ ~ ~ ~ ~ ~ }if(err)\{

{~ ~ ~ ~ ~ ~ ~ ~ }throw err;

{~ ~ ~ ~ ~ ~ }\}

{~ ~ ~ ~ ~ ~ }else\{

{~ ~ ~ ~ ~ ~ ~ ~
}db.collection("employee").findOne(\{"loginId":req.params.loginId,"password":req.params.password\},{~}

{~ ~ ~ ~ ~ ~ ~ ~ }(err, result) =\textgreater{} \{

{~ ~ ~ ~ ~ ~ ~ ~ ~ ~ }if (err) \{

{~ ~ ~ ~ ~ ~ ~ ~ ~ ~ ~ ~ }throw err;

{~ ~ ~ ~ ~ ~ ~ ~ ~ ~ }\}

{~ ~ ~ ~ ~ ~ ~ ~ ~ ~ }else\{

{~ ~ ~ ~ ~ ~ ~ ~ ~ ~ ~ ~ ~ ~ }res.send(result); {~ ~ ~ ~ ~ ~ ~ ~ ~ ~}

{~ ~ ~ ~ ~ ~ ~ ~ ~ ~ ~ ~ }\}

{~ ~ ~ ~ ~ ~ ~ ~ }\});

{~ ~ ~ ~ ~ ~ }\}

{~ ~ ~ ~ }\});

\});

module.exports = login;

\hfill    
\newpage

{\textbf{register.js}}

Create a file register.js and write the following code:

//this file acting as module (register)

//register module accepts the name, salary, loginId \& password from
angular

//register module persist the data "mongodb" database

\hfill    

let mongodb = require("mongodb");

let talentsprint = mongodb.MongoClient;

\hfill    

let register{~ }=
require("express").Router().post("/",(req,res)=\textgreater\{

{~ ~ ~ ~
}talentsprint.connect("mongodb://localhost:27017/crud",(err,db)=\textgreater{}

{~ ~ ~ ~ }\{

{~ ~ ~ ~ ~ ~ }if(err)

{~ ~ ~ ~ ~ ~ ~ ~ }throw err;

{~ ~ ~ ~ ~ ~ }else\{

{~ ~ ~ ~ ~ ~ ~ ~
}db.collection("employee").insertOne(\{"name":req.body.name,"salary":req.body.salary,"loginId":req.body.loginId,"password":req.body.password\},

{~ ~ ~ ~ ~ ~ ~ ~ }(err, result)=\textgreater{} \{

{~ ~ ~ ~ ~ ~ ~ ~ ~ ~ }if (err) throw err;

{~ ~ ~ ~ ~ ~ ~ ~ ~ ~ }res.send(\{message:"1 document inserted"\});

{~ ~ ~ ~ ~ ~ ~ ~ ~ ~ }db.close();

{~ ~ ~ ~ ~ ~ ~ ~ ~ }\});

{~ ~ ~ ~ ~ ~ ~ ~ }\}

{~ ~ ~ ~ ~ ~ }\});

{~ ~ ~ ~ ~ ~ }\});

module.exports = register;


\hfill    

\newpage
{\textbf{update.js}}

Create a file update.js and write the following code

//this file acting as module (update)

//update module accepts the id, name, loginId \& password from angular

//update module compares id in "mongodb" database and update the row

\hfill    

let mongodb = require("mongodb");

let talentsprint = mongodb.MongoClient;

\hfill    

let update{~ }=
require("express").Router().put("/",(req,res)=\textgreater\{

{~ ~ ~ ~
}talentsprint.connect("mongodb://localhost:27017/crud",(err,db)=\textgreater{}

{~ ~ ~ ~ }\{

{~ ~ ~ ~ ~ ~ }if(err)

{~ ~ ~ ~ ~ ~ ~ ~ }throw err;

{~ ~ ~ ~ ~ ~ }else\{ {~ ~ ~ ~ ~ ~ ~ ~}

var newvalues = \{ \$set: \{name:req.body.name, salary:req.body.salary\}
\};

{~ ~ ~ ~ ~ ~ ~ ~
}db.collection("employee").updateOne(\{loginId:req.body.loginId\},newvalues,
(err, result)=\textgreater{} \{

{~ ~ ~ ~ ~ ~ ~ ~ ~ ~ ~ ~ }if (err){~}

{~ ~ ~ ~ ~ ~ ~ ~ ~ ~ ~ ~ }throw err;

{~ ~ ~ ~ ~ ~ ~ ~ ~ ~ ~ ~ }else

{~ ~ ~ ~ ~ ~ ~ ~ ~ ~ ~ ~ }res.send(\{message:"1 document updated"\});

{~ ~ ~ ~ ~ ~ ~ ~ ~ ~ ~ ~ }db.close();

{~ ~ ~ ~ ~ ~ ~ ~ ~ ~ ~ }\});

{~ ~ ~ ~ ~ ~ ~ ~ ~ ~ }\}

{~ ~ ~ ~ ~ ~ ~ ~ }\});

{~ ~ ~ ~ ~ ~ ~ }\});

\hfill    

module.exports = update;

\hfill    
\newpage

{\textbf{delete.js}}

Create a file delete.js and write the following code

//this file acting as module (delete)

//delete module accepts the id from angular

//delete module compares id with "mongodb" database

//delete module delete the matching record

\hfill    

let mongodb = require("mongodb");

let talentsprint = mongodb.MongoClient;

\hfill    

let del{~ }=
require("express").Router().delete("/",(req,res)=\textgreater\{

{~ ~ ~ ~
}talentsprint.connect("mongodb://localhost:27017/miniproject",(err,db)=\textgreater{}

{~ ~ ~ ~ }\{

{~ ~ ~ ~ ~ ~ }if(err)

{~ ~ ~ ~ ~ ~ ~ ~ }throw err;

{~ ~ ~ ~ ~ ~ }else\{

{~ ~ ~ ~ ~ ~ ~ ~
}db.collection("employee").deleteOne(\{loginId:req.body.loginId\}, (err,
result)=\textgreater{} \{

{~ ~ ~ ~ ~ ~ ~ ~ ~ ~ ~ ~ }if (err){~}

{~ ~ ~ ~ ~ ~ ~ ~ ~ ~ ~ ~ }throw err;

{~ ~ ~ ~ ~ ~ ~ ~ ~ ~ ~ ~ }else

{~ ~ ~ ~ ~ ~ ~ ~ ~ ~ ~ ~ }res.send(\{message:"1 document deleted"\});

{~ ~ ~ ~ ~ ~ ~ ~ ~ ~ ~ ~ }db.close();

{~ ~ ~ ~ ~ ~ ~ ~ ~ ~ ~ }\});

{~ ~ ~ ~ ~ ~ ~ ~ ~ ~ }\}

{~ ~ ~ ~ ~ ~ ~ ~ }\});

{~ ~ ~ ~ ~ ~ ~ ~ }\});

\hfill    

module.exports = del;

\hfill    

\end{document}
