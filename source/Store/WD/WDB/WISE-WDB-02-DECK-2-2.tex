% Options for packages loaded elsewhere
\PassOptionsToPackage{unicode}{hyperref}
\PassOptionsToPackage{hyphens}{url}
%
\documentclass[
]{article}
\usepackage{amsmath,amssymb}
\usepackage{lmodern}
\usepackage{iftex}
\usepackage{graphicx}
\usepackage[margin=1.0in]{geometry}
\ifPDFTeX
  \usepackage[T1]{fontenc}
  \usepackage[utf8]{inputenc}
  \usepackage{textcomp} % provide euro and other symbols
\else % if luatex or xetex
  \usepackage{unicode-math}
  \defaultfontfeatures{Scale=MatchLowercase}
  \defaultfontfeatures[\rmfamily]{Ligatures=TeX,Scale=1}
\fi
% Use upquote if available, for straight quotes in verbatim environments
\IfFileExists{upquote.sty}{\usepackage{upquote}}{}
\IfFileExists{microtype.sty}{% use microtype if available
  \usepackage[]{microtype}
  \UseMicrotypeSet[protrusion]{basicmath} % disable protrusion for tt fonts
}{}
\makeatletter
\@ifundefined{KOMAClassName}{% if non-KOMA class
  \IfFileExists{parskip.sty}{%
    \usepackage{parskip}
  }{% else
    \setlength{\parindent}{0pt}
    \setlength{\parskip}{6pt plus 2pt minus 1pt}}
}{% if KOMA class
  \KOMAoptions{parskip=half}}
\makeatother
\usepackage{xcolor}
\IfFileExists{xurl.sty}{\usepackage{xurl}}{} % add URL line breaks if available
\IfFileExists{bookmark.sty}{\usepackage{bookmark}}{\usepackage{hyperref}}
\hypersetup{
  hidelinks,
  pdfcreator={LaTeX via pandoc}}
\urlstyle{same} % disable monospaced font for URLs
\setlength{\emergencystretch}{3em} % prevent overfull lines
\providecommand{\tightlist}{%
  \setlength{\itemsep}{0pt}\setlength{\parskip}{0pt}}
\setcounter{secnumdepth}{-\maxdimen} % remove section numbering
\ifLuaTeX
  \usepackage{selnolig}  % disable illegal ligatures
\fi

\author{}
\date{}

\begin{document}

{\textbf{Reading Data from JSON File and Print in Browser}}

{\textbf{}}\strut \\

Create a Data file with JSON Object as follows:{~}

\begin{center}
	\includegraphics*[width=4.48in, height=2.13in]{IMG-02-01}
\end{center}

Write NodeJS code to read data from the JSON file.

\begin{center}
	\includegraphics*[width=4.48in, height=2.13in]{IMG-02-02}
\end{center}

In the above code, line-3, you can see that we are using require()
method, which is taking the path of the JSON file as the parameter to
open it.

line-9, you can see forEach() is an array function from Node.js that is
used to iterate over items in a given array.

\hfill    

{Syntax:}

array\_name.forEach(function)

\hfill    

\textbf{Parameter}: This function takes a function (which is to be
executed) as a parameter.

\textbf{Return type}: The function returns array element after
iteration.

\begin{center}
	\includegraphics*[width=4.48in, height=2.13in]{IMG-02-03}
\end{center}

\end{document}
