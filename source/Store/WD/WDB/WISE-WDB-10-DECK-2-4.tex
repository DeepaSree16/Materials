
\documentclass{article}

\usepackage[utf8]{inputenc} 
\usepackage[english]{babel} 
\usepackage{amsmath}
\usepackage{amssymb}
\usepackage{txfonts}
\usepackage{mathdots}
\usepackage[classicReIm]{kpfonts}
\usepackage{graphicx}
\usepackage[margin=1.0in]{geometry}



\begin{document}


\noindent \textbf{useState() in React}

\noindent 

\noindent \\
It~is a Hook that allows you to maintain the state variables in functional components.  The useState hook is a special function that takes the initial state as an argument and returns an array of two entries.

\noindent 

\noindent 

\noindent \textbf{Declaring state in React}

\noindent \\
useState~is a named export from~react. To use it, you can write:

\noindent React.useState

\noindent \\
Or to import it just write~useState:

\noindent \\
import React, $\mathrm{\{}$ useState $\mathrm{\}}$ from 'react';

\noindent ~

\noindent 

\noindent 

\noindent \\
\textbf{Syntax:}

\noindent \\
const [state, setState] = useState(initialstate)

\noindent 

\noindent \\
\textbf{~}The first element is the initial state and the second one is a function that is used for updating the state.

\noindent 

\noindent \\
We can also pass a function as an argument if the initial state has to be computed. And the value returned by the function will be used as the initial state.

\noindent \\
Ex:-

\noindent Const [message,setMessage] = useState(`UI Sessions');

\noindent 

\noindent Const changeName = ()=$\mathrm{>}$$\mathrm{\{}$

\noindent    setMessage(`React Sessions');

\noindent $\mathrm{\}}$

\noindent 

\noindent 

\noindent 

\noindent 

\noindent 

\noindent 

\noindent 

\noindent 



\noindent \\
Example:-

\noindent 

\begin{center}
	\noindent \includegraphics*[width=6.25in, height=3.77in]{IMG-10-01}
\end{center}

\noindent 

\noindent 

\noindent 

\noindent 

\begin{center}
	\noindent \includegraphics*[width=6.27in, height=5.00in]{IMG-10-02}
\end{center}

\noindent 
\newpage
\noindent \\
In the above example it is not rendered the updated output to the browser because a normal function cannot re-render the output to the browser.  Here the useState() will come into the picture and make the changes using useState() as follows.

\noindent 

\noindent 

\noindent 

\noindent 

\begin{center}
	\noindent \includegraphics*[width=6.21in, height=3.78in]{IMG-10-03}
\end{center}
\noindent \\
\begin{center}
	\noindent \includegraphics*[width=6.25in, height=3.90in]{IMG-10-04}
\end{center}
\end{document}