
\documentclass{article}

\usepackage[utf8]{inputenc} 
\usepackage[english]{babel}
\usepackage{amsmath}
\usepackage{amssymb}
\usepackage{txfonts}
\usepackage{mathdots}
\usepackage[classicReIm]{kpfonts}
\usepackage{graphicx}
\usepackage[margin=1.0in]{geometry}




\begin{document}



\noindent \textbf{\underbar{MongoDB}}

\noindent \\ MongoDB is a document database designed for ease of development and scaling.

\noindent 

\noindent \\ A record in MongoDB is a document, which is a data structure composed of field and value pairs. MongoDB documents are similar to JSON objects. The values of fields may include other documents, arrays, and arrays of documents.

\noindent 

\noindent 

\noindent \\ \textbf{\underbar{The advantages of using documents are}}

\begin{enumerate}
\item \textbf{\underbar{ }}Documents (objects) correspond to native data types in many programming languages.

\item  Embedded documents and arrays reduce need for expensive joins.

\item  Dynamic schema supports fluent polymorphism.
\end{enumerate}

\noindent 

\noindent 

\noindent \\ MongoDB stores documents in collections. Collections are analogous to tables in relational databases. MongoDB provides high performance data persistence. In particular, it supports for embedded data models, reduces I/O activity on database system.

\noindent 

\noindent 

\noindent \\ 1) Make the database ready (mongodb)

\noindent     - mongodb database "NoSQL" database.

\noindent     - mongodb database is the JSON (document) Based database.

\noindent     - mongodb database is the light weight database.

\noindent     - mongodb database by default running on port no.27017

\noindent     - mongodb database follows the "mongodb" protocol.

\noindent 

\noindent 



\noindent \\ \textbf{\underbar{Installation of MongoDB}}

\noindent 

\noindent 
\subparagraph{Download the installer.}

\begin{enumerate}
\item \textbf{ } Download the MongoDB~Community~.msi~installer from the following link:\\ \\ https://www.mongodb.com/try/download/community?tck=docs\_server\&\_ga=2.206736150.408178942.1638815668-852836268.1638815668
\end{enumerate}

\noindent  

\begin{enumerate}
\item  In the~\textbf{Version}~dropdown, select the version of MongoDB to download.

\item  In the~\textbf{Platform}~dropdown, select~\textbf{Windows}.

\item  In the~\textbf{Package}~dropdown, select~\textbf{msi}.

\item  Click~\textbf{Download}.
\end{enumerate}

\noindent 


\noindent \\ \textbf{Run the MongoDB installer}

\noindent \\ For example, from the Windows Explorer/File Explorer:

\begin{enumerate}
\item Go to the directory where you downloaded the MongoDB installer (.msi~file). By default, this is your~Downloads~directory.

\item Double-click the~.msi~file.
\end{enumerate}
\newpage 
\noindent \textbf{Follow the MongoDB~Community~Edition installation wizard.}

\noindent \\ The wizard steps you through the installation of MongoDB and MongoDB Compass.

\begin{enumerate}
\item  \textbf{Choose Setup Type}
\end{enumerate}

\noindent \\ You can choose either the~\textbf{Complete}~(recommended for most users) or~\textbf{Custom}~setup type. The~\textbf{Complete}~setup option installs MongoDB and the MongoDB tools to the default location. The~\textbf{Custom}~setup option allows you to specify which executables are installed and where.

\begin{enumerate}
\item \textbf{Service Configuration}
\end{enumerate}

\noindent  Starting in MongoDB 4.0, you can set up MongoDB as a Windows service during the install or just install the binaries.

\noindent 

\begin{center}
	\noindent \includegraphics*[width=6.25in, height=4.52in]{IMG-04-01}
\end{center}

\noindent 

\noindent 

\begin{enumerate}
\item \begin{enumerate}
\item  Select~\textbf{Install MongoD as a Service}~MongoDB as a service.

\item  Select either:

\begin{enumerate}
\item   \textbf{Run the service as Network Service user}~(Default)
\end{enumerate}
\end{enumerate}
\end{enumerate}
\newpage
\noindent \\ This is a Windows user account that is built-in to Windows

\textbf{or}

\begin{enumerate}
\item \begin{enumerate}
\item \begin{enumerate}
\item  \textbf{Run the service as a local or domain user}

\begin{enumerate}
\item  For an existing local user account, specify a period (i.e.~.) for the~\textbf{Account Domain}~and specify the~\textbf{Account Name}~and the~\textbf{Account Password}~for the user.

\item  For an existing domain user, specify the~\textbf{Account Domain}, the~\textbf{Account Name}~and the~\textbf{Account Password}~for that user.
\end{enumerate}
\end{enumerate}

\item  \textbf{Service Name}. Specify the service name. Default name is~MongoDB. If you already have a service with the specified name, you must choose another name.

\item  \textbf{Data Directory}. Specify the data directory, which corresponds to the~--dbpath. If the directory does not exist, the installer will create the directory and sets the directory access to the service user.

\item  \textbf{Log Directory}. Specify the Log directory, which corresponds to the~--logpath. If the directory does not exist, the installer will create the directory and sets the directory access to the service user.
\end{enumerate}
\end{enumerate}

\noindent \textbf{}

\noindent \\ \textbf{Install MongoDB Compass}

\noindent \textit{Optional}. To have the wizard install~MongoDB Compass, select~\textbf{Install MongoDB Compass}~(Default).

\begin{enumerate}
\item  When ready, click~\textbf{Install}.
\end{enumerate}

\noindent 

\noindent 
\paragraph{Create database directory.}

\noindent \\ Create the~data directory~where MongoDB stores data. MongoDB's default data directory path is the absolute path~{\textbackslash}data{\textbackslash}db~on the drive from which you start MongoDB.

\noindent From the~\textbf{Command Interpreter}, create the data directories:

\noindent 
\paragraph{C:{\textbackslash}$\boldsymbol{\mathrm{>}}$md data}

\noindent 
\paragraph{C:{\textbackslash}$\boldsymbol{\mathrm{>}}$cd data}

\noindent 
\paragraph{C:{\textbackslash}data$\boldsymbol{\mathrm{>}}$md db}

\noindent 

\noindent 
\paragraph{Start your MongoDB database.}

\noindent \\ To start MongoDB, run~exe.

\begin{tabular}{|p{4.3in}|} \hline 
"C:{\textbackslash}Program Files{\textbackslash}MongoDB{\textbackslash}Server{\textbackslash}5.0{\textbackslash}bin{\textbackslash}mongod.exe" --dbpath="c:{\textbackslash}data{\textbackslash}db" \\ \hline 
\end{tabular}



\noindent 

\noindent \\ \textbf{\underbar{Run MongoDB}}

\noindent \\ 1. Open Folder Explorer

\noindent 2. Go to the following Path

\noindent    C:{\textbackslash}Program Files{\textbackslash}MongoDB{\textbackslash}Server{\textbackslash}5.0{\textbackslash}bin

\noindent 3. Click on Address Bar (Select the Path) and type CMD then press Enter

\noindent 4. Command Prompt will be opened in the above path.

\noindent 5. type mongod and press enter to initiate the server.

\noindent Server will be shutdown automatically

\noindent 

\noindent 

\noindent 

\begin{center}
	\noindent \includegraphics*[width=6.40in, height=3.17in]{IMG-04-02}
\end{center}

\noindent 

\noindent \\ 6. Now once again go to the mongodb bin path

\noindent \\   C:{\textbackslash}Program Files{\textbackslash}MongoDB{\textbackslash}Server{\textbackslash}5.0{\textbackslash}bin

\noindent \textbf{\underbar{}}

\begin{center}
	\noindent \textbf{\underbar{\includegraphics*[width=6.20in, height=2.03in]{IMG-04-03}}}
\end{center}



\noindent \\ \textbf{\underbar{Working on MongoDB}}

\noindent 

\noindent \\ \textbf{1. Show available/predefined databases}

\noindent \\ $\mathrm{>}$ show dbs

\noindent 

\noindent \\ \textbf{2. To create our own database}

\noindent \\ $\mathrm{>}$ use test;

\noindent $\mathrm{>}$ show dbs

\noindent 
\newpage
\noindent \\ \textbf{3. Create a table/collection under test db}

\noindent \\ $\mathrm{>}$ db.createCollection("student");

\noindent $\mathrm{>}$ show dbs

\noindent 

\noindent \\ \textbf{4. Add/insert row in student collection}

\noindent \\ Here the data will be stored in the form of JSON object.

\noindent 

\noindent \\ $\mathrm{>}$ db.student.insert($\mathrm{\{}$sname:"PASHA", course:"Java", fees:999.99$\mathrm{\}}$);   

\noindent $\mathrm{>}$ db.student.insert($\mathrm{\{}$sname:"Harsha", course:"Angular", fees:888.88$\mathrm{\}}$);

\noindent 

\noindent \\ \textbf{Display data from Student Table}

\noindent \\ $\mathrm{>}$ db.student.find();

\noindent 

\noindent 

\begin{center}
	\noindent \includegraphics*[width=7.08in, height=3.37in]{IMG-04-04}
\end{center}

\noindent 

\noindent 

\noindent 

\noindent 

\noindent 

\noindent 

\noindent 
\newpage
\noindent \\ \textbf{using pretty function}

\noindent \\ $\mathrm{>}$ db.student.find().pretty();

\noindent 

\noindent 

\begin{center}
	\noindent \includegraphics*[width=6.59in, height=2.99in]{IMG-04-05}
\end{center}

\noindent 

\noindent 

\noindent \\ \textbf{Add rows with Id}

\noindent  $\mathrm{>}$ db.student.insert($\mathrm{\{}$id:1, sname:"Indira", course:"Python", fees:777.77$\mathrm{\}}$);

\noindent $\mathrm{>}$ db.student.insert($\mathrm{\{}$id:1, sname:"Venkat", course:"Pega", fees:666.66$\mathrm{\}}$);

\noindent 

\noindent \\ \textbf{Add one more record by adding a new field}

\noindent  $\mathrm{>}$ db.student.insert($\mathrm{\{}$sname:"Vinay", course:"PEGA", fees:7777, mobile:9898958658$\mathrm{\}}$);

\noindent 
\newpage
\noindent \\ \textbf{Display the Records}

\noindent  $\mathrm{>}$ db.student.find().pretty();

\noindent 

\begin{center}
	\noindent \includegraphics*[width=6.68in, height=3.82in]{IMG-04-06}
\end{center}

\noindent 

\noindent 

\noindent 

\noindent 

\noindent 

\noindent \\ \textbf{Droping the Table}

\noindent  $\mathrm{>}$ db.student.drop();

\noindent $\mathrm{>}$ db.student.find().pretty();

\noindent $\mathrm{>}$ show collections;

\noindent 

\noindent $\mathrm{>}$ db.createCollection("student");

\noindent $\mathrm{>}$ db.student.insert($\mathrm{\{}$sname:"pasha", course:"Java", fees:999.99$\mathrm{\}}$);   

\noindent $\mathrm{>}$ db.student.insert($\mathrm{\{}$sname:"Harsha", course:"Angular", fees:888.88$\mathrm{\}}$);

\noindent $\mathrm{>}$ db.student.insert($\mathrm{\{}$sname:"Indira", course:"Python", fees:777.77$\mathrm{\}}$);

\noindent 

\noindent \\ \textbf{Insert Many}

\noindent  $\mathrm{>}$ db.student.insertMany([$\mathrm{\{}$sname:"Vinod", course:"NodeJS", fees:999.99$\mathrm{\}}$,$\mathrm{\{}$sname:"Ravi", course:"Java", fees:888.88$\mathrm{\}}$]);

\noindent \\ \textbf{Display the Records}

\noindent $\mathrm{>}$ db.student.find().pretty();

\noindent 

\noindent \\ \textbf{Display a particular record (Selected Record)}

\noindent $\mathrm{>}$ db.student.find($\mathrm{\{}$course: $\mathrm{\{}$\$eq:'Angular'$\mathrm{\}}$$\mathrm{\}}$);             

\noindent $\mathrm{>}$ db.student.find($\mathrm{\{}$fees: $\mathrm{\{}$\$gt:500$\mathrm{\}}$$\mathrm{\}}$);                      

\noindent $\mathrm{>}$ db.student.find($\mathrm{\{}$fees: $\mathrm{\{}$\$gte:500$\mathrm{\}}$$\mathrm{\}}$);                     

\noindent $\mathrm{>}$ db.student.find($\mathrm{\{}$fees: $\mathrm{\{}$\$lt:500$\mathrm{\}}$$\mathrm{\}}$);                     

\noindent $\mathrm{>}$ db.student.find($\mathrm{\{}$sname: $\mathrm{\{}$\$in: ["Pasha", "Harsha"]$\mathrm{\}}$$\mathrm{\}}$);  

\noindent $\mathrm{>}$ db.student.find($\mathrm{\{}$sname: $\mathrm{\{}$\$nin: ["Pasha", "Harsha"]$\mathrm{\}}$$\mathrm{\}}$);

\noindent 

\noindent 
\newpage
\noindent \\ \textbf{\underbar{Creating CRUD Database for Project}}

\noindent Show dbs;

\noindent db.createCollection(``employee'');

\noindent \\ $\mathrm{>}$ db.employee.insert($\mathrm{\{}$empId:''1000'', empName:''PASHA'', salary:''9999.9, loginId:''pasha@gmail.com'', \\ password:''password''$\mathrm{\}}$);   

\noindent 

\noindent \begin{center}
	\includegraphics*[width=6.24in, height=3.22in]{IMG-04-07}

\end{center}



\end{document}

