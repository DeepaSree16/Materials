
\documentclass{article}

\usepackage[utf8]{inputenc} 
\usepackage[english]{babel} 
\usepackage{amsmath}
\usepackage{amssymb}
\usepackage{txfonts}
\usepackage{mathdots}
\usepackage[classicReIm]{kpfonts}
\usepackage{graphicx}
\usepackage[margin=1.0in]{geometry}



\begin{document}
\noindent \textbf{Introduction to Reactjs}

\noindent 

\noindent \\
React is a declarative, efficient and flexible java script library for building user interface. React will efficiently update and render your components easily when your data changes. In order to learn ReactJS, you must have the knowledge on HTML and Java Script. It's `V' in the MVC architecture (mean View). React is an open-source, component based front end library responsible for the developing the User Interface (view layer) of the application. It is maintained by Facebook.

\noindent 

\noindent 

\begin{center}
	\noindent \includegraphics*[width=3.45in, height=2.12in]{IMG-06-05}
\end{center}

\noindent 

\noindent 

\noindent 

\noindent \\
\textbf{\underbar{Note}}: React is not a Framework, it is just a library developed by Facebook in 2011 in its newsfeed section, but it was officially released on \textbf{May-2013.} React was created by \textbf{Jordan} \textbf{Walke}, who is a software engineer at Facebook.

\noindent 

\noindent 

\noindent 

\noindent 

\noindent \\
Angular is a UI Framework but whereas React is a UI Library which resides on java script. React is not heavy as compared to Angular. React is a light weight component as it is a library. Whatever we can do with angular, the same thing we can do with React.  Angular contains services but React doesn't have. Angular is a product of google but React is a product of Facebook. Even though React is a library which provide very few things, but it supports additional third party libraries to carry out our various operations. Both Angular and ReactJS works on node server.

\noindent 

\noindent 

\noindent \\
The major difference we can find in the react, as it is not fully a framework, we can say React is a light weight component on top of the java script. As we know that React is working on top of the java script, React is not only java script, but is java script with xml, and we call it as JSX (JavaScript Syntax Extension) (JavaScript + JSX). React is component based library.

\noindent 

\noindent 

\noindent 

\noindent 

\noindent \textbf{}

\noindent \textbf{\underbar{}}

\noindent \textbf{\underbar{}}

\noindent \\
\textbf{\underbar{Why ReactJS:}}

\begin{enumerate}
\item \textbf{ }React is easy to learn

\item  React helps to build rich user interfaces

\item  React allows writing custom components

\item  React offers fast rendering

\item  React comes with useful developer toolset

\item  React offer's better code stability\textbf{\underbar{}}
\end{enumerate}

\noindent \textbf{}

\noindent \\
\textbf{\underbar{Development Environment}}

\noindent 

\noindent \\
\textbf{\underbar{ReactJS}}

\noindent The Official Website: https://reactjs.org/

\noindent 

\noindent \\
\textbf{\underbar{Visual Studio Code (IDE)}}

\noindent \\
The Official Website:  https://code.visualstudio.com/download

\noindent 

\noindent 

\noindent 

\noindent \textbf{}

\noindent \textbf{}

\noindent \\
\textbf{Installation (NodeJS):}


 \textbf{ }Open https://nodejs.org/en/, then click on recommended version to download the file.


\begin{center}
	\noindent \includegraphics*[width=3.49in, height=2.53in]{IMG-06-06}
\end{center}

\noindent 


  You can see the installation file is downloaded.


\begin{center}
	\noindent \includegraphics*[width=2.52in, height=1.17in]{IMG-06-07}
\end{center}

\noindent 


  Now click on the file to start the installation. Then Click on Next Button to start installation.


\begin{center}
	\noindent \includegraphics*[width=2.71in, height=2.14in]{IMG-06-08}
\end{center}

\noindent 


  Accept the Licence Agreement and click on next button.


\begin{center}
	\noindent \includegraphics*[width=2.88in, height=2.26in]{IMG-06-09}
\end{center}

 
   Click on Next Button.
 

\begin{center}
	\noindent \includegraphics*[width=3.53in, height=2.79in]{IMG-06-10}
\end{center}

\noindent 

 
   Click no Next Button.
 

\begin{center}
	\noindent \includegraphics*[width=3.48in, height=2.77in]{IMG-06-11}
\end{center}

\noindent 

\noindent 

 
   Click on Next Button.
 

\begin{center}
	\noindent \includegraphics*[width=3.51in, height=2.80in]{IMG-06-12}
\end{center}

\noindent 

 
   Click on Install Button to start the installation. Once you click on Install Button a popup will appear to confirm the installation, Just click on Yes Button.
 

\begin{center}
	\noindent \includegraphics*[width=3.39in, height=2.67in]{IMG-06-13}
\end{center}

\noindent 

 \newpage
\noindent You can see installation is started.
 

\begin{center}
	\noindent \includegraphics*[width=3.44in, height=2.71in]{IMG-06-14}
\end{center}

\noindent Click on finish to complete the installation.
 

\begin{center}
	\noindent \includegraphics*[width=3.67in, height=2.91in]{IMG-06-15}
\end{center}

\noindent \\
Done with installing the NodeJS.

\noindent \\
\newpage
\noindent \textbf{Installation (Visual Studio Code):}

 
  
\noindent \\
Download visual studio code system installer: https://code.visualstudio.com/download. Click on 64bit for system installer to download visual studio code.
 

\begin{center}
	\noindent \includegraphics*[width=6.21in, height=3.37in]{IMG-06-16}
\end{center}

\noindent You can see the visual studio code is downloaded, click on it to start the installation. Then you can see a popup to confirm the installation, just click on yes button.\\
   
 

\begin{center}
	\noindent \includegraphics*[width=3.01in, height=1.41in]{IMG-06-17}
\end{center}

\noindent 

\newpage
   Accept the licence agreement and click on next button.
 

\begin{center}
	\noindent \includegraphics*[width=3.93in, height=3.21in]{IMG-06-18}
\end{center}

\noindent Click on Next Button.
 

\begin{center}
	\noindent \includegraphics*[width=2.88in, height=2.38in]{IMG-06-19}
\end{center}

 
\noindent Click on next button.
 

\begin{center}
	\noindent \includegraphics*[width=2.72in, height=2.22in]{IMG-06-20}
\end{center}

 
\noindent Check on Desktop Shortcut and click on next button.
 

\begin{center}
	\noindent \includegraphics*[width=2.72in, height=2.23in]{IMG-06-21}
\end{center}

 
   Click on install button to start the installation.
 

\begin{center}
	\noindent \includegraphics*[width=2.60in, height=2.14in]{IMG-06-22}
\end{center}

 
   You can see the installation procedure is started.
 

\begin{center}
	\noindent \includegraphics*[width=3.16in, height=2.57in]{IMG-06-23}
\end{center}

\noindent 

\newpage 
\noindent Click on finish button to complete the installation. As launch visual studio code check box is already check. It will automatically launch the visual studio code software.
 

\begin{center}
	\noindent \includegraphics*[width=3.31in, height=2.76in]{IMG-06-24}
\end{center}

 
   Visual studio code is launched. Now you can start developing your projects.
 

\begin{center}
	\noindent \includegraphics*[width=5.48in, height=2.95in]{IMG-06-25}
\end{center}

\noindent 

\noindent \\
Done.

\noindent \\
\newpage
\noindent \textbf{Creating the Project in ReactJS}



 
\noindent \\
Go to the path where you want to create the ReactJS project. Here I am creating the ReactJS project in C: Drive
 

\begin{center}
	\noindent \includegraphics*[width=3.06in, height=2.22in]{IMG-06-26} \includegraphics*[width=3.14in, height=2.26in]{IMG-06-27}
\end{center}

\noindent 

 
   Go to start, search for NodeJs, and Right click on it and select Run as Administrator.
 

\begin{center}
	\noindent \includegraphics*[width=4.00in, height=2.57in]{IMG-06-28}
\end{center}

\noindent 

\noindent 

 
   NodeJS command prompt will be displayed. And you can see the default path as: \textbf{C:{\textbackslash}Windows{\textbackslash}system32$\boldsymbol{\mathrm{>}}$}
 

\noindent \textbf{}

\begin{center}
	\noindent \includegraphics*[width=4.54in, height=2.41in]{IMG-06-29}
\end{center}

\noindent Now change the path to the C: Drive, ReactJS folder path as shown in the images/image, it will navigate to your ReactJS folder (C:{\textbackslash}ReactJS). 
 

\begin{center}
	\noindent \includegraphics*[width=4.48in, height=2.36in]{IMG-06-30}
\end{center}

\noindent 

 
\noindent Now let us create the ReactJS Project. 
 

\begin{center}
	\noindent \includegraphics*[width=5.70in, height=0.68in]{IMG-06-31}
\end{center}

 
\noindent Enter the command: \textbf{npm install create-react-app demo}
 

\begin{center}
	\noindent \includegraphics*[width=4.26in, height=2.25in]{IMG-06-32}
\end{center}

\noindent 

\newpage
\noindent You can see the React Demo project is getting created.
 

\begin{center}
	\noindent \includegraphics*[width=4.32in, height=2.30in]{IMG-06-33}
\end{center}

\noindent \textbf{}

\noindent \\
\textbf{Note:} Wait for some time until the project is downloaded and installed (based on your internet speed and your PC/Laptop performance). Make sure that you are having the proper internet connection to download the project.

 
\noindent \\
After the installation you can see the following information:
 

\begin{center}
	\noindent \includegraphics*[width=5.20in, height=2.74in]{IMG-06-34}
\end{center}
\newpage
\noindent Now go the C: Driver and ReactJS folder, you can see the ReactJS Project Demo folder is created and all the project files are installed inside the demo folder.
 

\begin{center}
	\noindent \includegraphics*[width=3.48in, height=2.48in]{IMG-06-35}
\end{center}

\noindent  Open the demo folder you can the ReactJS project folder structure.
 

\begin{center}
	\noindent \includegraphics*[width=4.20in, height=3.04in]{IMG-06-36}
\end{center}

\noindent \\
Done with creating the ReactJS Project.

\noindent \\
\newpage
\textbf{Open the ReactJS Project in Visual Studio Code}

 
Open Visual Studio Code. Click on File  Open Folder.
 

\begin{center}
	\noindent \includegraphics*[width=4.97in, height=2.67in]{IMG-06-37}
\end{center}

\noindent \\
\\
Navigate to C: Drive, React JS folder, Select the demo folder, click on Select Folder button.
 

\begin{center}
	\noindent \includegraphics*[width=4.52in, height=2.83in]{IMG-06-38}
\end{center}
\newpage
\noindent You can see, your ReactJS demo project is opened in your Visual Studio Code.
 

\begin{center}
	\noindent \includegraphics*[width=5.51in, height=2.96in]{IMG-06-39}
\end{center}

 
\noindent Now go to View Menu  Click on Terminal, to show the terminal in the visual studio code.
 

\begin{center}
	\noindent \includegraphics*[width=5.18in, height=2.78in]{IMG-06-40}
\end{center}
\newpage
\noindent Now you can see the terminal at the bottom of the visual studio code. And by default, the current demo project path is already opened in the terminal.
 

\begin{center}
	\noindent \includegraphics*[width=5.37in, height=2.89in]{IMG-06-41}
\end{center}

\noindent To run the ReactJS demo project, From the Terminal type the command: ``npm start''
 

\begin{center}
	\noindent \includegraphics*[width=5.51in, height=2.96in]{IMG-06-42}
\end{center}

\newpage
\noindent You can see the project is getting compiled and executed.
 

\begin{center}
	\noindent \includegraphics*[width=5.67in, height=3.05in]{IMG-06-43}
\end{center}

\noindent You can see the output in your default web browser.
 

\begin{center}
	\noindent \includegraphics*[width=5.76in, height=3.12in]{IMG-06-44}
\end{center}

\noindent \textbf{}

\noindent \textbf{}

\noindent \\
\textbf{Note:} ReactJS project is executing in the localhost of the port no: 3000. 

\noindent \\
\textbf{URL: }http://localhost:3000/

\noindent 

\noindent Done.

\end{document}