\documentclass{book}
\usepackage{listings}
\usepackage{color}
\usepackage{graphicx}
\usepackage{booktabs}
\usepackage{fancyhdr}
\usepackage{amsmath}
\usepackage{hyperref}
\usepackage{csquotes}
\usepackage[english]{babel}
\pagestyle{fancy}
\fancyhf{}
\headheight = 0.1 in
%\rhead{\includegraphics[width=2cm, height=1cm]{logo}}
\lhead{\textbf{\Large{EIGENVALUES AND EIGENVECTORS}}}
\lfoot{COPYRIGHT ©TALENTSPRINT, 2020. ALL RIGHTS RESERVED.}
\rfoot{\thepage}

\begin{document}

Find the corresponding eigenvalues and eigenvectors for the below problem.
\begin{equation*}
	\begin{pmatrix}
	-2 & -4 & 2\\
	-2 & 1 & 2\\
	4 & 2 & 5
	\end{pmatrix}
\end{equation*}
The characteristic equation is
\begin{equation*}
det
	\begin{pmatrix}
	-2-\lambda & -4 & 2\\
	-2 & 1-\lambda & 2\\
	4 & 2 & 5-\lambda
	\end{pmatrix} = 0 
\end{equation*}
Expanding the determinant,
\begin{equation*}
	(-2-\lambda)([1-\lambda)(5-\lambda)-2 * 2] + 4 [(-2)*(5-\lambda) - 4 * 2] + 2[(-2) * 2 - 4(1-\lambda)] = 0
\end{equation*}
Expanding the brackets and simplifying:
\begin{equation*}
	-\lambda^3 + 4\lambda^2 - 27\lambda - 90 = 0,
\end{equation*}
or equivalently
\begin{equation*}
	\lambda^3 - 4\lambda^2 + 27\lambda + 90 = 0,
\end{equation*}
By trial and error, we find that
\begin{equation*}
	3^3 - 4 * 3^2 - 27 * 3 + 90 = 0,
\end{equation*}
and it follows from the Factor Theorem that $(\lambda¸-3)$  is a factor. Indeed,
\begin{equation*}
	\lambda^3 - 4\lambda^2 + 27\lambda + 90 = (\lambda - 3)(\lambda^2 - \lambda - 30)
\end{equation*}
and 
\begin{equation*}
	(\lambda - 3)(\lambda^2 - \lambda - 30) = (\lambda - 3)(\lambda + 5)(\lambda - 6)
\end{equation*}
meaning that the eigenvalues are 3, -5  and 6.\\
\\
We now go on to solve
\begin{equation*}
	\begin{pmatrix}
	-2-\lambda & -4 & 2\\
	-2 & 1-\lambda & 2\\
	4 & 2 & 5-\lambda
	\end{pmatrix}
	\begin{pmatrix}
	X\\
	Y\\
	Z
	\end{pmatrix}  = 
	\begin{pmatrix}
	0\\
	0\\
	0
	\end{pmatrix}
\end{equation*}
for each eigenvalue ¸. Now, every such system will have infinitely many solutions, because if e is an eigenvector, so is any multiple of e. So our strategy will be to try to find the eigenvector with X=1, and then if necessary scale up. (If there is no such eigenvector, we know that X must in fact be zero, and we instead look for the eigenvector with Y=1, and so on.)
\\
\\
\\
\\
\textbf{Eigenvector corresponding to eigenvalue 3}\\ \\
In the case ¸=3, we have
\begin{equation*}
	\begin{pmatrix}
	-5 & -4 & 2\\
	-2 & -2 & 2\\
	4 & 2 & 2
	\end{pmatrix}
	\begin{pmatrix}
	X\\
	Y\\
	Z
	\end{pmatrix}  = 
	\begin{pmatrix}
	0\\
	0\\
	0
	\end{pmatrix}
\end{equation*}
Setting X=1 gives, as our first two equations,
\begin{equation*}
	-5 - 4Y + 2Z = 0,
\end{equation*}
\begin{equation*}
	-2 - 2Y + 2Z = 0
\end{equation*}
Subtracting the first from the second:
\begin{equation*}
	3 + 2Y = 0,
\end{equation*}
and thus Y = $-\frac{3}{2}$.
\\
\\
Substituting back into the second equation,
\begin{equation*}
	-2 + 3 + 2Z = 0,
\end{equation*}
giving Z=$-\frac{1}{2}$.
\\
\\
Checking in the third equation,
\begin{equation*}
	4 - 3 - 1 = 0,
\end{equation*}
which works. This gives us the eigenvector
\begin{equation*}
	(1, -\frac{3}{2}, -\frac{1}{2}).
\end{equation*}
Once again, we can scale up by a factor of 2, to get
\begin{equation*}
	(2, -3, -1).
\end{equation*}
In the same way eigenvector corresponding to eigenvalue -5 is $(2, -1, 1)$ and to eigenvalue 6 is $(1, 6, 16)$.
\end{document}
