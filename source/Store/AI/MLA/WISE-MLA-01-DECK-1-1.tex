\documentclass{beamer}
\usetheme{Madrid}

\title{Introduction to Machine Learning}
\author{by Talentsprint Pvt.Ltd.}
\centering
\date{April 2020}
\begin{document}
\maketitle
\begin{frame}{Content}
\begin{itemize}
\item What is machine learning? 
\item Why use machine learning?
\item Growth of Machine Learning.
\item Application of machine Learning. 
\item Types of Machine Learning Systems.
\item Top 10 use cases of Machine Learning.
\item Main Challenges in Machine Learning.
\end{itemize}
\end{frame}
\begin{frame}{What is Machine learning?}
\framesubtitle{Why do we need to care about machine learning?}
\begin{itemize}
    \item Machine Learning is the science (and art) of programming computers so they can learn from data.\\
\vspace{10pt}
Here is a slightly more general definition:\\
    \item Machine Learning is the field of study that gives computers the ability to learn without being explicitly programmed.\\
—Arthur Samuel, 1959\\
\vspace{10pt}
And a more engineering-oriented one:
    \item A computer program is said to learn from experience E with respect to some task T and some performance measure P, if its performance on T, as measured by P, improves with experience E.\\
—Tom Mitchell, 1997
\end{itemize}
\end{frame}
\begin{frame}{Why use Machine learning?}
\begin{itemize}
    \item The Traditional Approach\\
\vspace{10pt}
    \includegraphics[width=\linewidth]{image.png}
\end{itemize}
\end{frame}
\begin{frame}{Contd..}
\begin{itemize}
    \item Machine Learning Approach\\
\vspace{10pt}
    \includegraphics[width=\linewidth]{image2.png}
\end{itemize}
\end{frame}
\begin{frame}{Growth of Machine Learning}
   \begin{itemize}
       \item Machine learning is preferred approach to
       \item Speech recognition, Natural language processing
       \item Computer vision
       \item Medical outcomes analysis
    \item Robot control
    \item Computational biology
    \item Improved machine learning algorithms
    \item Improved data capture, networking, faster computers
    \item Software too complex to write by hand
    \item New sensors / IO devices
\end{itemize}
\end{frame}

\begin{frame}{Applications of Machine Learning}
  \framesubtitle{Sample applications of machine learning:}
  The value of machine learning technology has been recognized by companies across several industries that deal with huge volumes of data. By leveraging insights obtained from this data, companies are able work in an efficient manner to control costs as well as get an edge over their competitors. This is how some sectors / domains are implementing machine learning -
  \begin{itemize}
      \item Financial Services
      \item Marketing and Sales
      \item Government
      \item Healthcare
      \item Transportation
      \item Oil and Gas
  \end{itemize}
\end{frame}
\begin{frame}{Types of Machine Learning Systems}
There are so many different types of Machine Learning systems that it is useful to classify them in broad categories based on:\\
\vspace{10pt}
\begin{itemize}
    \item Whether or not they are trained with human supervision (supervised, unsupervised, semi-supervised, and Reinforcement Learning).
    \vspace{10pt}
    \item Whether or not they can learn incrementally on the fly (online versus batch learning).
    \vspace{10pt}
    \item Whether they work by simply comparing new data points to known data points, or instead detect patterns in the training data and build a predictive model, much like scientists do (instance-based versus model-based learning)
\end{itemize}
\end{frame}
\begin{frame}{Contd..}
\begin{block}{Supervised/Unsupervised Learning}
ML systems can be classified according to the amount and type of supervision they get during training. There are four major categories: supervised, unsupervised, semi-supervised, and Reinforcement Learning.\\
\vspace{5pt}
\textbf{Supervised learning:}\\
\begin{itemize}
    \item In supervised learning, the training data you feed to the algorithm includes the desired solutions, called labels.
    \item A typical supervised learning task is classification. The spam filter is a good example of this: it is trained with many example emails along with their class (spam or ham), and it must learn how to classify new emails.
    \item Another typical task is to predict a target numeric value, such as the price of a car, given a set of features (mileage, age, brand, etc.) called predictors. This sort of task is called regression.
\end{itemize}
\end{block}
\end{frame}
\begin{frame}
\begin{block}{Contd..}
Here are some of the important supervised learning algorithms:\\
\begin{itemize}
    \item k - Nearest Neighbors.
    \item Linear Regression.
    \item Logistic regression.
    \item Support Vector Machines (SVMs)
    \item Decision Trees and Random Forests
    \item Neural networks
\end{itemize}
\textbf{Unsupervised learning:}\\
\begin{itemize}
    \item In unsupervised learning, as you might guess, the training data is unlabeled. The system tries to learn without a teacher.
\end{itemize}
Here are some of the most important unsupervised learning algorithms:
\begin{itemize}
    \item Clustering
    \begin{itemize}
        \item K-Means.
        \item DBSCAN.
        \item Hierarchical Cluster Analysis (HCA).
    \end{itemize}
\end{itemize}
\end{block}
\end{frame}
\begin{frame}
\begin{block}{Contd..}
\begin{itemize}
    \item Anomaly detection and novelty detection
        \begin{itemize}
            \item One-class SVM.
            \item DBSCAN.
            \item Isolation Forest.
        \end{itemize}
    \item Visualization and dimensionality reduction
        \begin{itemize}
            \item Principal Component Analysis (PCA).
            \item Kernel PCA.
            \item Locally Linear Embedding (LLE).
            \item t-distributed Stochastic Neighbor Embedding (t-SNE).
        \end{itemize}
    \item Association Rule Learning
        \begin{itemize}
            \item Apriori.
            \item Eclat.
        \end{itemize}
\end{itemize}
\textbf{Semisupervised learning:}\\
\begin{itemize}
    \item Some algorithms can deal with partially labeled training data, usually a lot of unlabeled data and a little bit of labeled data. This is called semisupervised learning.
\end{itemize}
\end{block}
\end{frame}
\begin{frame}
\begin{block}{Contd..}
\begin{itemize}
     \item Some photo-hosting services, such as Google Photos, are good examples of this. Once you upload all your family photos to the service, it automatically recognizes that the same person A shows up in photos 1, 5, and 11, while another person B shows up in photos 2, 5, and 7. This is the unsupervised part of the algorithm (clustering). Now all the system needs is for you to tell it who these people are. Just one label per person, 4 and it is able to name everyone in every photo, which is useful for searching photos.
    \item Most semisupervised learning algorithms are combinations of unsupervised and supervised algorithms. For example, deep belief networks (DBNs) are based on unsupervised components called restricted Boltzmann machines (RBMs) stacked on top of one another. RBMs are trained sequentially in an unsupervised manner, and then the whole system is fine-tuned using supervised learning techniques.
\end{itemize}
\end{block}
\end{frame}
\begin{frame}
\begin{block}{Contd..}
\textbf{Reinforcement learning:}\\
\begin{itemize}
    \item Reinforcement Learning is a very different beast. The learning system, called an agent in this context, can observe the environment, select and perform actions, and get rewards in return (or penalties in the form of negative rewards.
    \item It must then learn by itself what is the best strategy, called a policy, to get the most reward over time.
    \item A policy defines what action the agent should choose when it is in a given situation.
\end{itemize}
\end{block}
\begin{block}{Batch and Online Learning}
Another criterion used to classify Machine Learning systems is whether or not the system can learn incrementally from a stream of incoming data.\\
\end{block}
\end{frame}
\begin{frame}
\begin{block}{Contd..}
\textbf{Batch Learning:}\\
\begin{itemize}
    \item In batch learning, the system is incapable of learning incrementally: it must be trained using all the available data. \item This will generally take a lot of time and computing resources, so it is typically done offline. 
    \item First the system is trained, and then it is launched into production and runs without learning anymore; it just applies what it has learned. This is called offline learning.
\end{itemize}
\textbf{Online Learning}\\
\begin{itemize}
    \item In online learning, you train the system incrementally by feeding it data instances sequentially, either individually or by small groups called mini-batches. Each learning step is fast and cheap, so the system can learn about new data on the fly, as it arrives. It is great for systems that receive data as a continuous flow (e.g., stock prices) and need to adapt to change autonomously.
\end{itemize}
\end{block}
\end{frame}

\begin{frame}
\begin{block}{Instance-Based Versus Model-Based Learning}
One more way to categorize Machine Learning systems is by how they generalize. Most Machine Learning tasks are about making predictions. This means that the system needs to be able to generalize to examples it has never seen before.\\
\vspace{10pt}
There are two main approaches to generalization: instance-based learning and model-based learning.\\
\textbf{Instance-based Learning:}\\
\begin{itemize}
    \item This system learns the examples by heart, then
generalizes to new cases by comparing them to the learned examples (or a subset of them), using a similarity measure.
    \item This is called instance-based learning:.
\end{itemize}
\textbf{Model-based Learning}\\
\begin{itemize}
    \item Another way to generalize from a set of examples is to build a model of these examples, then use that model to make predictions. This is called model-based learning. 
\end{itemize}
\end{block}
\end{frame}

\begin{frame}{Top 10 use cases of Machine Learning}
\includegraphics[height=6.8cm]{image3.png}
\centering
\end{frame}

\begin{frame}{Main Challenges in Machine Learning}
\begin{itemize}
    \item Insufficient Training Data
    \item Non representative Training Data
    \item Poor Quality Data
    \item Irrelevant Features
    \item Overfitting
    \item Underfitting
\end{itemize}
\end{frame}
\begin{frame}
\huge{\centerline{The End}}
\end{frame}

\end{document}