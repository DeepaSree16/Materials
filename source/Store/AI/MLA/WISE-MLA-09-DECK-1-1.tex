\documentclass{beamer}
\usetheme{Madrid}
\usepackage{amsmath}
\usepackage{ragged2e}


\title{Introduction to Percentiles, Bootstrap and Confidence Intervals}
\author{by Talentsprint Pvt.Ltd.}
\centering
\date{July 2020}

\begin{document}
\maketitle
\begin{frame}{Content}
	\begin{itemize}
		\item Quantiles, Percentiles and Quartiles
		\item Bootstrap Method
		\item Configuration of Bootstrap
		\item Example - Bootstrap
		\item Confidence Intervals
		\item Interval for Classification Accuracy
	\end{itemize}
\end{frame}

\begin{frame}{Quantiles, Percentiles and Quartiles}
\begin{flushleft}
	Quantiles are the "cut points" to divide a distribution into continous intervals with equal probabilities. For example a quantile to divide a distribution into equal two parts, which means 50\% on each side. Thus, such quantile will be called Median. In similar way, based on number of equal partition a set of quantiles make they are collectively called
\\
\end{flushleft}
	\begin{enumerate}
		\item Quantile which partition distribution into 2 equal groups is called Median.
		\item Quantiles for 3 equal partitions are called Tertiles or Terciles.
		\item Quartiles for 4 equal partitions where we call them 1st(25 percentile), 2nd(next 25th percentile), 3rd(another 25th percentile) and last/4th Quartile(last or top 25th ).
		\item Quintiles(5), Sextiles(6), Septiles(7), Octiles(9), Deciles(10) and so on.
		\item Percentile for 100-Quantiles.
	\end{enumerate}
\begin{flushleft}
	
\end{flushleft}
\end{frame}

\begin{frame}{Bootstrap Method}
	\begin{itemize}
		\item The bootstrap method is a resampling technique used to estimate statistics on a population by sampling a dataset with replacement.
		\item In other words, it is a statistical technique for estimating quantities about a population by averaging estimates from multiple small data samples.
		\item The process for bulding one sample can be
		\begin{enumerate}
			\item Choose the size of the sample.
			\item While the size of the sample is less than the chosen size
			\begin{enumerate}	
				\item Randomly select an observation from the dataset
				\item Add it to the sample
			\end{enumerate} 
		\end{enumerate}
		\item The bootstrap method can be used to estimate a quantity of a population. This is done by repeatedly taking small samples, calculating the statistic, and taking the average of the calculated statistics.
	\end{itemize}
\end{frame}

\begin{frame}{Contd..}
	\begin{enumerate}
			\item Choose a number of bootstrap samples to perform.
			\item Choose a sample size.
			\item For each bootstrap sample
			\begin{enumerate}	
				\item Draw a sample with replacement with the chosen size.
				\item Calculate the statistic on the sample.
			\end{enumerate} 
			\item Calculate the mean of the calculated sample statistics.
	\end{enumerate}
	\begin{itemize}
		\item This procedure of using the bootstrap method to estimate the skill of the model can be summarized as follows:
		\begin{enumerate}
			\item Choose a number of bootstrap samples to perform.
			\item Choose a sample size.
			\item For each bootstrap sample
			\begin{enumerate}	
				\item Draw a sample with replacement with the chosen size.
				\item Fit a model on the data sample.
				\item Estimate the skill of the model on the out-of-bag sample.
			\end{enumerate} 
			\item Calculate the mean of the sample of model skill estimates.
		\end{enumerate}
	\end{itemize}
\end{frame}

\begin{frame}{Configuration of Bootstrap}
\begin{flushleft}
	There are two parameters that must be chosen when performing the bootstrap: \\
\vspace{10pt}
\textbf{Sample Size:}
	\begin{itemize}
		\item In machine learning, it is common to use a sample size that is the same as the original dataset.
		\item If the dataset is enormous and computational efficiency is an issue, smaller samples can be used, such as 50\% or 80\% of the size of the dataset.
	\end{itemize}
	\textbf{Repetitions:}

	\begin{itemize}
		\item The number of repetitions must be large enough to ensure that meaningful statistics, such as the mean, standard deviation, and standard error can be calculated on the sample.
		\item A minimum might be 20 or 30 repetitions. Smaller values can be used will further add variance to the statistics calculated on the sample of estimated values.
	\end{itemize}
\end{flushleft}
\end{frame}

\begin{frame}{Example - Bootstrap}
	\huge{\centerline{Refer IPython Notebook}}
\end{frame}

\begin{frame}{Confidence Intervals}
	\begin{itemize}
		\item Confidence intervals are a way of quantifying the uncertainty of an estimate. They can be used to add a bounds or likelihood on a population parameter, such as a mean, estimated from a sample of independent observations from the population.
		\item In other words, it is a bounds on the estimate of a population variable. It is an interval statistic used to quantify the uncertainty on an estimate.
		\item In applied machine learning, we may wish to use confidence intervals in the presentation of the skill of a predictive model.
		\item For example, a confidence interval could be used in presenting the skill of a classification model, which could be stated as:

Given the sample, there is a 95\% likelihood that the range x to y covers the true model accuracy.
	\end{itemize}
\end{frame}

\begin{frame}{Contd..}
	\begin{itemize}
		\item The choice of 95\% confidence is very common in presenting confidence intervals, although other less common values are used, such as 90\% and 99.7\%. In practice, you can use any value you prefer.
		\item Often, the larger the sample from which the estimate was drawn, the more precise the estimate and the smaller (better) the confidence interval.
		\begin{enumerate}
			\item \textbf{Smaller Confidence Interval:} A more precise estimate.
			\item \textbf{Larger Confidence Interval:} A less precise estimate.
		\end{enumerate}
		\item Confidence intervals belong to a field of statistics called estimation statistics that can be used to present and interpret experimental results instead of, or in addition to, statistical significance tests.
	\end{itemize}
\end{frame}

\begin{frame}{Interval for Classification Accuracy}
	\huge{\centerline{Refer Notebook}}
\end{frame}

\begin{frame}
\huge{\centerline{The End}}
\end{frame}
\end{document}