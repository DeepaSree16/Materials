\documentclass{book}
\usepackage{graphicx}
\usepackage{listings}
\usepackage{color}
\usepackage{graphicx}
\usepackage{booktabs}
\usepackage{fancyhdr}
\usepackage{amsmath}
\usepackage{hyperref}
\usepackage{csquotes}
\usepackage[english]{babel}
\usepackage{amssymb}
\usepackage{textcomp}
\usepackage{gensymb}
\usepackage{textcomp}
\usepackage{xfrac}
\pagestyle{fancy}
\fancyhf{}
\headheight = 0.1 in
%\rhead{\includegraphics[width=2cm, height=1cm]{logo}}
\lhead{\textbf{\Large{Assignment Solution}}}
\lfoot{COPYRIGHT ©TALENTSPRINT, 2020. ALL RIGHTS RESERVED.}
\rfoot{\thepage}

\begin{document}
\begin{enumerate}
    \item 
    Null Hypothesis $H_0$ : The population mean ($\mu_0$) $\leq$ 71.1\\
    Alternate Hypothesis $H_1$ : The population mean ($\mu_0$) $>$ 71.1\\\\
    Since it is a one tailed test,
    \begin{equation*}
        p=95.0\%=0.95
    \end{equation*}
    If we were interested in finding the level of significance, it would just be $\alpha=1-p$ which in this case is just $\alpha=0.05$.\\\\
    The degrees of freedom are sample size minus one.
    \begin{equation*}
        df = 7 - 1\\
    \end{equation*}
    \begin{equation*}
        df = 6
    \end{equation*}\\
    Determine $t_{0.95}$ also called $t_\alpha$
    \begin{equation*}
        t_\alpha=1.943
    \end{equation*}
    Calculate $t_{calc}$ also known as $t_c$
    \begin{align*}
        t_c &= \frac{\Bar{Y}-\mu}{\sfrac{s}{\sqrt{n}}}\\
        &= \frac{71.3-71.1}{\sfrac{0.214}{\sqrt{7}}}\\
        &= \frac{0.2}{0.0809}\\
        t_c &= 2.47
    \end{align*}
    Since $t_c > t_\alpha$ and since the $t_c$ lies in the \textit{Reject region}, we must reject the null hypothesis.\\\\
    Rejecting the null hypothesis means that the sample was NOT within the bounds of what we would find acceptable if the population mean were truly at $71.1\degree C$.\\
    
    \item To see if the results establish the firm's claim, let us see if they would lead to rejection of the hypothesis that the firm's cigarettes do not have an average nicotine content lower than 1.5 milligrams. That is, we should test\\\\
    $H_0 : \mu \geq 1.5$\\\\
    against the firm's claim of\\\\
    $H_1 : \mu \ < 1.5$\\\\
    Since the value of the test statistic is\\\\
    $\sqrt{n}\frac{\hat{X} - \mu_0}{\sigma} = \sqrt{20}\frac{1.42 - 1.5}{0.7} = -0.511$\\\\
    it follows that the p value is\\\\
    $p value = P\{Z \leq -0.511\} = 0.305$\\\\
    Since the p value exceeds 0.05, the foregoing data do not enable us to reject the null hypothesis and conclude that the mean content per cigarette is less than 1.5 milligrams. In other words, even though the evidence supports the cigarette producer's claim (since the average nicotine content of those cigarettes tested was indeed less than 1.5 milligrams), that evidence is not strong enough to prove the claim. This is because a result at least as supportive of the alternative hypothesis H1 as that obtained would be expected to occur 30.5 percent of the time when the mean nicotine content was 1.5 milligrams per cigarette.\\\\
Statistical hypothesis tests in which either the null or the alternative hypothesis states that a parameter is greater (or less) than a certain value are called one-sided tests.

We have assumed so far that the underlying population distribution is the normal distribution. However, we have only used this assumption to conclude that $\sfrac{\sqrt{n}(\hat{X} - \mu)}{\sigma}$  has a standard normal distribution. But by the central limit theorem this result will approximately hold, no matter what the underlying population distribution, as long as n is reasonably large. A rule of thumb is that a sample size of $n \geq 30$ will almost always suffice. Indeed, for many population distributions, a value of n as small as 4 or 5 will result in a good approximation. Thus, all the hypothesis tests developed so far can often be used even when the underlying population distribution is not normal.
\end{enumerate}
\end{document}