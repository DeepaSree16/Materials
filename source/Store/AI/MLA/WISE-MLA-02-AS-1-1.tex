\documentclass{book}
\usepackage{listings}
\usepackage{color}
\usepackage{graphicx}
\usepackage{booktabs}
\usepackage{fancyhdr}
\usepackage{amsmath}
\usepackage{hyperref}
\usepackage{csquotes}
\usepackage[english]{babel}
\pagestyle{fancy}
\fancyhf{}
\headheight = 0.95 in
%\rhead{\includegraphics[width=2cm, height=1cm]{logo}}
\lhead{\textbf{\Large{SYSTEM OF LINEAR EQUATIONS USING MATRICES}}}
\lfoot{COPYRIGHT ©TALENTSPRINT, 2020. ALL RIGHTS RESERVED.}
\rfoot{\thepage}

\begin{document}

Solve the system of linear equations using matrices.
\begin{align*}
  x - y + z &= 8 \\
  2x + 3y - z &= -2 \\
  3x - 2y - 9z &= 9
\end{align*}
\textbf{SOLUTION:}
First, we write the augmented matrix.
\begin{equation*}
	\begin{bmatrix}
	1 & -1 & 1 & 8\\
	2 & 3 & -1 & -2\\
	3 & -2 & -9 & 9
	\end{bmatrix}
\end{equation*}
Next, we perform row operations to obtain row-echelon form.
\begin{equation*}
	-2R1 + R2 = R2 \rightarrow
	\begin{bmatrix}
	1 & -1 & 1 & 8\\
	2 & 3 & -1 & -2\\
	3 & -2 & -9 & 9
	\end{bmatrix}
\end{equation*}
\begin{equation*}
	-3R1 + R3 = R3 \rightarrow
	\begin{bmatrix}
	1 & -1 & 1 & 8\\
	2 & 3 & -1 & -2\\
	3 & -2 & -9 & 9
	\end{bmatrix}
\end{equation*}
The easiest way to obtain a 1 in row 2 of column 1 is to interchange ${R_2}$ and ${R_3}$.
\begin{equation*}
	\text{Interchange}{R}_{2}\text{and}{R}_{3}\to \left[\begin{array}{rrrrrrr}\hfill 1& \hfill & \hfill -1& \hfill & \hfill 1& \hfill & \hfill 8\\ \hfill 0& \hfill & \hfill 1& \hfill & \hfill -12& \hfill & \hfill -15\\ \hfill 0& \hfill & \hfill 5& \hfill & \hfill -3& \hfill & \hfill -18\end{array}\right]
\end{equation*}

Then
\begin{equation*}
	-5R2 + R3 = R3 \rightarrow
	\begin{bmatrix}
	1 & -1 & 1 & 8\\
	0 & 1 & -12 & -15\\
	0 & 0 & -57 & 57
	\end{bmatrix}
\end{equation*}
\begin{equation*}
	-\frac{1}{57}R3 = R3 \rightarrow
	\begin{bmatrix}
	1 & -1 & 1 & 8\\
	0 & 1 & -12 & -15\\
	0 & 0 & 1 & 1
	\end{bmatrix}
\end{equation*}
he last matrix represents the equivalent system.
\begin{equation*}
\begin{array}{l}\text{ }x-y+z=8\hfill \\ \text{ }y - 12z=-15\hfill \\ \text{ }z=1\hfill \end{array}
\end{equation*}
Using back-substitution, we obtain the solution as (4,-3,1).
\end{document}
