\documentclass{book}
\usepackage{graphicx}
\usepackage{listings}
\usepackage{color}
\usepackage{graphicx}
\usepackage{booktabs}
\usepackage{fancyhdr}
\usepackage{amsmath}
\usepackage{hyperref}
\usepackage{csquotes}
\usepackage[english]{babel}
\usepackage{amssymb}
\usepackage{textcomp}
\pagestyle{fancy}
\fancyhf{}
\headheight = 0.9 in
%\rhead{\includegraphics[width=2cm, height=1cm]{logo}}
\lhead{\textbf{\Large{Hypothesis Testing Problems}}}
\lfoot{COPYRIGHT ©TALENTSPRINT, 2020. ALL RIGHTS RESERVED.}
\rfoot{\thepage}

\begin{document}
\subsection*{\textbf{Two-tailed test example}}
A premium golf ball production line must produce all of its balls to 1.615 ounces in order to
get the top rating (and therefore the top dollar). Samples are drawn hourly and checked. If the
production line gets out of sync with a statistical significance of more than 1\%, it must be shut down and repaired. This hour’s sample of 18 balls has a mean of 1.611 ounces and a standard deviation of 0.065 ounces. Do you shut down the line? \\ \\
\subsubsection*{Solution:}
\textbf{Step 1. Determine the null hypothesis.}\\\\
H0: The population mean (µ) = 1.615\\\\
HA: The population mean ≠ 1.615 (hence a 2-tailed test)\\\\
\textbf{Step 2. Draw the t-test diagram}
\begin{figure}[h!]
	\includegraphics[width=300pt, height=250pt]{image.png}
\end{figure}
\begin{figure}[h!]
	\includegraphics[width=400pt, height = 500pt]{image1.png}
\end{figure}
\begin{figure}[h!]
	\includegraphics[width=400pt, height = 550pt]{image2.png}
\end{figure}
\begin{figure}[h!]
	\includegraphics{image3.png}
\end{figure}
\end{document}