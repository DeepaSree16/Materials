\documentclass{book}
\usepackage{graphicx}
\usepackage{listings}
\usepackage{color}
\usepackage{graphicx}
\usepackage{booktabs}
\usepackage{fancyhdr}
\usepackage{amsmath}
\usepackage{hyperref}
\usepackage{csquotes}
\usepackage[english]{babel}
\usepackage{amssymb}
\usepackage{textcomp}
\usepackage{gensymb}
\usepackage{textcomp}
\pagestyle{fancy}
\fancyhf{}
\headheight = 0.1 in
%\rhead{\includegraphics[width=2cm, height=1cm]{logo}}
\lhead{\textbf{\Large{Assignment Problems}}}
\lfoot{COPYRIGHT ©TALENTSPRINT, 2020. ALL RIGHTS RESERVED.}
\rfoot{\thepage}

\begin{document}
\begin{enumerate}
    \item An alternator manufacturer must produce its alternators so that they are $95\%$ confident that it runs at less than $71.1\degree C$ under stress test in order to meet the production requirements for sale to the US government. The stress test is performed on random samples drawn from the production line on a daily basis. Today’s sample of $7$ alternators has a mean of $71.3\degree C$ and a standard deviation of $0.214\degree C$. Is there a production quality issue?
    \item All cigarettes presently being sold have an average nicotine content of at least 1.5 milligrams per cigarette. A firm that produces cigarettes claims that it has discovered a new technique for curing tobacco leaves that results in an average nicotine content of a cigarette of less than 1.5 milligrams. To test this claim, a sample of 20 of the firm's cigarettes was analyzed. If it were known that the standard deviation of a cigarette's nicotine content was 0.7 milligrams, what conclusions could be drawn, at the 5 percent level of significance, if the average nicotine content of these 20 cigarettes were 1.42 milligrams?
\end{enumerate}
\end{document}