\documentclass{book}
%\usepackage{listings}
\usepackage{amsmath}
\usepackage{color}
\usepackage{graphicx}
%\usepackage{booktabs}
\usepackage{fancyhdr}
\usepackage[english]{babel}
\pagestyle{fancy}
\fancyhf{}
%\rhead{\includegraphics[width=2cm, height=0.5cm]{logo}}
\lhead{Eigenvalues and Eigenvectors}
\lfoot{COPYRIGHT ©TALENTSPRINT, 2020. ALL RIGHTS RESERVED.}
\rfoot{\thepage}
\begin{document}


\section*{Eigenvalues and Eigenvectors}

      The subject of eigenvalues and eigenvectors will take up most of the rest of the course. We will again be working with square matrices.
 Eigenvalues are special numbers associated with a matrix and eigenvectors are special vectors.
    \subsection*{Eigenvectors and Eigenvalues}

      A matrix A acts on vectors x like a function does, with input x and output Ax. \emph{Eigenvectors} are vectors for which Ax is parallel
 to x.
In other words:
              
                   $Ax = \lambda x.$

 In this equation, x is an eigenvector of A and λ is an \emph{eigenvalue} of A.      

     \subsection*{Eigenvalue 0}
   
    If the eigenvalue $\lambda $ equals 0 then $Ax = 0x = 0.$ Vectors with eigenvalue 0 make up the nullspace of A; if A is singular, then $\lambda  = 0$ is an
eigenvalue of A.      

\paragraph*{Examples}

     Suppose P is the matrix of a projection onto a plane. For any x in the plane Px = x, so x is an eigenvector with eigenvalue 1. A vector x perpendicular to  the plane has Px = 0, so this is an eigenvector with eigenvalue λ = 0. The eigenvectors of P span the whole space (but this
is not true for every matrix).    

The matrix  
$\begin{bmatrix}
  0 & 1\\ 
  1 & 0
\end{bmatrix}$ 
has an eigenvector
$\begin{bmatrix}
  1 \\
  1
\end{bmatrix}$ 
with eigenvalue 1 and another eigenvector
$\begin{bmatrix}
   1 \\
  -1
\end{bmatrix}$
with eigenvalue − 1. These eigenvectors span the space. They are perpendicular because B = B T (as we will prove).

            $det ( A − \lambda I ) = 0$

An n by n matrix will have n eigenvalues, and their sum will be the sum of the diagonal entries of the matrix: a 11 + a 22 + · · · + a nn . This sum is the trace of the matrix. For a two by two matrix, if we know one eigenvalue we can use this fact to find the second.
Can we solve Ax = λx for the eigenvalues and eigenvectors of A? Both λ and x are unknown; we need to be clever to solve this problem:

 $Ax = \lambda x$
 $ ( A − \lambda I ) x = 0 $
In order for λ to be an eigenvector, A − λI must be singular. In other words, det ( A − λI ) = 0. We can solve this \emph{characteristic equation} for λ to get n solutions.

If we’re lucky, the solutions are distinct. If not, we have one or more repeated \emph{eigenvalues}.
Once we’ve found an eigenvalue λ, we can use elimination to find the nullspace of A − λI. The vectors in that nullspace are eigenvectors of A with eigenvalue λ.


\subsection*{Calculating eigenvalues and eigenvectors}
Let
$A =
\begin{bmatrix}. 
 3 & 1\\
 1 & 3
\end{bmatrix}$
Then:
$det ( A - \lambda I )=
\begin{vmatrix}
 3 − \lambda & 1\\
 1           & 3 - \lambda
\end{vmatrix}$

$= ( 3 - \lambda )^2 - 1$

$= \lambda^2 - 6λ + 8.$

Note that the coefficient 6 is the trace (sum of diagonal entries) and 8 is the determinant of A. In general, the eigenvalues of a two by two matrix are the solutions to:
                        
$ \lambda^2 - trace ( A ) · \lambda + det A = 0.$

Just as the trace is the sum of the eigenvalues of a matrix, the product of the eigenvalues of any matrix equals its determinant.

For
$A =
\begin{bmatrix},  
  3 & 1 \\                                                                                
  1 & 3                                                                                   
\end{bmatrix}$                                                                        
the eigenvalues are $\lambda_1 = 4$ and $\lambda_2 = 2$. We find the eigenvector $x_1=
\begin{bmatrix}
  1\\
  1
\end{bmatrix}$
for λ 1 = 4 in the nullspace of A − λ 1 I =
$\begin{bmatrix}           
  1 & 1\\                                                 
  1 & -1                                                   
 \end{bmatrix}$                                       
x 2 will be in the nullspace of A − 2I =
$\begin{bmatrix}
  1 & 1\\
  1 & 1
 \end{bmatrix}$
The nullspace is an entire line; x 2 could be any vector on that line. A natural choice is x 2 =
$\begin{bmatrix} 
 -1\\
  1
 \end{bmatrix}$
Note that these eigenvectors are the same as those of B =
$\begin{bmatrix}
  0 & 1\\                                                                                            
  1 & 0                                                                                              
 \end{bmatrix}$                                                                                         
.Adding 3I to the matrix B =
$\begin{bmatrix}
  0 & 1\\
  1 & 0                                            
 \end{bmatrix}$
added 3 to each of its eigenvalues and did not change its eigenvectors, because
$ Ax = ( B + 3I ) x = \lambda x + 3x = ( \lambda + 3 ) x. $ 
   \subsection*{A caution}

Similarly, if
$ Ax = \lambda x and Bx = αx, ( A + B ) x = ( \lambda + α ) x  $
It would be nice if the eigenvalues of a matrix sum were always the sums of the eigenvalues, but this is only true if A and B have the same eigenvectors. The eigenvalues of the product AB aren’t usually equal to the products
$ \lambda ( A ) \lambda ( b ) $ , either.

\subsection*{Complex eigenvalues}
The matrix Q =
$\begin{bmatrix} 
  0 & -1\\                
  1 & 0                 
 \end{bmatrix}$            
rotates every vector in the plane by 90 ◦ . It has trace 0 =
$ \lambda 1 + \lambda 2 $
and determinant 1 = 
$ \lambda 1 · \lambda 2 $
Its only real eigenvector is the zero vector; any other vector’s direction changes when it is multiplied by Q.

How will this affect our eigenvalue calculation?

$ det ( A − \lambda . I ) =
  \begin{vmatrix}
   − \lambda & 1\\
          1  & − \lambda
  \end{vmatrix}$

$ = \lambda 2 + 1 $

$ det ( A − λI ) = 0 $ 
has solutions 
$ λ 1 = i and λ 2 = − i $ 
If a matrix has a complex eigenvalue a + bi then the \emph{complex conjugate} a − bi is also an eigenvalue of that matrix.
Symmetric matrices have real eigenvalues.
For \emph{antisymmetri}c matrices like Q, for which A T = − A, all eigenvalues are imaginary (λ = bi).

\subsection*{Triangular matrices and repeated eigenvalues}

 For triangular matrices such as A = 
$\begin{bmatrix}
   3  & 1 \\
   0 & 3
  \end{bmatrix}$

the eigenvalues are exactly the 0 3  entries on the diagonal. In this case, the eigenvalues are 3 and 3:

$ det ( A − λ det I )=
\begin{vmatrix}
 3-\lambda & 1\\
   0       & 3-\lambda
\end{vmatrix}$
$ = ( 3 − \lambda )( 3 − \lambda )( = ( a 11 − \lambda )( a 22 − \lambda )) $
  = 0

so
$ \lambda 1 = 3 and \lambda 2 = 3 $
To find the eigenvectors, solve:

$ ( A − λI ) x = 
\begin{bmatrix}
 0 & 1\\
 0 & 0
\end{bmatrix}
x = 0 $ 
to get x1=
$ \begin{bmatrix}
   1\\
   0
\end{bmatrix} $

There is no independent eigenvector x 2

\end{document}
