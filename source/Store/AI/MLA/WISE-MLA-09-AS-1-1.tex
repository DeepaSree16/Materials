\documentclass{book}
\usepackage{graphicx}
\usepackage{listings}
\usepackage{color}
\usepackage{graphicx}
\usepackage{booktabs}
\usepackage{fancyhdr}
\usepackage{amsmath}
\usepackage{hyperref}
\usepackage{csquotes}
\usepackage[english]{babel}
\usepackage{amssymb}
\usepackage{textcomp}
\usepackage{xfrac}
\pagestyle{fancy}
\fancyhf{}
\headheight = 0.9 in
%\rhead{\includegraphics[width=2cm, height=1cm]{logo}}
\lhead{\textbf{\Large{Confidence Intervals Problems}}}
\lfoot{COPYRIGHT ©TALENTSPRINT, 2020. ALL RIGHTS RESERVED.}
\rfoot{\thepage}

\begin{document}
A manufacturer produces piston rings for an automobile engine. It is known that the ring diameter is approximately normally distributed and has a standard deviation $\sigma = 0.03 mm$. A random sample of $32$ piston rings has a mean diameter, $\Bar{x} = 74.036 mm$. Construct a $90\%$ confidence interval around the true population mean for piston ring diameter.\\\\
    \textbf{Solution:}\\
    $n=32,\\
    \Bar{x}=74.036,\\
    \sigma=0.03,\\
    \alpha=0.10,\\
    Z_{\sfrac{10}{2}}=Z_{0.05}=1.645$\\\\
    The Confidence Interval can be calculated using,
    \begin{equation*}
        \Bar{x}-\frac{Z_{\sfrac{\alpha}{2}}\sigma}{\sqrt{n}} \leq \mu \leq \Bar{x}+\frac{Z_{\sfrac{\alpha}{2}}\sigma}{\sqrt{n}}
    \end{equation*}
    \begin{equation*}
        74.036 - \frac{1.645(0.03)}{\sqrt{32}} \leq \mu \leq 74.036 + \frac{1.645(0.03)}{\sqrt{32}}
    \end{equation*}
    Thus the Confidence Interval is,
    \begin{equation*}
        74.0273 \leq \mu \leq 74.0447
    \end{equation*}
\end{document}