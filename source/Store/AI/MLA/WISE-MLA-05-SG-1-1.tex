\documentclass{book}
%\usepackage{listings}
\usepackage{amsmath}
\usepackage{color}
\usepackage{graphicx}
%\usepackage{booktabs}
\usepackage{fancyhdr}
\usepackage[english]{babel}
\pagestyle{fancy}
\fancyhf{}
%\rhead{\includegraphics[width=2cm, height=0.5cm]{logo}}
\lhead{Chain Rule}
\lfoot{COPYRIGHT ©TALENTSPRINT, 2020. ALL RIGHTS RESERVED.}
\rfoot{\thepage}
\begin{document}

\section*{Chain Rule} 

The product rule tells us how to find the derivative of a product of functions like f (x) · g(x). The composition or “chain” rule tells us how to find the derivative of a composition of functions like f (g(x)). Composition of functions is about substitution – you substitute a value forx into the formula for g, then you substitute the result into the formula for f . An example of a composition of two functions is $y = (\sin t)10$ (which is usually written as $y = \sin 10 t$).



One way to think about composition of functions is to use new variable names. For example, for the function
$y = \sin^{10} t$
we can say
$x = \sin^t$ 
and then 
$y = x^{10}$
.Notice that if you plug 
$x = \sin^t$ 
in to the formula for y you get back to 
$y = \sin^{10} t$
. It’s good practice to introduce new variables when they’re convenient, and this is one place where it’s very convenient. \par

So, how do we find the derivative of a composition of functions? We’re trying to find the slope of a tangent line; to do this we take a limit of slopes
$\frac{\Delta.y}{\Delta.t}$ 
of secant lines. Here y is a function of x, x is a function of t, and we want to know how y changes with respect to the original variable t. Here again using that intermediate variable x is a big help: \par

  
   $$\frac{\Delta.y}{\Delta.t}  = \frac{\Delta.y}{\Delta.x} \frac{\Delta.x}{\Delta.y}$$

because when we perform the multiplication, the small change $\delta x$ cancels.  The derivative of y with respect to t is

   $$\lim_{\Delta.x \to\infty}\frac{\Delta.y}{\Delta.t}$$

what happens when $\Delta t$ gets small? Because $x =\sin t$ is a continuous function, as $\Delta t$ approaches 0, $\Delta x$ also approaches zero.
It turns out that:

   $$\lim_{\Delta.x \to\infty}\frac{\Delta.y}{\Delta.t} = \frac{dy}{dt} = \frac{dy}{dx}\frac{dx}{dt}   \leftarrow  This is the Chain Rule $$   

                              
The derivative of a composition of functions is a product. In the example
$y = \sin t^{10}$
, we have the “inside function” $x = \sin t$ and the “outside function”
$y = x^{10}$
The chain rule tells us to take the derivative of y with respect to x and multiply it by the derivative of x with respect to t. \par
The derivative of
$y = x^{10}$
is
$\frac{dy}{dx} = y = 10x^9$
The derivative of $x = \sin t$ is
$\frac{dx}{dt} = \cos.t$
The chain rule tells us that
$\frac{d}{dt}.\sin^{10}.t = 10.x^9.\cos.t$
This is correct,but if a friend asked you for the derivative of
$\sin^{10}.t$ 
and you answered
$10x^9.\cos.t$
your friend wouldn’t know what x stood for. The last step in this process is to rewrite x in terms of t:
   
    $$\frac{d}{dt}.\sin^{10}.t  =  10\sin t^9.\cos.t = 10\sin ^9.t.\cos.t$$

Here is another way of writing the chain rule:
 
    $$\frac{d}{dx}.f(g(x)) = (f'(g(x))g'(x))$$

\end{document}
