\documentclass{book}
\usepackage{listings}
\usepackage{color}
\usepackage{graphicx}
\usepackage{booktabs}
\usepackage{fancyhdr}
\usepackage{amsmath}
\usepackage{hyperref}
\usepackage{csquotes}
\usepackage[english]{babel}
\pagestyle{fancy}
\fancyhf{}
\headheight = 0.95 in
%\rhead{\includegraphics[width=2cm, height=1cm]{logo}}
\lhead{\textbf{\Large{Google Colaboratory User Guide}}}
\lfoot{COPYRIGHT ©TALENTSPRINT, 2020. ALL RIGHTS RESERVED.}
\rfoot{\thepage}

\begin{document}

Getting started with Google Colaboratory (herein referred to as Colab) is a free Jupyter notebook environment which doesn\rq t require any setup and runs entirely on the cloud.

You will be working on your own Google Drives. Step by step details of this are given below:

\begin{enumerate}
\item \textbf{Setting Up Google Drive:}
	\begin{itemize}
		\item  Create a folder with a name of your choice. Eg:\lq Colab-docs\rq
	\end{itemize}
\item \textbf{Colab into your drive:}
	\begin{itemize}
\item Go to the folder \lq Colab-docs \rq (The folder you created)
\item Go to New \textgreater More \textgreater Connect More apps \textgreater Colab
\item Add the app into your drive by pressing connect button
	\end{itemize}
\item \textbf{Creating a new notebook:}
	\begin{itemize}
		\item Go to the folder \lq Colab-docs\rq (The folder you created)
		\item Then Right click \textgreater More \textgreater Colaboratory
		\item Rename notebook by means of clicking the file name
		\item Save the notebook
	\end{itemize}
\item \textbf{Using existing Experiment notebooks:} (To upload the notebook)
	\begin{itemize}
		\item Go to the link (\url{https://goo.gl/o5LJo8})
		\item Download the folder from the given link (Eg: ``Week\_0")
		\item Upload the downloaded folder to your drive
		\item Once uploaded Go to \textgreater folder \textgreater file \textgreater right click \textgreater open with \textgreater Colaboratory
	\end{itemize}
\item \textbf{Setting up GPU :} By default, all Experiments run on CPU. To run the Experiments on GPU, we need to change the runtime type to GPU using the following steps:
	\begin{itemize}
		\item Go to Edit \textgreater Notebook settings or Runtime \textgreater Change runtime type
		\item Select GPU as Hardware accelerator
	\end{itemize}
\item \textbf{To interrupt the execution:}
	\begin{itemize}
		\item Go to Runtime \textgreater interrupt execution
	\end{itemize}
\item \textbf{To mount your drive :}

 Run these codes first in order to install the necessary libraries and perform authorization. \\
			\begin{center} from google.colab import drive \end{center} 
				\begin{center}drive.mount('/content/drive/') \end{center}
					When you run the code, you will get a link. \textbf{Click} the link, \textbf{copy} verification code and \textbf{paste} it to text box. 


\end{enumerate}





\end{document}
