%\tolerance=10000
%\documentclass[prl,twocoloumn,preprintnumbers,amssymb,pla]{revtex4}
\documentclass[prl,twocolumn,showpacs,preprintnumbers,superscriptaddress]{revtex4}
\documentclass{article}
\usepackage{graphicx}
\usepackage{color}
\usepackage{dcolumn}
%\linespread{1.7}
\usepackage{bm}
%\usepackage{eps2pdf}
\usepackage{graphics}
\usepackage{pdfpages}
\usepackage{caption}
%\usepackage{subcaption}
\usepackage[demo]{graphicx} % omit 'demo' for real document
%\usepackage{times}
\usepackage{multirow}
\usepackage{hhline}
\usepackage{subfig}
\usepackage{amsbsy}
\usepackage{amsmath}
\usepackage{amsfonts}
\usepackage{amsthm}
\usepackage{float}
\documentclass{article}
\usepackage{amsmath,systeme}
\usepackage{tikz}

\sysalign{r,r}

% \textheight = 8.5 in
% \topmargin = 0.3 in

%\textwidth = 6.5 in
% \textheight = 8.5 in
%\oddsidemargin = 0.0 in
%\evensidemargin = 0.0 in

%\headheight = 0.0 in
%\headsep = 0.0 in
%\parskip = 0.2in
%\parindent = 0.0in

% \newcommand{\ket}[1]{\left|#1\right\rangle}
% \newcommand{\bra}[1]{\left\langle#1\right|}
\newcommand{\ket}[1]{| #1 \rangle}
\newcommand{\bra}[1]{\langle #1 |}
\newcommand{\braket}[2]{\langle #1 | #2 \rangle}
\newcommand{\ketbra}[2]{| #1 \rangle \langle #2 |}
\newcommand{\proj}[1]{| #1 \rangle \langle #1 |}
\newcommand{\al}{\alpha}
\newcommand{\be}{\beta}
\newcommand{\op}[1]{ \hat{\sigma}_{#1} }
\def\tred{\textcolor{red}}
\def\tgre{\textcolor{green}}


\theoremstyle{plain}
\newtheorem{theorem}{Theorem}

\newtheorem{lemma}[theorem]{Lemma}
\newtheorem{corollary}[theorem]{Corollary}
\newtheorem{proposition}[theorem]{Proposition}
\newtheorem{conjecture}[theorem]{Conjecture}

\theoremstyle{definition}
\newtheorem{definition}[theorem]{Definition}


\begin{document}
\begin{widetext}
\\
\\
\\

\begin{wrapfigure}
\centering
%\includegraphics[\textwidth]{TS_IISc.png}
\end{wrapfigure}
\begin{figure}[h!]
 \begin{right}
  \hfill\includegraphics[\textwidth, right]{TS_IISc.png}
 \end{right}
\end{figure}
%\noindent\textbf{1. Which is the metric used to evaluate the performance of Text to Speech Synthesis?}
\noindent\textbf{1. In which of the following acoustic models, do we use the non-RNN Acoustic Model?}
\\
\\
\\
%A. Robust Acoustic Model.
A. HMM based acoustic feature generation
\\
\\
B. Neural network based fast acoustic model
\\
\\
C. Neural network based robust acoustic model
\\
\\
\\
\textbf{Answer: B}
\\
\\
\\
\\
\textbf{2. In Statistical Parametric Speech Synthesis, speech is generated from stored exemplars rather than from probability distribution of the acoustic features, given the linguistic specification.}
\\
\\
\\
\noindent A. False
\\
\\
B. True 
\\
\\
\\
\textbf{Answer: A}
\\
\\
\\
\\
\textbf{3. Which of the following acoustic models are NOT examples of Autoregression model?}
\\
\\
\\
A. Deep Voice 3
\\
\\
B. FastSpeech 
\\
\\
C. Glow-TTS 
\\
\\
D. Tacotron
\\
\\
E. A, B, C and D
\\
\\
F. Only A and D
\\
\\
G. Only B and C
\\
\\
\\
\textbf{Answer: G}
\\
\\
\\
\\
\textbf{4. What are the Key components in Text to Speech Synthesis (TTS)?}
\\
\\
\\
A. Text Front-end.
\\
\\
B. Acoustic model.
\\
\\
C. Waveform generator or Vocoder.
\\
\\
D. Only A and B.
\\
\\
E. Only B and C.
\\
\\
F. A, B and C.
\\
\\
\\
\textbf{Answer: F}
\\
\\
\\
\\
\textbf{5. Which of the following is/are true for Concatenative Synthesis?}
\\
\\
\\
A. It is based on concatenating together small units to produce an utterance.
\\
\\
B. It is done using a time-domain joining algorithm.
\\
\\
C. Requires a large database of speech.
\\
\\
D. It is not really possible to do voice adaption.
\\
\\
E. All of the above.
\\
\\
\\
\textbf{Answer: E}
\\
\\
\\
\\
\\
\\
\end{widetext}
\end{document}