%\tolerance=10000
%\documentclass[prl,twocoloumn,preprintnumbers,amssymb,pla]{revtex4}
\documentclass[prl,twocolumn,showpacs,preprintnumbers,superscriptaddress]{revtex4}
\documentclass{article}
\usepackage{graphicx}
\usepackage{color}
\usepackage{dcolumn}
%\linespread{1.7}
\usepackage{bm}
%\usepackage{eps2pdf}
\usepackage{graphics}
\usepackage{pdfpages}
\usepackage{caption}
%\usepackage{subcaption}
\usepackage[demo]{graphicx} % omit 'demo' for real document
%\usepackage{times}
\usepackage{multirow}
\usepackage{hhline}
\usepackage{subfig}
\usepackage{amsbsy}
\usepackage{amsmath}
\usepackage{amsfonts}
\usepackage{amsthm}
\usepackage{float}
\documentclass{article}
\usepackage{amsmath,systeme}
\usepackage{tikz}

\sysalign{r,r}

% \textheight = 8.5 in
% \topmargin = 0.3 in

%\textwidth = 6.5 in
% \textheight = 8.5 in
%\oddsidemargin = 0.0 in
%\evensidemargin = 0.0 in

%\headheight = 0.0 in
%\headsep = 0.0 in
%\parskip = 0.2in
%\parindent = 0.0in

% \newcommand{\ket}[1]{\left|#1\right\rangle}
% \newcommand{\bra}[1]{\left\langle#1\right|}
\newcommand{\ket}[1]{| #1 \rangle}
\newcommand{\bra}[1]{\langle #1 |}
\newcommand{\braket}[2]{\langle #1 | #2 \rangle}
\newcommand{\ketbra}[2]{| #1 \rangle \langle #2 |}
\newcommand{\proj}[1]{| #1 \rangle \langle #1 |}
\newcommand{\al}{\alpha}
\newcommand{\be}{\beta}
\newcommand{\op}[1]{ \hat{\sigma}_{#1} }
\def\tred{\textcolor{red}}
\def\tgre{\textcolor{green}}


\theoremstyle{plain}
\newtheorem{theorem}{Theorem}

\newtheorem{lemma}[theorem]{Lemma}
\newtheorem{corollary}[theorem]{Corollary}
\newtheorem{proposition}[theorem]{Proposition}
\newtheorem{conjecture}[theorem]{Conjecture}

\theoremstyle{definition}
\newtheorem{definition}[theorem]{Definition}


\begin{document}
\begin{widetext}
\\
\\
\\

\begin{wrapfigure}
\centering
%\includegraphics[\textwidth]{TS_IISc.png}
\end{wrapfigure}
\begin{figure}[h!]
 \begin{right}
  \hfill\includegraphics[\textwidth, right]{TS_IISc.png}
 \end{right}
\end{figure}
%\noindent\textbf{1. Which is the metric used to evaluate the performance of Text to Speech Synthesis?}
\noindent\textbf{1. What is the size of the output image after an input image of size $327 \times 327$ is passed through a convolution layer of 48 filters of size $21 \times 21$, with no zero padding and stride of 3, followed by another convolution layer of 64 filters of size $3 \times 3$ with zero padding 1 and stride of 1?}. 
%What will be the volume of the final image?}
\\
\\
\\
A. $103 \times 103 \times 48$
\\
\\
B. $103 \times 103 \times 64$ 
\\
\\
C. $327 \times 327 \times 64$
\\
\\
D. $327 \times 327 \times 48$
\\
\\
\\
\textbf{Answer: B}
\\
\\
\\
\\
\textbf{2. What is the output of the following matrix after applying average pooling operation, $2 \times 2$, and stride = 2?}
\\
\\
\[ 
\begin{bmatrix}    5 & 7 & 9 & 8 \\     3 & 1 & 2 & 3\\ 4 & 9 & 5 & 7 \\ 8 & 3 & 8 & 6 \\ \end{bmatrix}
%\begin{bmatrix}    1 & 2 & 3 \\     4 & 5 & 6 \\ 7 & 8 & 9  \\ \end{bmatrix} \ast  
%\begin{bmatrix}    1 & 2 \\ 3 & 4  \\ \end{bmatrix} 
\]
\\
\\
\\
A. \begin{bmatrix}    4 & 6 \\  5 & 8  \\ \end{bmatrix}
\\
\\
\\
B. \begin{bmatrix}    4.5 & 6 \\  3 & 7.5  \\ \end{bmatrix}
\\
\\
\\
C. \begin{bmatrix}    4 & 5.5 \\  6 & 6.5  \\ \end{bmatrix}
\\
\\
\\
D. \begin{bmatrix}    5.5 & 7 \\  2 & 8  \\ \end{bmatrix}
\\
\\
\\
\\
\textbf{Answer: C}
\\
\\
\\
\\
\textbf{3. Consider a layer of a convolutional neural network with the following weight matrix and bias vector:
\\
\[ 
W = \begin{bmatrix}    \phantom{-} 0.4 & \phantom{-} 0.1 & \phantom{-} 0.1 \\     -0.5 & \phantom{-} 0.9 & \phantom{-} 0.4 \\ -0.8 & \phantom{-} 0.5 & \phantom{-} 0.1 \\ -0.6 & \phantom{-} 0.8 & -0.7  \\ \end{bmatrix} \ b = \begin{bmatrix}   \phantom{-} 0.1 \\ -0.1 \\ -0.2 \\ \phantom{-} 0.1 \end{bmatrix} \]
\\
The input to this layer is:
\\
\[
x = \begin{bmatrix}    \phantom{-} 0.4 \\ -1.0 \\ \phantom{-} 0.1 \end{bmatrix} \]
\\
If the activation function used is ReLU, what is the output $y$ of the layer?}
\\
\\
\\
A. $y$ = \begin{bmatrix}    \phantom{-} 0.17 \\ -1.16 \\ -1.01 \\ -1.01 \end{bmatrix} 
\\
\\
\\
B. $y$ = \begin{bmatrix}    \phantom{-} 0.17 \\ 0 \\ 0 \\ 0 \end{bmatrix}
\\
\\
\\
C. $y$ = \begin{bmatrix}    \phantom{-} 0.07 \\ -1.06 \\ -0.81 \\ -1.11 \end{bmatrix}
\\
\\
\\
D. $y$ = \begin{bmatrix}    \phantom{-} 0.07 \\ 0 \\ 0 \\ 0 \end{bmatrix}
\\
\\
\\
\\
\textbf{Answer: B}
\\
\\
\\
\\
%\textbf{4. Select the correct statement from the following in order of increasing number of parameters?}
%and what is the type of the neural network used for image segmentation?}
\textbf{4. Select the false statement from the following:}
\\
\\
\\
A. Most of the trainable parameters are in the convolutional layers in a CNN such as VGG-16.
%$\underline{\hspace{3cm}}$ VGGNet16 $<$ LeNet $<$ AlexNet $<$ ResNet150
\\
\\
B. The total number of trainable parameters in VGG-16 is approximately 138 million.
%AlexNet $<$ VGGNet16 $<$ LeNet $<$ ResNet150
\\
\\
C. Activation Function ReLU was first to be used in AlexNet to overcome the Vanishing Gradient (VG) problem.
%LeNet $<$ VGGNet16 $<$ AlexNet $<$ ResNet150
\\
\\
D. The gap between the pixel level information of an image and it's high level interpretation is popularly known as semantic gap.
%ResNet-152 gave the best Image Classifiation Results at 3.57%
%AlexNet $<$ ResNet150 $<$ LeNet $<$ VGGNet16 \\
\\
%E. 
%LeNet $<$ AlexNet $<$ VGGNet16 $<$ ResNet150
\\
\\
\\
\textbf{Answer: A}
\\
\\
\\
\\
\textbf{5. A filter of size $7 \times 7$ is applied to an image of size $7 \times 7$, with stride of 1. What should be the zero padding (P) used in the convolutional layer in order for the volume of the final image to preserve the size spatially?}
\\
\\
\\
A. Zero padding (P) = 3
\\
\\
B. Zero padding (P) = 2
\\
\\
C. Zero padding (P) = 1
\\
\\
D. None of the above
\\
\\
\\
\textbf{Answer: A}
\\
\\
\\
\\
\\
\\
\end{widetext}
\end{document}