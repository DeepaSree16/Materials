%\tolerance=10000
%\documentclass[prl,twocoloumn,preprintnumbers,amssymb,pla]{revtex4}
\documentclass[prl,twocolumn,showpacs,preprintnumbers,superscriptaddress]{revtex4}
\documentclass{article}
\usepackage{graphicx}
\usepackage{color}
\usepackage{dcolumn}
%\linespread{1.7}
\usepackage{bm}
%\usepackage{eps2pdf}
\usepackage{graphics}
\usepackage{pdfpages}
\usepackage{caption}
%\usepackage{subcaption}
\usepackage[demo]{graphicx} % omit 'demo' for real document
%\usepackage{times}
\usepackage{multirow}
\usepackage{hhline}
\usepackage{subfig}
\usepackage{amsbsy}
\usepackage{amsmath}
\usepackage{amsfonts}
\usepackage{amsthm}
\usepackage{float}
\documentclass{article}
\usepackage{amsmath,systeme}
\usepackage{tikz}

\sysalign{r,r}

% \textheight = 8.5 in
% \topmargin = 0.3 in

%\textwidth = 6.5 in
% \textheight = 8.5 in
%\oddsidemargin = 0.0 in
%\evensidemargin = 0.0 in

%\headheight = 0.0 in
%\headsep = 0.0 in
%\parskip = 0.2in
%\parindent = 0.0in

% \newcommand{\ket}[1]{\left|#1\right\rangle}
% \newcommand{\bra}[1]{\left\langle#1\right|}
\newcommand{\ket}[1]{| #1 \rangle}
\newcommand{\bra}[1]{\langle #1 |}
\newcommand{\braket}[2]{\langle #1 | #2 \rangle}
\newcommand{\ketbra}[2]{| #1 \rangle \langle #2 |}
\newcommand{\proj}[1]{| #1 \rangle \langle #1 |}
\newcommand{\al}{\alpha}
\newcommand{\be}{\beta}
\newcommand{\op}[1]{ \hat{\sigma}_{#1} }
\def\tred{\textcolor{red}}
\def\tgre{\textcolor{green}}


\theoremstyle{plain}
\newtheorem{theorem}{Theorem}

\newtheorem{lemma}[theorem]{Lemma}
\newtheorem{corollary}[theorem]{Corollary}
\newtheorem{proposition}[theorem]{Proposition}
\newtheorem{conjecture}[theorem]{Conjecture}

\theoremstyle{definition}
\newtheorem{definition}[theorem]{Definition}


\begin{document}
\begin{widetext}
\\
\\
\\

\begin{wrapfigure}
\centering
%\includegraphics[\textwidth]{TS_IISc.png}
\end{wrapfigure}
\begin{figure}[h!]
 \begin{right}
  \hfill\includegraphics[\textwidth, right]{TS_IISc.png}
 \end{right}
\end{figure}
%\noindent\textbf{1. Which is the metric used to evaluate the performance of Text to Speech Synthesis?}
\noindent\textbf{1. Atrous convolution used in UNET model is also known as:}
\\
\\
\\
%A. Robust Acoustic Model.
A. Up Convolution
\\
\\
B. Transposed Convolution 
\\
\\
C. Dilated Convolution
\\
\\
D. Down Convolution
\\
\\
\\
\textbf{Answer: C}
\\
\\
\\
\\
\textbf{2. What is the loss function while training the SimCLR?}
\\
\\
\\
\noindent A. Contrastive Loss Function
\\
\\
B. Pixel Wise Cross Entropy Loss 
\\
\\
C. Mean Square Error Loss
\\
\\
D. Binary Cross Entropy
\\
\\
\\
\textbf{Answer: A}
\\
\\
\\
\\
\textbf{3. State which of the following statement is FALSE while using UNET neural network?}
\\
\\
\\
A. $2 \times 2$ Max pooling layer decreases spatial feature resolution by $2$ in each dimension and $2 \times 2$ upconvolution layer increases the spatial feature resolution by $2$ in each dimension. \\
\\
B. $2 \times 2$ Max pooling layer increases spatial feature resolution by $2$ in each dimension and $2 \times 2$ upconvolution layer decreases the spatial feature resolution by $2$ in each dimension.
\\
\\
C. U-Net is a convolutional neural network that is designed for performing semantic segmentation.
%A. While using UNET for performing Image Segmentation, upsampling of the image is done prior to downsampling.
%\\
%\\
%B. While using UNET for performing Image Segmentation, downsampling of the image is done prior to upsampling.
%\\
%\\
%A. Upsampling of the image with a stride of $\frac{1}{2}$ means increasing the size/resolution of the input image by $2$. 
%\\
%\\
%D. Downsampling of the image with a stride of $\frac{1}{2}$ means increasing the size/resolution of the input image by 2. 
%\\
%\\
%B. Upsampling of the image with a stride of $2$ means increasing the size/resolution of the input image by $2$.
%\\
%\\
%C. Upsampling of the image with a stride of $\frac{1}{2}$ means decreasing the size/resolution of the input image by $2$. 
%\\
%\\
%D. Upsampling of the image with a stride of $2$ means decreasing the size/resolution of the input image by $2$. 
%\\
%\\
%F. Downsampling with a stride of $2$ means increasing the size/resolution of the input image by 2.
\\
\\
\\
\textbf{Answer: B}
\\
\\
\\
\\
%\textbf{3. SimCLR is an example of which of the following machine learning algorithms?}
%\\
%\\
%\\
%A. Unsupervised Learning
%\\
%\\
%B. Self Supervised Learning 
%\\
%\\
%C. Unsupervised Learning 
%\\
%\\
%D. Reinforcement Learning
%\\
%\\
%\\
%\textbf{Answer: B}
%\\
%\\
%\\
%\\
\textbf{4. What of the following statement(s) is/are TRUE regarding image segmentation?}
%and what is the type of the neural network used for image segmentation?}
\\
\\
\\
A. Main idea of image segmentation is to treat all the pixels jointly and get a segmentation map as output.
%and a fully convolutional neural network with fully connected layers is used for image segmentation.
%\\
%\\
%B. Main idea of image segmentation is to treat all the pixels jointly and get a segmentation map as output and a fully convolutional neural network with no fully connected layers is used for image segmentation.
\\
\\
B. Main idea image segmentation is to treat all the pixels independently and get a segmentation map as output.
%and a fully convolutional neural network with fully connected layers is used for image segmentation.
%\\
%\\
%D. Main idea image segmentation is to treat all the pixels independently and get a segmentation map as output and a fully convolutional neural network with no fully connected layers is used for image segmentation.
\\
\\
C. Image segmentation is  used to locate objects and boundaries (lines, curves, etc.) in images.
\\
\\
D. A and C
\\
\\
E. B and C
\\
\\
\\
\textbf{Answer: D}
\\
\\
\\
\\
\textbf{5. The mathematical equivalence of skip connection in residual block in the architecture of ResNet model is:}
\\
\\
\\
A. Summation
\\
\\
B. Convolution
\\
\\
C. Identity function
\\
\\
D. None of the above
\\
\\
\\
\textbf{Answer: C}
\\
\\
\\
\\
\\
\\
\end{widetext}
\end{document}