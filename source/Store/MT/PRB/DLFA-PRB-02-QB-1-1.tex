%\tolerance=10000
%\documentclass[prl,twocoloumn,preprintnumbers,amssymb,pla]{revtex4}
\documentclass[prl,twocolumn,showpacs,preprintnumbers,superscriptaddress]{revtex4}
\documentclass{article}
\usepackage{graphicx}
\usepackage{color}
\usepackage{dcolumn}
%\linespread{1.7}
\usepackage{bm}
%\usepackage{eps2pdf}
\usepackage{graphics}
\usepackage{pdfpages}
\usepackage{caption}
%\usepackage{subcaption}
\usepackage[demo]{graphicx} % omit 'demo' for real document
%\usepackage{times}
\usepackage{multirow}
\usepackage{hhline}
\usepackage{subfig}
\usepackage{amsbsy}
\usepackage{amsmath}
\usepackage{amsfonts}
\usepackage{amsthm}
\usepackage{float}
\documentclass{article}
\usepackage{amsmath,systeme}

\sysalign{r,r}

% \textheight = 8.5 in
% \topmargin = 0.3 in

%\textwidth = 6.5 in
% \textheight = 8.5 in
%\oddsidemargin = 0.0 in
%\evensidemargin = 0.0 in

%\headheight = 0.0 in
%\headsep = 0.0 in
%\parskip = 0.2in
%\parindent = 0.0in

% \newcommand{\ket}[1]{\left|#1\right\rangle}
% \newcommand{\bra}[1]{\left\langle#1\right|}
\newcommand{\ket}[1]{| #1 \rangle}
\newcommand{\bra}[1]{\langle #1 |}
\newcommand{\braket}[2]{\langle #1 | #2 \rangle}
\newcommand{\ketbra}[2]{| #1 \rangle \langle #2 |}
\newcommand{\proj}[1]{| #1 \rangle \langle #1 |}
\newcommand{\al}{\alpha}
\newcommand{\be}{\beta}
\newcommand{\op}[1]{ \hat{\sigma}_{#1} }
\def\tred{\textcolor{red}}
\def\tgre{\textcolor{green}}


\theoremstyle{plain}
\newtheorem{theorem}{Theorem}

\newtheorem{lemma}[theorem]{Lemma}
\newtheorem{corollary}[theorem]{Corollary}
\newtheorem{proposition}[theorem]{Proposition}
\newtheorem{conjecture}[theorem]{Conjecture}

\theoremstyle{definition}
\newtheorem{definition}[theorem]{Definition}


\begin{document}
\begin{widetext}
\\
\\
\\

\begin{wrapfigure}
\centering
%\includegraphics[\textwidth]{TS_IISc.png}
\end{wrapfigure}
\begin{figure}[h!]
 \begin{right}
  \hfill\includegraphics[\textwidth, right]{TS_IISc.png}
 \end{right}
\end{figure}
\\
\\
\\
\noindent\textbf{Let the random variable $X$ denote the time a person waits for an elevator to arrive. Suppose the longest one would need to wait for the elevator is 2 minutes, so that the possible values of $X$ (in minutes) are given by the interval $[0,2]$. A possible PDF for $X$ is given by:}
\\
\begin{align*}
    f(x) = \Bigg\{\begin{matrix}
x,&\  for\  0 \le x \le 1 \\
2 - x,&\  for\  1 < x \le 2 \\
0,&\  otherwise
\end{matrix}
\end{align*}
\\
\\
Given the $PDF$ above, answer questions 1 to 3 below.
\\
\\
\\
\textbf{1.} What is the expected value of $X$
\\
\\
\noindent A. $\frac{1}{3}$\\
\\
B. $\frac{2}{3}$\\
\\
C. $1$\\
\\
D. $\frac{3}{4}$
\\
\\
\\
\textbf{Answer: C}
\\
\\
\textbf{Solution:}
If $X$ is a continuous random variable with $PDF$ $f(x)$, then the expected value (or mean) of $X$ is given by
\begin{equation}
    \mu = \mu_{X} = E[X] = \int_{-\infty}^{\infty} x\cdot f(x) \,dx {}\nonumber
\end{equation}
\\
Applying the definition given in eq (28) above to our problem, we compute the expected value of $X$:
\begin{equation}
    E[X] = \int_{0}^{1} x\cdot x  \,dx + \int_{1}^{2} x\cdot (2 - x)  \,dx = \int_{0}^{1} x^2  \,dx + \int_{1}^{2} (2x - x^2)  \,dx = \frac{1}{3} + \frac{2}{3} = 1 {}\nonumber
\end{equation}
\\
\\
\\
\textbf{2.} What is the variance of $X$
\\
\\
\noindent A. $1$\\
\\
B. $\frac{1}{6}$\\
\\
C. $\frac{7}{6}$\\
\\
D. $\frac{3}{4}$
\\
\\
\\
\textbf{Answer: B}
\\
\\
\textbf{Solution:}
Now we calculate the variance and standard deviation of $X$, by first finding the expected value of $X^2$.
\begin{equation}
    E[X^2] = \int_{0}^{1} x^2\cdot x  \,dx + \int_{1}^{2} x^2\cdot (2 - x)  \,dx = \int_{0}^{1} x^3  \,dx + \int_{1}^{2} (2x^2 - x^3)  \,dx = \frac{1}{4} + \frac{11}{12} = \frac{7}{6} {}\nonumber
\end{equation}
\\
Thus we have
\begin{equation}
    Var(X) = E[X^2] - \mu^2 = \frac{7}{6} - 1 = \frac{1}{6} {}\nonumber
\end{equation}
\\
\\
\\
\\
\textbf{3.} What is the standard deviation of $X$
\\
\\
\noindent A. $\frac{1}{\sqrt{6}}$\\
\\
B. $\sqrt{6}$\\
\\
C. $6$\\
\\
D. $\frac{1}{6}$
\\
\\
\\
\textbf{Answer: A}
\\
\\
\textbf{Solution:}
$SD(X) = \sqrt{Var(x)} = \frac{1}{\sqrt{6}}$ 
\\
\\
\\
\textbf{4.} Given a matrix $X_{m \times n}$. What are the properties of the co-variance matrix of $X$?
\\
\\
\noindent A. It is symmetric\\
\\
B. It is positive semi definite\\
\\
C. Its main diagonal contains variances\\
\\
D. All of the above
\\
\\
\\
\textbf{Answer: D}
\\
\\
\textbf{Solution:}
In probability theory and statistics, a co-variance matrix (also known as auto co-variance matrix, dispersion matrix, variance matrix, or variance co-variance matrix) is a square matrix giving the co-variance between each pair of elements of a given random vector. Any co-variance matrix is symmetric and positive semi-definite and its main diagonal contains variances (i.e., the co-variance of each element with itself).
\\
\newpage
\noindent\textbf{5.} Let $X$ and $Y$ be two discrete random variables with joint probability mass functions $p_{XY}(1, 2) = p_{XY}(3, 4) = \frac{1}{2}$. What is the co-variance matrix of the following matrix of $X$ and $Y$?
\\
\\
A. \Sigma(X, Y) = \begin{bmatrix}
    1 & 1 \\
    1 & 1 \\
          
\end{bmatrix}
\\
\\
\\
B.\  \Sigma(X, Y) = \begin{bmatrix}
    2 & 2 \\
    2 & 2 \\
          
\end{bmatrix}
\\
\\
\\
C.\  \Sigma(X, Y) = \begin{bmatrix}
    2 & 2.5 \\
    2.5 & 3 \\
          
\end{bmatrix}
\\
\\
\\
D.\  \Sigma(X, Y) = \begin{bmatrix}
    1 & 2 \\
    3 & 4 \\
          
\end{bmatrix}
\\
\\
\\
\textbf{Answer: A}
\\
\\
\textbf{Solution:}\ 
$Marginal PMFs are calculated as follows:$
\\
\\
p_{X}(1) = \Sigma_{y}\ p_{XY}(1, y) = \frac{1}{2}
\\
\\
$Similarly$,\  p_{X}(3) =  \frac{1}{2}
\\
\\
p_{Y}(2) = \Sigma_{x}\ p_{XY}(x, 2) = \frac{1}{2}
\\
\\
p_{Y}(4) = \frac{1}{2}
\\
\\
\textbf{$Expectation$}:
\\
\\
E[X] = \frac{1}{2} \times 1 + \frac{1}{2} \times 3 = 2
\\
\\
E[Y] = \frac{1}{2} \times 2 + \frac{1}{2} \times 4 = 3
\\
\\
E[X^2] = \frac{1}{2} \times 1^2 + \frac{1}{2} \times 3^2 = 5
\\
\\
E[Y^2] = \frac{1}{2} \times 2^2 + \frac{1}{2} \times 4^2 = 10
\\
\\
\textbf{$Variance$}:
\\
\\
Var[X] = E[X^2] - (E[X])^2 = 5 - 2^2 = 5 - 4 = 1
\\
\\
Var[Y] = E[Y^2] - (E[Y])^2 = 10 - 3^2 = 10 - 9 = 1
\\
\\
\textbf{$Co-variance$:}
\\
\\
\begin{equation}
    Cov[X, Y] = E[\{X - E(X)\}\{Y - E(Y)\}] {}\nonumber
%\end{equation}
%\\
%\begin{equation}
    = E[X\cdot \{Y - E[Y]\} - E[X]\cdot \{Y - E[Y]\}] {}\nonumber
\end{equation}
%\begin{equation}
%    = E[\{X\cdot Y - X\cdot E(X)\}\{Y - E(Y)\}]  {}\nonumber
%\end{equation}
\begin{equation}
    = E[X\cdot Y - X\cdot E[Y] - E[X]\cdot Y + E[X]\cdot [Y] {}\nonumber
\end{equation}
\begin{equation}
    = E[X\cdot Y - E[X]\cdot E(Y) - E[X]\cdot E[Y] + E[X]\cdot [Y] {}\nonumber
\end{equation}
\begin{equation}
    = E[X\cdot Y] - E[X]\cdot  E(Y) - E[X]\cdot E[Y] + E[X]\cdot [Y] {}\nonumber
\end{equation}
\begin{equation}
    = E[X\cdot Y] - E[X]\cdot  E[Y]  {}\nonumber
\end{equation}
Now we calculate $E[X\cdot Y]$ as follows:
\\
\begin{equation}
    E[X\cdot Y] = \Sigma_{x} \Sigma_{y} xy f_{XY}(x, y) = (1 \times 2)\times \frac{1}{2} + (3 \times 4)\times \frac{1}{2} = 7 {}\nonumber
\end{equation}
\begin{equation}
    Cov[X, Y] = 7 - 2\times 3 = 7 - 6 = 1 {}\nonumber
\end{equation}
%or rearranging, we get
%\begin{equation}
%    Cov[X, Y] = \frac{1}{2}(1 \times 2 - 2 \times 3) + \frac{1}{2}(3 \times 4 - 2 \times 3) = 1 {}\nonumber
%\end{equation}
\\
\\
So: \Sigma(X, Y) = \begin{bmatrix}
    1 & 1 \\
    1 & 1 \\
          
\end{bmatrix}
\\
\\
\\
%\noindent\textbf{5.} What is the co-variance matrix of the following matrix $A$?
%\\
%\\
%\[
%A = \begin{bmatrix}
%    1 & 2 \\
%    3 & 4 \\
%          
%\end{bmatrix}
%\]
%\noindent A. CoVAR(A) = %\begin{bmatrix}
%    2 & 2 \\
%    2 & 2 \\
%          
%\end{bmatrix}
%\\
%\\
%\\
%B. CoVAR(A) = \begin{bmatrix}
%    1 & 3 \\
%    2 & 4 \\
%          
%\end{bmatrix}
%\\
%\\
%\\
%C. CoVAR(A) = \begin{bmatrix}
%    2 & 1 \\
%    4 & 3 \\
%          
%\end{bmatrix}
%\\
%\\
%\\
%D. CoVAR(A) = \begin{bmatrix}
%    4 & 3 \\
%    2 & 1 \\
%          
%\end{bmatrix}
%\\
%\\
%\\
%\\
%\textbf{Answer: A}
%\\
%\\
%\textbf{Solution}:
%Suppose that you want to find the co-variance of the following set:
%\\
%\\
%$X = 1, 3\  (mean = \hat x = 2)$
%\\
%\\
%$Y = 2, 4\  (mean = \hat y = 3)$
%\\
%\\
%Here, you have to use the following co-variance equation that is:
%The formula for computing the co-variance of the variables X and Y is
%\begin{equation}
%    COV(X, Y) = \frac{\sum_{i = 1}^{n} (X_{i} - \hat x)(Y_{i} - \hat y)}{n - 1} {}\nonumber
%\end{equation}  
%\\
%with $\hat x$ and $\hat y$ denoting the means of $X$ and $Y$, respectively.
%\\
%\\
%where $n = 5$ for this example.
%\\
%\\
%So we have
%\begin{equation}
%    COV(X, Y) = COV(Y, X) = \frac{(1 - 2)(2 - 3) + (3 - 2)(4 - 3)}{2 - 1} = 2 {}\nonumber
%\end{equation}
%\\
%For each variate $X_{j}$ , $j = 1, 2, 3, \dots , p$, define its sample variance as:
%\\
%\begin{equation}
%    S = \frac{1}{n - 1}\sum_{i = 1}^{n} (X_{i} - \hat x)^2 {}\nonumber
%\end{equation}
%\\
%So we have
%\begin{equation}
%    S = \frac{(1 - 2)^2 + (3 - 2)^2}{2 - 1} = 2 {}\nonumber
%\end{equation}
%\\
%Similarly we have each variate %$Y_{j}$
%\begin{equation}
%    S = \frac{(2 - 3)^2 + (4 - 3)^2}{2 - 1} = 2 {}\nonumber
%\end{equation}
%\\
%\\
%\\
%So we have the co-variance matrix of A as follows:
%\\
%\\
%$cov_{x,y} = \left[ %\begin{array}{cc}
%cov_{x,x} & cov_{x,y} \\
%cov_{y,x} & cov_{y,y} \\
%\end{array} \right]
%= \left[ \begin{array}{cc}
%\sigma^2_{x} & \sigma_{xy} \\
%\sigma_{yx} & \sigma^2_{y} \\
%\end{array} \right]$
%= \begin{bmatrix}
%    2 & 2 \\
%    2 & 2 \\
%          
%\end{bmatrix}
\\
\\
\\
\end{widetext}
\end{document}