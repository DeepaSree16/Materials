%\tolerance=10000
%\documentclass[prl,twocoloumn,preprintnumbers,amssymb,pla]{revtex4}
\documentclass[prl,twocolumn,showpacs,preprintnumbers,superscriptaddress]{revtex4}
\documentclass{article}
\usepackage{graphicx}
\usepackage{color}
\usepackage{dcolumn}
%\linespread{1.7}
\usepackage{bm}
%\usepackage{eps2pdf}
\usepackage{graphics}
\usepackage{pdfpages}
\usepackage{caption}
%\usepackage{subcaption}
\usepackage[demo]{graphicx} % omit 'demo' for real document
%\usepackage{times}
\usepackage{multirow}
\usepackage{hhline}
\usepackage{subfig}
\usepackage{amsbsy}
\usepackage{amsmath}
\usepackage{amsfonts}
\usepackage{amsthm}
\usepackage{float}
\documentclass{article}
\usepackage{amsmath,systeme}

\sysalign{r,r}

% \textheight = 8.5 in
% \topmargin = 0.3 in

%\textwidth = 6.5 in
% \textheight = 8.5 in
%\oddsidemargin = 0.0 in
%\evensidemargin = 0.0 in

%\headheight = 0.0 in
%\headsep = 0.0 in
%\parskip = 0.2in
%\parindent = 0.0in

% \newcommand{\ket}[1]{\left|#1\right\rangle}
% \newcommand{\bra}[1]{\left\langle#1\right|}
\newcommand{\ket}[1]{| #1 \rangle}
\newcommand{\bra}[1]{\langle #1 |}
\newcommand{\braket}[2]{\langle #1 | #2 \rangle}
\newcommand{\ketbra}[2]{| #1 \rangle \langle #2 |}
\newcommand{\proj}[1]{| #1 \rangle \langle #1 |}
\newcommand{\al}{\alpha}
\newcommand{\be}{\beta}
\newcommand{\op}[1]{ \hat{\sigma}_{#1} }
\def\tred{\textcolor{red}}
\def\tgre{\textcolor{green}}


\theoremstyle{plain}
\newtheorem{theorem}{Theorem}

\newtheorem{lemma}[theorem]{Lemma}
\newtheorem{corollary}[theorem]{Corollary}
\newtheorem{proposition}[theorem]{Proposition}
\newtheorem{conjecture}[theorem]{Conjecture}

\theoremstyle{definition}
\newtheorem{definition}[theorem]{Definition}


\begin{document}
\begin{widetext}
\\
\\
\\

\begin{wrapfigure}
\centering
%\includegraphics[\textwidth]{TS_IISc.png}
\end{wrapfigure}
\begin{figure}[h!]
 \begin{right}
  \hfill\includegraphics[\textwidth, right]{TS_IISc.png}
 \end{right}
\end{figure}
\\
\\
\\
\\
\\
\noindent\textbf{1. If A and B are two events such that $P(A) = \frac{3}{8}$, $P(B) = \frac{5}{8}$ and $P(A\cup B) = \frac{3}{4}$, then $P(B|A) = $?}
\\
\\
\\
A. $\frac{2}{3}$
\\
\\
B. $\frac{3}{5}$
\\
\\
C. $\frac{4}{5}$
\\
\\
D. $\frac{1}{5}$
\\
\\
\\
\textbf{Answer: A}
\\
\\
\textbf{Solution}:
The general probability addition rule for the union of two events states that:
\\
\begin{equation}
    P(A\cup B) = P(A) + P(B) - P(A\cap B) {}\nonumber
\end{equation}
Given: $P(A) = \frac{3}{8}$, $P(B) = \frac{5}{8}$ and $P(A\cup B) = \frac{3}{4}$, we have $P(A\cap B) = \frac{1}{4}$
\\
\\
The probability that events $A$ and $B$ both occur is equal to the probability that event $A$ occurs times the probability that event $B$ occurs, given that $A$ has already happened.
\begin{equation}
    P(B|A) = \frac{P(A\cap B)}{P(A)} {}\nonumber
\end{equation}
We have $P(A\cap B) = \frac{1}{4}$ and $P(A) = \frac{3}{8}$, so we have from the above equation: $P(B|A) = \frac{2}{3}$
\\
\\
\\
\textbf{2. An urn contains 5 red balls and 2 green balls. Two balls are drawn one after the other without replacement. What is the probability that the second ball is red?}
\\
\\
\\
A. $\frac{10}{21}$\\
\\
B. $\frac{5}{21}$\\
\\
C. $\frac{2}{21}$\\
\\
D. $\frac{5}{7}$
\\
\\
\\
\textbf{Answer: D}
\\
\\
\textbf{Solution}:
The sample space is $\Omega = \{rr, rg, gr, gg\}$.
\\
\\
Let $R_{1}$ be the event ‘the first ball is red’, $G_{1}$ = ‘first ball is green’, $R_{2}$ = ‘second ball is red’, $G_{2}$ = ‘second ball is green’. We are asked to find $P(R_{2})$.
\\
\\
The fast way to compute this is just like $P(S_{2})$ in the card example above. Every ball is equally likely to be the second ball. Since 5 out of 7 balls are red, $P(R_{2})$ = $\frac{5}{7}$.
Let’s compute this same value using the law of total probability. First, we’ll find the conditional probabilities. This is a simple counting exercise.
\\
\\
$P(R2|R1) = \frac{4}{6}$, $P(R2|G1) = \frac{5}{6}$.
\\
\\
Since $R_{1}$ and $G_{1}$ partition $\Omega$ the law of total probability says:
\begin{equation}
    P(R_{2}) = P(R_{2}|R_{1})P(R_{1}) + P(R_{2}|G_{1})P(G_{1}) {}\nonumber
\end{equation}
\begin{equation}
    P(R_{2}) = \frac{4}{6}\cdot \frac{5}{7} + \frac{5}{6}\cdot \frac{2}{7} = \frac{30}{42} = \frac{5}{7} {}\nonumber
\end{equation}
\\
\\
\\
\textbf{3. You roll two fair dice. Find the probability that the first die is a 4 given that the sum is 7.}
\\
\\
\\
A. $\frac{5}{36}$\\
\\
B. $\frac{1}{6}$\\
\\
C. $\frac{1}{5}$\\
\\
D. $\frac{1}{36}$
\\
\\
\\
\textbf{Answer: B}
\\
\\
\textbf{Solution}:
Let A denote the event that the first die shows $4$ and $B$ denote the event that the sum of the dice is 7.
Notice that for any number the first die shows, there is only one number the second die can show to make
the sum 7 (e.g. if the first die shows 5 then the second die must show 2 to make the sum 7). There are six ways in which we can get a 7. They are {(1, 6), (6, 1), (2, 5), (5, 2), (4, 3), (3, 4)}, so the probability of getting a 7 is $\frac{6}{36}.$
\\
\\
So $P(B) = \frac{6}{36}$ 
\\
\\
Now, if the first die shows 4 there is only one way to make the sum of both dice equal 7 which means $P(A\cap B) = \frac{1}{36}$
\\
\\
Therefore, the probability that the first die shows 4 given that the sum is 7 is:
\begin{equation}
P(A|B) = \frac{P(A\cap B)}{P(B)} = \frac{1/36}{1/6} = \frac{1}{6} {}\nonumber
\end{equation}
\\
\\
\\
\newpage
\\
\noindent\textbf{4. Determine the mean of the random variable X having the following probability distribution.}
\\
\\
%\begin{table}[H]
\begin{center}
\begin{tabular}{|c|c|c|c|c|c|c|c|c|c|c|}
%\hline \parbox{.5in}{\ \\$X = x$\\ \ \\} & \parbox{.35in}{\ \\$1$\\ \ \\} &\parbox{1.in}{\ \\$2$\\ \ \\}&\parbox{1.in}{\ \\$3$\\ \ \\}  \\
\hline \Large$X = x$  & \Large$1$ & \Large$2$ &  \Large$3$  & \Large$4$ & \Large$5$ & \Large$6$ & \Large$7$ & \Large$8$ & \Large$9$ & \Large$10$\\
\hline \Large$P(x)$  & \Large$0.15$ & \Large$0.10$ &  \Large$0.10$ & \Large$0.06$ & \Large$0.08$ & \Large$0.01$ & \Large$0.05$ & \Large$0.02$ & \Large$0.30$ & \Large$0.2$\\
\hline
\end{tabular} 
\end{center}
%\end{table}
\\
%\begin{tabular}{c|c|c|c}
%$x$ & 1 & 2 & 3\\ \hline 
%\\[-1em]
%$f(x)$ & 1 & 2 & 3
%\end{tabular}
\\
\\

%\\
\noindent A. $6.56$\\
\\
B. $50.38$\\
\\
C. $7.09$\\
\\
D. None of the above
\\
\\
\\
\textbf{Answer: A}
\\
\\
\textbf{Solution}:
The mean of the random variable $X$ = $E(X)$ $=$ $\sum xP_{X}(x)$.
\begin{equation}
    = (1 * 0.15) + (2 * 0.1) + (3 * 0.1) + (4 * 0.06) + (5 * 0.08) + (6 * 0.01) + (7 * 0.05) + (8 * 0.02) + (9 * 0.3) + (10 * 0.2) {}\nonumber
\end{equation}
\begin{equation}
    = 6.56{}\nonumber
\end{equation}
\\
\\
\\
\textbf{5. Let X be a binomial random variable with parameters (12, 0.5). Where $n = 12$ and $p = 0.5.$ Find the variance and the standard deviation of X.}
\\
\\
\\
\noindent A. Variance $Var(X) = \sqrt{3}$, Standard deviation $\sigma_{X} = 3$\\
\\
B. Variance $Var(X) = \sqrt{3}$, Standard deviation $\sigma_{X} = \sqrt{3}$\\
\\
C. Variance $Var(X) = 3$, Standard deviation $\sigma_{X} = 9$\\
\\
D. Variance $Var(X) = 3$, Standard deviation $\sigma_{X} = \sqrt{3}$
\\
\\
\\
\textbf{Answer: D}
\\
\\
\textbf{Solution}:
Variance $Var(X) = npq = np(1 - p) = 6(1 - 0.5) = 3.$
\\
\\
The standard deviation is $\sigma_{X} = \sqrt{npq} = \sqrt{3}$
\\
\\
\\
\end{widetext}
\end{document}