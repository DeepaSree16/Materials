%\tolerance=10000
%\documentclass[prl,twocoloumn,preprintnumbers,amssymb,pla]{revtex4}
\documentclass[prl,twocolumn,showpacs,preprintnumbers,superscriptaddress]{revtex4}
\documentclass{article}
\usepackage{graphicx}
\usepackage{color}
\usepackage{dcolumn}
%\linespread{1.7}
\usepackage{bm}
%\usepackage{eps2pdf}
\usepackage{graphics}
\usepackage{pdfpages}
\usepackage{caption}
%\usepackage{subcaption}
\usepackage[demo]{graphicx} % omit 'demo' for real document
%\usepackage{times}
\usepackage{multirow}
\usepackage{hhline}
\usepackage{subfig}
\usepackage{amsbsy}
\usepackage{amsmath}
\usepackage{amsfonts}
\usepackage{amsthm}
\usepackage{float}
\documentclass{article}
\usepackage{amsmath,systeme}

\sysalign{r,r}

% \textheight = 8.5 in
% \topmargin = 0.3 in

%\textwidth = 6.5 in
% \textheight = 8.5 in
%\oddsidemargin = 0.0 in
%\evensidemargin = 0.0 in

%\headheight = 0.0 in
%\headsep = 0.0 in
%\parskip = 0.2in
%\parindent = 0.0in

% \newcommand{\ket}[1]{\left|#1\right\rangle}
% \newcommand{\bra}[1]{\left\langle#1\right|}
\newcommand{\ket}[1]{| #1 \rangle}
\newcommand{\bra}[1]{\langle #1 |}
\newcommand{\braket}[2]{\langle #1 | #2 \rangle}
\newcommand{\ketbra}[2]{| #1 \rangle \langle #2 |}
\newcommand{\proj}[1]{| #1 \rangle \langle #1 |}
\newcommand{\al}{\alpha}
\newcommand{\be}{\beta}
\newcommand{\op}[1]{ \hat{\sigma}_{#1} }
\def\tred{\textcolor{red}}
\def\tgre{\textcolor{green}}


\theoremstyle{plain}
\newtheorem{theorem}{Theorem}

\newtheorem{lemma}[theorem]{Lemma}
\newtheorem{corollary}[theorem]{Corollary}
\newtheorem{proposition}[theorem]{Proposition}
\newtheorem{conjecture}[theorem]{Conjecture}

\theoremstyle{definition}
\newtheorem{definition}[theorem]{Definition}


\begin{document}
\begin{widetext}
\\
\\
\\

\begin{wrapfigure}
\centering
%\includegraphics[\textwidth]{TS_IISc.png}
\end{wrapfigure}
\begin{figure}[h!]
 \begin{right}
  \hfill\includegraphics[\textwidth, right]{TS_IISc.png}
 \end{right}
\end{figure}
\\
\\
\\
\noindent\textbf{1. When can we say two nonzero vectors are orthogonal?}
\\
\\
A. When their dot product is zero
\\
\\
B. When their cross product is zero
\\
\\
C. When their sum is zero
\\
\\
D. None of the above
\\
\\
\\
\textbf{Answer: A}
\\
\\
\textbf{Solution}:
Two nonzero vectors are perpendicular, or orthogonal, if and only if their dot product is equal to zero.
%\textbf{1. Find the eigenvalues of following matrix:}
%\[
%A = \begin{bmatrix}    17 & 8  \\     8 & 17    \\ \end{bmatrix}
%\]
%\\
%\\
%A. $-25, 9$\\
%\\
%B. $5, 3$\\
%\\
%C. $25, -9$\\
%\\
%D. $25, 9$
%\\
%\\
%\\
%\textbf{Answer: D}
%\\
%\\
%\textbf{Solution}:
%To find out the eigenvalues of matrix $A$, we write the characteristic equation as follows:
%
%\[
%\begin{vmatrix}    A - \lambda \cdot I \end{vmatrix} = \begin{vmatrix} \begin{bmatrix}    17 & %8 \\     8 & 17   \\ \end{bmatrix} - \begin{bmatrix}    \lambda & 0 \\     0 & \lambda  \\ %\end{bmatrix}\end{vmatrix} = 0
%\]
%\\
%\[
%\begin{vmatrix}   \begin{bmatrix}    17 - \lambda & 8  \\     8 & 17 - \lambda  \\ %\end{bmatrix}\end{vmatrix} = 0
%\]
%\\
%\\
%Solving we get $(17 - \lambda)(17 - \lambda) - 64 = 0$
%\\
%\\
%So the eigenvalues are $\lambda_{1} = 25$ and $\lambda_{2} = 9$
\\
\\
\\
%\textbf{2. Find the eigenvectors of following matrix:}
%\[
%A = \begin{bmatrix}    2 & -1 \\     -1 & 2   \\ \end{bmatrix}
%\]
%\\
%\\
%A. $\begin{bmatrix}   \frac{1}{\sqrt{2}} & \frac{1}{\sqrt{2}} \\ -\frac{1}{\sqrt{2}} & \frac{1}{\sqrt{2}} \end{bmatrix}$\\
%\\
%\\
%B. $\begin{bmatrix}   \frac{1}{\sqrt{2}} & \frac{1}{\sqrt{2}} \\ \frac{1}{\sqrt{2}} & \frac{1}{\sqrt{2}} \end{bmatrix}$\\
%\\
%\\
%C. $\begin{bmatrix}   \frac{1}{2} & \frac{1}{2} \\ \frac{1}{2} & \frac{1}{2} \end{bmatrix}$\\
%\\
%\\
%D. $\begin{bmatrix}   \frac{1}{2} & \frac{1}{2} \\ \frac{1}{2} & -\frac{1}{2} \end{bmatrix}$\\
%\\
%\\
%\\
%\textbf{Answer: A}
%\\
%\\
%\textbf{Solution}
%Calculating as above, the two eigenvalues of matrix $A$ are $\lambda_{1} = 3$ and $\lambda_{2} = 1$
%\\
%\\
%Let us find the eigenvector associated first with $\lambda_{1} = 3$
%\\
%\\
%We have $A \cdot v_{1} = \lambda_{1} \cdot \ u_{1}$ or $(A - \lambda_{1}) \cdot \ u_{1} = 0$
%\\
%\[ or
%\begin{bmatrix}    2 - 3 & -1 \\     -1 & 2 - 3   \\ \end{bmatrix} \begin{bmatrix}    u_{1,1} \\     u_{1,2}  \\ \end{bmatrix}= \begin{bmatrix}    0 \\     0  \\ \end{bmatrix}
%\]
%\\
%\\
%We have: $-u_{1,1} - u_{1,2} = 0$ or $u_{1,1} = - u_{1,2}$
%\\
%\\
%Normalizing, we get the following eigenvector associated with eigenvalue $\lambda_{1} = 3$ as:
%\\
%\[
%\begin{bmatrix}    u_{1,1} \\     u_{1,2}  \\ \end{bmatrix}= \begin{bmatrix}    %\frac{1}{\sqrt{2}} \\     -\frac{1}{\sqrt{2}}  \\ \end{bmatrix}
%\]
%\\
%\\
%Now let us find the eigenvector associated first with $\lambda_{2} = 1$
%\\
%So we have $A \cdot u_{2} = \lambda_{2} \cdot \ u_{2}$ or $(A - \lambda_{2}) \cdot \ u_{2} = 0$
%\\
%\[
%\begin{bmatrix}    2 - 1 & -1 \\     -1 & 2 - 1   \\ \end{bmatrix} \begin{bmatrix}    u_{2,1} \\     u_{2,2}  \\ \end{bmatrix}= \begin{bmatrix}    0 \\     0  \\ \end{bmatrix}
%\]
%\\
%We have: $u_{2,1} - u_{2,2} = 0$ or $u_{2,1} =  u_{2,2}$
%\\
%\\
%Normalizing, we get the following eigenvector associated with eigenvalue $\lambda_{2} = 1$ as:
%\[
%\begin{bmatrix}    u_{2,1} \\     u_{2,2}  \\ \end{bmatrix}= \begin{bmatrix}    \frac{1}{\sqrt{2}} \\     \frac{1}{\sqrt{2}}  \\ \end{bmatrix}
%\]
%\textbf{2. Which of the following triangular matrices is NOT invertible?}
%\\
%\\
%1. \begin{bmatrix}    1 & 5 & 3 & 4 \\     0 & 2 & 1 & 6 \\ 0 & 0 & 3 & 5  \\ 0 & 0 & 0 & 1 \end{bmatrix}
%\\
%\\
%\\
%2. \begin{bmatrix}    1 & 5 & 3 & 4 \\     0 & 0 & 1 & 6 \\ 0 & 0 & 2 & 5  \\ 0 & 0 & 0 & 1 \end{bmatrix}
%\\
%\\
%\\
%3. \begin{bmatrix}    1 & 0 & 0 & 0 \\     5 & 2 & 0 & 0 \\ 3 & 1 & 3 & 0  \\ 4 & 6 & 5 & 1 \end{bmatrix}
%\\
%\\
%\\
%4. \begin{bmatrix}    1 & 0 & 0 & 0 \\     5 & 2 & 0 & 0 \\ 3 & 1 & 0 & 0  \\ 4 & 6 & 5 & 1 \end{bmatrix}
%\\
%\\
%\\
%A. 1 Only
%\\
%\\
%B. 3 Only
%\\
%\\
%C. 2 Only
%\\
%\\
%D. Both 2 and 4 
%\\
%\\
%\\
%\newpage
%\textbf{Answer: D}
%\\
%\\
%\textbf{Solution:} 
% A triangular matrix is invertible if and only if all diagonal entries are nonzero.
\textbf{2. Given the following matrix A. What can be deduced about it?}
\\
\\
\[ A = 
\begin{bmatrix}    0 & 5 & 3 \\     -5 & 0 & -8 \\ -3 & 8 & 0  \\ \end{bmatrix} \]
%\[ B = 
%\begin{bmatrix}    0 & -2 & 45 \\     2 & 0 & 4 \\ -45 & -4 & 0 \\ \end{bmatrix} \]
A. $A$ is a symmetric matrix
\\
\\
B. $A$ is a skew-symmetric matrix
\\
\\
C. $A$ is an orthogonal matrix
\\
\\
D. None of the above
\\
\\
\\
\textbf{Answer: B}
\\
\\
\textbf{Solution:} 
 A matrix is skew-symmetric if and only $A$ = -$A^T$.
\\
\\
\\
\textbf{3. The determinant of the following matrix is:}
\[
A = \begin{bmatrix}    2 & 3 & -1 \\     4 & -3 & -1 \\ 1 & -3 & 3  \\ \end{bmatrix}
\]
\\
\\
A. $54$
\\
\\
B. $27$
\\
\\
C. $-54$
\\
\\
D. $81$
\\
\\
\\
%\newpage
\textbf{Answer: C}
\\
\\
\textbf{Solution:} 
The determinant of the matrix is: $|A| = 2([-3 * 3] - [-1 * -3]) - 3*([4 * 3] - 1 * [-1 * 1]) - 1 * ([4 * -3] - [-3 * 1]) = -54$
\\
\\
\\
%\textbf{4. What is the formula for computing the Singular Value Decomposition of a matrix of size $m \times n$?}
%\\
%\\
%A. $U_{m \times m}\Sigma_{m \times n} V^T_{n \times n}$
%\\
%\\
%B. $V_{m \times m}\Sigma_{m \times n} U^T_{n \times n}$
%\\
%\\
%C. $U_{n \times n}\Sigma_{m \times n} V^T_{m \times m}$
%\\
%\\
%D. $V_{n \times n}\Sigma_{m \times m} U^T_{m \times m}$
%\\
%\\
%\\
%\textbf{Answer: A}
%\\
%\\
%\textbf{Solution:} 
%The SVD theorem states: $A_{m \times n} = U_{m \times m}\Sigma_{m \times n} V^T_{n \times n}$
\textbf{4. What is the resultant matrix when an upper triangular matrix of size $m \times m$ and a lower triangular matrix of size $m \times m$ are multiplied?}
\\
\\
A. A lower triangular matrix of size $m \times m$
\\
\\
B. A diagonal matrix of size $m \times m$
\\
\\
C. An upper triangular matrix of size $m \times m$
\\
\\
D. A full matrix of size $m \times m$
\\
\\
\\
\textbf{Answer: D}
\\
\\
\textbf{Solution:} 
%The SVD theorem states: $A_{m \times n} = U_{m \times m}\Sigma_{m \times n} V^T_{n \times n}$
\\
\\
\\
\textbf{5. What is the inverse of the following matrix?}
\\
%\[
%\begin{bmatrix}    2 & 3 & -1 \\     4 & -3 & -1 \\ 1 & -3 & 3     \\ \end{bmatrix}
%\]
\[ A= 
\begin{bmatrix}    2 & 3 \\     5 & 7 \\ \end{bmatrix}
\]
\\
\\
%A. $\begin{bmatrix}   \frac{13}{54} & -\frac{7}{54} & \frac{1}{27} \\ \frac{2}{9} & \frac{1}{9} & \frac{1}{9} \\ \frac{1}{6} & -\frac{1}{6} & \frac{1}{3} \end{bmatrix}$\\
A. $\begin{bmatrix}   7 & -5 \\ -3 & 2 \end{bmatrix}$\\
\\
\\
%B. $\begin{bmatrix}    \frac{2}{9} & \frac{1}{9} & \frac{1}{9} \\     \frac{13}{54} & -\frac{7}{54} & \frac{1}{27} \\ \frac{1}{6} & -\frac{1}{6} & \frac{1}{3}   \end{bmatrix}$
B. $\begin{bmatrix}   -7 & 3 \\ 5 & -2 \end{bmatrix}$\\
\\
\\
C. $\begin{bmatrix}   7 & -3 \\ -5 & 2 \end{bmatrix}$\\
%C. $\begin{bmatrix}   \frac{1}{6} & -\frac{1}{6} & \frac{1}{3} \\     \frac{13}{54} & -\frac{7}{54} & \frac{1}{27} \\ \frac{2}{9} & \frac{1}{9} & \frac{1}{9}   \end{bmatrix}$\\
\\
\\
D. Does not exist\\
\\
\\
\\
\textbf{Answer: B}
\\
\\
\textbf{Solution:}
First, find the determinant of $2 \times  2$ $A$ Matrix and then find it’s minor, cofactors and adjoint and insert the results in the Inverse Matrix formula given below:
\begin{equation}
A^{-1}=\frac{1}{|A|} Adj(A) {}\nonumber
\end{equation}
\\
\[
\begin{vmatrix}    A \end{vmatrix} =\begin{vmatrix}    2 & 3 \\     5 & 7  \\  \end{vmatrix} = 14 - 15 = -1 \neq 0 
\]
\\
Therefore $A$ is non-singular
\\
\\
Therefore, the system has the unique solution $X=A^{-1}B$
\\
\\
We have the determinant of $1 \times  1$ minor matrices as follows:
\\
$A_{11} = 7$,\\
$A_{12} = -5$,\\
$A_{21} = -3$,\\
$A_{22} = 2$,\\

\\
Therefore, we now have
\[
adj(A) = \begin{bmatrix}    A_{11} & A_{21}  \\     A_{12} & A_{22}  \\ \end{bmatrix} = \begin{bmatrix}    7 & -3 \\   -5 & 2 \\  \end{bmatrix}
\]
\\
\\
And
\[
A^{-1} =   \frac{adj(A)}{\begin{vmatrix} A \end{vmatrix}} = -\frac{1}{1} \begin{bmatrix}   7 & -3 \\     -5 & 2 \\  \end{bmatrix} = \begin{bmatrix}    -7 & 3 \\     5 & -2   \\ \end{bmatrix}
\]
\\
\\
\\
\end{widetext}
\end{document}