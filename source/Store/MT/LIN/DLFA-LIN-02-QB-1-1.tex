%\tolerance=10000
%\documentclass[prl,twocoloumn,preprintnumbers,amssymb,pla]{revtex4}
\documentclass[prl,twocolumn,showpacs,preprintnumbers,superscriptaddress]{revtex4}
\documentclass{article}
\usepackage{graphicx}
\usepackage{color}
\usepackage{dcolumn}
%\linespread{1.7}
\usepackage{bm}
%\usepackage{eps2pdf}
\usepackage{graphics}
\usepackage{pdfpages}
\usepackage{caption}
%\usepackage{subcaption}
\usepackage[demo]{graphicx} % omit 'demo' for real document
%\usepackage{times}
\usepackage{multirow}
\usepackage{hhline}
\usepackage{subfig}
\usepackage{amsbsy}
\usepackage{amsmath}
\usepackage{amsfonts}
\usepackage{amsthm}
\usepackage{float}
\documentclass{article}
\usepackage{amsmath,systeme}

\sysalign{r,r}

% \textheight = 8.5 in
% \topmargin = 0.3 in

%\textwidth = 6.5 in
% \textheight = 8.5 in
%\oddsidemargin = 0.0 in
%\evensidemargin = 0.0 in

%\headheight = 0.0 in
%\headsep = 0.0 in
%\parskip = 0.2in
%\parindent = 0.0in

% \newcommand{\ket}[1]{\left|#1\right\rangle}
% \newcommand{\bra}[1]{\left\langle#1\right|}
\newcommand{\ket}[1]{| #1 \rangle}
\newcommand{\bra}[1]{\langle #1 |}
\newcommand{\braket}[2]{\langle #1 | #2 \rangle}
\newcommand{\ketbra}[2]{| #1 \rangle \langle #2 |}
\newcommand{\proj}[1]{| #1 \rangle \langle #1 |}
\newcommand{\al}{\alpha}
\newcommand{\be}{\beta}
\newcommand{\op}[1]{ \hat{\sigma}_{#1} }
\def\tred{\textcolor{red}}
\def\tgre{\textcolor{green}}


\theoremstyle{plain}
\newtheorem{theorem}{Theorem}

\newtheorem{lemma}[theorem]{Lemma}
\newtheorem{corollary}[theorem]{Corollary}
\newtheorem{proposition}[theorem]{Proposition}
\newtheorem{conjecture}[theorem]{Conjecture}

\theoremstyle{definition}
\newtheorem{definition}[theorem]{Definition}


\begin{document}
\begin{widetext}
\\
\\
\\
\\

\begin{wrapfigure}
\centering
%\includegraphics[\textwidth]{TS_IISc.png}
\end{wrapfigure}
\begin{figure}[h!]
 \begin{right}
  \hfill\includegraphics[\textwidth, right]{TS_IISc.png}
 \end{right}
\end{figure}
\\
\\
\\
{\Large
\noindent\textbf{1. Find the eigenvalues of following matrix:}
\[
A = \begin{bmatrix}    17 & 8  \\     8 & 17    \\ \end{bmatrix}
\]
\\
\\
A. $-25, 9$\\
\\
B. $5, 3$\\
\\
C. $25, -9$\\
\\
D. $25, 9$
\\
\\
\\
\textbf{Answer: D}
\\
\\
\textbf{Solution:}
To find out the eigenvalues of matrix $A$, we write the characteristic equation as follows:
%
\[
\begin{vmatrix}    A - \lambda \cdot I \end{vmatrix} = \begin{vmatrix} \begin{bmatrix}    17 & 8 \\     8 & 17   \\ \end{bmatrix} - \begin{bmatrix}    \lambda & 0 \\     0 & \lambda  \\ \end{bmatrix}\end{vmatrix} = 0
\]
\\
\[
\begin{vmatrix}   \begin{bmatrix}    17 - \lambda & 8  \\     8 & 17 - \lambda  \\ \end{bmatrix}\end{vmatrix} = 0
\]
\\
\\
Solving we get $(17 - \lambda)(17 - \lambda) - 64 = 0$
\\
\\
So the eigenvalues are $\lambda_{1} = 25$ and $\lambda_{2} = 9$
\\
\\
\\
\textbf{2. Find the eigenvectors of following matrix:}
{\Large\[
A = \begin{bmatrix}    2 & -1 \\     -1 & 2   \\ \end{bmatrix}
\]}
\\
\\
\\
A. {\Large $\begin{bmatrix}   \frac{1}{\sqrt{2}}  \\ \\ -\frac{1}{\sqrt{2}} \end{bmatrix} , \begin{bmatrix} \frac{1}{\sqrt{2}} \\ \\ \frac{1}{\sqrt{2}} \end{bmatrix}$ }\\
\\
\\
B. {\Large $\begin{bmatrix}   \frac{1}{\sqrt{2}}  \\ \\ \frac{1}{\sqrt{2}} \end{bmatrix} , \begin{bmatrix} \frac{1}{\sqrt{2}} \\ \\ \frac{1}{\sqrt{2}} \end{bmatrix}$ }\\
\\
\\
C. {\Large $\begin{bmatrix}   \frac{1}{2}  \\  \\ \frac{1}{2} \end{bmatrix} , \begin{bmatrix} \frac{1}{2} \\ \\  \frac{1}{2} \end{bmatrix}$ }\\
\\
\\
D. {\Large $\begin{bmatrix}   \frac{1}{2} \\ \\ \frac{1}{2} \end{bmatrix} ,  \begin{bmatrix}  \frac{1}{2} \\ \\ -\frac{1}{2} \end{bmatrix}$ }\\
\\
\\
\\
\\
\textbf{Answer: A}
\\
\\
\textbf{Solution:}
Calculating as above, the two eigenvalues of matrix $A$ are $\lambda_{1} = 3$ and $\lambda_{2} = 1$
\\
\\
Let us find the eigenvector associated first with $\lambda_{1} = 3$
\\
\\
We have $A \cdot v_{1} = \lambda_{1} \cdot \ u_{1}$ or $(A - \lambda_{1}) \cdot \ u_{1} = 0$
\\
\[ or
\begin{bmatrix}    2 - 3 & -1 \\     -1 & 2 - 3   \\ \end{bmatrix} \begin{bmatrix}    u_{1,1} \\     u_{1,2}  \\ \end{bmatrix}= \begin{bmatrix}    0 \\     0  \\ \end{bmatrix}
\]
\\
\\
We have: $-u_{1,1} - u_{1,2} = 0$ or $u_{1,1} = - u_{1,2}$
\\
\\
Normalizing, we get the following eigenvector associated with eigenvalue $\lambda_{1} = 3$ as:
\\
\[
\begin{bmatrix}    u_{1,1} \\  \\   u_{1,2}  \\ \end{bmatrix}= \begin{bmatrix}    \frac{1}{\sqrt{2}} \\   \\  -\frac{1}{\sqrt{2}}  \\ \end{bmatrix}
\]
\\
\\
Now let us find the eigenvector associated first with $\lambda_{2} = 1$
\\
So we have $A \cdot u_{2} = \lambda_{2} \cdot \ u_{2}$ or $(A - \lambda_{2}) \cdot \ u_{2} = 0$
\\
\[
\begin{bmatrix}    2 - 1 & -1 \\     -1 & 2 - 1   \\ \end{bmatrix} \begin{bmatrix}    u_{2,1} \\     u_{2,2}  \\ \end{bmatrix}= \begin{bmatrix}    0 \\     0  \\ \end{bmatrix}
\]
\\
We have: $u_{2,1} - u_{2,2} = 0$ or $u_{2,1} =  u_{2,2}$
\\
\\
Normalizing, we get the following eigenvector associated with eigenvalue $\lambda_{2} = 1$ as:
\[
\begin{bmatrix}    u_{2,1} \\  \\   u_{2,2}  \\ \end{bmatrix}= \begin{bmatrix}    \frac{1}{\sqrt{2}} \\  \\   \frac{1}{\sqrt{2}}  \\ \end{bmatrix}
\]
Combining eigenvectors calculated above for both the eigenvalues, we get the 
\\
\\
\\
\textbf{3. What is the formula for computing the Singular Value Decomposition of a matrix $P$ of size $m \times n$, where $m \ne n$?}
\\
\\
A. $P_{m \times n}$ = $U_{m \times m}\Sigma_{m \times n} V^T_{n \times n}$
\\
\\
B. $P_{m \times n}$ = $V_{m \times n}\Sigma_{n \times n} U^T_{n \times m}$
\\
\\
C. $P_{m \times n}$ = $U_{n \times n}\Sigma_{n \times m} V^T_{m \times m}$
\\
\\
D. $P_{m \times n}$ = $V_{n \times n}\Sigma_{m \times m} U^T_{m \times m}$
\\
\\
\\
\textbf{Answer: A}
\\
\\
\textbf{Solution:} 
The SVD theorem states: $P_{m \times n} = U_{m \times m}\Sigma_{m \times n} V^T_{n \times n}$
\\
\\
\\
\textbf{4. Find the number of solutions of the following pair of linear equations:} 
\\
\begin{equation}
   4x - 6y = 10 {}\nonumber \\
\end{equation}
\begin{equation}
  -8x + 12y = 10 {}\nonumber
\end{equation}
\\
\\
A. $2$\\
\\
B. $1$\\
\\
C. $0$\\
\\
D. Infinite
\\
\\
\\
\textbf{Answer: C}
\\
\\
\textbf{Solution:} 
A system has no solution if the equations are inconsistent, they are contradictory. A system of linear equations $ax + by + c = 0$ and $dx + ey + g = 0$ will have a unique solution if the two lines represented by the equations $ax + by + c = 0$ and $dx + ey + g = 0$ intersect at a point. Since the lines $4x - 6y = 10$ and $-8x + 12y = 10$ are parallel, they will NEVER intersect and so they do not have any solution.
\\
\\
\\
\textbf{5. What is the rank of the following matrix $A$:} 
\\
\[
A = \begin{bmatrix}    1 & 2 & 1 \\    4 & 5 & 6 \\ 6 & 9 & 8   \\ \end{bmatrix}
\]
\\
\\
A. $0$\\
\\
B. $2$\\
\\
C. $1$\\
\\
D. $3$
\\
\\
\\
\textbf{Answer: B}
\\
\\
\textbf{Solution:} 
Maximum number of linearly independent rows in a matrix (or linearly independent columns) is called Rank of that matrix.
To find if rows of matrix are linearly independent, we have to check if none of the row vectors (rows represented as individual vectors) is linear combination of other row vectors.
\\
\\
$a_{1} = [1\  2\  1]$ 
\\
\\
$a_{2} = [4\  5\  6]$
\\
\\
$a_{3} = [6\  9\  8]$
\\
\\
Turns out vector $a_{3}$ is a linear combination of vector $a_{1}$ and $a_{2}$.
\begin{equation}
    2a_{1} + a_{2} = a_{3} {}\nonumber
\end{equation}
\\
So, matrix $A$ is not linearly independent. But, row vector $a_{1}$ and $a_{2}$ are linearly independent among each other.
\\
\\
For matrix A, rank is 2 (row vector $a_{1} = [1\  2\  1]$ and $a_{2} = [4\  5\  6]$ are linearly independent).
\\
\\
\\
}
\end{widetext}
\end{document}