\documentclass[aspectratio=169,14pt,usenames,dvipsnames]{beamer}
\usetheme{TalentSprint}
\usepackage[utf8]{inputenc}
\usepackage{graphics}
\usepackage{ragged2e}
\usepackage{amsfonts}
\usepackage{xcolor}
\usepackage{mathtools}
\usepackage{tcolorbox}
\usepackage{setspace}
\usepackage{lmodern}
\definecolor{swe}{rgb}{0.19, 0.73, 0.56}
\definecolor{lblue}{RGB}{190,200,198}
\title[Nested Loop]{Nested Loop}

\newcommand\tab[1][1cm]{\hspace*{#1}}
\begin{document}

{\1
\begin{frame} \vspace{35pt}

\subtitle{TalentSprint}
\maketitle
\end{frame}
}


\begin{frame}{Learning Objectives}
At the end of this video you should able to:
\begin{itemize}
\item understand and use nested loop statement.
\item understand how nested loop statement
works.
\item write programs using nested loops
\end{itemize}
\end{frame}


\begin{frame}{Nested Loop}
\begin{itemize}
    \item A loop within a loop.
    \item A inner loop within the body of an outer
loop.
\item An inner loop will execute only if the outer
loop is true.
\item After termination of inner loop, the control
goes to outer loop
\end{itemize}
\end{frame}



\begin{frame}{Syntax}
\begin{lstlisting}
for (init; bool\_expression; increment) \{\\
\tab statement\_out;\\
\tab while (boolean\_expression\_in) \{\\
\tab \tab statement\_in;\\
\tab \tab statement\_in;\\
\tab \}\\
\tab statement\_out;\\
\}\\
\end{lstlisting}
Note: We can use one loop (of any kind) inside
another loop (of any kind)
\end{frame}



\begin{frame}{problem Statement}
print all three digit palindromes.
\end{frame}

\begin{frame}{solution}
\begin{lstlisting}
for (int num = 100; num $<$= 999; num++) \{\\
\tab int n= num;\\
\tab int rev\_n = 0\\
\tab while (n $>$ 0) \{\\
\tab \tab int d = n \% 10;\\
\tab \tab rev\_n =(rev\_n * 10 ) +d;\\
\tab \tab n /=10;\\
\tab \}\\
\tab if (num == rev\_n) \{\\
\tab \tab System.out.println(num);\\
\tab \}\\
\}\\
\end{lstlisting}
\end{frame}
\begin{frame}{Problem statement}
Print the below pattern as shown:\\
\begin{lstlisting}
1\\
1 2\\
1 2 3\\
1 2 3 4\\
1 2 3 4 5\\
\end{lstlisting}
\end{frame}

\begin{frame}{Solution}
\begin{lstlisting}
int n = 5;\\
for (int i = 1; i $<$= n; i++) \{\\
\tab for (int j = 1; j $<$ i; j++) \{\\
\tab \tab System.out.print(j + " ");\\
\tab \}\\
\tab System.out.println();\\
\}\\
\end{lstlisting}
\end{frame}

\begin{frame}{Problem Statement}
Print the below pattern as shown:\\

\hspace{5em} 1\\
\hspace{4em} 1\hspace{0.2em} 2\\
\hspace{3em} 1\hspace{0.2em} 2 \hspace{0.2em}3\\
\hspace{2em} 1\hspace{0.2em} 2\hspace{0.2em} 3\hspace{0.2em} 4\\
\hspace{1em} 1\hspace{0.2em} 2\hspace{0.2em} 3\hspace{0.2em} 4\hspace{0.2em} 5\\
\end{frame}

\begin{frame}{Solution}
\begin{lstlisting}
int n = 5;\\
for (int i=1;i $<$= n;i++)\{\\
\tab for (int k = n; k $>$ i; k- -) \{\\
\tab \tab System.out.print(" ");\\
\tab \}\\
\tab for (int j = 1; j $<$= i; j++) \{\\
\tab \tab System.out.print(j + " ");\\
\tab \}\\
\tab System.out.println();\\
\}\\
\end{lstlisting}
\end{frame}


\begin{frame}{Summary}
\begin{itemize}
    \item A loop inside an another loop is known as
nested loop.
\item After termination of an inner loop, the
control goes to outer loop.
\item We can have any type of loop inside
another loop of any type.
\end{itemize}
\end{frame}
\end{document}
