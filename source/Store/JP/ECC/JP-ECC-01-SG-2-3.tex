\documentclass[aspectratio=169,14pt,usenames,dvipsnames]{beamer}
\usetheme{TalentSprint}
\usepackage[utf8]{inputenc}
\usepackage{graphics}
\usepackage{ragged2e}
\usepackage{amsfonts}
\usepackage{xcolor}
\usepackage{mathtools}
\usepackage{tcolorbox}
\usepackage{setspace}
\usepackage{lmodern}
\definecolor{swe}{rgb}{0.19, 0.73, 0.56}
\definecolor{lgreen}{RGB}{190,200,198}
\title[Conditional Statement-If]{Conditional Statement-If}

\newcommand\tab[1][1cm]{\hspace*{#1}}
\begin{document}

{\1
\begin{frame} \vspace{35pt}

\subtitle{TalentSprint}
\maketitle
\end{frame}
}


\begin{frame}{Learning Objectives}
At the end of this video you should able to:
\begin{itemize}
\item Understand the need of conditional statements.
\item Understand the Types of conditional statements.
\item Use Conditional Statement- If
\end{itemize}
\end{frame}


\begin{frame}{Structured Programming Theorem}

\begin{block}{Sequential execution:}
\tab \tab Executes block of code in a sequence.
\end{block}
\end{frame}

\begin{frame}{Structured Programming Theorem}
\begin{block}{Sequential execution:}
\tab \tab Executes block of code in a sequence.
\begin{block}{Sequential execution:}
\tab \tab Executes block of code,based on the decision.
\end{block}
\end{block}
\end{frame}

\begin{frame}{Structured Programming Theorem}
\begin{block}{Sequential execution:}
\tab \tab Executes block of code in a sequence.
\begin{block}{Sequential execution:}
\tab \tab Executes block of code,based on the decision.
\begin{block}{Sequential execution:}
\tab \tab Executes block of code repeatedly , for required number \tab \tab of times.
\end{block}
\end{block}
\end{block}
\end{frame}

\begin{frame}{Conditional Statement}
\begin{itemize}
    \item Controls the flow of program execution.
\end{itemize}
\end{frame}

\begin{frame}{Conditional Statement}
\begin{itemize}
    \item Controls the flow of program execution.
    \item Types of Conditional Statements:
    \begin{itemize}
        \item if
        \item if else
        \item nested if
        \item if else ladder
    \end{itemize}
\end{itemize}
\end{frame}


\begin{frame}{Syntax-if}
\begin{lstlisting}
if (boolean\_expression)\{\\
\tab statement;\\
\tab statement;\\
\}\\
\end{lstlisting}
\end{frame}

\begin{frame}{Syntax-if}
\begin{lstlisting}
if (boolean\_expression)\{\\
\tab statement;\\
\tab statement;\\
\}\\
\end{lstlisting}\\
Statement(s) in 'if' block executes only if boolean\_expression is 'true'.
\end{frame}

\begin{frame}{Problem Statement}
Given the purchase amount , calculate amount to be paid after discount.\\
\begin{center}
\begin{tabular}{|c|c|}
\hline
\textbf{Discount} & \textbf{Range} \\
\hline
25\% & min 1000\\
\hline
\end{tabular}
\end{center}
\end{frame}

\begin{frame}{Solution}
\begin{lstlisting}
int p\_amt = 15000;\\
intd\_amt = 0;\\
if (p\_amt $>$= 10000) \{\\
\tab d\_amt = (p\_amt * 25) / 100;\\
\}\\
System.out.println(p\_amt - d\_amt);\\
\end{lstlisting}
\end{frame}

\begin{frame}{Problem Statement}
Accept a number and display its absolute value.
\end{frame}

\begin{frame}{Problem Statement}
Accept a number and display its absolute value.
\begin{block}{Absolute value of a number is the distance of the number on the number line from zero.}
\end{block}
\end{frame}

\begin{frame}{Solution}
\begin{lstlisting}
int num = -27;\\
if (num $<$ 0) \{\\
\tab num = -num;\\
\}\\
System.out.println(num);\\
\end{lstlisting}
\end{frame}


\begin{frame}{Problem Statement}
Given two integer values,print their sum .If the sum is even, double the sum.
\end{frame}

\begin{frame}{solution}
\begin{lstlisting}
int n1 = 15;\\
int n2 = 30;\\
int sum\_nums = n1 + n2;\\
if (sum\_nums \% 2 == 0) \{\\
\tab sum\_nums *=2;\\
\}\\
System.out.println(sum\_nums);
\end{lstlisting}
\end{frame}

\begin{frame}{Summary}
\begin{itemize}
    \item Conditional statements control the flow of
execution of the program.
\item The expression returns either true or false.
\item The block of code in if executes only when
if condition returns true.
\end{itemize}
\end{frame}
\end{document}