\documentclass[aspectratio=169,14pt,usenames,dvipsnames]{beamer}
\usetheme{TalentSprint}
\usepackage[utf8]{inputenc}
\usepackage{graphics}
\usepackage{ragged2e}
\usepackage{amsfonts}
\usepackage{xcolor}
\usepackage{mathtools}
\usepackage{tcolorbox}
\usepackage{setspace}
\usepackage{lmodern}
\definecolor{swe}{rgb}{0.19, 0.73, 0.56}
\definecolor{lblue}{RGB}{190,200,198}
\title[Looping Statements - for]{Looping Statements - for}

\newcommand\tab[1][1cm]{\hspace*{#1}}
\begin{document}

{\1
\begin{frame} \vspace{35pt}

\subtitle{TalentSprint}
\maketitle
\end{frame}
}


\begin{frame}{Learning Objectives}
At the end of this video you should able to:
\begin{itemize}
\item Understand how for loop works.
\item Write porgrams using for loop.
\end{itemize}
\end{frame}


\begin{frame}{for}
Allows to efficiently write a loop that needs to
execute a specific number of times.
\end{frame}



\begin{frame}{Syntax}
\begin{lstlisting}
for (init; boolean\_expression; inc/dec)  \{ \\
\tab statement;\\
\tab statement;\\
\}\\
\end{lstlisting}\\
Note:The statements in the loop will execute
until the expression becomes false.
\end{frame}



\begin{frame}{Problem Statement}
Print the multiplication table for an given
number till 10.
\end{frame}

\begin{frame}{Solution}
\begin{lstlisting}
int n = 6;\\
for (int i=1;i $<$ 10; i++) \{\\
\tab int r= n* i;\\
\tab System.out.println(n + " * " + i + " = " + r);\\

\end{lstlisting}\\
\end{frame}

\begin{frame}{Problem Statement}
Print the sum of factors of the given number
\end{frame}

\begin{frame}{Solution}
\begin{lstlisting}
int num = 280;\\
int sum\_fact = num + 1;\\
for (int f=2;f $<$ num / 2;f++)\{\\
\tab if (num \% f == 0) \{\\
\tab \tab sum\_fact += f;\\
\tab \}\\
\}\\
System.out.println(sum\_fact);
\end{lstlisting}
\end{frame}


\begin{frame}{Summary}
\begin{itemize}
    \item for loop is used when you want to execute
the block of code fixed number of times.
\item The initialization statement will execute
only once.
\end{itemize}
\end{frame}
\end{document}