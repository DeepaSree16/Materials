\documentclass[aspectratio=169,14pt,usenames,dvipsnames]{beamer}
\usetheme{TalentSprint}
\usepackage[utf8]{inputenc}
\usepackage{graphics}
\usepackage{ragged2e}
\usepackage{amsfonts}
\usepackage{xcolor}
\usepackage{mathtools}
\usepackage{tcolorbox}
\usepackage{setspace}
\usepackage{lmodern}
%\definecolor{swe}{rgb}{0.19, 0.73, 0.56}
%\definecolor{lblue}{RGB}{190,200,198}
\title[Conditional Statement-Switch-case ]{Conditional Statement-Switch-case}

\newcommand\tab[1][1cm]{\hspace*{#1}}

\begin{document}
{\1
\begin{frame} \vspace{35pt}

\subtitle{TalentSprint}
\maketitle
\end{frame}
}

\begin{frame}{Learning Objectives}
At the end of this video you should able to:
\begin{itemize}
\item understand and use switch case statement.
\item write programs using switch case.
\end{itemize}
\end{frame}

\begin{frame}{Switch..Case}
Transfers control to one of the several
statements, depending on the value.
\end{frame}



\begin{frame}{Syntax}
\begin{lstlisting}
switch (variable) \{\\
\tab case value\_1:\\
\tab \tab statement;\\
\tab \tab break;\\
\tab case value\_n:\\
\tab \tab statement;\\
\tab \tab break;\\
\tab default:\\
\tab \tab statement;
\}\\
break will move the control out from the switch
\end{lstlisting}
\end{frame}

\begin{frame}{Problem statement}
Accept the day of the week and display the day,
assumes that day 1 is Sunday.\\
1 --$>$ Sunday\\
2 --$>$ Monday\\
.....\\
7 --$>$ Saturday\\
\end{frame}

\begin{frame}{Solution}

\begin{lstlisting}
int day = 3;\\
switch(day) \{\\
\tab case 1: S.o.pln("Sunday");\\
\tab \tab break;\\
\tab case 2: S.o.pln("Monday");\\
\tab \tab break;\\
\tab case 3: S.o.pln("Tuesday");\\
\tab \tab break;\\
\tab case 4: S.o.pln("Wednesday");\\
\tab \tab break;\\
\tab case 5: S.o.pln("Thursday");\\
\tab \tab break;\\
\tab case 6: S.o.pln("Friday");\\
\tab \tab break;\\
\tab case 7: S.o.pln("Saturday");\\
\tab \tab break;\\
\tab default: S.o.pln("Invalid option");\\
\}\\
\end{lstlisting}
\end{frame}

\begin{frame}{Problem Statement}
Create a simple calculator for addition,
substraction, multiplication and division.
\end{frame}

\begin{frame}{Solution}
\begin{lstlisting}
int num1 = 20;\\
int num2 = 40;\\
char operator = ‘*’;\\
switch(operator) \{\\
\tab case ‘+’: S.o.pln(num1 + num2);\\
\tab \tab break;\\
\tab case ‘-’: S.o.pln(num1 - num2);\\
\tab \tab break;\\
\tab case ‘*’: S.o.pln(num1 * num2);\\
\tab \tab break;\\
\tab case ‘/’: S.o.pln(num1 / num2);\\
\tab \tab break;\\
\tab default: S.o.pln("Invalid operator");\\
\}\\
\end{lstlisting}
\end{frame}

\begin{frame}{Problem statement}
Check the character is vowel or consonant
using switch.
\end{frame}

\begin{frame}{Solution}
\begin{lstlisting}
switch (ch) \{\\
\tab case ‘a’:\\
\tab case ‘e’:\\
\tab case ‘i’:\\
\tab case ‘o’:\\
\tab case ‘u’: S.o.pln("Vowel");\\
\tab \tab \tab break;\\
\tab default: S.o.pln("Consonant");\\
\}\\
\end{lstlisting}
\end{frame}


\begin{frame}{Summary}
\begin{itemize}
    \item Transfers control to one of the several
statements.
\item if none of the case value matches, default
block will execute.
\end{itemize}
\end{frame}
\end{document}