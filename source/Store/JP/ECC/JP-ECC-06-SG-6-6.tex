\documentclass[aspectratio=169,14pt,usenames,dvipsnames]{beamer}
\usetheme{TalentSprint}
\usepackage[utf8]{inputenc}
\usepackage{graphics}
\usepackage{ragged2e}
\usepackage{amsfonts}
\usepackage{xcolor}
\usepackage{mathtools}
\usepackage{tcolorbox}
\usepackage{setspace}
\usepackage{lmodern}
\definecolor{swe}{rgb}{0.19, 0.73, 0.56}
\definecolor{lblue}{RGB}{190,200,198}
\title[String - substring method]{String - substring method}

\newcommand\tab[1][1cm]{\hspace*{#1}}

\begin{document}
{\1
\begin{frame} \vspace{35pt}

\subtitle{TalentSprint}
\maketitle
\end{frame}
}

\begin{frame}{Learning Objectives}
At the end of this video you should able to:
\begin{itemize}
\item understand substring method.
\item implement program using substring.
\end{itemize}
\end{frame}

\begin{frame}{substring}
“Hello World”\\
“World” → is a sub-string of “Hello World”\\
\end{frame}


\begin{frame}{substring}
“Hello World”\\
“World” → is a sub-string of “Hello World”\\
\vspace{1em}
“India is a greatest country in the world.”\\
“greatest country” → is also a sub-string of the above
string\\
\end{frame}


\begin{frame}{substring methods}
Returns a new string that is a part of the string.\\
\vspace{1em}
\textbf{Types of substring methods}
\begin{center}
    String substring (int startIndex)\\
\tab \tab \hspace{1.2em}  String substring (int startIndex, int endIndex)\\
\end{center}
\end{frame}


\begin{frame}{substring (int startIndex)}
\tab \tab S t r i n g s t r = " H ell o World " ;\\
\tab \tab S t r i n g s u b s t r = s t r . s u b s t r i n g ( 6 ) ;\\
\vspace{1.5em}
\tab \tab \tab \tab \tab substr contains “World”
\end{frame}


\begin{frame}{Problem Statement}
Define a string and print the second half of the string
using substring method.
\end{frame}  


\begin{frame}{Solution}
\begin{lstlisting}
class DataTypesDemo \{\\
\tab public static void main ( String [ ] args ) \{\\
\tab \tab i n t a = 1 0;\\
\tab \tab String str = " Hello World ! " ;\\
\tab \tab System .out.println( " Sub String : " + str .\\
\tab substring ( ( str.length( ) / 2 ) ) ) ;\\
\tab \}\\
\}\\
\end{lstlisting}
\end{frame}

\begin{frame}{Problem Statement}
Write program to display the given name in the below
pattern.\\
\textbf{Example:}\\
Name: TALENT\\
Output:\\
T\\
TA\\
TAL\\
TALE\\
TALEN\\
TALENT\\
\end{frame}

\begin{frame}{Solution}
\begin{lstlisting}
class DataTypesDemo \{\\
\tab public static void main ( String [ ]args ) \{\\
\tab \tab String str= "TALENT" ;\\
\tab \tab for(int i=1 ; i <= str.length( );i++)\\
\tab \tab \tab System.out.println(str.substring(0,i));\\
\tab \}\\
\}\\
\end{lstlisting}
\end{frame}


\begin{frame}{Summary}
\begin{itemize}
    \item String has two substring methods.
    \item substring (int index), returns substring of the string
from the given index position till the end.
\item substring (int statrIndex, int endIndex), returns
substring of the string from startIndex to endIndex.
\end{itemize}
\end{frame}
\end{document}