\documentclass[aspectratio=169,14pt,usenames,dvipsnames]{beamer}
\usetheme{TalentSprint}
\usepackage[utf8]{inputenc}
\usepackage{graphics}
\usepackage{ragged2e}
\usepackage{amsfonts}
\usepackage{xcolor}
\usepackage{mathtools}
\usepackage{tcolorbox}
\usepackage{setspace}
\usepackage{lmodern}
\definecolor{swe}{rgb}{0.19, 0.73, 0.56}
\definecolor{lblue}{RGB}{190,200,198}
\title[Procedural programs using static methods]{Procedural programs using static methods}

\newcommand\tab[1][1cm]{\hspace*{#1}}

\begin{document}
{\1
\begin{frame} \vspace{35pt}

\subtitle{TalentSprint}
\maketitle
\end{frame}
}

\begin{frame}{Learning Objectives}
At the end of this video you should able to:
\begin{itemize}
\item divide problem statement into smaller units.
\end{itemize}
\end{frame}

\begin{frame}{Problem Statement}
Print sum of all three digit palindromes, which
contains all even digits.
\end{frame}



\begin{frame}{Solution}

\begin{block}{Methods Used}\\
\begin{itemize}
    \item getReverse(int)
    \item boolean isPalindrome(int)
    \item boolean isEven(int)
    \item boolean isAll3DigitsEven(int)
    \item int sumOfAll3DEvenPalindromes()
\end{itemize}
\end{block}
\end{frame}

\begin{frame}{Solution}
\begin{lstlisting}
int getReverse(int num) \{\\
\tab int revNum = 0;\\
\tab while (num $>$ 0) \{\\
\tab \tab int rem = num \% 10;\\
\tab \tab revNum = revNum * 10 + rem;\\
\tab \tab num /= 10;
\tab \}\\
\tab return revNum;\\
\}\\
\end{lstlisting}
\end{frame}

\begin{frame}{Solution}
\begin{lstlisting}
boolean isPalindrome(int n) \{\\
\tab return (n == getReverse(n));\\
\}\\
\end{lstlisting}
\end{frame}

\begin{frame}{Solution}

\begin{lstlisting}
boolean isEven(int n) \{\\
\tab return (n \% 2 == 0);\\
\}\\
\end{lstlisting}
\end{frame}

\begin{frame}{Solution}
\begin{lstlisting}
boolean isAll3DigitsEven(int num) \{\\
\tab while (num $>$ 0) \{\\
\tab \tab int rem = num \% 10;\\
\tab \tab if (!isEven(rem))\\
\tab \tab \tab return false;\\
\tab \tab num /= 10;\\
\tab \}\\
\end{lstlisting}
\end{frame}

\begin{frame}{Solution}
\begin{lstlisting}
int sumOfAll3DEvenPalindromes() \{\\
\tab int sum = 0;\\
\tab for (int i = 201; i $<$= 899; i++) \{\\
\tab \tab if (isPalindrome(i) \&\&\\
\tab \tab \tab \tab isAll3DigitsEven(i))\\
\tab \tab \tab sum += i;\\
\tab \}\\
\tab return sum;\\
\}\\
\end{lstlisting}
\end{frame}

\begin{frame}{Solution - main method}
\begin{lstlisting}
public class SumOfEvenDigsPal \{\\
\tab public static void main(String[] args) \{\\
\tab \tab int sum = sumOfAll3DEvenPalindromes();\\
\tab \tab System.out.println(sum);\\
\tab \}\\
\}\\
\end{lstlisting}
\end{frame}

\begin{frame}{Summary}
Four different forms of methods are:\\
\begin{itemize}
    \item Method is a block of code which can be
used to perform specific task.
\item modularizing the code will increase the
readability.
\item each method can be implemented and
tested separately.
\end{itemize}
\end{frame}
\end{document}

