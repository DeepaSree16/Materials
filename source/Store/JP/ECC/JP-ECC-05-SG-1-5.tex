\documentclass[aspectratio=169,14pt,usenames,dvipsnames]{beamer}
\usetheme{TalentSprint}
\usepackage[utf8]{inputenc}
\usepackage{graphics}
\usepackage{ragged2e}
\usepackage{amsfonts}
\usepackage{xcolor}
\usepackage{mathtools}
\usepackage{tcolorbox}
\usepackage{setspace}
\usepackage{lmodern}
\definecolor{swe}{rgb}{0.19, 0.73, 0.56}
\definecolor{lgreen}{RGB}{190,200,198}
\title[Introduction to methods]{Introduction to methods}

\newcommand\tab[1][1cm]{\hspace*{#1}}
\begin{document}

{\1
\begin{frame} \vspace{35pt}

\subtitle{TalentSprint}
\maketitle
\end{frame}
}


\begin{frame}{Learning Objectives}
At the end of this video you should able to:
\begin{itemize}
\item Understand the need of modularization of
the program.
\item Write and use the user-defined methods.
\end{itemize}
\end{frame}


\begin{frame}{modularization}
Breaking down a large program into smaller
pieces of code.
\end{frame}

\begin{frame}{modularization}
Breaking down a large program into smaller
pieces of code.
\begin{block}{Advantages:}
\begin{itemize}
    \item Easier to Debug
    \item Reusable code
    \item Readability 
    \item Reliability
\end{itemize}
\end{block}
\end{frame}

\begin{frame}{Method}
A method is a piece of code that performs a
particular task.
\end{frame}

\begin{frame}{Method Signature}
\begin{block}{General form of methods:}\\
\begin{lstlisting}
modifier returnType\\
\tab \tab nameOfMethod(parameter List) \{\\
\tab method body\\
\}
\end{lstlisting}
\end{block}
\end{frame}

\begin{frame}{Key Points}
\begin{block}{modifiers:}such as public, private, static and others you will learn later.
\end{block}
\begin{block}{return type:}
the data type of the value returned by the method, or void if the method does not return a value.
\end{block}
\begin{block}{method name:}
The rules for variable names apply to method names as well.
\end{block}
\begin{block}{parameter list:}
Input values passed to the method enclosed by parentheses.
\end{block}
\begin{block}{method body:}
The method’s code, the actual
task to be performed by the method.
\end{block}
\end{frame}


\begin{frame}{Problem Statement}
Define a method which returns the greatest
number among three numbers.
\end{frame}

\begin{frame}{Syntax-if}
\begin{lstlisting}
public static int getGreatest\\
\tab \tab \tab \tab (int a, int b, int c) \{\\
\tab int max = a;\\
\tab if (b $>$ max)\\
\tab \tab max $=$ b;\\
\tab if (c $>$ max)\\
\tab \tab max $=$ c;\\
\tab return max;\\
\}\\
\end{lstlisting}\\
\end{frame}

\begin{frame}{Problem Statement}
Define a method which returns the sum of all
even numbers from 1 to given limit.
\end{frame}

\begin{frame}{Solution}
\begin{lstlisting}
public static int evenSum(int limit) \{\\
\tab int sum = 0;\\
\tab for (int n = 1; n$<$= limit; n++) \{\\
\tab \tab  if (n \% 2 == 0) \{\\
\tab \tab \tab sum+= n;\\
\tab \tab  \}\\
\tab  \}\\
\tab return sum;\\
\}\\
\end{lstlisting}
\end{frame}

\begin{frame}{Problem Statement}
Define a method to check if the given number is
prime. Return true if number is prime, else
return false.
\end{frame}


\begin{frame}{Solution}
\begin{lstlisting}
public static int isPrime(int limit) \{\\
\tab if (num == 1)\\
\tab \tab return false;\\
\tab for (int i = 2; i$<$= num / 2; i++) \{\\
\tab \tab  if (n \% i == 0)\\
\tab \tab \tab return false;\\
\tab  \}\\
\tab return true;\\
\}\\
\end{lstlisting}
\end{frame}



\begin{frame}{Summary}
\begin{itemize}
    \item Method is a block of code which can be
used to perform specific tasks.
\item Modularizing the code will increase
readability.
\item Each method can be implemented and
tested separately.
\end{itemize}
\end{frame}
\end{document}