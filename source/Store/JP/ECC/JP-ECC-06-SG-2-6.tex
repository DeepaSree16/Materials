\documentclass[aspectratio=169,14pt,usenames,dvipsnames]{beamer}
\usetheme{TalentSprint}
\usepackage[utf8]{inputenc}
\usepackage{graphics}
\usepackage{ragged2e}
\usepackage{amsfonts}
\usepackage{xcolor}
\usepackage{mathtools}
\usepackage{tcolorbox}
\usepackage{setspace}
\usepackage{lmodern}
\definecolor{swe}{rgb}{0.19, 0.73, 0.56}
\definecolor{lblue}{RGB}{190,200,198}
\title[Commonly use String methods]{Commonly use String methods}

\newcommand\tab[1][1cm]{\hspace*{#1}}

\begin{document}
{\1
\begin{frame} \vspace{35pt}

\subtitle{TalentSprint}
\maketitle
\end{frame}
}

\begin{frame}{Learning Objectives}
At the end of this video you should able to:
\begin{itemize}
\item learn Immutable String class.
\item learn commonly use String methods.
\end{itemize}
\end{frame}

\begin{frame}{Immutable Class}
\begin{block}{Immutable}\\
\vspace{0.8em}
An Immutable object is a kind of object whose state
cannot be modified after it is created.
\end{block}
\end{frame}

\begin{frame}{Immutable Class}
\begin{block}{Immutable}\\
An Immutable object is a kind of object whose state
cannot be modified after it is created.
\end{block}
\vspace{0.8em}
\begin{block}{Immutable Object }\\
If an object is known to be immutable, the object
reference can be shared.
\end{block}
\end{frame}

\begin{frame}{Immutable Class}
\begin{block}{Immutable}\\
An Immutable object is a kind of object whose state
cannot be modified after it is created.
\end{block}
\begin{block}{Immutable Object }\\
If an object is known to be immutable, the object
reference can be shared.
\end{block}
\begin{block}{\textcolor{blue}{Example}}\\
Boolean, Byte, Character, Double, Float, Integer, Long,
Short, and String are immutable classes in Java,
\end{block}
\end{frame}


\begin{frame}{String is Immutable Class}
\begin{lstlisting}
String str1 = "Hellow";\\
String str2 = str1";\\
\vspace{1em}
String str1 + "World"; (OR)\\
String str1.concat("World");\\
\vspace{1em}
str1 = str1.concat("World");\\
\end{lstlisting}
\end{frame}



\begin{frame}{Operations on String}
\textbf{\textcolor{red}{int}} \textbf{\textcolor{blue}{length()}} \\
Returns No. of characters in a String\\
\vspace{1em}
\textbf{\textcolor{red}{string}} \textbf{\textcolor{blue}{trim()}} \\
Remove leading and trailing spaces for a String\\
\vspace{1em}
\textbf{\textcolor{red}{String}} \textbf{\textcolor{blue}{concat(String)}} \\
Join two strings and return it\\
\end{frame}


\begin{frame}{Operations on String}
\textbf{\textcolor{red}{boolean}} \textbf{\textcolor{blue}{startsWith(String prefix)}} \\
Join two strings and return it\\
\vspace{1em}
\textbf{\textcolor{red}{boolean}} \textbf{\textcolor{blue}{endsWith(String suffix)}} \\
Returns true if String ends with given suffix\\
\end{frame}


\begin{frame}{Example}
\begin{block}{int length()}
\begin{lstlisting}
String name = "Praveen Kumar";\\
int len = name.length();\\
\end{lstlisting}
\end{block}
\begin{block}{String trim()}
\begin{lstlisting}
String name = " Praveen Kumar ";\\
String str1 = name.trim();\\
\end{lstlisting}
\end{block}
\end{frame}

\begin{frame}{Example}
\begin{block}{String concat(String)}
\begin{lstlisting}
String FirstName = "Praveen";\\
String LastName = " Kumar";\\
String str2 = FirstName.concat(LastName);\\
\end{lstlisting}
\end{block}
\end{frame}

\begin{frame}{Examples}
\begin{block}{boolean startsWith(String prefix)}\\
\begin{lstlisting}\\
String Str3 = new String("Welcome to Talentsprint.com");\\
boolean result1 = Str3.startsWith("Welcome");\\
boolean result2 = Str3.startsWith("Hellow");\\
boolean result3 = Str3.startsWith("Telentsprint",11);\\
\end{lstlisting}
\end{block}

\end{frame}
\begin{frame}{Examples}
\begin{block}{boolean endsWith(String suffix)}\\
\begin{lstlisting}\\
String Str4 = new String("This is really wonderful image!!");\\
boolean result4 = str4.endsWith("image!!");\\
boolean result5 = str4.endsWith("really");\\
\end{lstlisting}
\end{block}
\vspace{0.5em}
\color{red} Write code with the above examples, compile and execute
the program.
\end{frame}


\begin{frame}{Summary}
\begin{itemize}
    \item Learning Immutable String class.
\item Learning commonly using String methods.
\end{itemize}
\end{frame}
\end{document}
