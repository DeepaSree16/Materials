\documentclass[aspectratio=169,14pt,usenames,dvipsnames]{beamer}
\usetheme{TalentSprint}
\usepackage[utf8]{inputenc}
\usepackage{graphics}
\usepackage{ragged2e}
\usepackage{amsfonts}
\usepackage{xcolor}
\usepackage{mathtools}
\usepackage{tcolorbox}
\usepackage{setspace}
\usepackage{lmodern}
\definecolor{swe}{rgb}{0.19, 0.73, 0.56}
\definecolor{lgreen}{RGB}{190,200,198}
\title[Variable and Data Types]{Variable and Data Types}

\newcommand\tab[1][1cm]{\hspace*{#1}}
\begin{document}

{\1
\begin{frame} \vspace{35pt}

\subtitle{TalentSprint}
\maketitle
\end{frame}
}


\begin{frame}{Learning Objectives}
At the end of this video you should able to:
\begin{itemize}
\item Learn what is a variable
\item Learn different data types in Java
\item Learn how to declare and initialize variable;
\end{itemize}
\end{frame}

\begin{frame}{Variable}
\begin{block}{variable}
A variable is the name of memory location where the data is stored.
\end{block}
\end{frame}


\begin{frame}{Variable Naming Convention}
\textbf{Key Points}
\begin{itemize}
\item Variable names can have alpha-numerics.
\item Variable names should start with either alphabet or underscore $($\_$)$
\item Variable names should not contain any special character except under-score $($\_$)$,including white space.
\item If variable name contains more than one word every first Character from second word should be in uppercase or seperate words with underscore $($\_$)$
\end{itemize}
\end{frame}

\begin{frame}
\frametitle{Naming Convention Example}
\begin{columns}
\column{0.5\textwidth}
\textbf{valid variable names}
\begin{itemize}
\item name
\item number$5$
\item \_number
\item firstName
\item first\_name
\end{itemize}

\column{0.5\textwidth}
\textbf{Invalid variable names}
\begin{itemize}
\item $@$name
\item first$-$Name
\item $5$number

\item first name
\item number\$
\end{itemize}

\end{columns}
\end{frame}












\begin{frame}{Data Types}
Data types represent the set of different values that can be stored in the variable
\end{frame}




\begin{frame}{Types of Data Types}
\begin{columns}
\column{0.3\textwidth}
\textbf{Primitive Types}
\begin{itemize}
\item Integer
\begin{itemize}
\item byte
\item short
\item int
\item long
\end{itemize}

\item Floating Point
\begin{itemize}
\item float
\item double
\end{itemize}
\end{itemize}

\column{0.3\textwidth}
\textbf{Primitive Types}
\begin{itemize}
\item Character
\begin{itemize}
\item char
\end{itemize}
\item Boolean
\begin{itemize}
\item boolean
\end{itemize}
\end{itemize}

\column{0.4\textwidth}
\textbf{Non-Primitive Types}
\begin{itemize}
\item String
\item Array
\end{itemize}
\end{columns}
\end{frame}


\begin{frame}{Syntax}
\textbf{Declaration$:$}\\
datatype variable\_name$;$
\end{frame}

\begin{frame}{Syntax}
\textbf{Declaration$:$}\\

\begin{lstlisting}
datatype variable\_name$;$\\
\tab int number $;$
\end{lstlisting}
\end{frame}

\begin{frame}{Syntax}
\textbf{Declaration$:$}\\
\begin{lstlisting}
datatype variable\_name$;$\\
\tab int number $;$
\end{lstlisting}\\
\textbf{Initialization$:$}\\
datatype variable\_name$=$value$;$
\end{frame}


\begin{frame}{syntax}
\textbf{Declaration$:$}\\
\begin{lstlisting}
datatype variable name;\\
\tab int number ;
\end{lstlisting}\\
\textbf{Initialization$:$}\\
\begin{lstlisting}
datatype variable\_name = value ;\\
\tab int number = 50; \\
\tab float salary = 5000.of;\\
\tab boolean flag = false;\\
\end{lstlisting}
\end{frame}

\begin{frame}{problem statement}
Declare and initialize two variables of type interger and print their values.
\end{frame}
\begin{frame}{Solution}
\begin{lstlisting}
class DataTypesDemo\{\\
\tab public static void main(string[] args)\{\\
\tab \tab int num1 = 10;\\
\tab \tab int num2 = 15;\\
\tab \tab system.out.println("num1 = " + num1);\\
\tab \tab  system.out.println("num2 = " + num2);\\
 \tab   \}\\
\}\\
\end{lstlisting}
\end{frame}

\begin{frame}{Problem Statement}
Declare and initialize variable of type int , float,double and print their values.
\end{frame}

\begin{frame}{Solution}
\begin{lstlisting}
class PrintData\{\\
\tab public static void main(String[] args)\{\\
\tab \tab int num = 10;\\
\tab \tab float fValue = 15f;\\
\tab \tab double dValues = 100.0;\\
\tab \tab system.out.println("Integer value = " + num);\\
\tab \tab system.out.println("Float value = " + fValue);\\
\tab \tab system.out.println("Double value = " + dValue);\\
\end{lstlisting}
\end{frame}


\begin{frame}{Summary}
\begin{itemize}
    \item A variable should be declared and initialized before using it in the program.
    \item There are 8 primitive data types in Java.
    \item Float values should have post fix 'f'.
    
\end{itemize}
\end{frame}

\end{document}