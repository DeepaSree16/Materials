\documentclass[aspectratio=169,14pt,usenames,dvipsnames]{beamer}
\usetheme{TalentSprint}
\usepackage[utf8]{inputenc}
\usepackage{graphics}
\usepackage{ragged2e}
\usepackage{amsfonts}
\usepackage{xcolor}
\usepackage{mathtools}
\usepackage{tcolorbox}
\usepackage{setspace}
\usepackage{lmodern}
\definecolor{swe}{rgb}{0.19, 0.73, 0.56}
\definecolor{lblue}{RGB}{190,200,198}
\title[Conditional Statement-if else ladder]{Conditional Statement-if else ladder}

\newcommand\tab[1][1cm]{\hspace*{#1}}
\begin{document}

{\1
\begin{frame} \vspace{35pt}

\subtitle{TalentSprint}
\maketitle
\end{frame}
}


\begin{frame}{Learning Objectives}
At the end of this video you should able to:
\begin{itemize}
\item understand and use if else ladder.
\item write programs using if else ladder.
\end{itemize}
\end{frame}


\begin{frame}{If else ladder}
\begin{itemize}
    \item Is a multipath decision control structure.
    \item Executes only one block of statement(s).
\end{itemize}
\end{frame}



\begin{frame}{Syntax}
\begin{lstlisting}
if (A\_boolean\_expression) \{\\
\tab statement\_A;\\
\} else if (B\_boolean\_expression) \{\\
\tab statement\_B;\\
\} else if (C\_boolean\_expression) \{\\
\tab statement\_C;\\
\} else \{\\
\tab statement\_D;\\
\}\\
\end{lstlisting}
\end{frame}



\begin{frame}{problem Statement}
Find the grade of a student based on the
average mark of his/her semester.
\end{frame}

\begin{frame}{problem Statement}
Find the grade of a student based on the
average mark of his/her semester.
\begin{center}
\begin{tabular}{|c|c|}
\hline 
\textbf{Average Marks} & \textbf{Grade}\\
\hline
$>$= 60 & A\\
\hline
$>$= 45 and $<$ 60 & B\\
\hline
$>$= 30 & C\\
\hline
Otherwise & Fail\\
\hline
\end{tabular}
\end{center}
\end{frame}
    


\begin{frame}{Solution}
\begin{lstlisting}
int avgmarks = 55;\\
if (avgmarks $>$= 60) \{\\
\tab System.out.println("Grade A");\\
\} else if (avgmarks $>$= 45) \{\\
\tab System.out.println("Grade B");\\
\} else if (avgmarks $>$= 30) \{\\
\tab System.out.println("Grade C");\\
\} else \{\\
\tab System.out.println("Fail");\\
\}\\
\end{lstlisting}
\end{frame}

\begin{frame}{Problem Statement}
Given the purchased amount, calculate
amount to pay after discount.\\
\begin{center}
\begin{tabular}{|c|c|}
\hline 
\textbf{Average Marks} & \textbf{Grade}\\
\hline
$>$= 60 & A\\
\hline
$>$= 45 and $<$ 60 & B\\
\hline
$>$= 30 & C\\
\hline
Otherwise & Fail\\
\hline
\end{tabular}
\end{center}
\end{frame}

\begin{frame}{Solution}
\begin{lstlisting}
int avgmarks = 55;\\
if (avgmarks $>$= 60) \{\\
\tab System.out.println("Grade A");\\
\} else if (avgmarks $>$= 45) \{\\
\tab System.out.println("Grade B");\\
\} else if (avgmarks $>$= 30) \{\\
\tab System.out.println("Grade C");\\
\} else \{\\
\tab System.out.println("Fail");\\
\}\\
\end{lstlisting}
\end{frame}

\begin{frame}{Problem statement}
Given the purchased amount, calculate
amount to pay after discount.
\begin{center}
\begin{tabular}{|c|c|}
\hline 
\textbf{Discount} & \textbf{Range}\\
\hline
10\% & lessthan 15000\\
\hline
20\% & between 15000 to 20000\\
\hline
30\% & greaterthan 20000\\
\hline
\end{tabular} 
\end{center}
\end{frame}

\begin{frame}{Solution}
\begin{lstlisting}
int p\_amt = 18500;\\
int d\_amt = 0;\\
if (p\_amt $<$$ 15000) \{\\
\tab d\_amt = (p\_amt * 10) / 100;\\
\} else if (p\_amt $>$= 15000 \&\& p\_amt $<$= 20000) \{\\
\tab d\_amt = (p\_amt * 20) / 100;\\
\} else \{\\
\tab d\_amt = ((p\_amt * 30) / 100;\\
\}\\
System.out.println(p\_amt - d\_amt);\\
\end{lstlisting}
\end{frame}


\begin{frame}{Summary}
\begin{itemize}
    \item if else is used to execute one amoung
multiple options.
\item if any of the conditions is true then the
corresponding block of code will execute.
\item if none are true else block code will execute.

\end{itemize}
\end{frame}
\end{document}

