\documentclass[aspectratio=169,14pt,usenames,dvipsnames]{beamer}
\usetheme{TalentSprint}
\usepackage[utf8]{inputenc}
\usepackage{graphics}
\usepackage{ragged2e}
\usepackage{amsfonts}
\usepackage{xcolor}
\usepackage{mathtools}
\usepackage{tcolorbox}
\usepackage{setspace}
\usepackage{lmodern}
\definecolor{swe}{rgb}{0.19, 0.73, 0.56}
\definecolor{lblue}{RGB}{190,200,198}
\title[for-each Loop]{for-each Loop}

\newcommand\tab[1][1cm]{\hspace*{#1}}

\begin{document}
{\1
\begin{frame} \vspace{35pt}

\subtitle{TalentSprint}
\maketitle
\end{frame}
}

\begin{frame}{Learning Objectives}
At the end of this video you should able to:
\begin{itemize}
\item understand purpose of using for-each loop.
\item learn limitations of for-each loop.
\item undaertand printing array elements using for-each
loop
\end{itemize}
\end{frame}

\begin{frame}{for-each}
\begin{itemize}
\item When we work with iteration over the arrays Java
provided \textcolor{blue}{for-each} loop.
\item It is mainly used to traverse array elements.
\item Less error prone code.
\item Improved readability.
\item Less number of variables to clean up.
\end{itemize}
\end{frame}



\begin{frame}{Can’t use in the following cases}

\begin{itemize}
    \item To remove elements as you traverse arrays
    \item To modify the current index in an array
    \item To iterate over multiple arrays
\end{itemize}
\end{frame}

\begin{frame}{General form}
\begin{lstlisting}
for(data\_type variable : array) \{\\
\tab //Statments;\\
\tab //Statments;\\
\}\\
\end{lstlisting}
\end{frame}

\begin{frame}{Problem Statement}
Write a program to print list of array values using for-each
loop.
\end{frame}

\begin{frame}{Solution}
\begin{center}
\begin{lstlisting}
int []numbers = \{10, 20, 30, 40, 50\};\\
for (int x : numbers) \{\\
\tab System.out.print(x);\\
\}\\
\end{lstlisting}
\end{center}
\vspace{0.8em}
\textbf{Here Array value read from same datatype one by one and
printing at console.}
\end{frame}

\begin{frame}{Problem Statement}
wirte a program to print String array values using for-each
loop.
\end{frame}


\begin{frame}{Solution}
\begin{lstlisting}
String emp\_names[] = \{ "James","King","Blake","Jone","Murray" \};\\
for (String name : emp\_names) \{\\
\tab System.out.print(name);\\
\}\\
\end{lstlisting}
\end{frame}


\begin{frame}{Summary}
\begin{itemize}
    \item Understood when to use for-each loop in program.
\item Understood pritning the elements from starting to till
size-1 locaiton from the array.
\end{itemize}
\end{frame}
\end{document}

