\documentclass[aspectratio=169,14pt,usenames,dvipsnames]{beamer}
\usetheme{TalentSprint}
\usepackage[utf8]{inputenc}
\usepackage{graphics}
\usepackage{ragged2e}
\usepackage{amsfonts}
\usepackage{xcolor}
\usepackage{mathtools}
\usepackage{tcolorbox}
\usepackage{setspace}
\usepackage{lmodern}
\definecolor{swe}{rgb}{0.19, 0.73, 0.56}
\definecolor{lgreen}{RGB}{190,200,198}
\title[Arithmetic and Relational Operators]{Arithmetic and Relational Operators}

\newcommand\tab[1][1cm]{\hspace*{#1}}
\begin{document}

{\1
\begin{frame} \vspace{35pt}

\subtitle{TalentSprint}
\maketitle
\end{frame}
}


\begin{frame}{Learning Objectives}
At the end of this video you should able to:
\begin{itemize}
\item Understand the type of Operators.
\item Differentiate between = and == operator.
\item Perform Operations on Variables
\end{itemize}
\end{frame}

\begin{frame}{Operator}
A symbol that represents an Operation.
\begin{block}{Example}
sum $=$ num$1$\ $+$ num$2$\\
$"="$ and $"+"$ are Operators
\end{block}
\end{frame}


\begin{frame}{Arithmetic Operators}
Let a $=$ 25 and b $=$ 10\\

\begin{tabular}{|c|c|c|}
\hline 
\textbf{Operator} & \textbf{Meaning} & \textbf{Example}\\

\hline
$+$ & Addition & a$+$b$==>$ 35\\
\hline
$-$ & Subtraction & a$-$b$==>$ 15\\
\hline
$*$ & Multiplication & a$*$b$==>$ 250\\
\hline
$/$ & Division & a$/$b $==>$2\\
\hline
$\%$ & Module & a$\%$b $==>$5\\
\hline
\end{tabular}
\end{frame}

\begin{frame}{Problem Statement}
\begin{lstlisting}
num = 245;\\
u\_digit = num \% 10;\\
h\_digit = num / 100;\\
num /= 10;\\
t\digit = num \% 10;\\
sum u\_digit + t\_digit + h\_digit;\\
system.out.println(sum);\\
\end{lstlisting}
\end{frame}


\begin{frame}{Relational Operators}
Let a $ = $ 25 and b $ = $ 12
\begin{tabular}{|c|c|c|}
\hline 
\textbf{Operator} & \textbf{Meaning} & \textbf{Example}\\
\hline
$>$ & Greater than & a $>$ 10==$>$ true\\
\hline
 $>$= & Greater than or equal to & a$>$= 25  ==$>$  true\\
\hline
$<$ & Less than & b $<$ a==$>$ true\\
 \hline
$<$= & Less than or equal to & a$<$= 25  ==$>$  true\\
 \hline
== & Equals & b == 10==$>$true\\
\hline
!= & Not Equals & b!=12 ==$>$ false\\
\hline
\end{tabular}
\end{frame}

\begin{frame}{Unary Operators}
\begin{tabular}{|c|c|}
\hline
\textbf{Operator} & \textbf{Meaning} \\
\hline
++ & Increment / Unary Plus\\
\hline
- - & Decrements / Unary Minus\\
\hline
\end{tabular}
\end{frame}

\begin{frame}{Unary Operators}
\begin{columns}
\column{0.5\textwidth}\\
\begin{block}{X $=$ 3}
z $=$ ++x

\end{block}
\column{0.5\textwidth}
==$>$ z is 4,x is 4

\end{columns}
\end{frame}

\begin{frame}{Unary Operators}
\begin{columns}
\column{0.5\textwidth}\\
\begin{block}{X $=$ 3}
z $=$ ++x\\
z $=$ x++\\
\end{block}
\column{0.5\textwidth}
==$>$ z is 4,x is 4\\
==$>$ z is 3,x is 4\\
\end{columns}
\end{frame}

\begin{frame}{Unary Operators}
\begin{columns}
\column{0.5\textwidth}\\
\begin{block}{X $=$ 3}
z $=$ ++x\\
z $=$ x++\\
z $=$ x- -\\
\end{block}
\column{0.5\textwidth}
==$>$ z is 4,x is 4\\
==$>$ z is 3,x is 4\\
==$>$ z is 3,x is 2\\
\end{columns}
\end{frame}

\begin{frame}{Unary Operators}
\begin{columns}
\column{0.5\textwidth}\\
\begin{block}{X $=$ 3}
z $=$ ++x\\
z $=$ x++\\
z $=$ x- -\\
z $=4$ - -x
\end{block}
\column{0.5\textwidth}
==$>$ z is 4,x is 4\\
==$>$ z is 3,x is 4\\
==$>$ z is 3,x is 2\\
==$>$ z is 2,x is 2\\
\end{columns}
\end{frame}

\begin{frame}{Unary Operators}
\begin{columns}
\column{0.5\textwidth}\\
\begin{block}{X $=$ 3}
z $=$ ++x\\
z $=$ x++\\
z $=$ x- -\\
z $=4$ - -x
\end{block}
\column{0.5\textwidth}
==$>$ z is 4,x is 4\\
==$>$ z is 3,x is 4\\
==$>$ z is 3,x is 2\\
==$>$ z is 2,x is 2\\
\end{columns}
\color{red}x++ and x- - ==$>$ Post Increment / Decrement\\
\color{red}++x and - -x ==$>$ Post Increment / Decrement\\
\end{frame}


\begin{frame}{Problem Statement}
Given the purchase amount, calculate the
amount to be paid after discount. Discount is
10 percent on purchase amount.
\end{frame}

\begin{frame}{Solution}
\begin{lstlisting}
amt\_purchsd = 20000;\\
discount\_amt = (amt\_purchsd * 10) / 100;\\
System.out.println(amt\_purchsd -\\
discount\_amt);\\
\end{lstlisting}
\end{frame}

\begin{frame}{Summary}
\begin{itemize}
    \item Arithmetic Operators are used to perform
arithmetic operations.
\item Relational Operators are used to check the
relation between two values and result to
true or false.
\item “=” is an assignment operator, used to
assign the value.
\item “==” is a comparision operator, used to
compare the values.
\end{itemize}
\end{frame}
\end{document}