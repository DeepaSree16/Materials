\documentclass[aspectratio=169,14pt,usenames,dvipsnames]{beamer}
\usetheme{TalentSprint}
\usepackage[utf8]{inputenc}
\usepackage{graphics}
\usepackage{ragged2e}
\usepackage{amsfonts}
\usepackage{xcolor}
\usepackage{mathtools}
\usepackage{tcolorbox}
\usepackage{setspace}
\usepackage{lmodern}
%\definecolor{swe}{rgb}{0.19, 0.73, 0.56}
%\definecolor{lblue}{RGB}{190,200,198}
\title[Different Forms of methods]{Different Forms of methods}

\newcommand\tab[1][1cm]{\hspace*{#1}}

\begin{document}
{\1
\begin{frame} \vspace{35pt}

\subtitle{TalentSprint}
\maketitle
\end{frame}
}

\begin{frame}{Learning Objectives}
At the end of this video you should able to:
\begin{itemize}
\item understand and write different forms of
methods
\end{itemize}
\end{frame}

\begin{frame}{Forms of Methods}
\begin{itemize}
    \item without arguments and return type
    \item with arguments and without return type
    \item without arguments and with return type
    \item with arguments and return type
\end{itemize}
\end{frame}



\begin{frame}{Method-I}
without arguments and return type\\
\begin{block}{Example}\\
\textbf{Method to print the address}\\
\begin{lstlisting}
public void printAddress() \{\\
\tab S.o.pln("House/Flat No: 6-53/4-A");\\
\tab S.o.pln("Road No: 03");\\
\tab S.o.pln("MP Colony, New Delhi");\\
\tab S.o.pln("Pin Code: 110005");\\
\}\\
\end{lstlisting}
\end{block}

\end{frame}

\begin{frame}{Method - II}
with arguments without return type
\begin{block}{Example}\\
\textbf{print even numbers in the range}\\
\begin{lstlisting}
public void printEvens(int s\_val, int e\_val) \{\\
\tab int n = s\_val;\\
\tab while (n $<$= e\_val) \{\\
\tab \tab if (n \% 2 == 0) \{\\
\tab \tab \tab System.out.println(n);\\
\tab \tab \}\\
\tab \tab n++;\\
\tab \}\\
\}\\
\end{lstlisting}
\end{block}
\end{frame}

\begin{frame}{Method - III}
without arguments with return type
\begin{block}{Example}\\
\textbf{return the sum of first ten natural numbers}\\
\begin{lstlisting}
public int sumOfFirst10() \{\\
\tab int sum\_n = 0;\\
\tab for (int n = 1; n $<$= 10; n++) \{\\
\tab \tab sum\_n += n;\\
\tab \}\\
\tab return sum\_n;\\
\}\\
\end{lstlisting}
\end{block}
\end{frame}

\begin{frame}{Method - IV}
with arguments and return type
\begin{block}{Example}\\
\textbf{return sum of digits of given number}\\
\begin{lstlisting}
public int sumOfDigits(int num) \{\\
\tab int sum\_d = 0;\\
\tab while (num $>$$ 0) \{\\
\tab \tab sum\_d = num \% 10;\\
\tab \tab num /= 10;\\
\tab \}\\
\tab return sum\_d;\\
\}\\
\end{lstlisting}
\end{block}
\end{frame}

\begin{frame}{Summary}
Four different forms of methods are:\\
\begin{itemize}
    \item without arguments and return type.

\item with arguments and without return type.
\item without arguments and with return type
\item with arguments and return type
\end{itemize}
\end{frame}
\end{document}
