\documentclass[aspectratio=169,14pt,usenames,dvipsnames]{beamer}
\usetheme{TalentSprint}
\usepackage[utf8]{inputenc}
\usepackage{graphics}
\usepackage{ragged2e}
\usepackage{amsfonts}
\usepackage{xcolor}
\usepackage{mathtools}
\usepackage{tcolorbox}
\usepackage{setspace}
\usepackage{lmodern}
\definecolor{swe}{rgb}{0.19, 0.73, 0.56}
\definecolor{lblue}{RGB}{190,200,198}
\title[Introduction to an Arrays]{Introduction to an Arrays}

\newcommand\tab[1][1cm]{\hspace*{#1}}

\begin{document}
{\1
\begin{frame} \vspace{35pt}

\subtitle{TalentSprint}
\maketitle
\end{frame}
}

\begin{frame}{Learning Objectives}
At the end of this video you should able to:
\begin{itemize}
\item understand need of array.
\item know types arrays
\item declare and initialize an array
\item access the Array elements
\end{itemize}
\end{frame}

\begin{frame}{Array}
\begin{itemize}
\item array is collection of heterogeneous values
\item array has limited in size
\item array size can’t increase or decrease at run time
\item Index starts with ’0’ and ends at ’size-1’ location
\end{itemize}
\end{frame}



\begin{frame}{Types of Arrays}

\begin{itemize}
    \item Single Dimensional Array
    \item Multi Dimensional Array
\end{itemize}
\end{frame}

\begin{frame}{Declaration}
\textbf{Single dimension array can declared as follows:}
\begin{lstlisting}
datatype[] variable\_name;\\
datatype []variable\_name;\\
datatype variable\_name[];\\
\vspace{0.8em}
int[] stud\_id;\\
int stud\_id[];\\
int []stud\_id;\\
\end{lstlisting}
\textcolor{red}{Note:} Here variable acts like a reference, does not have memory of its own.
\end{frame}

\begin{frame}{Definition}
\textbf{datatype variableName = new dataype[size];}\\
\vspace{0.5em}
\textcolor{blue}{datatype:} type of value to be define\\
\vspace{0.5em}
\textcolor{blue}{variableName:} identifier to store the value\\
\vspace{0.5em}
\textcolor{blue}{new:} used to allocate dynamic memory allocation for variable\\
\vspace{0.5em}
\textcolor{blue}{Size:} Number of elements to store in a variable\\
\vspace{0.8em}
\textcolor{red}{Note:} Allocates variable memory physically in the
memory location with address.
\end{frame}

\begin{frame}{Initialization}
\begin{center}
\begin{lstlisting}
\testbf int[] marks = new int[5];\\ 
\vspace{0.8em}
int marks[0] = 10;\\
int marks[1] = 20;\\
int marks[2] = 30;\\
int marks[3] = 40;\\
int marks[4] = 50;\\
\end{lstlisting}
\end{center}
\end{frame}

\begin{frame}{Initialization}
\begin{center}
\textbf{marks  ==$>$}
\begin{tabular}{|c|c|c|c|c|}
\hline
     10 & 20 & 30 & 40 & 50 \\
\hline
    0 & 1 & 2 & 3 & 4\\
\hline
\end{tabular}\\
\vspace{0.8em}
\textbf{(or)}\\
\vspace{0.8em}
\textbf{int [] marks = {10, 20, 30, 40, 50};}
\end{center}
\end{frame}

\begin{frame}{Problem Statement}
Write a program to store 5 values into an array and
display.
\end{frame}

\begin{frame}{Solution}
\begin{lstlisting}
int num[] = \{45, 78, 89, 34, 58\};\\
\tab for (int i = 0; i < 5; i++)\\
\tab \tab S.O.P(num[i]);\\
\end{lstlisting}
\vspace{0.8em}
\textcolor{red}{Note:} Instead of constant value to repeat the loop, in Java \textcolor{blue}{‘length’} keyword can be used with array to repeat from ‘0th’ index to ’size-1’ location.
\end{frame}

\begin{frame}{Problem Statement}
Write a program to count even numbers in a given array.
\end{frame}

\begin{frame}{Solution}
\begin{lstlisting}
int num[] = \{45, 78, 89, 34, 58, 56, 23, 36\};\\
int count = 0;\\
for (int i = 0; i $<$ num.length; i++) \{\\
\tab if( num [i] \% 2 == 0)\\
\tab count++;\\
\}\\
S.O.P(count);\\
\end{lstlisting}
\vspace{0.8em}
\textbf{Write code, compile and execute the program}
\end{frame}

\begin{frame}{With other datatypes}
\textbf{Array can be implemented similarly with other datatypes
also as follows.}\\
\vspace{0.8em}
\begin{lstlisting}
float age[] = \{34.5, 35.78, 89.4, 23.5\};\\
char grades[] = \{’A’, ’B’, ’C’, ’D’\};\\
String names[] = \{"James", "Warner", "Turner"\};\\
\end{lstlisting}
\end{frame}

\begin{frame}{Summary}
\begin{itemize}
    \item Understood purpose of array in program.
\item Declaration and Defining of single dimensional array.
\item Understood Purpose of using \textcolor{blue}{length} with array.
\item Understood arrays with other datatypes as well.
\end{itemize}
\end{frame}
\end{document}