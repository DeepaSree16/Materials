\documentclass[aspectratio=169,14pt,usenames,dvipsnames]{beamer}
\usetheme{TalentSprint}
\usepackage[utf8]{inputenc}
\usepackage{graphics}
\usepackage{ragged2e}
\usepackage{amsfonts}
\usepackage{xcolor}
\usepackage{mathtools}
\usepackage{tcolorbox}
\usepackage{setspace}
\usepackage{lmodern}
\usepackage{graphicx}
\definecolor{swe}{rgb}{0.19, 0.73, 0.56}
\definecolor{lblue}{RGB}{190,200,198}
\title[Reading and Writing Data]{Reading and Writing Data}

\newcommand\tab[1][1cm]{\hspace*{#1}}

\begin{document}
{\1
\begin{frame} \vspace{35pt}

\subtitle{TalentSprint}
\maketitle
\end{frame}
}

\begin{frame}{Learning Objectives}
At the end of this video you should able to:
\begin{itemize}
\item Read contents of a file.
\item Write into a file.
\end{itemize}
\end{frame}


\begin{frame}{IO Hierarchy}
\begin{center}
\resizebox{0.4\textwidth}{!}{
\includegraphics{Images/rea_and_writing.png}
}
\end{center}
\end{frame}

\begin{frame}{To read(Binary)}
\textbf{FileInputStream fis = new FileInputStream(String);}\\
\vspace{1em}\\
\textbf{Note : String represents the address of a file.}\\
\tab \tab \tab \tab \textcolor{blue}{(or)}\\
\textbf{FileInputStream fis = new FileInputStream(File);}\\
\vspace{1em}
\textbf{Note: File represents the file object.}\\
\end{frame}


\begin{frame}{To read(Binary) ..}
\textbf{int n = fis.read();}\\
\textbf{Reads a character and returns an ASCII value.}\\
\vspace{1em}\\
\textbf{fis.close();}\\
\textbf{Close the opened resource.}\\
\vspace{1em}\\
\textbf{Note: Both the above methods throws}\\
\textbf{IOException.}\\
\end{frame}


\begin{frame}{To write(Binary)}
\textbf{FileOutputStream fos = new FileOutputStream(String);}\\
\tab \tab \tab \tab \textcolor{blue}{(or)}\\
\textbf{FileOutputStream fos = new FileOutputStream(File);}
\tab \tab \tab \tab \textcolor{blue}{(or)}\\
\textbf{FileOutputStream fos = new}\\
\textbf{FileOutputStream(String,boolean);}\\
\vspace{1em}\\
\textbf{Note: boolean true append content to the file.}
\end{frame}


\begin{frame}{To write(Binary) . . .}
\textbf{fos.write(n);}\\
\textbf{Write the specified byte to the file and ’n’}\\
\textbf{represents byte.}\\
\vspace{1em}
\textbf{fos.close();}\\
\textbf{Close the opened resource.}\\
\vspace{1em}
\textbf{Note: Both the above methods throws}\\
\textbf{IOException.}\\
\end{frame}  


\begin{frame}{To read(Character)}
\textbf{FileReader fr = new FileReader(file);}\\
\textbf{Note: file represents object of type File.}\\
\end{frame}

\begin{frame}{To read(Character)}
\textbf{FileReader fr = new FileReader(file);}\\
\textbf{Note: file represents object of type File.}\\
\vspace{1em}
\textbf{BufferedReader br = new BufferedReader(fr));}\\
\textbf{Note: One can read char by char or line by line.}\\
\end{frame}

\begin{frame}{To read(Character)}
\textbf{FileReader fr = new FileReader(file);}\\
\textbf{Note: file represents object of type File.}\\
\vspace{1em}
\textbf{BufferedReader br = new BufferedReader(fr));}\\
\textbf{Note: One can read char by char or line by line.}\\
\vspace{1em}
\textbf{int n = br.read();}\\
\textbf{Note: Read the character and returns the ASCII value.}\\
\end{frame}

\begin{frame}{To read(Character)}
\textbf{String s = br.readLine();}\\
\textbf{Note: Reads the entire line and retuns the string.}\\
\vspace{1em}
\textbf{Note: Both the read methods throws IOException.}\\
\end{frame}

\begin{frame}{To write(Character)}
\textbf{FileWriter fw = new FileWriter(file);}\\
\textbf{Note: file represents object of type File.}\\
\end{frame}

\begin{frame}{To write(Character)}
\textbf{FileWriter fw = new FileWriter(file);}\\
\textbf{Note: file represents object of type File.}\\
\vspace{1em}
\textbf{BufferedWriter bw = new BufferedWriter(fw);}\\
\textbf{Note: One can write byte or string to a file.}\\
\end{frame}

\begin{frame}{To write(Character)}
\textbf{FileWriter fw = new FileWriter(file);}\\
\textbf{Note: file represents object of type File.}\\
\vspace{1em}
\textbf{BufferedWriter bw = new BufferedWriter(fw);}\\
\textbf{Note: One can write byte or string to a file.}\\
\vspace{1em}
\textbf{bw.write(String);}\\
\textbf{Note: Write the string to a file.}\\
\end{frame}

\begin{frame}{To write(Character)}
\textbf{FileWriter fw = new FileWriter(file);}\\
\textbf{Note: file represents object of type File.}\\
\vspace{1em}
\textbf{BufferedWriter bw = new BufferedWriter(fw);}\\
\textbf{Note: One can write byte or string to a file.}\\
\vspace{1em}
\textbf{bw.write(String);}\\
\textbf{Note: Write the string to a file.}\\
\vspace{1em}
\textbf{Note: Write method throws IOException.}\\
\end{frame}


\begin{frame}{To Read and Write Objects}
\textbf{To write and read objects of a class from a file,
the respective class must implement interface
Serializable.}
\end{frame}

\begin{frame}{To read(Object)}
\textbf{FileInputStream fis = new FileInputStream(String)};\\
\textbf{Note: String represents the File object.}\\
\end{frame}

\begin{frame}{To read(Object)}
\textbf{FileInputStream fis = new FileInputStream(String)};\\
\textbf{Note: String represents the File object.}\\
\vspace{1em}
\textbf{ObjectInputStream ois = new ObjectInputStream(fis);}
\end{frame}

\begin{frame}{To read(Object)}
\textbf{FileInputStream fis = new FileInputStream(String)};\\
\textbf{Note: String represents the File object.}\\
\vspace{1em}
\textbf{ObjectInputStream ois = new ObjectInputStream(fis);}\\
\vspace{1em}
\textbf{ClassName obj = (ClassName)ois.readObject();}\\
\textbf{Note: Reads object from file and returns it.}\\
\end{frame}

\begin{frame}{To read(Object)}
\textbf{FileInputStream fis = new FileInputStream(String)};\\
\textbf{Note: String represents the File object.}\\
\vspace{1em}
\textbf{ObjectInputStream ois = new ObjectInputStream(fis);}\\
\vspace{1em}
\textbf{ClassName obj = (ClassName)ois.readObject();}\\
\textbf{Note: Reads object from file and returns it.}\\
\vspace{1em}
\textbf{ois.close();}\\
\textbf{Note: To close the resource.}\\
\end{frame}

\begin{frame}{To write(Object)}
\textbf{FileOutputStream fos = new FileOutputStream(String);}\\
\textbf{Note: String represents the File object.}\\
\vspace{1em}
\textbf{ObjectOutputStream oos = new}\\
\textbf{ObjectOutputStream(fos);}\\
\end{frame}

\begin{frame}{To write(Object)}
\textbf{FileOutputStream fos = new FileOutputStream(String);}\\
\textbf{Note: String represents the File object.}\\
\vspace{1em}
\textbf{ObjectOutputStream oos = new}\\
\textbf{ObjectOutputStream(fos);}\\
\vspace{1em}
\textbf{oos.writeObject(object);}\\
\textbf{Note: Object is of type Serializable.}\\
\end{frame}

\begin{frame}{To write(Object)}
\textbf{FileOutputStream fos = new FileOutputStream(String);}\\
\textbf{Note: String represents the File object.}\\
\vspace{1em}
\textbf{ObjectOutputStream oos = new}\\
\textbf{ObjectOutputStream(fos);}\\
\vspace{1em}
\textbf{oos.writeObject(object);}\\
\textbf{Note: Object is of type Serializable.}\\
\vspace{1em}
    \textbf{ois.close();}\\
\textbf{Note: To close the resource.}\\
\end{frame}
\end{document}