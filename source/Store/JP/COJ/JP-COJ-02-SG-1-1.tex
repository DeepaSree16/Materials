\documentclass[aspectratio=169,14pt,usenames,dvipsnames]{beamer}
\usetheme{TalentSprint}
\usepackage[utf8]{inputenc}
\usepackage{graphics}
\usepackage{ragged2e}
\usepackage{amsfonts}
\usepackage{xcolor}
\usepackage{mathtools}
\usepackage{tcolorbox}
\usepackage{setspace}
\usepackage{lmodern}
\usepackage{blindtext}
\usepackage{lipsum}
%\usepackage[a4paper, total={10in, 12in}]{geometry}

\definecolor{swe}{rgb}{0.19, 0.73, 0.56}
\definecolor{lblue}{RGB}{190,200,198}
\title[Generalization and Specialization of Objects]{Generalization and
Specialization of Objects}

\newcommand\tab[1][1cm]{\hspace*{#1}}
%\newcommand\Fontvi{\fontsize}{8}{8.2}\selectfont
\begin{document}
{\1
\begin{frame} \vspace{35pt}

\subtitle{TalentSprint}
\maketitle
\end{frame}
}

\begin{frame}{Learning Objectives}
At the end of this video you should able to:
\begin{itemize}
\item Identify relationship between objects.
\end{itemize}
\end{frame}

\begin{frame}{Inheritance}
\begin{lstlisting}
class Player \{\\
\tab private String name;\\
\tab public String getName() \{\\
\tab \tab return name;\\
\tab \}\\
\tab public void setName(String name) \{\\
\tab \tab this.name = name;\\
\tab \}\\
\}\\
\end{lstlisting}
\end{frame}

\begin{frame}{Inheritance}
\begin{lstlisting}
class Batsman extends Player \{\\
\tab private int runsScored;\\
\tab private int ballsFaced;\\
\tab + Getter and Setter Methods\\
\tab public float calcStrikeRate() \{\\
\tab \tab String outputString = new DecimalFormat("\#.\#\#")\\
\tab \tab .format\\
\tab \tab \tab (((float)runsScored/(float)ballsFaced)* 100);\\
\tab \tab return Float.parseFloat(outputString);\\
\tab \}\\
\}\\
\end{lstlisting}
\end{frame}

\begin{frame}{Inheritance}
\begin{lstlisting}
class Bowler extends Player \{\\
\tab private int ballsBowled;\\
\tab private int wickets;\\
\tab + Getter and Setter Methods\\
\tab public float calcStrikeRate() \{\\
\tab \tab String outputString = new DecimalFormat("\#.\#\#")\\
\tab \tab .format\\
\tab \tab \tab (((float)ballsBowled/(float)wickets)* 100);\\
\tab \tab return Float.parseFloat(outputString);\\
\tab \}\\
\}\\
\end{lstlisting}
\end{frame}

\begin{frame}{Inheritance}
\begin{lstlisting}
\textbf{Batsman batsman = new Batsman();}\\
\textbf{batsman.setName("Virat Kohli");}\\
\textbf{batsman.setRunsScored(79);}\\
\textbf{batsman.setBallsFaced(47);}\\
\vspace{0.5em}
System.out.println(batsman.getName()); --$>$ Virat Kohli\\
System.out.println(batsman.getRunsScored()); --$>$ 79 \\
System.out.println(batsman.getBallsFaced()); --$>$ 47 \\
System.out.println(batsman.calcStrikeRate()); --$>$ 168.09 \\
\end{lstlisting}
\end{frame}


\begin{frame}{Inheritance}
\begin{lstlisting}
\textbf{Bowler bowler = new Bowler();}\\
\textbf{bowler.setName("R Ashwin");}\\
\textbf{bowler.setBallsBowled(60);}\\
\textbf{bowler.setWickets(2);}\\
\vspace{0.5em}
System.out.println(bowler.getName());--$>$ R Ashwin\\
System.out.println(bowler.getBallsBowled());--$>$60\\
System.out.println(bowler.getWickets()); --$>$ 2\\
System.out.println(bowler.calcStrikeRate()); --$>$ 30.00\\
\end{lstlisting}
\end{frame}



\begin{frame}{super}
\begin{lstlisting}
class Player \{\\
\tab private String name;\\
\tab Player(String name) \{\\
\tab \tab this.name = name;\\
\tab \}\\
\tab public String getName() \{\\
\tab \tab return name;\\
\tab \}\\
\}\\
\end{lstlisting}
\end{frame}


\begin{frame}{super}
\begin{lstlisting}
class Batsman extends Player \{\\
\tab private int runsScored;\\
\tab private int ballsFaced;\\
\tab public Batsman(String name, int runsScored, int ballsFaced) \{\\
\tab \tab super(name);\\
\tab \tab this.runsScored = runsScored;\\
\tab \tab this.ballsFaced = ballsFaced;\\
\tab \}\\
\end{lstlisting}
\end{frame}


\begin{frame}{super}
\begin{lstlisting}
\tab + Getter Methods\\
\tab public float calcStrikeRate() \{\\
\tab \tab String outputString = new DecimalFormat("\#.\#\#")\\
\tab \tab .format\\
\tab \tab \tab (((float)runsScored/(float)ballsFaced)* 100);\\
\tab \tab return Float.parseFloat(outputString);\\
\tab \}\\
\}\\
\end{lstlisting}
\end{frame}


\begin{frame}{super}
\begin{lstlisting}
class Bowler extends Player \{\\
\tab private int ballsBowled;\\
\tab private int wickets;\\
\tab public Bowler(String name, int ballsBowled, int wickets) \{\\
\tab \tab super(name);\\
\tab \tab this.ballsBowled = ballsBowled;\\
\tab \tab this.wickets = wickets;\\
\tab \}\\
\tab + Getter Methods\\
\end{lstlisting}
\end{frame}



\begin{frame}{super}
\begin{lstlisting}
\tab public float calcStrikeRate() \{\\
\tab \tab String outputString = new DecimalFormat("\#.\#\#")\\
\tab \tab .format\\
\tab \tab \tab (((float)ballsBowled/(float)wickets)* 100);\\
\tab \tab return Float.parseFloat(outputString);\\
\tab \}\\
\}\\
\end{lstlisting}
\end{frame}

\begin{frame}{super}
\begin{lstlisting}
\textbf{Batsman batsman = new Batsman("Virat Kohli", 79, 47);}\\
\vspace{1em}
System.out.println(batsman.getName()); --$>$ Virat Kohli\\
System.out.println(batsman.getRunsScorec()); --$>$ 79\\
System.out.println(batsman.getBallsFaced()); --$>$ 47\\
System.out.println(batsman.calcStrikeRate()); --$>$ 168.09\\
\end{lstlisting}
\end{frame}

\begin{frame}{super}
\begin{lstlisting}
\textbf{Bowler bowler = new Bowler("R Ashwin", 60, 2);}\\
\vspace{1em}
System.out.println(bowler.getName()); --$>$ R Ashwin\\
System.out.println(bowler.getBallsBowled();--$>$ 60\\
System.out.println(bowler.getWickets()); --$>$ 2\\
System.out.println(bowler.calcStrikeRate()); --$>$ 30.00\\
\end{lstlisting}
\end{frame}

\begin{frame}{Abstract}
\begin{lstlisting}
abstract class Player \{\\
\tab private String name;\\
\tab public Player(String name) \{\\
\tab \tab this.name = name;\\
\tab \}\\
\tab public String getName() \{\\
\tab \tab return name;\\
\tab \}\\
\tab abstract public float calcStrikeRate();\\
\}\\
\end{lstlisting}
\end{frame}

\begin{frame}{Abstract}
\begin{lstlisting}
class Batsman extends Player \{\\
\tab private int runsScored;\\
\tab private int ballsFaced;\\
\tab public Batsman(String name, int runsScored, int ballsFaced) \{\\
\tab \tab super(name);\\
\tab \tab this.runsScored = runsScored;\\
\tab \tab this.ballsFaced = ballsFaced;\\
\}\\
\end{lstlisting}
\end{frame}

\begin{frame}{Abstract}
\begin{lstlisting}
\tab + Getter Methods\\
\tab public float calcStrikeRate() \{\\
\tab \tab String outputString = new DecimalFormat("\#.\#\#")\\
\tab \tab .format\\
\tab \tab \tab (((float)runsScored/(float)ballsFaced)* 100);\\
\tab \tab return Float.parseFloat(outputString);\\
\tab \}\\
\}\\
\end{lstlisting}
\end{frame}

\begin{frame}{Abstract}
\begin{lstlisting}
class Bowler extends Player \{\\
\tab private int ballsBowled;\\
\tab private int wickets;\\
\tab public Bowler(String name, int ballsBowled, int wickets) \{\\
\tab \tab super(name);
\tab \tab this.ballsBowled = ballsBowled;\\
\tab \tab this.wickets = wickets;\\
\tab \}\\
\end{lstlisting}
\end{frame}

\begin{frame}{Abstract}
\begin{lstlisting}
\tab + Getter Methods\\
\tab public float calcStrikeRate() \{\\
\tab \tab String outputString = new DecimalFormat("\#.\#\#")\\
\tab \tab .format\\
\tab \tab \tab (((float)ballsBowled/(float)wickets)* 100);\\
\tab \tab return Float.parseFloat(outputString);\\
\tab \}\\
\}\\
\end{lstlisting}
\end{frame}

\begin{frame}{Abstract}
\begin{lstlisting}
\textbf{Batsman batsman = new Batsman("Virat Kohli", 79, 47);}\\
\vspace{1em}
System.out.println(batsman.getName()); --$>$ Virat Kohli\\
System.out.println(batsman.getRunsScored()); --$>$ 79\\
System.out.println(batsman.getBallsFaced()); --$>$ 47\\
System.out.println(batsman.calcStrikeRate()); --$>$ 169.08\\
\end{lstlisting}
\end{frame}

\begin{frame}{Abstract}
\begin{lstlisting}
\textbf{Bowler bowler = new Bowler("R Ashwin", 60, 2);}\\
\vspace{1em}
System.out.println(bowler.getName()); --$>$ R Ashwin\\
System.out.println(bowler.getBallsBowled();--$>$ 60\\
System.out.println(bowler.getWickets()); --$>$ 2\\
System.out.println(bowler.calcStrikeRate()); --$>$ 30.00\\
\end{lstlisting}
\end{frame}

\begin{frame}{Abstract}
\begin{lstlisting}
public interface Shape \{\\
\tab float area();\\
\}\\
\vspace{1em}
\textbf{Note: By default the methods in the interface are public
and abstract}
\end{lstlisting}
\end{frame}

\begin{frame}{Abstract}
\begin{lstlisting}
class Square implements Shape\{\\
\tab private int length;\\
\tab private int breadth;\\
\tab public Square(int length, int breadth) \{\\
\tab \tab this.length = length;\\
\tab \tab this.breadth = breadth;\\
\tab \}\\
\tab public float area() \{\\
\tab \tab return(length * breadth);\\
\tab \}\\
\}\\
\end{lstlisting}
\end{frame}

\begin{frame}{Abstract}
\begin{lstlisting}
\textbf{Shape square = new Square(8, 12);}\\
\vspace{1em}
System.out.println(square.area());--$>$96.00
\end{lstlisting}
\end{frame}
\end{document}
