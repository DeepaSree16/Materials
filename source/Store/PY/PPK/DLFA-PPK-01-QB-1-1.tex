%\tolerance=10000
%\documentclass[prl,twocoloumn,preprintnumbers,amssymb,pla]{revtex4}
\documentclass[prl,twocolumn,showpacs,preprintnumbers,superscriptaddress]{revtex4}
\documentclass{article}
\usepackage{graphicx}
\usepackage{color}
\usepackage{dcolumn}
%\linespread{1.7}
\usepackage{bm}
%\usepackage{eps2pdf}
\usepackage{graphics}
\usepackage{pdfpages}
\usepackage{caption}
%\usepackage{subcaption}
\usepackage[demo]{graphicx} % omit 'demo' for real document
%\usepackage{times}
\usepackage{multirow}
\usepackage{hhline}
\usepackage{subfig}
\usepackage{amsbsy}
\usepackage{amsmath}
\usepackage{amsfonts}
\usepackage{amsthm}
\usepackage{float}
\documentclass{article}
\usepackage{amsmath,systeme}

\sysalign{r,r}

% \textheight = 8.5 in
% \topmargin = 0.3 in

%\textwidth = 6.5 in
% \textheight = 8.5 in
%\oddsidemargin = 0.0 in
%\evensidemargin = 0.0 in

%\headheight = 0.0 in
%\headsep = 0.0 in
%\parskip = 0.2in
%\parindent = 0.0in

% \newcommand{\ket}[1]{\left|#1\right\rangle}
% \newcommand{\bra}[1]{\left\langle#1\right|}
\newcommand{\ket}[1]{| #1 \rangle}
\newcommand{\bra}[1]{\langle #1 |}
\newcommand{\braket}[2]{\langle #1 | #2 \rangle}
\newcommand{\ketbra}[2]{| #1 \rangle \langle #2 |}
\newcommand{\proj}[1]{| #1 \rangle \langle #1 |}
\newcommand{\al}{\alpha}
\newcommand{\be}{\beta}
\newcommand{\op}[1]{ \hat{\sigma}_{#1} }
\def\tred{\textcolor{red}}
\def\tgre{\textcolor{green}}


\theoremstyle{plain}
\newtheorem{theorem}{Theorem}

\newtheorem{lemma}[theorem]{Lemma}
\newtheorem{corollary}[theorem]{Corollary}
\newtheorem{proposition}[theorem]{Proposition}
\newtheorem{conjecture}[theorem]{Conjecture}

\theoremstyle{definition}
\newtheorem{definition}[theorem]{Definition}


\begin{document}
\begin{widetext}
\\
\\
\\

\begin{wrapfigure}
\centering
%\includegraphics[\textwidth]{TS_IISc.png}
\end{wrapfigure}
\begin{figure}[h!]
 \begin{right}
  \hfill\includegraphics[\textwidth, right]{TS_IISc.png}
 \end{right}
\end{figure}
\\
\\
\\
\noindent\textbf{1. The stages in a Visualization Process are:}
\\
\\
\\
\noindent A. Display hardware and software
\\
\\
\\
B. Collection and storage of data
\\
\\
\\
C. Preprocessing for transforming data into an understandable form
\\
\\
\\
D. All the above
\\
\\
\\
\textbf{Answer: D}
\\
\\
\textbf{Solution: The four stages in a Visualization Process are:}
\\
\\
1. Collection and storage of data
\\
2. Preprocessing for transforming data into an understandable form
\\
3. Display hardware and software
\\
4. Human perceptual and cognitive 
system.
\\
\\
\\
\\
\textbf{2. When is the need to visualize data?}
\\
\\
\\
\noindent A. To augment human capabilities\\
\\
\\
B. To replace human capabilities\\
\\
\\
C. When humans need details, but are unclear what questions to ask\\
\\
\\
D. Both A and C
\\
\\
\\
\textbf{Answer: D}
\\
\\
\textbf{Solution:} The need to visualize data arises when:
\\
\\
1. There is a need to augment (not replace) human capabilities
\\
2. No automatic solution exists, and humans needs details but unclear what questions to ask
\\
\\
\\
\\
\\
\textbf{3. Which of the following Data Visualization Technique can be used for Effective Visualization of Multi-dimensional data?}
\\
\\
\\
\\
\noindent A. Reingold-Tilford Layout\\
\\
\\
B. Scatterplot Matrix\\
\\
\\
C. Pie Chart\\
\\
\\
D. Hyperbolic Trees
\\
\\
\\
\textbf{Answer: B}
\\
\\
\textbf{Solution:}
Scatterplot Matrix is a Data Visualization Technique can be used for Effective Visualization of Multi-dimensional data.
\\
\\
\\
\\
\textbf{4. Which of the following is intended for the visualization of hierarchical data in the form of nested rectangles?}  
\\
\\
\\
A. Bar Chart
\\
\\
\\
B. Parallel Coordinates
\\
\\
\\
C. Treemap Chart
\\
\\
\\
D. None of the above
\\
\\
\\
\textbf{Answer: C}
\\
\\
\textbf{Solution:}
Treemap Chart is intended for the visualization of hierarchical data in the form of nested rectangles. . 
\\
\\
\\
\\
\newpage
\noindent\textbf{5. What are the successful application of Treemaps?}
\\
\\
\\
\noindent A. Use visual properties properly
\\
\\
\\
B.  Provide excellent interactivity
\\
\\
\\
C. Make appearance more usable
\\
\\
\\
D. All the above
\\
\\
\\
\textbf{Answer: D}
\\
\\
\textbf{Solution:}\ 
The following are the successful applications of Treemaps:
\\
\\
1. Make appearance more usable
\\
2. Use visual properties properly
\\
3. When exact numbers aren't very important
\\
4. Provide excellent interactivity 
\\
\\
\\
\end{widetext}
\end{document}