\documentclass[14pt]{beamer}
\title{Python 101 :: Unit 1}
\subtitle{Session 1}
\date{}
\author[TS]{TalentSprint}
\usefonttheme{serif}
\usepackage{bookman}
\usepackage{hyperref}
\usepackage[T1]{fontenc}
\usepackage{graphicx}
\usecolortheme{orchid}
\beamertemplateballitem

\usepackage{listings}
%\usebackgroundtemplate{\includegraphics[width=\paperwidth]{logo}}
\definecolor{mygreen}{rgb}{0,0.6,0}
\definecolor{mygray}{rgb}{0.5,0.5,0.5}
\definecolor{mymauve}{rgb}{0.58,0,0.82}

\lstset{ %
  backgroundcolor=\color{white},   % choose the background color
  basicstyle=\small,        % size of fonts used for the code
  breaklines=true,                 % automatic line breaking only at whitespace
  captionpos=b,                    % sets the caption-position to bottom
  commentstyle=\color{mygreen},    % comment style
  %escapeinside={\%*}{*)},          % if you want to add LaTeX within your code
  keywordstyle=\color{blue},       % keyword style
  stringstyle=\color{mymauve},     % string literal style
  showstringspaces=false,
}
   
\begin{document}
    \begin{frame}
        \titlepage
    \end{frame}

    \begin{frame}
        \frametitle{Topics for this Session}
        \begin{itemize}
            \item Histroy
            \item What is Python?
            \item Features
            \item Language Characteristic
            \item Applications
        \end{itemize}
    \end{frame}
    
    \begin{frame}
        \frametitle{History of Python}
        \begin{itemize}
            \item Developed in late 1980s by Guido van Rossum
            \item First release at the National Research Institute for Mathematics and Computer Science in the Netherlands
            \item Derived from ABC, C, Unix shell,etc
        \end{itemize}
    \end{frame}
    
    \begin{frame}
        \frametitle{History of Python}
        \begin{itemize}
            \item The Python Software Foundation, a non-profit organization devoted to language since 2001.
            \item Python 2.x - with many new features
            \item Python 3.x - with major features and backward incompatible
        \end{itemize}
    \end{frame}
    \begin{frame}
        \frametitle{What is Python?}
        \begin{itemize}
            \item High-level interpreted language
            \item Emphasizes code readability
            \item Supports various programming paradigms including:
                \begin{itemize}
                    \item functional
                    \item procedural
                    \item object oriented styles
                \end{itemize}
        \end{itemize}
    \end{frame}
    \begin{frame}
        \frametitle{Notable Features}
        \begin{itemize}
            \item Elegant and simple syntax
            \item Easier to read
            \item Free and Open Source
            \item Extensible
            \item Embeddable
        \end{itemize}
    \end{frame}
    \begin{frame}
        \frametitle{Notable Features}
        \begin{itemize}

            \item Large standard library - batteriesincluded
            \item Portable - Runs on many operating systems including, Linux, Unix, MacOs and Windows
            \item Excellent Unicode support
        \end{itemize}
    \end{frame}
    \begin{frame}
        \frametitle{Language Characteristic}
        \begin{itemize}
            \item Higher level containers like list are part of core language
            \item Dynamically typed
            \item Supports OOPs with classes and multiple inheritance
            \item Automatic memory management system
            \item Good Exception handling system
        \end{itemize}
    \end{frame}
    \begin{frame}
        \frametitle{Applications}
        \begin{itemize}
            \item Python - Widely used professional language
            \item Used by organizations such as
                \begin{itemize}
                    \item Google
                    \item NASA
                    \item Industrial Light and Magic
                    \item Cerenova
                    \item ABN Amro Bank, etc.
                \end{itemize}
        \end{itemize}
    \end{frame}
    \begin{frame}
        \frametitle{Applications}
        \begin{itemize}
            \item Important tools, frameworks and application written in Python
            \begin{itemize}
                \item Web development
                \item Scientific Computations
                \item Education
                \item GUI Toolkits
            \end{itemize}

        \end{itemize}
    \end{frame}
    \begin{frame}
        \frametitle{Applications}
        \begin{itemize}
            \item Many choices for Web Development
                \begin{itemize}
                    \item Full-stack frameworks such as Django, Pyramid and Zope
                    \item Micro-frameworks such as Flask and Bottle
                    \item Advanced Content Management System such as Plone
                \end{itemize}
        \end{itemize}
    \end{frame}
    \begin{frame}
        \frametitle{Applications}
        \begin{itemize}
            \item  Python’s standard Library supports many internet protocols
                \begin{itemize}
                    \item HTML, XML, JSON
                    \item Email processing
                    \item FTP, IMAP, sockets, etc.
                \end{itemize}
            \item Requests - a powerful HTTP client library
            \item BeautifulSoup - HTML parser
            \item Feedparser - RSS/Atom feeds parser
            \item Twisted Python - for asynchronous network programming
        \end{itemize}
    \end{frame}
    \begin{frame}
        \frametitle{Applications}
        \begin{itemize}
            \item Scientific Computation - Python the de-facto standard toolkit for scientific and engineering computing
                \begin{itemize}
                    \item Scipy - collection of packages for mathematics, science and engineering
                    \item Pandas - for data analysis and modeling
                    \item IPython - for interactive computation and exploration platform
                \end{itemize}
        \end{itemize}
    \end{frame}
    \begin{frame}
        \frametitle{Applications}
        \begin{itemize}
            \item GUI Toolkits - collection of wide variety of graphical interface
                \begin{itemize}
                    \item Tk GUI - included with basic installation
                    \item wxWidgets - powerful multiplatform toolkit
                    \item Kivy - for multi-touch applications
                    \item Qt - used via pyqt or pyside
                \end{itemize}
        \end{itemize}
    \end{frame}
\end{document}

