\documentclass[14pt]{beamer}
\title{Python 101 :: Unit 3}
\subtitle{Session 1}
\date{}
\author[TS]{TalentSprint}
\usefonttheme{serif}
\usepackage{bookman}
\usepackage{hyperref}
\usepackage[T1]{fontenc}
\usepackage{graphicx}
\usecolortheme{orchid}
\beamertemplateballitem

\usepackage{listings}

\definecolor{mygreen}{rgb}{0,0.6,0}
\definecolor{mygray}{rgb}{0.5,0.5,0.5}
\definecolor{mymauve}{rgb}{0.58,0,0.82}

\lstset{ %
  backgroundcolor=\color{white},   % choose the background color
  basicstyle=\small,        % size of fonts used for the code
  breaklines=true,                 % automatic line breaking only at whitespace
  captionpos=b,                    % sets the caption-position to bottom
  commentstyle=\color{mygreen},    % comment style
  %escapeinside={\%*}{*)},          % if you want to add LaTeX within your code
  keywordstyle=\color{blue},       % keyword style
  stringstyle=\color{mymauve},     % string literal style
  showstringspaces=false,
}
   
\begin{document}
    \begin{frame}
        \titlepage
    \end{frame}
    \begin{frame}
        \frametitle{Topics for the Session}
        \begin{itemize}
            \item While Loop
            \item Strings
            \item Code Reading
        \end{itemize}
    \end{frame}
    \begin{frame}[containsverbatim]
        \frametitle{While Loop}
        \begin{itemize}
            \item While loops are known as indefinite or conditional loops.
            \item They will keep reiterating a block of code defined inside it until the desired condition is met.
        \end{itemize}
	\begin{block}{}
        \begin{lstlisting}[language=Python]
            while <Expression>:
                statements
        \end{lstlisting}
	\end{block}
    \end{frame}
    \begin{frame}[containsverbatim]
        \frametitle{Problem 1}
        \begin{lstlisting}[language=Python]
            i = 1
            while i <= 10:
                print(i)
                i = i + 1
        \end{lstlisting}
    \end{frame}
    \begin{frame}
        \frametitle{Problem 1 :: Solution}
        Prints first 10 natural numbers
    \end{frame}
    \begin{frame}[containsverbatim]
        \frametitle{Problem 2}
        \begin{lstlisting}[language=Python]
            a, b = 0, 1
            print(a)
            print(b)
            while b < 10:
                a, b = b, a + b
                print(b)
        \end{lstlisting}
    \end{frame}
    \begin{frame}
        \frametitle{Problem 2 :: Solution}
        Prints a fibonacci series
    \end{frame}

    \begin{frame}[containsverbatim]
        \frametitle{Problem 3}
        \begin{lstlisting}[language=Python]
            num = 10
            count = 0
            while num != 0:
                count = count + 1
                num = num // 10
            print(count)
        \end{lstlisting}
    \end{frame}

    \begin{frame}
        \frametitle{Problem 3 :: Solution}
        Program to count total digits in a given integer / number
    \end{frame}
    \begin{frame}[containsverbatim]
        \frametitle{Problem 4}
        \begin{lstlisting}[language=Python]
            number = 1
            limit = 10
            while number <= limit :
                if number % 2 == 0:
                    print (number)
            number = number + 1
        \end{lstlisting}
    \end{frame}
    \begin{frame}
        \frametitle{Problem 4 :: Solution}
        Program to print even numbers from 1 to the given limit
    \end{frame}

    \begin{frame}[containsverbatim]
        \frametitle{Problem 5}
        \begin{lstlisting}[language=Python]
            i = 1
            while i <= 4:
                print ("*" * i)
                i = i + 1
        \end{lstlisting}
    \end{frame}
    \begin{frame}
        \frametitle{Problem 5 :: Solution}
        Prints the following pattern

        * 

        **

        ***

        ****
    \end{frame}
    \begin{frame}[containsverbatim]
        \frametitle{Problem 6}
        \begin{lstlisting}[language=Python]
            i = 1
            while i <= 10:
                print (24 * i)
                i = i + 1
        \end{lstlisting}
    \end{frame}
    \begin{frame}
        \frametitle{Problem 6 :: Solution}
        Print multiplication table of 24
    \end{frame}
    \begin{frame}[containsverbatim]
        \frametitle{What is a String?}
        \begin{itemize}
            \item A sequence of characters represented in single or double quotes
            \item Strings are Immutable
            \item \alert{Example:}  
        \end{itemize}
	    \begin{lstlisting}[language=Python]
            greet = "Hello, Everyone" 
            greet = 'Hello, Everyone'
        \end{lstlisting}
    \end{frame}
    \begin{frame}
        \frametitle{Why strings are made immutable in Python?}
        \begin{description}
            \item [Performance] Takes less time to allocate the memory for Immutable objects, since their memory size is fixed
            \item [Security] Any attempt to modify the string will lead to the creation of the new object in memory and hence ID changes will be tracked easily.
        \end{description}
    \end{frame}
    \begin{frame}[containsverbatim]
        \frametitle{Length of a String}
        \alert{Example:}  
        \begin{lstlisting}[language=Python]
            greet = 'Hello, Everyone'
            print(len(greet))
        \end{lstlisting}
    \end{frame}
    \begin{frame}[containsverbatim]
        \frametitle{Triple quoted Strings}
        \begin{itemize}
            \item In somecases, when you need to include really long string using triple quoted string is useful.
            \item \alert{Example:}
        \end{itemize}
        \begin{lstlisting}[language=Python]
            message = ''' Example for triple quote '''
        \end{lstlisting}
        \textbf{Note:} Triple quoted strings are also used as Docstrings.
    \end{frame}
    \begin{frame}[containsverbatim]
        \frametitle{Access a Character from a String}
        \textbf{Approach 1} Using While Loop
        
        \alert{Example:} 
        \begin{lstlisting}[language=Python]
            greet = 'Hello, Everyone'
            lengthofstring = len(greet)
            i = 0
            while i < lengthofstring:
                print(greet[i], end=" ")
                i = i + 1
        \end{lstlisting}
    \end{frame}
    \begin{frame}[containsverbatim]
        \frametitle{Access a Character from a String}
        \textbf{Approach 2} Using Slicing
        \begin{itemize}
            \item Both positive and negative indexing is possible in Python
            \item \alert{Example:}
        \end{itemize}
        \begin{lstlisting}[language=Python]
            greet = 'Hello, Everyone'
            # Prints first character from a string
            print(greet[0]) 
            # Prints last character from a string
            print(greet[-1]) 
        \end{lstlisting}
    \end{frame}
    \begin{frame}[containsverbatim]
        \frametitle{Slicing the Strings}
        \begin{lstlisting}[language=Python]
            word = 'Python Course'
            # Prints the word
            word[: :]
            # Access the string from 0th to 8th Element
            word[0: :9 1]
            # Access the entire string in the step size of 2
            word[: : 2]
            # Access the character from string -6 index to 
            # the end of the string
            word[-6: :]
        \end{lstlisting}
    \end{frame}
    \begin{frame}[containsverbatim]
        \frametitle{Concatenation of Strings}
        \begin{itemize}
            \item \emph{+} is used as a concatenation operator
            \item \alert{Example:}
                
        \end{itemize}
	    \begin{lstlisting}[language=Python]
            course = 'Python'
            print(course + 'Programming')
        \end{lstlisting}
    \end{frame}
    \begin{frame}[containsverbatim]
        \frametitle{Repeating the Strings}
        \begin{itemize}
            \item \emph{*} is used a repetition operater
            \item \alert{Example:}
               
        \end{itemize}
        \begin{lstlisting}[language=Python]
            course = 'Python'
            print(course * 2)
        \end{lstlisting}
    \end{frame}
    \begin{frame}[containsverbatim]
        \frametitle{Removing Spaces from a String}
        \begin{lstlisting}[language=Python]
            course = "   Python    "
            # Removes spaces from the left side
            print(course.lstrip())
            # Removes spaces from the right side
            print(course.rstrip())
            # Removes spaces from both the sides
            print(course.strip())
        \end{lstlisting}
    \end{frame}

    \begin{frame}[containsverbatim]
        \frametitle{Problem 7}
        \begin{lstlisting}[language=Python]
            word = "Python is easy programming language"
            middleIndex = len(word) // 2
            middleThree = word[middleIndex-1:middleIndex+2]
            print(middleThree)
        \end{lstlisting}
    \end{frame}

    \begin{frame}
        \frametitle{Problem 7 :: Solution}
        Given a string of odd length greater 7, creates a new string made of the middle three chars of a given String
    \end{frame}
    \begin{frame}[containsverbatim]
        \frametitle{Problem 8}
        \begin{lstlisting}[language=Python]
            word = "Hello, welcome to Python 101!!!"
            word = word.lower()
            word = word.capitalize()
            print(word.endswith('!'))
            print(word.startswith('H'))
        \end{lstlisting}
    \end{frame}
    \begin{frame}
        \frametitle{Problem 8 :: Solution}
        Problem checks whether given string is ending with  \emph{!} and starting with \emph{H}.
    \end{frame}
\end{document}
