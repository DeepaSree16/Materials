\documentclass[14pt]{beamer}
\title{Python 101 :: Unit 2}
\subtitle{Session 2}
\date{}
\author[TS]{TalentSprint}
\usefonttheme{serif}
\usepackage{bookman}
\usepackage{hyperref}
\usepackage[T1]{fontenc}
\usepackage{graphicx}
\usecolortheme{orchid}
\beamertemplateballitem

\usepackage{listings}
%\usebackgroundtemplate{\includegraphics[width=\paperwidth]{logo}}
\definecolor{mygreen}{rgb}{0,0.6,0}
\definecolor{mygray}{rgb}{0.5,0.5,0.5}
\definecolor{mymauve}{rgb}{0.58,0,0.82}

\lstset{ %
  backgroundcolor=\color{white},   % choose the background color
  basicstyle=\small,        % size of fonts used for the code
  breaklines=true,                 % automatic line breaking only at whitespace
  captionpos=b,                    % sets the caption-position to bottom
  commentstyle=\color{mygreen},    % comment style
  %escapeinside={\%*}{*)},          % if you want to add LaTeX within your code
  keywordstyle=\color{blue},       % keyword style
  stringstyle=\color{mymauve},     % string literal style
  showstringspaces=false,
}
   

\begin{document}
    \begin{frame}
        \titlepage
    \end{frame}

    \begin{frame}
        \frametitle{Topics for this Session}
        \begin{itemize}
            \item If statement
        \end{itemize}
    \end{frame}
    \begin{frame}
        \frametitle{If statement}
        \begin{itemize}
            \item The Python \emph{if} statement is used to implement decision.
                \begin{block}{}
                    \begin{lstlisting}[language=Python]
                        if <Expression>:
                            statements
                    \end{lstlisting}
                \end{block}
            \item The body is a sequence of one or more statements intended under \emph{if} heading.
            \item The body is executed if the expression(exp) is True otherwise it is skipped.
        \end{itemize}
    \end{frame}
    \begin{frame}
        \frametitle{Problems}
        \begin{itemize}
            \item Write a program to check whether given number is greater than 34. If true then display message as "Greet" 
            \item Write a program to check whether given number is equal to sum of its individual digits
            \item Write a prorgam to check whether given number is positive or negative
        \end{itemize}
    \end{frame}
    \begin{frame}
        \frametitle{The else clause}
        Sometimes, you want to evaluate a condition and take one path if it is true but specify an alternative path if it is not. This is accomplished with an else clause:
        \begin{block}{}
            \begin{lstlisting}[language=Python]
                if <Expression>:
                    statements
                else:
                    statements
            \end{lstlisting}
        \end{block}
    \end{frame}
    \begin{frame}
        \frametitle{Problems}
        \begin{itemize}
            \item Write a program to find largest number among two numbers
            \item Write a program that takes a number as input and display ``Hurray!! you have written perfect code" if it's less than or equal to zero, otherwise displays ``OOPs!! sorry you have missed some part in the code
            \item Given two numbers, write a program to print `True' if the sum of both numbers is less than 100. Otherwise print `False'.
            \item Write a program to check whether given number is pallindrome or not. A palindromic number is a number that is the same when written forwards or backwards. (Example 121)
        \end{itemize}
    \end{frame}
    \begin{frame}
        \frametitle{The elif Clause}
        There is also syntax for branching execution based on several alternatives. For this, use one or more elif clauses.
        \begin{block}{}
            \begin{lstlisting}[language=Python]
                if <Expression>:
                    statements
                elif <Expression>:
                    statements
                else:
                    statements
            \end{lstlisting}
        \end{block}
    \end{frame}
    \begin{frame}
        \frametitle{Problems}
        \begin{itemize}
            \item Write a program to find largest among three numbers
            \item Write a program which outputs "Sorry!! I am bug" if number is less than zero, "OOPs try to fix the bug soon" if number is less than 100, and "Hurry!! you have fixed the bug" otherwise.
            \item Given two integers, a and b, output ``True" if \emph{a} can be divided evenly by \emph{b}. Outputs ``False" otherwise.
            \item Write a program that outputs \emph{0} if the input is 1, and outputs \emph{1} if the input is 0, other outputs \emph{-1}
        \end{itemize}
    \end{frame}
\end{document}
        
              
        
        
        
        
        
        
        
        
        
        
        
        
        
        
        
        
        
        
        
        
        
        
        
        
        
        
        
        
        
        
        
        
        
        
        
        
        
        
        
        
        
        
        
        
        
        
        
        
        
        
        
        
        
        
        
        
        
        
        
        
        
        
        
        
        
        
        
        
        
        
        
        
        
        
        
        
        
        
        
        
        
        
        
        
        
        
        
        
        
        
        
        
        
        
        
        
        
        
        
        
        
        
        
        
