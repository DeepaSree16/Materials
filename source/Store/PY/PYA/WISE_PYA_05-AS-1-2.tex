\documentclass{exam}
\usepackage[utf8]{inputenc}
\begin{document}
\begin{questions}
    \question What is the output of the following code?

    \begin{verbatim}
         mix = ["hello", "55", "world"]
         print(mix[0] + mix[-1])
    \end{verbatim}
    \begin{oneparchoices}
        \choice Error
        \choice hello world
        \choice \textbf{helloworld}
    \end{oneparchoices}

    \question What is the output of the following code?

    \begin{verbatim}
        fibonacci = [1, 1, 2, 3, 5, 8, 13, 21]
        print(fibonacci[1:3])
    \end{verbatim}
    \begin{oneparchoices}
        \choice [1, 1, 2, 3]
        \choice [1, 2, 3]
        \choice \textbf{[1, 2]}
        \choice [1, 1, 2]
    \end{oneparchoices}

   \question What is the output of the following code?

    \begin{verbatim}
        x = 1
        if x == 0:
            print ('x is 0')
        elif x == 1:
            print('x is 1')

    \end{verbatim}
    \begin{oneparchoices}
        \choice x is 0
        \choice {x is 1}
        \choice error
        \choice no output
    \end{oneparchoices}

   \question  What are the two main types of loops that we have learned about in Python?

    \begin{oneparchoices}
        \choice IF Function & Variables
        \choice \textbf{FOR and WHILE Loops}
        \choice Forever and Fixed Loop
        \choice Data Types and Numbers
    \end{oneparchoices}
   \question What is the output of the following code?

    \begin{verbatim}
        print("Hello" + 2 + "world")
    \end{verbatim}
    \begin{oneparchoices}
        \chocie Hello 2 world
        \choice Hello2world
        \choice \textbf{Error}
        \choice HelloHelloworld
    \end{oneparchoices}

   \question What is the output of the following code?

    \begin{verbatim}
        x = 0
        for letter in "HelloWORLD":
            if letter.isupper():
                x = x + 1
        print(x)
    \end{verbatim}
    \begin{oneparchoices}
        \choice \textbf{6}
        \choice 10
        \choice 5
        \choice 1
    \end{oneparchoices}

   \question What is the output of the following code?

    \begin{verbatim}
    \end{verbatim}
    \begin{oneparchoices}
    \end{oneparchoices}

   \question What is the output of the following code?

    \begin{verbatim}
        fruits = ["apple", "orange"]
        fruits.append("grapes")
        print(fruits)

    \end{verbatim}
    \begin{oneparchoices}
        \choice ['apple', 'orange', 'grapes']
        \choice ['apple', 'orange']
        \chocie ['grapes']
    \end{oneparchoices}

   \question What is the output of the following code?

    \begin{verbatim}
        fruits = ['apple', 'orange', 'grapes']
        for fruit in fruits:
            print("Yummy!!")
    \end{verbatim}
    \begin{oneparchoices}
        \choice error
        \choice no output
        \choice Yummy!! gets printed thrice
        \choice Yummy!! gets printed once
    \end{oneparchoices}

   \question What is the output of the following code?

    \begin{verbatim}
        fruits = ['apple', 'orange', 'grapes']
        fruits.append('pineapple')
        print(fruits)
    \end{verbatim}
    \begin{oneparchoices}
        \choice Adds pineapple to the beginning of the list
        \choice Replaces grapes with pineapple
        \choice Replaces apple with pineapple
        \choice Adds pineapple to the end of the list
    \end{oneparchoices}
    
   \question What is the output of the following code?

    \begin{verbatim}
        mix = [4, 66, 88, 44, 4]
        mix.remove(4)
        print(mix)
    \end{verbatim}
    \begin{oneparchoices}
        \choice [4, 66, 88, 44, 4]
        \choice [4, 66, 88, 44]
        \choice [66, 88, 44, 4]
        \choice [66, 88, 44]
    \end{oneparchoices}


\end{questions}
\end{document}
