\documentclass[14pt]{beamer}
\title{Python 101 :: Unit 3}
\subtitle{Session 2}
\date{}
\author[TS]{TalentSprint}
\usefonttheme{serif}
\usepackage{bookman}
\usepackage{hyperref}
\usepackage[T1]{fontenc}
\usepackage{graphicx}
\usecolortheme{orchid}
\beamertemplateballitem

\usepackage{listings}
%\usebackgroundtemplate{\includegraphics[width=\paperwidth]{logo}}
\definecolor{mygreen}{rgb}{0,0.6,0}
\definecolor{mygray}{rgb}{0.5,0.5,0.5}
\definecolor{mymauve}{rgb}{0.58,0,0.82}

\lstset{ %
  backgroundcolor=\color{white},   % choose the background color
  basicstyle=\small,        % size of fonts used for the code
  breaklines=true,                 % automatic line breaking only at whitespace
  captionpos=b,                    % sets the caption-position to bottom
  commentstyle=\color{mygreen},    % comment style
  %escapeinside={\%*}{*)},          % if you want to add LaTeX within your code
  keywordstyle=\color{blue},       % keyword style
  stringstyle=\color{mymauve},     % string literal style
  showstringspaces=false,
}
   
\begin{document}
    \begin{frame}
        \titlepage
    \end{frame}
    \begin{frame}
        \frametitle{Topics for the Session}
        \begin{itemize}
            \item Functions
            \item Code Reading
        \end{itemize}
    \end{frame}
    \begin{frame}[containsverbatim]
        \frametitle{Functions}
        \begin{itemize}
            \item A function is a group of statements that exist within a
program for the purpose of performing a specific task.
            \item \alert{Syntax}
        \end{itemize}
        \begin{lstlisting}[language=Python]
            def functionname (arguments):
                body
        \end{lstlisting}
    \end{frame}
    \begin{frame}
        \frametitle{Functions \& Arguments}
        \begin{itemize}
            \item Defined using \alert{def} keyword
            \item Function naming
            \item Input parameters
            \item Function body - as block of statements
            \item Docstring - First line of the block
            \item Return statement
            \begin{itemize}
                \item May return value
                \item Python None value is returned by default
                \item May return multiple values
            \end{itemize}    
    \end{itemize}
    \end{frame}
    \begin{frame}[containsverbatim]
        \frametitle{Example}
        \begin{lstlisting}[language=Python]
            def greet(name):
                """This function greets to
                the person passed in as
                parameter"""
                return "Hello, " + name + ". Good morning!"
            # Function Call
            print(greet("World"))
        \end{lstlisting}
    \end{frame}
    \begin{frame}[containsverbatim]
        \frametitle{Problem 1}
        \begin{lstlisting}[language=Python]
            def k_to_k(n, k):
                return n == k ** k
        \end{lstlisting}
    \end{frame}
    \begin{frame}
        \frametitle{Problem 1 :: Solution}
        Program returns \emph{True} if k ** k == n for input (n, k).
    \end{frame}
    \begin{frame}[containsverbatim]
        \frametitle{Problem 2}
        \begin{lstlisting}[language=Python]
            def is_plural(word):
                word = word.lower()
                return word.endswith('s')
        \end{lstlisting}
    \end{frame}
    \begin{frame}
        \frametitle{Problem 2 :: Solution}
        Program check whether given word is plural or not. A plural word is one that ends in "s".
    \end{frame}
    \begin{frame}[containsverbatim]
        \frametitle{Problem 3}
        \begin{lstlisting}[language=Python]
            def digitLetterCount (word):
                lettercount = 0
                digitcount = 0
                i = 0
                while i < len(word):
                    if word[i].isdigit():
                        digitcount = digitcount + 1
                    else:
                        lettercount = lettercount + 1
                    i = i + 1
                return lettercount, digitcount
        \end{lstlisting}
    \end{frame}
    \begin{frame}
        \frametitle{Problem 3 :: Solution}
        Program calculates the number of letters and digits
    \end{frame}
    \begin{frame}
        \frametitle{Problem 4}
        Create a function that takes in three arguments (prob, prize, pay) and returns true if $prob * prize > pay$; otherwise return false.
    \end{frame}
    \begin{frame}[containsverbatim]
        \frametitle{Problem 4 :: Solution}
        \begin{lstlisting}[language=Python]
            def profitable_gamble(prob, prize, pay):
                return prob * prize > pay
        \end{lstlisting}
    \end{frame}
    \begin{frame}
        \frametitle{Problem 5}
        Tom is a very methodic guy that loves geometry and pizza: he loves them so much that, before eating a pizza, he calculates its radius and its height. Now, he wants to know from you the total volume of pizza that he swallowed!

        You are given the two parameters that Tom measured: Radius and Height.
    \end{frame}
    \begin{frame}
        \frametitle{Problem 5}
        He tells you that if you multiply the height for the square of the radius and multiply the result for the mathematical constant $\pi$ (Pi), you will obtain the total volume of the pizza. 
        
        \emph{Implement a function that returns the volume of the pizza as a whole number, rounding it to the nearest integer.}
    \end{frame}
    \begin{frame}[containsverbatim]
        \frametitle{Problem 5 :: Solution}
        \begin{lstlisting}[language=Python]
            import math
            def vol_pizza(radius, height):
                return round(radius ** 2 * height * math.pi)
        \end{lstlisting}
    \end{frame}
\end{document}
