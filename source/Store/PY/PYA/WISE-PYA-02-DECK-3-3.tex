\documentclass[14pt]{beamer}
\title{Python 101 :: Unit 2}
\subtitle{Session 3}
\date{}
\author[TS]{TalentSprint}
\usefonttheme{serif}
\usepackage{bookman}
\usepackage{hyperref}
\usepackage[T1]{fontenc}
\usepackage{graphicx}
\usecolortheme{orchid}
\beamertemplateballitem

\usepackage{listings}
%\usebackgroundtemplate{\includegraphics[width=\paperwidth]{logo}}
\definecolor{mygreen}{rgb}{0,0.6,0}
\definecolor{mygray}{rgb}{0.5,0.5,0.5}
\definecolor{mymauve}{rgb}{0.58,0,0.82}

\lstset{ %
  backgroundcolor=\color{white},   % choose the background color
  basicstyle=\small,        % size of fonts used for the code
  breaklines=true,                 % automatic line breaking only at whitespace
  captionpos=b,                    % sets the caption-position to bottom
  commentstyle=\color{mygreen},    % comment style
  %escapeinside={\%*}{*)},          % if you want to add LaTeX within your code
  keywordstyle=\color{blue},       % keyword style
  stringstyle=\color{mymauve}, % string literal style
  showstringspaces=false,
}
   
   
\begin{document}
    \begin{frame}
        \titlepage
    \end{frame}
    \begin{frame}
        \frametitle{Topics for the Session}
        \begin{itemize}
            \item Code Reading and Writing
        \end{itemize}
    \end{frame}
    \begin{frame}[containsverbatim]
        \frametitle{Problem 1}
        \begin{lstlisting}[language=Python]
            miles = float(input())
            kilometers = miles * 1.609344
            print(kilometers)
        \end{lstlisting} 
    \end{frame}
    \begin{frame}
        \frametitle{Problem 1 :: Solution}
        \small Convert miles to kilometers
    \end{frame}
    \begin{frame}[containsverbatim]
        \frametitle{Problem 2}
        \begin{lstlisting}[language=Python]
            mass1 = float(input())
            mass2 = float(input())
            r = float(input())
            G = 6.673 * (10 ** -11)
            force =(G * mass1 * mass2) / (r ** 2)
            print(round(force, 5),"N")
        \end{lstlisting}
    \end{frame}
    \begin{frame}
        \frametitle{Problem 2 :: Solution}
        \small Compute gravitational force between two objects
        \begin{center}
        $F = G\; \frac{Mm}{r^{2}}$, Where G is $6.674 \; 10^{-11} N(m/kg)^{2}$
        \end{center}
    \end{frame}
    \begin{frame}[containsverbatim]
        \frametitle{Problem 3}
        \begin{lstlisting}[language=Python]
            num = 1122
            if 9 < num < 99:
                print("Two digit number")
            elif 99 < num < 999:
                print("Three digit number")
            elif 999 < num < 9999:
                print("Four digit number")
            else:
                print("number is <= 9 or >= 9999")
        \end{lstlisting}
    \end{frame}
    \begin{frame}
        \frametitle{Problem 3 :: Solution}
        The program checks whether given number is two or three or four digit number.
    \end{frame}
    \begin{frame}[containsverbatim]
        \frametitle{Problem 4}
        \begin{lstlisting}[language=Python]
            year = int(input())
            if year % 4 == 0 and year % 100 != 0:
                print("Hurray")
            elif year % 100 == 0:
                print("Oops")
            elif year % 400 ==0:
                print("Hurray")
            else:
                print("Oops")
        \end{lstlisting}
    \end{frame}
    \begin{frame}
        \frametitle{Problem 4 :: Solution}
        The program prints "Hurray" if the given year is leap year otherwise prints "Oops"
    \end{frame}
    \begin{frame}[containsverbatim]
        \frametitle{Problem 5}
        \begin{lstlisting}[language=Python]
        cp = int(input())
        sp = int(input())
        if sp > cp:
            print(sp - cp)
        elif cp > sp:
            print(cp - sp)
        else:
            print(0)
        \end{lstlisting}
    \end{frame}
    \begin{frame}
        \frametitle{Problem 5 :: Solution}
        Calculate profit or loss
    \end{frame}
    \begin{frame}[containsverbatim]
        \frametitle{Problem 6}
        \begin{lstlisting}[language=Python]
            side1 = int(input())
            side2 = int(input())
            side3 = int(input())
            if (side1 == side2) & (side2 == side3):
                print("Equal")
            elif (side1 == side2) | (side1 == side3) | (side2 == side3):
                print("Isosceles")
            else:
                print("Scalene")
        \end{lstlisting}
    \end{frame}
    \begin{frame}
        \frametitle{Problem 6 :: Solution}
        The program to checks whether a triangle is Equilateral, Isosceles or Scalene
    \end{frame}
\end{document}
