\documentclass[14pt]{beamer}
\title{Python 101 :: Unit 4}
\subtitle{Session 2}
\date{}
\author[TS]{TalentSprint}
\usefonttheme{serif}
\usepackage{bookman}
\usepackage{hyperref}
\usepackage[T1]{fontenc}
\usepackage{graphicx}
\usecolortheme{orchid}
\beamertemplateballitem

\usepackage{listings}
%\usebackgroundtemplate{\includegraphics[width=\paperwidth]{logo}}
\definecolor{mygreen}{rgb}{0,0.6,0}
\definecolor{mygray}{rgb}{0.5,0.5,0.5}
\definecolor{mymauve}{rgb}{0.58,0,0.82}

\lstset{ %
  backgroundcolor=\color{white},   % choose the background color
  basicstyle=\small,        % size of fonts used for the code
  breaklines=true,                 % automatic line breaking only at whitespace
  captionpos=b,                    % sets the caption-position to bottom
  commentstyle=\color{mygreen},    % comment style
  %escapeinside={\%*}{*)},          % if you want to add LaTeX within your code
  keywordstyle=\color{blue},       % keyword style
  stringstyle=\color{mymauve},     % string literal style
  showstringspaces=false,
}
   
\begin{document}
    \begin{frame}
        \titlepage
    \end{frame}
    \begin{frame}
        \frametitle{Topics for the session}
        \begin{itemize}
            \item List and for Loop
        \end{itemize}
    \end{frame}

    \begin{frame}[containsverbatim]
       \frametitle{List}
       \begin{itemize}
        \item List is an ordered collection of items
        \item Items could be any sequential or non sequential data types
        \item \alert{Example:}
       \end{itemize}
       \begin{lstlisting}[language=Python]
        mix = ['Red', 1, 2, 45, -678, 'Pink'] 
       \end{lstlisting}
    \end{frame}

    \begin{frame}[containsverbatim]
        \frametitle{Creating a List}
        \begin{lstlisting}[language=Python]
            mix = ['Red', 1, 2, 45, -678, 'Pink']
            # Creats a empty list
            mix = [] # or
            mix = list()
        \end{lstlisting}
    \end{frame}

    \begin{frame}[containsverbatim]
        \frametitle{Accessing List Elements}
        \begin{lstlisting}[language=Python]
            mix = ['Red', 1, 2, 45, -678, 'Pink']
            # prints third element from the list
            print(mix[2])
            # prints last element from the list
            print(mix[-1])
        \end{lstlisting}
    \end{frame}
    \begin{frame}[containsverbatim]
        \frametitle{Accessing List Elements}
        \textbf{Note:} Lists are mutable. So we can modify an element of a list by assigning a new value. For example:

        \begin{lstlisting}[language=Python]
            mix = ['Red', 1, 2, 45, -678, 'Pink']
            # changing value of fourth element in the list as 67
            mix[3] = 67
            print(mix) # outputs ['Red', 1, 2, 67, -678, 'Pink']
        \end{lstlisting}
    \end{frame}
    \begin{frame}[containsverbatim]
        \frametitle{Accessing List Elements - Slicing and striding}
        \begin{lstlisting}[language=Python]
            mix = ['Red', 1, 2, 45, -678, 'Pink']
            # Prints all the elements from the list
            print(mix[0:]) # or
            print(mix[:])
            # Prints second and third elements from the list
            print(mix[1:3])
        \end{lstlisting}
    \end{frame}
    \begin{frame}[containsverbatim]
        \frametitle{Accessing List Elements - for loop}
        \begin{lstlisting}[language=Python]
            mix = [3, 66, -99, "hello", "k", 5]
            for element in mix:
                print(element)
        \end{lstlisting}
    \end{frame}
    \begin{frame}[containsverbatim]
        \frametitle{List Methods}
        \textbf {append(value)}: appends a new element to the end of the list.
        
        \alert{Example:} 
        \begin{lstlisting}[language=Python]
            mix = []
            # Appending 65, 866, 7
            mix.append(65)
            mix.append(866)
            mix.append(7)
            # Appending Hello
            mix.append("Hello")
            print(mix) # Outputs: [65, 866, 7, "Hello"]
        \end{lstlisting}
    \end{frame}

    \begin{frame}[containsverbatim]
        \frametitle{List Methods}
        \textbf{extend(enumerable):} extends the list by appending elements from another enumerable.

        \alert{Example:}
        \begin{lstlisting}[language=Python]
            mix = [1, 355, 88, -5]
            # Extend list by appending hello, 77
            mix.extend("hello")
            mix.extend(77) # outputs [1, 355, 88, -5, "hello", 77]
        \end{lstlisting}
    \end{frame}
    \begin{frame}[containsverbatim]
        \frametitle{List Methods}
        \textbf{remove(value)} : removes the first occurrence of the specified value. If the provided value cannot be found, a ValueError is raised.

        \alert{Example:}
        \begin{lstlisting}[language=Python]
            mix = [1, 12, 1, 77, 99, -11]
            #remove 11
            mix.remove(1)
            print(mix) #Outputs- [12, 1, 77, 99, -11]
        \end{lstlisting}
    \end{frame}
    \begin{frame}
        \frametitle{List methods}
        \begin{itemize}
            \item \textbf{mix.sort()} : sorts the elements of the list in place. That is it mutates the list.
            \item \textbf{mix.reverse()} : reverses the order of elements in the list. Again mutates the list.
            \item \textbf{mix.count(x)} : returns the number of times the value x occurs in the list.
        \end{itemize}
    \end{frame}
\end{document}
