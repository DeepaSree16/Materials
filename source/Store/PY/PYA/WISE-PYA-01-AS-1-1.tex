\documentclass{exam}
\usepackage[utf8]{inputenc}
\begin{document}
\begin{questions}
    \question Python was created by \_\_\_\_\_\_\_\_
    
    \begin{oneparchoices}
        \choice James Gosling \\
        \choice Bill Gates \\
        \choice Steve Jobs \\
        \choice \textbf{Guido van Rossum} \\
        \choice Google 
    \end{oneparchoices}

    \question Python was released publicly in \_\_\_\_\_\_\_\_

    \begin{oneparchoices}
        \choice 1941 \\
        \choice 1971 \\ 
        \choice 1981 \\
        \choice \textbf{1991}
    \end{oneparchoices}

    \question Which of the following statements is true?

    \begin{oneparchoices}
        \choice Python 3 is a newer version, but it is backward compatible with Python 2 \\
        \choice \textbf{Python 3 is a newer version, but it is not backward compatible with Python 2} \\
        \choice A Python 2 program can always run on a Python 3 interpreter\\
        \choice A Python 3 program can always run on a Python 2 interpreter
    \end{oneparchoices}

    \question To start Python from the command prompt, use the command \_\_\_\_\_\_\_\_ 

    \begin{oneparchoices}
        \choice \textbf{python} \\
        \choice run python \\
        \choice Python \\
        \choice start python
    \end{oneparchoices}

    \question Python syntax is case-sensitive

    \begin{oneparchoices}
        \choice \textbf{True}
        \choice False
    \end{oneparchoices}

    \question Which of the following symbols are used for comments in Python?
    
    \begin{oneparchoices}
        \choice * \\
        \choice \textbf{\#} \\
        \choice " \\
        \choice \$
    \end{oneparchoices}

    \question What is the output of the below code? \\

        a = 8.6 \\
        b = 2 \\
        print (a // b)\\

    \begin{oneparchoices}
        \choice \textbf{4.0} \\
        \choice 4 \\
        \choice Error \\
        \choice 4.3 
    \end{oneparchoices}

    \question A Python paragraph comment uses the style \_\_\_\_\_\_\_

    \begin{oneparchoices}
        \choice \$ comments \$  \\
        \choice \# comments \# \\ 
        \choice \textbf {``` comments '''}
    \end{oneparchoices}

    \question What is the result of 45 / 4?

    \begin{oneparchoices}
        \choice \textbf{11.25} \\
        \choice 11 \\
        \choice 13 \\
        \choice 10
    \end{oneparchoices}

    \question In the expression 45 / 4, the values on the left and right of the / symbol are called \_\_\_\_\_\_\_

    \begin{oneparchoices}
        \choice \textbf{operands} \\
        \choice operators \\
        \choice parameters \\
        \choice arguments
    \end{oneparchoices}

    \question 25 \% 3  is \_\_\_\_\_\_

    \begin{oneparchoices}
        \choice 0 \\
        \choice 8.3 \\
        \choice 3 \\
        \choice 8
        \choice \textbf{1}
    \end{oneparchoices}

    \question 4 ** 3 is \_\_\_\_\_\_\_

    \begin{oneparchoices}
        \choice 14 \\
        \choice \textbf{64} \\
        \choice 81 \\
        \choice 12
    \end{oneparchoices}

    \question 25 ** 5.00 is \_\_\_\_\_\_\_

    \begin{oneparchoices}
        \choice \textbf{9765625.0} \\
        \choice {625625.0} \\
        \choice {97665625}
    \end{oneparchoices}

    \question 15 * 3 ** 2 is \_\_\_\_\_\_\_

    \begin{oneparchoices}
        \choice \textbf{135} \\
        \choice 153 \\
        \choice 531 \\
        \choice 90
    \end{oneparchoices}

    \question Which of the following yield 17?

    \begin{oneparchoices}
        \choice 88 // 5 \\
        \choice 88 / 5 \\
        \choice 88.0 / 5 \\
        \choice 88 \% 5
    \end{oneparchoices}

    \question What will be the output of the below code? 

    number = 5
    number = number + 4
    print(number)

    \begin{oneparchoices}
        \choice 4 \\
        \choice \textbf{9}
        \choice 5
    \end{oneparchoices}

    \question In Python, a variable must be declared before it is assigned a value:

    \begin{oneparchoices}
        \choice True
        \choice \textbf{False}
    \end{oneparchoices}

    \question Which of the following statements assigns the value 11999 to the variable number in Python:

    \begin{oneparchoices}
        \choice \textbf{number = 11999}
        \choice number - 11999
        \choice set number 11999
        \choice number equal 11999
    \end{oneparchoices}

    \question In Python, a variable may be assigned a value of one type, and then later assigned a value of a different type:

    \begin{oneparchoices}
        \choice \textbf{True}
        \choice False
    \end{oneparchoices}

    \question Which of the following are valid Python variable names:

    \begin{oneparchoices}
        \choice \textbf{number}
        \choice 44number
        \choice for
    \end{oneparchoices}

    \question what is the output of the following code?

        print(type(56789))

    \begin{oneparchoices}
        \choice \textbf{$< class 'int'>$} \\
        \choice $< class 'set' >$ \\
        \choice $< class 'long' >$ \\
        \choice $< class 'complex' >$
    \end{oneparchoices}

    \question what is the output of the following code?

        print(type(1J))

    \begin{oneparchoices}
        \choice $<class 'int'>$ \\
        \choice $<class 'string'>$ \\
        \choice $<class 'long'>$ \\
        \choice \textbf{$<class 'complex'>$}
    \end{oneparchoices}

    \question what is the output of the following code?

        print(type(1/2))

    \begin{oneparchoices}
        \choice \textbf{$<class 'float'>$} \\
        \choice $<class 'int'>$ \\
        \choice $<class 'complex'>$ \\
        \choice $<class 'ratio'>$ 
    \end{oneparchoices}

    \question what gets printed?

        print('\textbackslash x67')

    \begin{oneparchoices}
        \choice \textbf{g} \\
        \choice \textbackslash x67 \\
        \choice 67
    \end{oneparchoices}

    \question what gets printed?

        print(0xB + 0xb)

    \begin{oneparchoices}
        \choice \textbf{22} \\
        \choice 20 \\
        \choice (0xB + 0xb) \\
        \choice gh \\
    \end{oneparchoices}
\end{questions}
\end{document}
