% Fundamentals of Programming
% Python Quick Reference
% (c) TalentSprint, Chennai
%
%
\documentclass{article}
\usepackage[landscape]{geometry}
\usepackage[pdftex]{color}
\usepackage{url}
\usepackage{multicol}
\usepackage{listings}
\usepackage[T1]{fontenc}
\usepackage[sfdefault]{AlegreyaSans} 
\newcommand{\code}[1]{{\color{blue}#1}}
\newcommand{\inl}[1]{\emph{\color{blue}#1}}
\newcommand{\warn}[1]{{\color{red}#1}}
\advance\topmargin-.8in
\advance\textheight2in
\advance\textwidth3.0in
\advance\oddsidemargin-1.45in
\advance\evensidemargin-1.45in
\parindent0pt
\parskip2pt

\newcounter{parnum}
\newcommand{\N}{%
   \noindent\refstepcounter{parnum}
    %\makebox[1.5em][l]
    {\textbf{\arabic{parnum}. }}}

% Section break, dictates column widths?
\newcommand{\hr}{\centerline{\rule{3.5in}{1pt}}}
% Adjust gap to affect spacing, page count
%\newcommand{\sect}[1]{\hr\par\ vspace*{2pt}\textbf{#1}\par}
\newcommand{\sect}[1]{\hr\par\textbf{\N #1}\par}
% Mandatory indentation on subsidiary lines
\newcommand{\skipin}{\hspace*{12pt}}
% \newcommand{\inp}[1]{\color{blue}{\textbf{$>$$>$$>$ #1}}}
% \newcommand{\out}[1]{\color{blue}{\textbf{#1}}}

\begin{document}
\begin{multicols*}{3}
\begin{center}
\textbf{Quick Reference} \\
\textbf{Python-3.x}\\
\end{center}

% backup over center environment gap
\vspace{-2ex}

%\begin{enumerate}
%*********************************************
\sect{Storing values in variables}
\code{x = 5} stores the integer \inl{5} in \inl{x}\\
\code{y = 2.5} stores the float \inl{2.5} in \inl{y}\\
\code{s = ``Hello World''} stores string \inl{Hello World} in {\inl s}\\\\
%*********************************************
\sect{Boolean Constants}
The Boolean constants are \inl{True} and \inl{False}.\\
\warn{Note the capitalization.}\\\\
%*************************************************
\sect{Arithmetic Operations}
\code{x + y} computes the sum of \inl{x} and \inl{y}\\
\code{x - y} computes the value of \inl{y} subtracted from \inl{x}\\
\code{x * y} computes the product of \inl{x} and \inl{y}\\
\code{x ** y} computes \inl{x} raised to \inl{y}\\
\code{x \% y} computes the remainder when \inl{x} is divided by \inl{y}\\
\code{x / y} computes the float value of \inl{x} divided by \inl{y}.\\
\warn{17 / 4 gives 4.25}\\
\code{x // y} computes the quotient when \inl{x} is divided by \inl{y}\\
\warn{17 // 4 gives 4}\\\\
%*********************************************
\sect{Comparison Operations}
\skipin Returns Boolean values \code{True} or \code{False}\\
\code{x == y} checks if \inl{x} is equal to \inl{y}\\
%\code{x == y} checks if the values in \inl{x} and \inl{y} are equal\\
\code{x != y} checks if \inl{x} is not equal to \inl{y}\\
\code{x $>$ y} checks if value in \inl{x} is greater than \inl{y}\\
\code{x $>$ y} checks if value in \inl{x} is greater than that in \inl{y}\\
\code{x $\geq$ y} checks if \inl{x} is greater than or equal to 
\inl{y}\\
\code{x $<$ y} checks if value in \inl{x} is less than that in \inl{y}\\
\code{x $\leq$ y} checks if value in \inl{x} is less than or equal to \inl{y}\\
%\code{x $<$ 6 $>$ y} checks if the value is in between \inl{x} and \inl{y}\\
\code{x $<$ y $<$ z} checks if \inl{y} is in between \inl{x} and \inl{z}\\\\
%*********************************************
\sect{Logical Operations}
\code{x == 5 and y != 7} returns \code{True} if both conditions are \code{True}\\
\code{x == 5 or y != 7} returns \code{True} if either condition is \code{True}\\
\code{not x $>$ 7} \code{not} negates the condition\\
%****************************************************
\\\\\\
\sect{Membership Operators}
\code{x in y } results in \code{True} if x is a member of sequence y.\\
\code{x not in y} results in \code{True} if x is not a member of sequence y.\\
%****************************************************
\sect{Identity Operators}
\code{x is y} Evaluates to \code{True} if the variables on either side of the operator point to the same object.\\
\code{x is not y} Evaluates to \code{False} if the variables on either side of the operator point to the same object.\\
%****************************************************
\sect{Conversions}
\code{int(``65'')} gives the integer \inl{65}\\
\code{int(65.75)} gives the integer \inl{65}\\
\code{float(``65.75'')} gives the float \inl{65.75}\\
\code{float(65)} gives the float \inl{65.0}\\
\code{str(65)} gives the string \inl{``65''}\\
\code{str(65.75)} gives the string \inl{``65.75''}\\
\warn{ int(``65.75'') gives an error}\\
%*********************************************
\sect{Indentation}
{\color{blue}\begin{lstlisting}
In Python blocks are identified by 
indentation.
statement 1:
    statement 2
    statement 3
\end{lstlisting}}
\warn{ statement 1 must end in a colon. It can be an \emph{if statement}, \emph{while statement}, \emph{for statement} or a \emph{def statement}}

Similarly,
{\color{blue}\begin{lstlisting}
statement 1
    statement 2
        statement 3
        statement 4
        statement 5
    statement 6
\end{lstlisting}}
\warn{Use only 4 spaces for an indent.}\\\\
%*********************************************
\sect{Simple Input}
\code{x = input()} for taking input.\\
\code{x = input("Enter number: ")} display a prompt while taking input.\\
\warn{The value given by input is always a string.}\\\\\\
%*********************************************
\sect{Simple Output}
\code{print(x)} print the value in \inl{x} and a new line.\\
\code{prin(x, y)} print the value in \inl{x} and a space.\\
\code{print(x,y, sep=``...'')} prints the values of \inl{x}, \inl{y} separated by ``...'' instead of the default space.
\code{print(x, y, sep=``\t'', end = ``::'')} prints the values of \inl{x}, \inl{y} seperated by a tab and instead of ending with a newline\\\\
%*********************************************
\sect{if statement}
{\color{blue}\begin{lstlisting}
if x > 0:
    print(``positive'')
\end{lstlisting}}

%*********************************************
\sect{if...else statement}
{\color{blue}\begin{lstlisting}
if x > 0:
    print(``positive'')
else:
    print(''not positive'')
\end{lstlisting}}

%*********************************************
\sect{if...elif statement}
{\color{blue}\begin{lstlisting}
if x > 0:
    print(``positive'')
elif x < 0:
    print(``negative)
else:
    print(``Zero'')
\end{lstlisting}}

%*********************************************
\sect{while statement}
{\color{blue}\begin{lstlisting}
x = 1
while x < 10:
    print(``The value of x is'', x)
    x += 1
\end{lstlisting}}
Prints \inl{x} value from 1 to 9\\
%*********************************************
\sect{Defining Strings}
\code{s = "I am a string"}\\
\skipin enclosed in double quotes.\\
\code{s = 'He said "Good Morning", to the class'}\\
\skipin use single quotes if there is a double quote in the string.\\
\code{s = "It's time"}\\
\skipin use double quotes if there is a single quote in the string.\\\\\\
%*********************************************
\sect{Accessing characters in strings}
\code{s[0]} accesses the first character in the string \inl{s}.\\
\code{s[4]} accesses the fifth character in the string \inl{s}.\\
\warn{Indexing starts with 0 for the first character.}\\
\code{s[-1]} accesses the last character in the string \inl{s}.\\
\code{s[-2]} accesses the last but one character in \inl{s}.\\
%\code{s[-5]} accesses the fifth character from the last.\\
\warn{Negative indexing starts with -1 from last.}\\\\
%*********************************************
\sect{Slicing strings}
\code{s = ``Hello World''}\\
\code{s[3:]} returns \inl{``lo World''}\\
\skipin substring from character with index 3 to end.\\
\code{s[:7]} returns \inl{``Hello W''}\\
\skipin substring from start to character with index 6.\\
\code{s[3:7]} returns \inl{``lo W''}\\
\skipin substring from character with index 3 to character with index 6.\\
\code{s[2:-2]} returns ``llo Wor''\\
\skipin substring from third character to the third character from the end.\\\\
%*********************************************
\sect{string methods}
\code{s = ``Hello'' + `World''} stores \inl{HelloWorld} in \inl{s}.\\
\code{len(s)} length of the string \inl{s}\\
\code{"ell" in s} checks for the presence of \inl{``ell''} in \inl{s}.\\
\code{s.lower()} returns \inl{``helloworld''}\\
\skipin a new string with characters of \inl{s}, in lower case.\\
\code{s.upper()} returns \inl{``HELLOWORLD''}\\
\skipin a new string with characters of \inl{s}, in upper case.\\
\code{s.replace(``l'', ``m'')} returns \inl{``Hemmo Wormd''}\\
\skipin a new string with all the \inl{l}  replaced with \inl{m}.\\
\code{s.split()} returns \inl{[``Hello'', ``World'']} \\
\skipin a list of words in the string.\\
%\code{s.split(";")} returns a list of strings obtained by splitting \inl{s} on semicolon\\
\warn{All the above operations return new strings. The original
string remains unaltered.}\\
%*********************************************
\\
\sect{range function}
\code{range(8)} returns list of numbers from \inl{0} to \inl{7}.\\
%\code{range(3, 13)} returns a list of numbers from \inl{3} to \inl{12}.\\
\code{range(3, 13, 2)} returns odd numbers from \inl{3} to \inl{12}.\\
\warn{range returns a ``generator'', convert it to list to see the values,\\   Example: \code{print(list(range(8)))}}\\\\
%*********************************************
\sect{Defining functions}
{\color{blue}\begin{lstlisting}
def add_one(x):
    return x + 1
\end{lstlisting}}
defines the \inl{add\_one} function that takes one argument and 
returns the value of argument plus one.

{\color{blue}\begin{lstlisting}
def getMax(x, y):
    if x > y:
        return x
    return y
\end{lstlisting}}
defines the \inl{getMax} function that takes two arguments and returns the
greater one from them.\\\\
%*********************************************
\sect{Calling functions}
\code{add\_one(5)} returns 6.\\
\code{x = add\_one(8)} stores the value 9 in x.\\
\code{x = add\_one(x)} increments x by one.\\
\code{y = getMax(4, 8)} stores the return value \inl{8} in \inl{y}.\\
\code{biggest = getMax(biggest, currentValue)}\\\\
%*********************************************
\sect{lists}
\code{pr = [2, 3, 5, 7, 11, 13]} creates the list \inl{pr}.\\
\code{len(pr)} returns the length of the list, \inl{6}\\
\code{15 in pr} checks for the presence of \inl{15} in the list \inl{pr}.\\
\code{pr + [17, 19, 23]} adds the lists and returns a new list.\\
%*********************************************
\sect{slicing lists}
\code{pr[0]} accesses the first item, \inl{2}.\\
\code{pr[-4]} accesses the fourth item from end, \inl{5}.\\
\code{pr[2:]} accesses \inl{[5, 7, 11, 13]}\\
\skipin list of items from third to last.\\
\code{pr[:4]} accesses \inl{[2, 3, 5, 7]}\\
\skipin list of items from first to fourth.\\
\code{pr[2:4]} accesses \inl{[5, 7]}\\
\skipin list of items from third to fifth.\\
\code{pr[1::2]} accesses \inl{[3, 7, 13]}\\
\skipin alternate items, starting from the second item.\\
%*********************************************
\\\\\\\\\\\\
\sect{list methods}
\code{pr.append(17)} adds \inl{17} at the end of the list \inl{pr}.\\
\skipin \inl{pr} becomes \inl{[2, 3, 5, 7, 11, 13, 17]}\\
\code{pr.extend([17, 19, 21])} appends \inl{17}, \inl{19}, \inl{21}\\
\skipin \inl{pr} becomes \inl{[2, 3, 5, 7, 11, 13, 17, 19, 21]}\\
\warn{Operations mentioned above modify the list itself.}\\
%*********************************************
\sect{for loop}
{\color{blue}\begin{lstlisting}
for i in pr:
    print(i)
\end{lstlisting}}
iterates over the list \inl{pr} one item at a time.\\
%*********************************************
\sect{dictionaries}
\code{mm2num = \{``jan'': 1, ``feb'': 2, ``mar'': 4\}}\\
\skipin creates the dictionary \inl{mm2num}\\
\code{mm2num[``feb'']} gives the corresponding value, \inl{2}\\
\code{mm2num[``mar''] = 3} \\
\skipin changes the value for the key \inl{`mar''} to \inl{3}\\
\code{mm2num[``apr''] = 4} \\
\skipin creates the key \inl{``apr''} with \inl{4} as the value\\
\code{mm2num.values()} returns list of values, \inl{[1, 2, 3, 4]}\\
\code{mm2num.keys()} returns list of keys,\\
\skipin \inl{[``jan'', ``feb'', ``mar'', ``apr'']}\\
%*********************************************
\sect{sets}
\code{prs = set([2, 3, 2, 5, 3, 7, 7, 2, 3])}\\
\skipin creates the set \inl{set([2, 3, 5, 7])} and stores in \inl{prs}.\\
\code{ods = set([1, 3, 5, 9, 3, 7, 7, 9, 3])}\\
\skipin creates the set \inl{set([1, 3, 5, 7, 9])} and stores in \inl{ods}.\\
\code{prs $|$ ods} gives the union of the sets, \inl{set([1, 2, 3, 5, 7, 9])}\\
\code{prs \& ods} gives the intersection of the sets, \inl{set([3, 5, 7])}\\
\code{ods - prs} gives the difference of sets\\
\skipin items in \inl{ods} that are not in \inl{prs}, which is 
\inl{set([1, 9])}\\
\code{ods \^{} prs} gives the symmetric difference\\
\skipin items in \inl{ods} or in \inl{prs} but not in both, 
\inl{set([1, 2, 9])}\\
%*********************************************
\sect{Reading from files}
{\color{blue}\begin{lstlisting}
fileLoc = ``/home/tsprint/primes.txt''
for line in open(fileLoc):
    prime = int(line)
    print(prime * prime)
\end{lstlisting}}
\warn{Data in the file is read as a {\bf string} line by line.}\\
%*********************************************
\hr
%\end{enumerate}
\end{multicols*}

\end{document}
