\documentclass{exam}
\usepackage[utf8]{inputenc}
\usepackage{verbatim}
\begin{document}
\begin{questions}
    \question What is the output of the below code?
    \begin{verbatim}
        x = 5
        if true:
            print(x)
    \end{verbatim}
    \begin{oneparchoices}
        \choice \textbf{It will throw an Error} \\
        \choice It will display 5 \\
        \choice Nothing gets displayed
    \end{oneparchoices}

    \question What will be the output of below code?
    \begin{verbatim}
        a = print("hello")
    \end{verbatim}
    \begin{oneparchoices}
        \choice It will throw an Error \\
        \choice \textbf{It will display hello} \\
        \choice Nothing gets displayed
    \end{oneparchoices}
    \question Choose a statement to use Python If Else statement

    \begin{oneparchoices}
        \choice else if is compulsory to use with if statement. \\
        \choice else is compulsory to use with if statement. \\
        \choice \textbf{else or else if is optional with if statement.} \\
        \choice None of the above
    \end{oneparchoices}
    \question What is the output of the below code?

    \begin{verbatim}
        if 4 > 5:
            print("Hurray!!!")
        elif 4 > 2:
            print("OOps")
    \end{verbatim}
    \begin{oneparchoices}
        \choice It will throw an Error \\
        \choice It will display Hurray!!! \\
        \choice \textbf{It will display OOps} \\
        \choice Nothing gets displayed
    \end{oneparchoices}

    \question What is the output of the below code?

    \begin{verbatim}
        i = 5
        if i == i - 5 > 4:
            print("Hey, You are great")
    \end{verbatim}
    \begin{oneparchoices}
        \choice \textbf{Nothing gets displayed} \\
        \choice It will thrown an Error \\
        \choice It will display Hey, You are great \\
    \end{oneparchoices}

    \question What is the output of the below code?

    \begin{verbatim}
        a = 100
        if a > 40:
            print("Cricket")
        elif a > 60:
            print("Tennis")
        elif a > 80:
            print("Table Tennis")
    \end{verbatim}

    \begin{oneparchoices}
        \choice Nothing gets displayed \\
        \choice It will throw an Error \\
        \choice It will display Cricket, Tennis, Table Tennis \\
        \choice \textbf{It will display Cricket}
    \end{oneparchoices}

    \question What is the output of the below code?

    \begin{verbatim}
        a = 4
        b = 5
        c = 6
        if a * b // c:
            print("Great Job!!")
        else:
            print("Try Again!!")
    \end{verbatim}

    \begin{oneparchoices}
        \choice It will throw an Error \\
        \choice It will display Try Again!! \\
        \choice Nothing gets displayed \\
        \choice \textbf{It will display Great Job!!}
    \end{oneparchoices}
    
    \question What is the output of the below code?

    \begin{verbatim}
        if ('\0'):
            print("Hello")
        else:
            print("Hi")
    \end{verbatim}
    \begin{oneparchoices}
        \choice It will throw an Error \\
        \choice It will display Hi \\
        \choice \textbf{It will display Hello} \\
        \choice Nothing will get displayed
    \end{oneparchoices}

    \question What is the output of the below code?

    \begin{verbatim}
        fruit = 'Orange'
        if (fruit == 'orange'):
            print("Cool Fruit")
        else:
            print("Good Fruit")
    \end{verbatim}

    \begin{oneparchoices}
        \choice Nothing will be displayed \\
        \choice It will throw an Error \\
        \choice It will display Cool Fruit \\
        \choice \textbf{It will display Good Fruit}
    \end{oneparchoices}

    \question How many choices are possible when using a single if-else statement?

    \begin{oneparchoices}
        \choice 1 \\
        \choice 3 \\
        \choice 5 \\
        \choice \textbf{2}
    \end{oneparchoices}

    \question What is the output of the below code?
    
    \begin{verbatim}
        average = 4
        if average > 5:
            print("Over")
        elif average < 2:
            print("Under")
        print("Limit")
    \end{verbatim}

    \begin{oneparchoices}
        \choice It will throw an Error \\
        \choice \textbf{It will display Limit} \\
        \choice It will display Over \\
        \choice It will display Under
    \end{oneparchoices}

    \question What is the output of the below code?

    \begin{verbatim}
        fruitsCount = 88
        if fruitsCount < 100:
            fruitsCount = fruitsCount - 50
            if fruitsCount > 30:
                print(fruitsCount)
        else:
            print(fruitsCount)
    \end{verbatim}

    \begin{oneparchoices}
        \choice \textbf{38} \\
        \choice 88 \\
        \choice 50 \\
        \choice 75
    \end{oneparchoices}

    \question What is the output of the below code?

    \begin{verbatim}
        x = 9
        if x < 10:
            x = x + 5
        else:
            x = x - 5
        print(x)
    \end{verbatim}

    \begin{oneparchoices}
        \choice 9 \\
        \choice 15 \\
        \choice \textbf{14} \\
        \choice 4
    \end{oneparchoices}

    \question What is the output of the below code?

    \begin{verbatim}
        if 1:
            print("1 is truthy!")
        else:
            print("???")
    \end{verbatim}

    \begin{oneparchoices}
        \choice It will throw an Error \\
        \choice It will display ??? \\
        \choice Nothing will get displayed \\
        \choice \textbf{It will display 1 is truthy!}
    \end{oneparchoices}

    \question What is the output of the below code?

    \begin{verbatim}
        if 0:
            print("huh?")
        else:
            print("0 is falsy!")
    \end{verbatim}

    \begin{oneparchoices}
        \choice It will throw an Error \\
        \choice It will display huh? \\
        \choice Nothing will get displayed \\
        \choice \textbf{It will display 0 is falsy!}
    \end{oneparchoices}

    \question What is the output of the below code?

    \begin{verbatim}
        hair_color = "red"
        if 3 > 2:
            if hair_color == "pink":
                print("You rock!")
            else:
                print("Boring")
    \end{verbatim}

    \begin{oneparchoices}
        \choice It will display You rock! \\
        \choice \textbf{It will display Boring} \\
        \choice It will throw an Error \\
        \choice Nothing will get displayed
    \end{oneparchoices}

    \question What is the output of the below code?

    \begin{verbatim}
        if 5 > 10:
            print("fan")
        elif 8 != 9:
            print("glass")
        else:
            print("cream")
    \end{verbatim}

    \begin{oneparchoices}
        \choice It will display fan \\
        \choice It will display cream \\
        \choice \textbf{It will display glass} \\
        \choice Nothing gets displayed
    \end{oneparchoices}

    \question What is the output of the below code?

    \begin{verbatim}
        if 2 == 2:
            print("ice cream is tasty!")
    \end{verbatim}

    \begin{oneparchoices}
        \choice \textbf{It will display ice cream is tasty!} \\
        \choice It will throw an Error \\
        \choice Nothing gets displayed
    \end{oneparchoices}

    \question In a Python program, a control structure:

    \begin{oneparchoices}
        \choice Defines program-specific data structures \\
        \choice Manages the input and output of control characters \\
        \choice Dictates what happens before the program starts and after it terminates \\
        \choice \textbf{Directs the order of execution of the statements in the program}
    \end{oneparchoices}

    \question What signifies the end of a statement block or suite in Python?
    \begin{oneparchoices}
        \choice end \\
        \choice \[ \\
        \choice \textbf{A line that is indented less than the previous line} \\
        \choice A comment
    \end{oneparchoices}

    \question What is the output of the below code?

    \begin{verbatim}
        a, b = 4, 5
        if a < b:
            m = a
        m = b
        print(m)
    \end{verbatim}

    \begin{oneparchoices}
        \choice It will output 5
        \choice \textbf{It will output 4}
        \choice Nothing gets printed
        \choice It will output 4 and 5
    \end{oneparchoices}

\end{questions}
\end{document}
