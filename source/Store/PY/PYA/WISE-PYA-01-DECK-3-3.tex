\documentclass[14pt]{beamer}
\title{Python 101 :: Unit 1}
\subtitle{Session 3}
\date{}
\author[TS]{TalentSprint}
\usefonttheme{serif}
\usepackage{bookman}
\usepackage{hyperref}
\usepackage[T1]{fontenc}
\usepackage{graphicx}
\usecolortheme{orchid}
\beamertemplateballitem

\usepackage{listings}
%\usebackgroundtemplate{\includegraphics[width=\paperwidth]{logo}}
\definecolor{mygreen}{rgb}{0,0.6,0}
\definecolor{mygray}{rgb}{0.5,0.5,0.5}
\definecolor{mymauve}{rgb}{0.58,0,0.82}

\lstset{ %
  backgroundcolor=\color{white},   % choose the background color
  basicstyle=\small,        % size of fonts used for the code
  breaklines=true,                 % automatic line breaking only at whitespace
  captionpos=b,                    % sets the caption-position to bottom
  commentstyle=\color{mygreen},    % comment style
  %escapeinside={\%*}{*)},          % if you want to add LaTeX within your code
  keywordstyle=\color{blue},       % keyword style
  stringstyle=\color{mymauve},     % string literal style
  showstringspaces=false,
}
   

\begin{document}
    \begin{frame}
        \titlepage
    \end{frame}
    \begin{frame}
        \frametitle{Topics for this Session}
        \begin{itemize}
            \item Datatypes
            \item Building Blocks of Python
        \end{itemize}
    \end{frame}
    \begin{frame}
        \frametitle{Datatypes}
        \begin{itemize}
            \item Classification:
                \begin{itemize}
                    \item Mutable
                    \item Immutable
                \end{itemize}
            \item Numeric
            \item Boolean
            \item String
        \end{itemize}
    \end{frame}
    \begin{frame}
        \frametitle{Mutable Verus Immutable}
        \begin{itemize}
            \item Everything in Python is a object
            \item A mutable object can change its state or contents and immutable objects cannot.
        \end{itemize}
    \end{frame}
    \begin{frame}
        \frametitle{Numeric Types}
        Python supports four different numeric types:
        \begin{itemize}
            \item int (signed integers)
                \begin{description}
                    \item [Example:]  0, 3, 567, -666
                \end{description}
            \item long (can also be represented in octal and hexadecimal)
                \begin{description}
                    \item [Example:] 5555555555555555555555555555L
                \end{description}
            \item float (floating point real numbers)
                \begin{description}
                    \item [Example:] 1.0, 455.78, 888888.888
                \end{description}
            \item complex
                \begin{description}
                    \item [Example:] 3 + 4j, 5J
                \end{description}
        \end{itemize}
    \end{frame}
    \begin{frame}
        \frametitle{Boolean Types}
        \begin{itemize}
            \item Type of built-in values true, false
            \item \textbf{Example: } $14 == 2$
        \end{itemize}
    \end{frame}
    \begin{frame}
        \frametitle{Strings}
        \begin{itemize}
            \item Strings are enclosed in single or double quotation marks.
            \item \textbf{Example:}  "apple", 'hello'
        \end{itemize}
    \end{frame}
    \begin{frame}
      \frametitle{Building Blocks of Python}
      \begin{itemize}
                    \item Variables and Naming convention rules
                    \item Keywords
                    \item Operators
                    \item Statements
                    \item Comments
                \end{itemize}
    \end{frame}
    \begin{frame}
        \frametitle{Variables and Naming convention rules}
        \begin{itemize}
            \item Must start with a alphabet (a-z A-Z) or underscore (\_)
            \item Other characters can be alphabets, digits or underscores
            \item Very case sensitive
            \item Meaningful and readable names for identifiers
        \end{itemize}
    \end{frame}
    \begin{frame}
        \frametitle{Keywords}
        \begin{itemize}
            \item Reserved identifiers
            \item Cannot be used as identifiers in program
            \item Some of them are
                \begin{itemize}
                    \item and, as, assert, break, class, continue, def, del
                    \item elif, else, except, finally, for, from, global
                    \item if, import, in, is, lambda, not, or, pass, raise, return
                    \item try, while, with, yeild, True, False, None
                \end{itemize}
        \end{itemize}
    \end{frame}
    \begin{frame}
        \frametitle{Operators}
        \begin{itemize}
            \item Arithmetic
            \item Logical
            \item Comparison
        \end{itemize}
    \end{frame}
    \begin{frame}
        \frametitle{Operators}
        \begin{itemize}
            \item Chaining comparison - a$<$b$>$c
            \item Membership
                \begin{itemize}
                    \item a is b / a is not b
                    \item a in b / a not in b
                \end{itemize}
            \item Logical and membership
                \begin{itemize}
                    \item a and b
                    \item a or b
                    \item a not b
                \end{itemize}
        \end{itemize}
    \end{frame}
    \begin{frame}
        \frametitle{Statements}
        \begin{itemize}
            \item Assigning Values
            \item Storing result of any operations
            \item Conversions
        \end{itemize}
    \end{frame}
    \begin{frame}
        \frametitle{Comments}
        \begin{itemize}
            \item \emph{\#} - for single line comments
            \item \emph{```comment```} - for multi-line comments
        \end{itemize}
    \end{frame}
    \begin{frame}
        \begin{center}
            \emph{Let's Practice!!!}
        \end{center}
    \end{frame}
\end{document}
