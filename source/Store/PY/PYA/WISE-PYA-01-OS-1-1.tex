\documentclass{article}
\usepackage{listings}
\usepackage{color}
\usepackage[english]{babel}
\usepackage[utf8]{inputenc}
\usepackage{fancyhdr}
\usepackage{graphicx}
\pagestyle{fancy}
\fancyhf{}
\rhead{\includegraphics[width=2cm, height=1cm]{logo}}
\lhead{Python Programming :: Unit 1 :: WorkSheet}
\lfoot{COPYRIGHT ©TALENTSPRINT, 2020. ALL RIGHTS RESERVED.}
\rfoot{\thepage}
\begin{document}

\section{Introduction: Workbook}

\subsection{Review Questions}

\paragraph{1. The \rule{2cm}{0.15mm} types in Python are the built-in types for which we cannot change their content once created.}

\paragraph{2. \texttt{int} and \texttt{float} data types are Immutable type. State True / False.}

\paragraph{3. In python bool data type is a subclass of \rule{1cm}{0.15mm}.}

\paragraph{4. The bool data type values True and False are represented as \rule{2cm}{0.15mm} and \rule{2cm}{0.15mm}.}

\paragraph{5. \rule{2cm}{0.15mm} is a mutable data type.}

\paragraph{6. Write a statement to assign the value 10 and 20 to a and b variables.\\
 \rule{8cm}{0.15mm}.}

\paragraph{7. Write a statement to assign the value 50 to a, b and c variables.
\rule{8cm}{0.15mm}.}

\paragraph{ 8. x = “65”. Write a statement to convert x to int type.\\
 \rule{2cm}{0.15mm}.}


\subsection{Exercises}
\paragraph{• Write the expected output, or errors if any, for each of the following programs in the box provided below each program.}
\paragraph{• Then execute the programs and check your answers.}
\paragraph{• Also answer the questions given below.}
\subsubsection{Program 1}
\begin{verbatim}
1  x = 10
2  y = 20
3  print ( x )
4  print ( y )
\end{verbatim}
\subsubsection{Program 2}
\begin{verbatim}
1  a = 40
2  b = 50
3  c = a + b
4  print ( c )
\end{verbatim}

\subsubsection{Program 3}
\begin{verbatim}
1  x = 30
2  y = 30
3  print ( x == y )
\end{verbatim}
\paragraph{\textbf{Q1:} What will be the output if the value of x is greater than y.}
\paragraph{\textbf{Q2:} What will be the output if the value of x is less than y.}

\subsubsection{Program 4}
\begin{verbatim}
1  a = 45
2  b = 65
3. a, b = b, a
3  print(a, b)
\end{verbatim}

\subsubsection{Program 5}
\begin{verbatim}
1  a = 37
2  b = a / 3
3  print (b)
\end{verbatim}
\paragraph{\textbf{Q1:} What will be the output, if line 2 is replaced by b = a // 3?}

\subsubsection{Program 6}
\begin{verbatim}
1  n = 5 
2  r = n * 5
3  print (r)
\end{verbatim}
\paragraph{\textbf{Q1:}What will be the output, if line 2 is replaced with r = n ** 5?}

\subsubsection{Program 7}
\begin{verbatim}
1  a = 242
2  print( a % 10 == a // 1 0 0 )
\end{verbatim}

\paragraph{\textbf{Q1:}What will be the output, if line 2 is replaced with }
\begin{verbatim}
print(a % 10 == a / 100)?}
\end{verbatim}

\subsubsection{Program 8}
\begin{verbatim}
1  m1= 65 
2  m2 = 65
3  m3 = 65
2  print(m1 >= 65 and m2 >= 65 and m3 >= 6 5 )
\end{verbatim}
\paragraph{\textbf{Q1:} What will be the output if m2 = 60?}
\paragraph{\textbf{Q2:} What will be the output if values of m1, m2, and m3 are 50?}
\paragraph{\textbf{Q3:} What will be the output if in line 4 and is replaced by \texttt{or}?}
\paragraph{\textbf{Q4:} What will be the output if in line 4, and is replaced by \texttt{or}, as well as values 50, 40 and 70 for m1, m2 and m3?}

\subsection{Additional Exercises}
\paragraph{1. Write a program to calculate total marks and average of three subject marks.}
\paragraph{2. Write a program that prints the power of one number to the other.}
\paragraph{3. Write a program to print the sum of all individual digits of any three digits number.}
\paragraph{4. Write a program which prints \texttt{True} if the number is Arrmstrong (OnlyThree Digits Numbers).}
\end{document}

