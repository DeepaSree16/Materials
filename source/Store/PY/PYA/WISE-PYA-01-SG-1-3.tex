\documentclass[]{book}
\usepackage[english]{babel}
\usepackage[utf8]{inputenc}
\usepackage{fancyhdr}
\usepackage{graphicx}
\pagestyle{fancy}
\fancyhf{}
\rhead{\includegraphics[width=2cm, height=1cm]{logo}}
\lhead{Python Programming :: Unit 1}
\lfoot{COPYRIGHT ©TALENTSPRINT, 2020. ALL RIGHTS RESERVED.}
\rfoot{\thepage}
\begin{document}
    
    \section*{History}
        Python was developed in the late 1980’s by Guido van Rossum. The first release was version 0.90 in February, 1991, at the National Research Institute for Mathematics and Computer Science in the Netherlands. The Institute is more popularly known by the Dutch name and initials Centrum Wiskunde and Informatica–CWI.
        
        It is derived from many languages like ABC, C, Unix shell and other scripting languages. But its primary influence in the early stages was ABC and the desire to design an easy to learn, first language for learning programming.
        
        Since 2001, the Python Software Foundation (PSF) a non-profit organization, created specifically to own Python-related Intellectual Property, directs the development, evangelising and directions of the language.
        
        Guido van Rossum continues to be the prime mover and is fondly referred to as BDFL–Benevolent Dictator For Life
    
    \section*{What is Python?}
    Python is a high-level, interpreted, interactive and object-oriented language. Its hallmark is an elegant syntax that enables writing very easy to read programs.
    \subsection*{Notable features}
    \begin{itemize}
        \item Uses an elegant syntax, making the programs you write easier to read.
        \item Makes it easy to get your program working – ideal for prototype development and other ad-hoc programming tasks, without compromising maintainability.
        \item Comes with a large standard library that supports many common programming tasks such as connecting to web servers, searching text with regular expressions, reading and modifying files. Often referred to as batteries included feature of Python.
        \item Runs on many different computers and operating systems: Windows, MacOS, many brands of Unix, OS/2, ...
        \item Has excellent Unicode support.
    \end{itemize}
    \section* {Language characteristics}
        \begin{itemize}
            \item All basic data types are available: numbers (floating point, complex, and unlimited-length
            \item Higher level containers such as lists, dictionaries and sets are part of the core language.
            \item Python supports object-oriented programming with classes and multiple inheritance.
            \item Modules and packages are the mechanisms to design, build, and distribute applications and libraries.
            \item Exception handling is available.
            \item Data types are strongly and dynamically typed. Mixing incompatible types (e.g. attempting to
            \item add a string and a number) causes an exception to be raised, so errors are caught sooner.
            \item Provides advanced programming features such as generators and list comprehensions.
            \item Automatic memory management frees you from having to manually allocate and free memory in your code.
        \end{itemize}
    \section*{Applications}
    Python has now become a widely used professional language; it is used by organizations such as Google, NASA, Industrial Light and Magic, Cerenova, ABN Amro Bank . . .
    
    The following is a partial list of tools, frameworks and applications developed in python
    \begin{description}
        \item [Web Development Python] offers many choices for web development:
        \begin{itemize}
            \item Full-stack frameworks such as Django, Pyramid, and Zope.
            \item Micro-frameworks such as Flask and Bottle.
            \item Advanced content management systems such as Plone.
            \item Python’s standard library supports many Internet protocols:
                \begin{itemize}
                    \item HTML, XML, JSON.
                    \item E-mail processing.
                    \item FTP, IMAP, sockets . . .
                \end{itemize}
            \item Requests, a powerful HTTP client library.
            \item BeautifulSoup, a ‘fault-tolerant’ HTML parser.
            \item Feedparser for parsing RSS/Atom feeds.
            \item Paramiko implements the SSH2 protocol.
            \item Twisted Python, for asynchronous network programming.
        \end{itemize}
    \item [Scientific Computation Python] is widely used in scientific and engineering computing. In fact it has become the de-facto standard toolkit for such work.
        \begin{itemize}
            \item SciPy is the collection packages for mathematics, science, and engineering.
            \item Pandas is a data analysis and modeling library.
            \item IPython is currently leading the effort at providing an interactive computation and exploration platform.
        \end{itemize}
    \item [Education Python] is ideal for teaching programming, both at the introductory level and in more advanced courses. The most famous of the MIT’s courses, 6.00x, switched to using Python a few years back.
    \item [GUI Toolkits] Python has wide variety of graphical interface libraries.
        \begin{itemize}
            \item The Tk GUI library is included with Python.
            \item wxWidgets is a powerful big multiplatform gui toolkit
            \item Kivy, for writing multitouch applications, on say Android
            \item Qt can be used via pyqt or pyside libraries
        \end{itemize}
    \end{description}
\end{document}
