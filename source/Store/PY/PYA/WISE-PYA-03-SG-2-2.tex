\documentclass{book}
\title{Strings}
\date{}
\author{TS}
\usepackage[utf8]{inputenc}
\usepackage{listings}
\usepackage{color}
\usepackage{graphicx}
\usepackage{booktabs}
\usepackage{fancyhdr}
\addtolength{\headheight}{1.5cm} % make more space for the header
\pagestyle{fancyplain} % use fancy for all pages except chapter start
\rhead{\includegraphics[height=1.3cm]{TS.png}} % right logo
\renewcommand{\headrulewidth}{0pt} % remove rule below header

\begin{document}

\section*{Functions}

\hrulefill

\paragraph{A function is a block of organized, reusable code that is used to perform a single, related action.
Functions provide better modularity for your application and a high degree of code reuse.}
\paragraph{As you already know, Python gives you many built-in functions like \textit{print()}. You can also
create your own functions. These functions are called \textit{user-defined} functions.}

\subsection*{Defining a Funcion}
\paragraph{You can define functions to provide the required functionality. Here are simple rules to define a
function in Python.}

\begin{itemize}
\item Function blocks begin with the keyword \textit{def} followed by the function name and parentheses ().
\item Any input parameters or arguments should be placed within these parentheses.
\item The code block within every function starts with a colon (:) and is indented.
\item The statement \textit{return [expression]} exits a function, optionally passing back an expression to the
caller. A \textit{return} statement with no arguments is the same as return None.
\end{itemize}
\paragraph{Syntax}
\begin{verbatim}
def functionName():
    statement1
    statement2
    return [expression]
\end{verbatim}

\subsection*{Calling a Function}
\paragraph{Defining a function only gives it a name, specifies the parameters that are to be included in the
function and structures the blocks of code.}
\paragraph{Once the basic structure of a function is finalized, you can execute it by calling it from another
function or directly from the Python prompt.}
\begin{verbatim}
>>> functionName()
\end{verbatim}

\paragraph{Example 1: Function to print the sum of all even numbers in a given range.}

\begin{verbatim}
def isEven(num):
    return (num % 2 == 0)

def sumOfEvenNumbers(st_val, limit):
    sumEvenNums = 0
    for num in range(st_val, limit+1):
        if (isEven(num)):
            sumEvenNums += num
    return sumEvenNums
    
print(sumOfEvenNumbers(100, 999))
\end{verbatim}

\begin{figure}[h!]
  \includegraphics[width=\linewidth]{sumOfEvenNumbers.png}
  \caption{sumOfEvenNumbers program output}
  \label{fig:sumOfEvenNumbers}
\end{figure}

\paragraph{Example 2: Function to print all palindromes with all even digits in a given range}

\begin{verbatim}
def reverse(num):
    rev_num = 0
    while (num > 0):
        rem = num % 10
        rev_num = (rev_num * 1 0 ) + rem
        num //= 10
    return rev_num

#functio n to check the number is palindrome 
def isPalindrome(num):
    return(reverse(num) == num)

#function to check all digits are even
def allEvenDigits(num):
    while(num > 0):
        rem = num % 10
        if (rem % 2 != 0):
            return False
        num //= 10
    return True

def allPalindromes(n1,n2):
    for num in range(n1, n2 + 1):
        if (isPalindrome(num) and allEvenDigits(num)):
            print(num)

# function call
allPalindromes (100 ,999)
\end{verbatim}

\begin{figure}[h!]
  \includegraphics[width=\linewidth]
   {allEvenDigitPalindromes.png}
  \caption{allEvenDigitPalindromes program output}
  \label{fig:allEvenDigitPalindromes output}
\end{figure}


\subsection *{Returning Multiple Values}
\paragraph{A function can return exactly one value, or we should better say one object. An object can be a value of any type, integer, float or boolean. A function can also return a list or a tuple. So, if we have to return more than one value, we can use list or tuple for returning multiple values.}
\paragraph{Example:}
\paragraph{A program to print all two digit perfect square numbers.}

\begin{verbatim}
def isPerfectSquare(n):
    ’’’ This function returns
      a number and a boolean value’’’
      f = 1
    while f * f < n :
        f += 1
    return f, f * f == n
    
def generatePerfectSquares(LO, HI):
    for num in range(LO, HI):
        val, status = isPerfectSquare(num)
        if status :
            print(num)

generatePerfectSquares(10, 100)
\end{verbatim}

\begin{figure}[h!]
  \includegraphics[width=\linewidth]
   {perfectSquares.png}
  \caption{perfectSquares program output}
  \label{fig:perfectSquaresoutput}
\end{figure}

\end{document}
