\documentclass{exam}
\usepackage[utf8]{inputenc}
\begin{document}
\begin{questions}
    \question  What is the output of this code? \\
    \begin{verbatim}
        a, b = 4, 5
        a = a ^ b
        b = b ^ a
        a = a ^ b
        print(a)
    \end{verbatim}
    \begin{oneparchoices}
        \choice \textbf{5} \\
        \choice 4 \\
        \choice 1 \\
    \end{oneparchoices}
    \question What is the name of datatype for character in Python?

    \begin{oneparchoices}
        \choice char \\
        \choice chr \\
        \choice character \\
        \choice \textbf{python do not have any datatype for character they are treated as strings} 
    \end{oneparchoices}

    \question Python is written in which language?

    \begin{oneparchoices}
        \choice \textbf{C} \\
        \choice C++ \\
        \choice Java \\
    \end{oneparchoices}

    \question What is used to define a block of code in Python?

    \begin{oneparchoices}
        \choice Parenthesis \\
        \choice \textbf{Indentation} \\
        \choice Curly braces \\
        \choice None of these
    \end{oneparchoices}

    \question What is the maximum possible length of an identifier?

    \begin{oneparchoices}
        \choice 31 Characters \\
        \choice 26 Characters \\
        \choice 150 Characters \\
        \choice \textbf{none of the mentioned}
    \end{oneparchoices}

    \question Which of the following is invalid?

    \begin{oneparchoices}
        \choice \_number \\
        \choice \_number\_ \\
        \choice \_number = 6 \\
        \choice \textbf{none of the mentioned}
    \end{oneparchoices} 

    \question keywords in Python should be Lowercase. Is this statement is right?

    \begin{oneparchoices}
        \choice True \\
        \choice \textbf{False}
    \end{oneparchoices}

    \question Which of the following is true for variable names in Python?

    \begin{oneparchoices}
        \choice \textbf{unlimited length} \\
        \choice all private members must have leading and trailing underscores \\
        \choice underscore and ampersand are the only two special characters allowed \\
        \choice none of the mentioned
    \end{oneparchoices}

    \question Which of the following is an invalid statement?

    \begin{oneparchoices}
        \choice number = 45 \\
        \choice a, b, c = 1, 2, 3 \\
        \choice abc = 123 \\
        \choice \textbf{a b c = 1 2 3}
    \end{oneparchoices}

    \quetsion Which of the following cannot be a variable?

    \begin{oneparchoices}
        \choice \_number\_ \\
        \choice \_number \\
        \choice number\\
        \choice \textbf{in}
    \end{oneparchoices}

    \question Which is the correct operator for power($x^ y$)?

    \begin{oneparchoices}
        \choice $x * y$\\
        \choice \textbf{$x ** y$}\\
        \choice $x ^ y$\\
        \choice $x ^^ y$
    \end{oneparchoices}

    \question Which one of these is floor division?

    \begin{oneparchoices}
        \choice \backslash\\
        \choice \textbf{\backslash \backslash}\\
        \choice \%\\
    \end{oneparchoices}

    \question What is the average value of the code that is executed below ?

    \begin{verbatim}
        number1 = 78
        number2 = 56
        number3 = (number1 + number2) / 2
    \end{verbatim}

    \begin{oneparchoices}
        \choice \textbf{67.0}\\
        \choice 78\\
        \choice 13\\
        \choice 56
    \end{oneparchoices}
    
    \quetsion What is the output of print 6.7 + 0.2 == 0.7?

    \begin{oneparchoices}
        \choice \textbf{False}\\
        \choice True
    \end{oneparchoices}

    \question Which of the following is not a complex number?

    \begin{oneparchoices}
        \choice 2 + 6J\\
        \choice k = 88 + 3j\\
        \choice number = complex(6,7)\\
        \choice \textbf{number = 77 + 8i} 
    \end{oneparchoices}

    \question Which of the following is incorrect?

    \begin{oneparchoices}
        \choice x = 0b101\\
        \choice x = 0x4f5\\
        \choice x = 99999999999999L\\
        \choice \textbf{x = 0\_3964}
    \end{oneparchoices}

    \question What does $2 ^ 21$ evaluate to?

    \begin{oneparchoices}
        \choice \textbf{23}\\
        \choice 21\\
        \choice 42\\
        \choice 441
    \end{oneparchoices}

    \question The value of the expressions 4/(3*(2-1)) and 4/3*(2-1) is the same. State whether true or false.

    \begin{oneparchoices}
        \choice \textbf{True}\\
        \choice False\\
        \choice Maybe\\
        \choice Can't say
    \end{oneparchoices}

    \question The value of the expression: 4 + 3 \% 5

    \begin{oneparchoices}
        \choice 4\\
        \choice 2\\
        \choice \textbf{7}\\
        \choice 0
    \end{oneparchoices}

    \question What is the output of the following expression:

            print(4.00 / (2.0 + 2.0))

     \begin{oneparchoices}
         \choice 1.00\\
         \choice \textbf{1.0}\\
         \choice Error\\
         \choice 1\\
     \end{oneparchoices}

     \question Consider the expression given below. The value of X is:

            X = 2 + 9 * ((3 * 12)- 8) / 10

     \begin{oneparchoices}
         \choice \textbf{27.2}\\
         \choice 30.0\\
         \choice 30.8\\
         \choice 27.5
     \end{oneparchoices}

     \question What is the value of the expression:

            float(4 + int(2.39) \% 2)

     \begin{oneparchoices}
         \choice 5.0 \\
         \choice 5 \\
         \choice 4 \\
         \choice \textbf{4.0}
     \end{oneparchoices}

     \question What is the value of the expression:

            $4 + 2 ** 5 // 10$

     \begin{oneparchoices}
         \choice 77 \\
         \choice \textbf{7} \\
         \choice 3 \\
         \choice 0 \\
     \end{oneparchoices}

     \question The expression 2 ** 2 ** 3 is evaluates as: (2 ** 2) ** 3. State whether this statement is true or false.

     \begin{oneparchoices}
         \choice True \\
         \choice \textbf{False} \\
         \choice Maybe \\
         \choice Can't Say
     \end{oneparchoices}

     \question The output of the snippet of code shown below?

            bool()
     \begin{oneparchoices}
         \choice True \\
         \choice \textbf{False} \\
         \choice Error
     \end{oneparchoices}

     \question What is the output of below expression?

           $not(44 < 5)$

     \begin{oneparchoices}
         \choice \textbf{True} \\
         \choice False
     \end{oneparchoices}

     \question The output of the line of code shown below is:

            $not(10 < 20) and not(10 > 30)$

     \begin{oneparchoices}
        \choice True \\
        \choice \textbf{False}\\
        \choice Error\\
        \choice No output
     \end{oneparchoices}

     \question What is the output of this expression, $3 * 1 ** 3$?

     \begin{oneparchoices}
         \choice \textbf{3} \\
         \choice 1 \\
         \choice 9
         \choice 33

     \end{oneparchoices}

     \question What are the values of the following expressions:

     \begin{verbatim}
        2 ** (3 ** 2)
        
        (2 ** 3) ** 2
        
        2 ** 3 ** 2
     \end{verbatim}

     \begin{oneparchoices}
         \choice 64, 512, 64
         \choice 64, 64, 64
         \choice 512, 512, 512
         \choice \textbf{512, 64, 512}
     \end{oneparchoices}

     \question What is the result of the expression (x \& y) if x=15 and y=12

     \begin{oneparchoices}
         \choice b1101
         \choice b1101
         \choice \textbf{12}
         \choice 1101
     \end{oneparchoices}

     \question Which of the following represents the bitwise XOR operator?

     \begin{oneparchoices}
         \choice \textbf{\^}
         \choice |
         \choice ||
     \end{oneparchoices}

     \question Bitwise \_\_\_\_\_\_\_\_ gives 1 if either of the bits is 1 and 0 when both of the bits are 1.

     \begin{oneparchoices}
         \choice OR
         \choice AND
         \choice \textbf{XOR}
         \choice NOT
     \end{oneparchoices}

     \question The result of the expression $4 ^ 12$

     \begin{oneparchoices}
         \choice 2
         \choice 3
         \choice \textbf{8}
         \choice 48
     \end{oneparchoices}


\end{questions}
\end{document}
