\documentclass[14pt]{beamer}
\title{Python 101 :: Unit 2}
\subtitle{Session 1}
\date{}
\author[TS]{TalentSprint}
\usefonttheme{serif}
\usepackage{bookman}
\usepackage{hyperref}
\usepackage[T1]{fontenc}
\usepackage{graphicx}
\usecolortheme{orchid}
\beamertemplateballitem

\usepackage{listings}
%\usebackgroundtemplate{\includegraphics[width=\paperwidth]{logo}}
\definecolor{mygreen}{rgb}{0,0.6,0}
\definecolor{mygray}{rgb}{0.5,0.5,0.5}
\definecolor{mymauve}{rgb}{0.58,0,0.82}

\lstset{ %
  backgroundcolor=\color{white},   % choose the background color
  basicstyle=\small,        % size of fonts used for the code
  breaklines=true,                 % automatic line breaking only at whitespace
  captionpos=b,                    % sets the caption-position to bottom
  commentstyle=\color{mygreen},    % comment style
  %escapeinside={\%*}{*)},          % if you want to add LaTeX within your code
  keywordstyle=\color{blue},       % keyword style
  stringstyle=\color{mymauve},     % string literal style
  showstringspaces=false,
}
   

\begin{document}
    \begin{frame}
        \titlepage
    \end{frame}

    \begin{frame}
        \frametitle{Topics for this Session}
        \begin{itemize}
            \item Vim Editor
            \item Creating, running Python files
        \end{itemize}
    \end{frame}

    \begin{frame}
        \frametitle{What is Vim?}
        \begin{itemize}
            \item It's a ``modal" text editor based on the vi editor written by Bill Joy in the 1970s for a version of UNIX 
        \end{itemize}
    \end{frame}

    \begin{frame}
        \frametitle{Modes of Vim}
        Vim has two basic modes. They are 
        \begin{itemize}
            \item \textbf{Insert mode} in which you write text as if in normal text editor.
            \item \textbf{Normal mode} which provides you efficient ways to navigate and manipulate text.
        \end{itemize}
        \textbf{Note:} To change between modes, use \emph{Esc} for normal mode and \emph{i} for insert mode
    \end{frame}

    \begin{frame}
        \frametitle{How to use Vim?}
        \begin{itemize}
            \item Go to Terminal. To open an existing file or to create a new file use the below command: \\
                \begin{lstlisting}
                    vim filename.py
                \end{lstlisting}
            \item To enter Insert Mode simply type: \emph{i} and you write your code in the file
            \item To save the file and to quit from the vim use below command:
                \begin{lstlisting}
                    :wq
                \end{lstlisting}
            \item If you want to quit from the vim without saving the file name then use below command
                \begin{lstlisting}
                    :q!
                \end{lstlisting}
        \end{itemize}
    \end{frame}
    \begin{frame}
        \frametitle{Creating, Running Python files}
        \begin{itemize}
            \item Open the file using \emph{vim HelloWorld.py} and type the following line and save the file.
                \begin{lstlisting}[language=Python]
                    print("HelloWorld!!")
		        \end{lstlisting}
            \item Now type the below command in terminal and press Enter Key to Execute.
		        \begin{lstlisting}
                    python HelloWorld.py
		        \end{lstlisting}
        \end{itemize}
    \end{frame}
    \begin{frame}
        \frametitle{Problems}
        \begin{itemize}
            \item Write a program which takes two numbers as input and displays sum of those numbers
            \item Write a program to calculate area of circle
            \item Write a program to calculate area of triangle
            \item Write a program to calculate area of rectangle
        \end{itemize}
    \end{frame}
\end{document}
