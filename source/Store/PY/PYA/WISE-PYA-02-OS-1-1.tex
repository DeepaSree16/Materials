\documentclass{article}
\usepackage{listings}
\usepackage{color}
\usepackage{graphicx}
\usepackage{booktabs}
\usepackage{fancyhdr}
\usepackage[english]{babel}
\pagestyle{fancy}
\fancyhf{}
\rhead{\includegraphics[width=2cm, height=1cm]{logo}}
\lhead{Python Programming :: Unit 2}
\lfoot{COPYRIGHT ©TALENTSPRINT, 2020. ALL RIGHTS RESERVED.}
\rfoot{\thepage}
\begin{document}

\section{ifElse: Workbook}

\subsection{Exercises}
\paragraph{• Write the expected output, or errors if any, for each of the following programs in the box provided below each program.}
\paragraph{• Then execute the programs and check your answers.}
\paragraph{• Also answer the questions given below.}

\subsubsection{Program 1}
\begin{verbatim}
1  num = int(input(‘‘Enter The Number:’’))
2  if num % 2 == 0 :
3      print(‘‘Even Number’’)
4  else:
5      print(‘ ‘Odd Number’’)
\end{verbatim}

\framebox(400,50){}

\paragraph{Q1: What will be the output for a value 33?}
\paragraph{Q2: What will be the output for a value 56?}

\subsubsection{Program 2}
\begin{verbatim}
1  a = int (input(‘ ‘ Enter the First Number: ’ ’))
2  b = int (input(‘ ‘ Enter the Second Nummber: ’ ’ ))
3  c = int (input(‘ ‘ Enter the Third Number: ’ ’ ))
4  if a > b and a > c :
5      print ( ‘ ‘A is Greater’’)
6  elif b > c:
7      print (‘ ‘B is Greater’’)
8  else:
9      print ( ‘‘C is Greater’‘)
\end{verbatim}
\framebox(400,50){}

\paragraph{Q1: What will be the output if the value of a, b and c are 10, 20 and 30?}
\paragraph{Q2: What will be the output if the value of a, b and c are equal to 55?}
\paragraph{Q3: Modify the if statement without using the and operator.}

\subsubsection{Program 3}
\begin{verbatim}
1  num = 1
2  limit = 20
3  sum_num = 0
4  while num <= limit:
5      if num % 3 == 0:
6          sum_num += num
7      num += 1
8  print(sum_num)
\end{verbatim}

\framebox(400,50){}

\subsubsection{Program 4}
\begin{verbatim}
1  num = 0
2  limit = 20
3  sum_d = 0
4  while num <= limit:
5      num += 1
6      if num % 2 == 0:
7          continue
8      sum_d += num
9  print(sum_d)
\end{verbatim}

\framebox(400,50){}

\subsubsection{Program 5}
\begin{verbatim}
1 nums = [22, 33, 44, 55, 66, 77, 88, 99]
2 n = int(input(‘‘Enter The Number:’’))
3 if n in nums:
4     print(‘‘Element Exists’’)
5 else:
6     print(‘‘Element Does Not Exist’’)
\end{verbatim}

\framebox(400,50){}

\paragraph{Q1: What will be the output if the input value is 33?}
\paragraph{Q2: What will be the output if the input value is 100?}

\subsubsection{Additional Exercises}
\paragraph{1. Given a year, write a program to check if the year is a leap year or not.}
\paragraph{2. Given a month, write a program to find number of days in that month.}
\paragraph{3. Write a program to print the greatest value amoung three values.}
\paragraph{4. Write a program to check if a given number is a palidrome number.}
\paragraph{5. Write a program to print the collatz sequence for a given number.\\
\textit{Note: To get a Collatz sequence from a number, if it’s even, divide
it by two, and if it’s odd, multiply it by three and add one. Continue
the operation on the result of the previous operation until the number
becomes 1.}}
\paragraph{6. Write a program to find the product of two given numbers using Russian multiplication.}
\paragraph{7. Write a program to check if the given number is perfect number or not.}
\paragraph{8. Write a program to check if the given number is an prime number or not.}
\paragraph{9. Write a program to print the sum of all even digits in a given number.}

\end{document}
