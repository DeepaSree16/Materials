\documentclass[14pt]{beamer}
\title{Python 101 :: Unit 5}
\subtitle{Session 1}
\date{}
\author[TS]{TalentSprint}
\usefonttheme{serif}
\usepackage{bookman}
\usepackage{hyperref}
\usepackage[T1]{fontenc}
\usepackage{graphicx}
\usecolortheme{orchid}
\beamertemplateballitem

\usepackage{listings}
%\usebackgroundtemplate{\includegraphics[width=\paperwidth]{logo}}
\definecolor{mygreen}{rgb}{0,0.6,0}
\definecolor{mygray}{rgb}{0.5,0.5,0.5}
\definecolor{mymauve}{rgb}{0.58,0,0.82}

\lstset{ %
  backgroundcolor=\color{white},   % choose the background color
  basicstyle=\small,        % size of fonts used for the code
  breaklines=true,                 % automatic line breaking only at whitespace
  captionpos=b,                    % sets the caption-position to bottom
  commentstyle=\color{mygreen},    % comment style
  %escapeinside={\%*}{*)},          % if you want to add LaTeX within your code
  keywordstyle=\color{blue},       % keyword style
  stringstyle=\color{mymauve},     % string literal style
  showstringspaces=false,
}
   
\begin{document}
    \begin{frame}
        \titlepage
    \end{frame}
    \begin{frame}
        \frametitle{Topics for the session}
        \begin{itemize}
            \item Code reading and writing
        \end{itemize}
    \end{frame}
    \begin{frame}
        \frametitle{Problem 1}
        \alert{Given a list of integers, return the difference between the largest and smallest integers in the list.}
    \end{frame}
    \begin{frame}[containsverbatim]
        \frametitle{Problem 1 :: Solution}
        \begin{lstlisting}[language=Python]
            def difference(nums):
                return max(nums) - min(nums)
        \end{lstlisting}
    \end{frame}
    \begin{frame}
        \frametitle{Problem 2}
        \alert{Write a function that returns True if two rooks can attack each other, and False otherwise.}

        \begin{itemize}
            \item Assume no blocking pieces.
            \item Two rooks can attack each other if they share the same row (letter) or column (number).
        \end{itemize}
    \end{frame}

    \begin{frame}[containsverbatim]
        \frametitle{Problem 2 :: Solution}
        \begin{lstlisting}[language=Python]
            def can_capture(rooks):
                A, B = rooks
                return A[0] == B[0] or A[1] == B[1]
        \end{lstlisting}
    \end{frame}

    \begin{frame}
        \frametitle{Problem 3}
        \alert{Create a function that returns the minimum number of elements removed to make the sum of all elements in a list even.}
        
        \textbf{Note:} If the sum is already even, return 0
    \end{frame}

    \begin{frame}[containsverbatim]
        \frametitle{Problem 3 :: Solution}
        \begin{lstlisting}[language=Python]
            def minimum_removals(lst):
                return sum(lst) % 2
        \end{lstlisting}
    \end{frame}

    \begin{frame}
        \frametitle{Problem 4}
        \alert{Write a Program to count the number of strings where the string length is 2 or more and the first and last character are same from a given list of strings.}
    \end{frame}
    
    \begin{frame}[containsverbatim]
        \frametitle{Problem 4 :: Solution}
        \begin{lstlisting}[language=Python]
            def match_words(words):
                counter = 0
                for word in words:
                    if len(word) > 1 and word[0] == word[-1]:
                        counter += 1
                return counter
        \end{lstlisting}
    \end{frame}

    \begin{frame}
        \frametitle{Problem 5}
        \alert{Write a function that partitions the list into two sublists: one with all even integers, and the other with all odd integers. Return your result in the following format: [[evens], [odds]]}

        \begin{itemize}
            \item Return two empty sublists if the input is an empty list.
            \item Keep the same relative ordering as the original list.
        \end{itemize}
    \end{frame}

    \begin{frame}[containsverbatim]
        \frametitle{Problem 5 :: Solution}

        \begin{lstlisting}[language=Python]
            def even_odd_partition(lst):
                evens = []
                odds = []
                for ele in lst:
                    if ele % 2 == 0:
                        evens.append(i)
                    else:
                        odds.append(i)
            return [evens,odds]
        \end{lstlisting}
    \end{frame}
\end{document}
