\documentclass[14pt]{beamer}
\title{Python 101 :: Unit 5}
\subtitle{Session 3}
\date{}
\author[TS]{TalentSprint}
\usefonttheme{serif}
\usepackage{bookman}
\usepackage{hyperref}
\usepackage[T1]{fontenc}
\usepackage{graphicx}
\usecolortheme{orchid}
\beamertemplateballitem

\usepackage{listings}
%\usebackgroundtemplate{\includegraphics[width=\paperwidth]{logo}}
\definecolor{mygreen}{rgb}{0,0.6,0}
\definecolor{mygray}{rgb}{0.5,0.5,0.5}
\definecolor{mymauve}{rgb}{0.58,0,0.82}

\lstset{ %
  backgroundcolor=\color{white},   % choose the background color
  basicstyle=\small,        % size of fonts used for the code
  breaklines=true,                 % automatic line breaking only at whitespace
  captionpos=b,                    % sets the caption-position to bottom
  commentstyle=\color{mygreen},    % comment style
  %escapeinside={\%*}{*)},          % if you want to add LaTeX within your code
  keywordstyle=\color{blue},       % keyword style
  stringstyle=\color{mymauve},     % string literal style
  showstringspaces=false,
}
   
\begin{document}
    \begin{frame}
        \titlepage
    \end{frame}
    \begin{frame}
        \frametitle{Problem 1}
        \alert{Write a program which will find all such numbers which are divisible by 7 but are not a multiple of 5, between 2000 and 3200 (both included).}
    \end{frame}
    \begin{frame}[containsverbatim]
        \frametitle{Problem 1 :: Solution}
        \begin{lstlisting}[language=Python]
            def divisibleby7 ():
                numbersdivisibleby7 = []
                for number in range(2000, 3201):
                    if (number % 7 == 0) and (number % 5 != 0):
                        numbersdivisibleby7.append(number)
                return numbersdivisibleby7
        \end{lstlisting}
    \end{frame}

    \begin{frame}
        \frametitle{Problem 2}
        You're in the midst of creating a typing game.
        
        \alert{Create a function that takes in two lists: the list of user-typed words, and the list of correctly-typed words and outputs a list containing 1s (correctly-typed words) and -1s (incorrectly-typed words).}
    \end{frame}

    \begin{frame}[containsverbatim]
        \frametitle{Problem 2 :: Solution}
        \begin{lstlisting}[language=Python]
            def correct_stream(user, correct):
                for i in range(len(user)):
                    if user[i] != correct[i]: 
                        user[i] = -1
                    else: 
                        user[i] = 1
                return user
        \end{lstlisting}
    \end{frame}
\end{document}
