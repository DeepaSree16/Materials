\documentclass{exam}
\usepackage[utf8]{inputenc}
\begin{document}
\begin{questions}
    \question What Will Be The Output Of The Following Code Snippet?
    
    \begin{verbatim}
        fruits = ["apples", "oranges", "grapes", "mangoes","pineapple"]
        print(fruits[::1])
    \end{verbatim}

    \begin{oneparchoices}
        \choice "apple" \\
        \choice \textbf{["apples", "oranges", "grapes", "mangoes","pineapple"]} \\
        \choice ['pineapple', 'mangoes', 'grapes', 'oranges', 'apples']
    \end{oneparchoices}

    \question What Will Be The Output Of The Following Code Snippet?

    \begin{verbatim}
        numbers = [1, 2, 3, 4, 5, 6, 7, 8]
        print(numbers[3:0:-1])
    \end{verbatim}

    \begin{oneparchoices}
        \choice \textbf{[4, 3, 2]} \\
        \choice [1, 2, 3, 4, 5, 6, 7, 8] \\
        \choice [4 5 6]
    \end{oneparchoices}

    \question What is used to separate elements in a list?

    \begin{oneparchoices}
        \choice space \\
        \choice semicolon \\
        \choice \textbf{comma}
    \end{oneparchoices}

    \question What is a list in Python?

    \begin{oneparchoices}
        \choice A statement that can't be changed
        \choice A collection of variables
        \choice \textbf{A collection of values stored in a variable}
    \end{oneparchoices}

    \question Which code you will use to append values to a list?

    \begin{oneparchoices}
        \choice \textbf{variable.append()}
        \choice variable.extend()
        \choice variable.count()
    \end{oneparchoices}

    \question What is the index of Apple in the list?

    \begin{verbatim}
        fruits = ["Orange", "Apple", "Grapes"]
    \end{verbatim}
    
    \begin{oneparchoices}
        \choice 0 \\
        \choice \textbf{1} \\
        \choice 2
    \end{oneparchoices}

    \question What Will Be The Output Of The Following Code Snippet?

    \begin{verbatim}
        for num in range(1, 5):
            print(num)
    \end{verbatim}
    \begin{oneparchoices}
        \choice \textbf{Prints numbers from 1 to 5} \\
        \choice Prints numbers from 1 to 4 \\
        \choice Prints numbers from 2 to 4
    \end{oneparchoices}

    \question How do you delete a element from a list?
     
    \begin{oneparchoices}
        \choice variable.delete() \\
        \choice variable.append() \\
        \choice \textbf{varaible.remove()}
    \end{oneparchoices}

    \question List are indexed by an \_\_\_\_\_\_

    \begin{oneparchoices}
        \choice \textbf{Integer} \\
        \choice Strings \\
        \choice decimal \\
        \choice fractions
    \end{oneparchoices}

    \question A list is an \_\_\_\_\_\_ datatype

    \begin{oneparchoices}
        \choice \textbf{sequence} \\
        \choice decision \\
        \choice non-sequence
    \end{oneparchoices}

    \question List are \_\_\_\_\_

    \begin{oneparchoices}
        \choice Immutable \\
        \choice \textbf{Mutable} \\
        \choice Both
    \end{oneparchoices}

    \question What Will Be The Output Of The Following Code Snippet?

    \begin{verbatim}
        fruits = []
        fruits.append("apple")
        fruits.append("orange")
        fruits.append("grapes")
        print(fruits[-1])
    \end{verbatim}

    \begin{oneparchoices}
        \choice apple \\ 
        \choice orange \\
        \choice \textbf{grapes}
    \end{oneparchoices}

    \question What Will Be The Output Of The Following Code Snippet?

    \begin{verbatim}
        mix = [1, 2, 4, 77, -99, 5, "h", "e","l","m","n"]
        print(len(mix))
    \end{verbatim}

    \begin{oneparchoices}
        \choice 5 \\
        \choice 10 \\
        \choice \textbf{11}
    \end{oneparchoices}
\end{questions}
\end{document}
