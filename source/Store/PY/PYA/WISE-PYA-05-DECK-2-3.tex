\documentclass[14pt]{beamer}
\title{Python 101 :: Unit 5}
\subtitle{Session 2}
\date{}
\author[TS]{TalentSprint}
\usefonttheme{serif}
\usepackage{bookman}
\usepackage{hyperref}
\usepackage[T1]{fontenc}
\usepackage{graphicx}
\usecolortheme{orchid}
\beamertemplateballitem

\usepackage{listings}
%\usebackgroundtemplate{\includegraphics[width=\paperwidth]{logo}}
\definecolor{mygreen}{rgb}{0,0.6,0}
\definecolor{mygray}{rgb}{0.5,0.5,0.5}
\definecolor{mymauve}{rgb}{0.58,0,0.82}

\lstset{ %
  backgroundcolor=\color{white},   % choose the background color
  basicstyle=\small,        % size of fonts used for the code
  breaklines=true,                 % automatic line breaking only at whitespace
  captionpos=b,                    % sets the caption-position to bottom
  commentstyle=\color{mygreen},    % comment style
  %escapeinside={\%*}{*)},          % if you want to add LaTeX within your code
  keywordstyle=\color{blue},       % keyword style
  stringstyle=\color{mymauve},     % string literal style
  showstringspaces=false,
}
   
\begin{document}
    \begin{frame}
        \titlepage
    \end{frame}
    \begin{frame}
        \frametitle{Problem 1}
        \alert{Create a function that filters out a list to include numbers who only have a certain number of digits.}
        \begin{itemize}
            \item If no numbers of the specified digit length exist, return an empty list.
            \item If all numbers in the list have the specified digit length, return original list.
            \item The sub-list returned should have the same relative order as the original list.
        \end{itemize}
    \end{frame}
    \begin{frame}[containsverbatim]
        \frametitle{Probelm 1 :: Solution}
        \begin{lstlisting}[language=Python]
            def filter_digit_length(lst, num):
                new_list = []
                for item in lst:
                    if len(str(item)) == num:
                        new_list.append(item)
                return new_list
        \end{lstlisting}
    \end{frame}

    \begin{frame}
        \frametitle{Problem 2}
        \alert{Given a sorted list of numbers, remove any numbers that are divisible by 13. Return the amended list.}
    \end{frame}

    \begin{frame}[containsverbatim]
        \frametitle{Problem 2 :: Solution}
        \begin{lstlisting}[language=Python]
            def unlucky_13(nums):
                numdivisibleby13 = []
                for ele in nums:
                    if (ele % 13) != 0:
                        numdivisibleby13 = numdivisibleby13 + [ele]
                return numdivisibleby13
        \end{lstlisting}
    \end{frame}

    \begin{frame}
        \frametitle{Problem 3}
        \alert{Write a program to find factors of a number}
    \end{frame}

    \begin{frame}[containsverbatim]
        \frametitle{Problem 3 :: Solution}
        \begin{lstlisting}[language=Python]
            def get_factors(number):
                factors = []
                for factor in range(1, number + 1):
                    if number % factor == 0:
                        factors.append(factor)
                return factors
        \end{lstlisting}
    \end{frame}
\end{document}
