\documentclass{book}
\title{Strings}
\date{}
\author{TS}
\usepackage{listings}
\usepackage{color}
\usepackage{graphicx}

\usepackage{fancyhdr}
\addtolength{\headheight}{1.5cm} % make more space for the header
\pagestyle{fancyplain} % use fancy for all pages except chapter start
\rhead{\includegraphics[height=1.3cm]{TS.png}} % right logo
\renewcommand{\headrulewidth}{0pt} % remove rule below header


\begin{document}
\section*{Strings}

\hrulefill

\paragraph{
A string in Python is a sequence of characters. For Python to recognize a sequence of characters, like \textit{hello}, as a string, it must be enclosed in quotes. The string can be enclosed in single or double quotes. }

\paragraph{Example: }

\begin{verbatim}
>>> ‘‘Hello World’’ 
‘Hello World’ 
>>> ‘Hello World’ 
‘Hello World’ 
\end{verbatim}

\paragraph{Note: Python interpreter displays the string with single quotes.}

\paragraph{Having two sorts of quotes can be useful in certain circumstances. If you want text itself to include quotes of one type you can define it surrounded by other type. }

\paragraph{Example: }

\begin{verbatim}
>>> print(‘‘This is ‘Use of Single Quotes’.’’) 
This is ‘Use of Sungle Quotes’. 
>>> print(‘This is ‘‘Use of Double Quotes’’.’) 
This is ‘‘Use of Double Quotes’’. 
\end{verbatim}

\subsection*{Triple Quoted String Literals}

\paragraph{
Strings delimited by single or double quote character are required to lie within a single line. It is sometimes convenient to have a multi-line string, which can be delimited with triple quotes}

\paragraph{Example: }

\begin{verbatim}
>>> str1 = ’’’Hello
... "Good Morning!!!"
... Have a cup of coffee.’’’
>>> print(str1)
Hello
"Good Morning!!!"
Have a cup of coffee.
\end{verbatim}

\section-*{String Operations}

\subsection*{Concatenation}

\paragraph{The plus operation with strings means concatenate the strings. Python looks at the type of operands before deciding what operation is associated with the +.}

\paragraph{ The plus ( + ) sign is the string concatenation operator which is used to combine number of
strings and returns the new string.}

\paragraph{Example:}

\begin{verbatim}
>>> str1 = "Hello World"
>>> print(str1 + " ’Can Be Joined’") # Prints concatenated string
Hello World ’Can Be Joined’
\end{verbatim}

\subsection*{Repetition Operator}

\paragraph{The asterisk( * ) sign is the repetition operator which is use to repeat the string as many times as specified.}

\paragraph{Example:}

\begin{verbatim}
>>> str1 = "Hello World"
>>> print(str1 * 3)   # Prints string 3 times
Hello WorldHello WorldHello World
\end{verbatim}

\subsection*{Accessing Characters In String}

\paragraph{We can access a character from the string by specifying the index of the character. Index starts from
’0’ indicates the beginning of the string and working their way from -1 at the end.}

\paragraph{For Example:}

\begin{verbatim}
>>> str1 = "Strings In Python"
>>> print(str1)     # Prints complete string
Strings In Python
>>> print(str1[0])  # Prints first character of the string
S
>>> print(str1[5])  # Prints sixth character of the string
g
>>> print(str1[-1]) # Prints the last character of the string
n
>>> print(str1[-3]) # Prints the third character from the last
h
>>> print(str1[-8]) # Prints the eigth character from the last
n
\end{verbatim}

\subsection*{Slicing the String}

\paragraph{We can access the subsets of a string using slice operator ([:]) with indexes starting at 0 in the
beginning of the string and working their way from -1 at the end.}

\paragraph{For Example:}

\begin{verbatim}
>>> str2 = "Slicing the String"
>>> print(str2[4:])  # Prints string starting from the 5th character 
ing the String
>>> print(str2[:4])  # Prints the first four characters
Slic
>>> print(str2[2:8]) # Prints characters starting from 3rd to 7th
icing
>>> printt(str2[4:-3]) # Prints characters from 5th to 3rd character from last
ing the Str
\end{verbatim}

\section*{String Methods}

\paragraph{Strings have their own set of functions. In this section we will go through few of them:}

\subsection*{len()}

\paragraph{The \textit{len()} function returns the length of a string as an integer. \texttt{len("String")}}

\paragraph{Example}

\begin{verbatim}
>>> len("Hello World") # This returns the value as 11.
>>>
>>> name = "TalentSprint"
>>> len(name) # 12
\end{verbatim}

\subsection*{lower()}

\paragraph{The \textit{lower()} function converts all uppercase letters in string to lowercase and return the new string.}

\begin{verbatim}
>>> s1 = "PYTHON"
>>> s1.lower()
’python’
\end{verbatim}

\subsection*{upper()}

\paragraph{The \textit{upper()} function converts lowercase letters in string to uppercase and return the new string.}

\begin{verbatim}
>>> s2 = "python"
>>> s2 = name.upper()
>>> print(s2)
’PYTHON’
\end{verbatim}

\subsection*{replace()}

\paragraph{The function \textit{replace()} returns a copy of the string with all occurrences of substring old replaced by new.}

\paragraph{Syntax:}

\paragraph{
str.replace(old, new)
old - This is the old substring to be replaced
new - This is new substring, which would replace old substring.}

\paragraph{Example:}

\begin{verbatim}
>>> str = "This is example for replace function"
>>> str.replace(’is’, ’was’)
’Thwas was example for replace function’
\end{verbatim}

\subsection*{split()}

\paragraph{The function \textit{split()} is used to split on the whitespaces (blanks, newline) and returns the list of sub strings as items.}

\paragraph{Example:}

\begin{verbatim}
>>> str1 = "Engineering Student"
>>> str1.split()
[’Engineering’, ’Student’]
>>> str1.split("i")
[’Eng’, ’neer’, ’ng Student’]
\end{verbatim}

\subsection*{strip()}

\paragraph{By using \textit{strip()} function, It returns a copy of the string with the leading and trailing characters removed.}

\paragraph{Example:}

\paragraph{As it treats the argument as a set of characters. In this example, we specify all digits, and some punctuation chars.}

\begin{verbatim}
>>> value = "543210=Data,123"
# strip all digits
# Also remove equals sign and comma.
>>> result = value.strip("0123456789=,")
>>> print(result)
Data
\end{verbatim}

\end{document}

