\documentclass{exam}
\usepackage[utf8]{inputenc}
\begin{document}
\begin{questions}
    \question Python is named after \_\_\_\_\_

    \begin{oneparchoices}
        \choice I know! A famous fish!
        \choice I think it is named after a type of snake that ate the inventor.
        \choice I think it is something to do with a British comedy group in the 80s...
    \end{oneparchoices}
    \question What is the output of the following code?

    \begin{verbatim}
        print("Hello", "World!!")
    \end{verbatim}
    \begin{oneparchoices}
        \choice HelloWorld!!
        \choice Hello world!!
        \chocie Hello World!!
    \end{oneparchoices}

    \question What is the output of the following code?

    \begin{verbatim}
        word = "Notebook"
        print(word[4])
    \end{verbatim}
    \begin{oneparchoices}
        \choice b
        \choice e
        \choice o
        \choice N
    \end{oneparchoices}

    \question stores a piece of data, and gives it a specific name

    \begin{oneparchoices}
        \choice variable
        \choice whitespace
        \choice interpreter
        \choice modulo
    \end{oneparchoices}
   \question What is the output of the following code?

    \begin{verbatim}
        fruits = ['apple', 'mango', 'grapes']
        atribs = ['big', 'tasty', 'red']
        for fruit in fruits[::-1]:
            for atri in atribs:
                print(atri, fruit)
    \end{verbatim}
    \begin{oneparchoices}
        \choice error
        \choice atri, fruit
        \choice big grapes \\
           tasty grapes \\
           red grapes \\
           big mango \\
           tasty mango \\
           red mango \\
           big apple\\ 
           tasty apple \\
           red apple
    \end{oneparchoices}

   \question What is the output of the following code?

    \begin{verbatim}
        def select(x):
            return x % 3 == 0 or x % 5 == 0
        LIMIT = 10
        total = 0
        for n in range(LIMIT):
            if select(n):
                total = total + n
                print(total)
    \end{verbatim}
    \begin{oneparchoices}
        \choice 23
        \choice 2323
        \chocie 44
        \choice error
    \end{oneparchoices}

   \question What is the output of the following code?

    \begin{verbatim}
        word = "Python Programming"
        print(list(word))
    \end{verbatim}
    \begin{oneparchoices}
        \choice Error
        \choice Python Programming
        \choice ['P', 'y', 't', 'h', 'o', 'n', ' ', 'P', 'r', 'o', 'g', 'r', 'a', 'm', 'm', 'i', 'n', 'g']
    \end{oneparchoices}

   \question What is the output of the following code?

    \begin{verbatim}
        fruits = ['apple', 'orange', 'grapes']
        print(fruits[-1][-1])
    \end{verbatim}
    \begin{oneparchoices}
        \choice grapes
        \choice s
        \choice Error
    \end{oneparchoices}

   \question What is the output of the following code?

    \begin{verbatim}
        veggies = ['carrot', 'broccoli', 'potato', 'asparagus']
        print(sorted(veggies))

    \end{verbatim}
    \begin{oneparchoices}
        \choice ['carrot', 'broccoli', 'potato', 'asparagus']
        \choice ['asparagus']
        \choice Error
        \choice ['asparagus', 'broccoli', 'carrot', 'potato']
    \end{oneparchoices}

\end{questions}
\end{document}
