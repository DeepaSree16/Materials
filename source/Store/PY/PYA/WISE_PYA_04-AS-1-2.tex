\documentclass{exam}
\usepackage[utf8]{inputenc}
\begin{document}
\begin{questions}
    \question What is the output of the below code?
    
    \begin{verbatim}
        s = 'foo'
        t = 'bar'
        print('barf' in 2 * (s + t))
    \end{verbatim}

    \begin{oneparchoices}
        \choice \textbf{True} \\
        \choice False
    \end{oneparchoices}

    \question All of the following expressions produce the same result except one. Which one?
    \begin{verbatim}
        s = 'foobar'
    \end{verbatim}

    \begin{oneparchoices}
        \choice s[::5] \\
        \choice \textbf{s[::-5]} \\
        \choice s[0] + s[-1]
    \end{oneparchoices}

    \question What is the output of the below code?

    \begin{verbatim}
        print('$100 $200 $300'.count('$'),
              '$100 $200 $300'.count('$', 5, 10),
              '$100 $200 $300'.count('$', 5)
        )
    \end{verbatim}

    \begin{oneparchoices}
        \choice \textbf{3 1 2} \\
        \choice 3 2 1 \\
        \choice 3 1 1 \\
        \choice 3 1 0
    \end{oneparchoices}

    \question A while loop in Python is used for what type of iteration?

    \begin{oneparchoices}
        \choice discriminant \\
        \choice definite \\
        \choice \textbf{indeterminate} \\
        \choice indefinite
    \end{oneparchoices}

    \question Which of the following is a valid way to start a function in Python?

    \begin{oneparchoices}
        \choice \textbf{def someFunction():} \\
        \choice function someFunction() \\
        \choice def someFunction() \\
        \choice function someFunction()
    \end{oneparchoices}

    \question Which of the following is a valid way to start a while loop in Python?

    \begin{oneparchoices}
        \choice while loop $a < 10$ \\
        \choice \textbf{while $a < 10:$} \\
        \choice while $(a < 10)$ \\
        \choice while loop $a < 10:$ 
    \end{oneparchoices}

    \question What is the output of the below code?

    \begin{verbatim}
        i = 5
        while False:
            if i % 0O11 == 0:
                print(i)
            i = i +  1
        print(i)
    \end{verbatim}

    \begin{oneparchoices}
        \choice 5 6 7 8 9 10 \\ 
        \choice 5 6 7 8 \\
        \choice \textbf{5} \\ 
        \choice error
    \end{oneparchoices}

    \question What is the output of the below code?

    \begin{verbatim}
        num = 1
        LIMIT = 10
        while num < LIMIT:
            if num % 2 == 0:
                print(num)
            num = num + 2
        print(num)
    \end{verbatim}

    \begin{oneparchoices} 
        \choice \textbf{11} \\
        \choice 2 4 6 8 10 \\ 
        \choice 1 2 3 4 5 6 \\
        \choice 1 3 5 7 9 11 
    \end{oneparchoices}

    \question What is the output of the below code?

    \begin{verbatim}
        print('Ab!2'.swapcase())
    \end{verbatim}

    \begin{oneparchoices}
        \choice AB!@ \\
        \choice ab12 \\
        \choice \textbf{aB!2} \\
        \choice aB1@  
    \end{oneparchoices}

    \question What is the output of the below code?

    \begin{verbatim}
        def change(num):
            num = 0
            num = num + 1
            return num
        print(change(1))
    \end{verbatim}

    \begin{oneparchoices}
        \choice 0 \\
        \choice \textbf{1} \\
        \choice 2
        \choice error
    \end{oneparchoices}

    \question What is the output of the below code?

    \begin{verbatim}
        def display(b, n):
            while n > 0:
                print(b, end="")
                n = n - 1
        display('z',3)
    \end{verbatim}

    \begin{oneparchoices}
        \choice \textbf{zzz} \\
        \choice zz \\
        \choice error \\
        \choice Infinite loop   
    \end{oneparchoices}
    
    \question What is the output of the below code?

    \begin{verbatim}
        print('abc'.islower())
    \end{verbatim}
    \begin{oneparchoices}
        \choice \textbf{True} \\
        \choice False \\
        \choice error \\
        \choice none
    \end{oneparchoices}

    \question What is the output of the below code?

    \begin{verbatim}
        def test1(param):
            return param
        def test2(param): 
            return param * 2
        def test3(param):
            return param + 3
        result = test1(test2(test3(1)))
        print(result)
    \end{verbatim}

    \begin{oneparchoices}
        \choice error \\
        \choice result \\
        \choice 1 4 8 \\
        \choice 8 \\
        \choice \textbf{1}
    \end{oneparchoices}


\end{questions}
\end{document}
