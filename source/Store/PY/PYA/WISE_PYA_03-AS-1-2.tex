\documentclass{exam}
\usepackage[utf8]{inputenc}
\begin{document}
\begin{questions}
    \question  In Python, a decision can be made by using if else statement.

    \begin{oneparchoices}
        \choice \textbf{True} \\
        \choice False
    \end{oneparchoices}
    \question Checking multiple conditions in Python requires elif statements.

    \begin{oneparchoices}
        \choice \textbf{True} \\
        \choice False
    \end{oneparchoices}

    \question If the condition is evaluated to true, the statement(s) of if block will be executed, otherwise the statement(s) in else block(if else is specified) will be executed.

    \begin{oneparchoices}
        \choice \textbf{True} \\
        \choice False
    \end{oneparchoices}

    \question What will be output of the below code?

    \begin{verbatim}
        if True & False & True | False:
            print("Hurray!!")
        else:
            print("Oops")
    \end{verbatim}

    \begin{oneparchoices}
        \choice It will throw an Error \\
        \choice It will display Hurray!  \\
        \choice \textbf{It will display Oops} \\
        \choice  Nothing will be displayed
    \end{oneparchoices}

    \question What is the output of the below code?

    \begin{verbatim}
        if bool(10):
            print("Ten")
        else:
            print("Zero")
    \end{verbatim}
    \begin{oneparchoices}
        \choice \textbf{It will display Ten} \\
        \choice It will display Zero \\
        \choice It will throw an Error \\
        \choice Nothing gets displayed 
    \end{oneparchoices}
    
      \question What is the output of the below code?

      \begin{verbatim}
        print("hello" "world")
      \end{verbatim}

      \begin{oneparchoices}
          \choice \textbf{It will display helloworld} \\
          \choice It will display hello world \\
          \choice It will throw an Error \\
          \choice Nothing gets displayed 
      \end{oneparchoices}

      \question What is the output of the below code?

      \begin{verbatim}
        print(0xB + 0xC + 0xD)
      \end{verbatim}

      \begin{oneparchoices}
          \choice \textbf{36} \\
          \choice 33 \\
          \choice 30 \\
          \choice 38
      \end{oneparchoices}


      \question What is the output of the below code?

      \begin{verbatim}
        age = 9
        if 0 < age < 10:
            print("kid")
        else:
            print("You are not a kid anymore")
      \end{verbatim}

      \begin{oneparchoices}
          \choice It will throw an Error \\
          \choice It will display You are not a kid anymore \\
          \choice \textbf{It will display kid} \\
          \choice Nothing gets displayed
      \end{oneparchoices}

      \question What is the output of the below code?

      \begin{verbatim}
        if (6 * 2 < 20):
            a = "Hello" * 2
        else:
            a = "Hello" * 5
        print(a)
      \end{verbatim}

      \begin{oneparchoices}
          \choice It will throw an Error \\
          \choice Nothing gets displayed \\
          \choice It will display Hello \\
          \choice  \textbf{It will display HelloHello}
      \end{oneparchoices}

      \question What is the output of the below code?

      \begin{verbatim}
        age = 14
        if age < 18:
            print("You are not eligible to Vote")
        print("Hey You are eligible to Vote")
      \end{verbatim}

      \begin{oneparchoices}
          \choice It will throw an Error \\
          \choice It will display Hey You are not eligible to Vote \\
          \choice \textbf{It will display You are not eligible to Vote} \\
          \choice Nothing will be displayed
      \end{oneparchoices}

      \question What is the output of the below code?

      \begin{verbatim}
        a = 4
        b = 6
        if a % b:
            print("Equal")
        print("Not Equal")
      \end{verbatim}

      \begin{oneparchoices}
          \choice It will throw an Error \\
          \choice \textbf{It will display Equal} \\
          \choice  Nothing gets displayed \\
          \choice It will display Not Equal
      \end{oneparchoices}

      \question What is the output of the below code?

      \begin{verbatim}
        length = 4
        breadth = 4
        if length == breadth:
            print("It's a square")
        print(It's a Rectangle)
      \end{verbatim}

      \begin{oneparchoices}
          \choice It will throw an Error \\
          \choice Nothing is displayed \\
          \choice It will display It's a Rectangle \\
          \choice \textbf{It will display It's a Square}
      \end{oneparchoices}

      \question What is the output of the below code?

      \begin{verbatim}
        a = 4
        b = 6
        if a % b == 0:
            print("Equal")
        print("Not Equal")
      \end{verbatim}

      \begin{oneparchoices}
          \choice It will throw an Error \\
          \choice \textbf{It will display Not Equal} \\
          \choice  Nothing gets displayed \\
          \choice It will display Equal
      \end{oneparchoices}

      \question What is the output of the below code?

      \begin{verbatim}
        celsius = 42
        f = (celsius * 1.8) + 32
        print(round(f,2))
      \end{verbatim}

      \begin{oneparchoices}
        \choice \textbf{107.6} \\
        \choice 107.601 \\
        \choice 42.0 \\
        \choice 42.78
      \end{oneparchoices}

      \question What is the output of the below code?

      \begin{verbatim}
        a = 6
        if 0 < a < 10:
            a = a ^ 7
            a = a ** 3
            print(a)
      \end{verbatim}
    
      \begin{oneparchoices}
          \choice \textbf{1} \\
          \choice 0 \\
          \choice 343 \\
          \choice 216
      \end{oneparchoices}

      \question What is the output of the below code?

      \begin{verbatim}
        number1 = 88
        number2 = 99
        number1, number2 = number2, number1
        number3 = number1 ** 3
        number4 = number3 % 5
        print(number4)
      \end{verbatim}

      \begin{oneparchoices}
        \choice \textbf{4} \\
        \choice 2 \\
        \choice 1 \\
        \choice 3
      \end{oneparchoices}

      \question The character that must be at the end of the line if.

      \begin{oneparchoices}
          \choice \textbf{:} \\
          \choice ; \\
          \choice . \\
          \choice ,
      \end{oneparchoices}

      \question What does If statement do?

      \begin{oneparchoices}
          \choice \textbf{makes decision} \\
          \choice repeat statements \\
          \choice stores the value in a variable
      \end{oneparchoices}

      \question An if statement can be executed multiple times.

      \begin{oneparchoices}
          \choice \textbf{False} \\
          \choice True
      \end{oneparchoices}

      \question Is it possible to execute both the statements under if and else at the same time

      \begin{oneparchoices}
          \choice True \\
          \choice \textbf{False}
      \end{oneparchoices}

      \question The \emph{else} is option in python if statements

      \begin{oneparchoices}
          \choice \textbf{True} \\
          \choice False
      \end{oneparchoices}

      \question The relational operators cannot be used within a if condition
      
      \begin{oneparchoices}
          \choice True \\
          \choice \textbf{False}
      \end{oneparchoices}

      \question The code inside \emph{if} statement is executed only once

      \begin{oneparchoices}
          \choice \textbf{True}\\
          \choice False
      \end{oneparchoices}

      \question There can be an \emph{else} statement without an if statement

      \begin{oneparchoices}
          \choice True\\
          \choice \textbf{False}
      \end{oneparchoices}

      \question When the if condition is false, it exists the program

      \begin{oneparchoices}
          \choice True \\
          \choice \textbf{False}
      \end{oneparchoices}

      \question The program generates an error message if the \emph{else} statement is not defined

      \begin{oneparchoices}
          \choice True \\
          \choice \textbf{False}
      \end{oneparchoices}

      \question What is the syntax error with the below code?

      \begin{verbatim}
        print("Welcome!!!");
      \end{verbatim}

      \begin{oneparchoices}
          \choice \textbf{Colon is not needed} \\
          \choice print is spelt incorrectly \\
          \choice No brackets around "Welcome" \\
          \choice p should be capital 
      \end{oneparchoices}
\end{questions}
\end{document}
