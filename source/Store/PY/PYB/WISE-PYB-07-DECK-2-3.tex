\documentclass[14pt]{beamer}
\title{Python 102 :: Week 7}
\date{}
\author[TS]{TalentSprint}
\usefonttheme{serif}
\usepackage{bookman}
\usepackage{hyperref}
\usepackage[T1]{fontenc}
\usepackage{graphicx}
\usecolortheme{orchid}
\beamertemplateballitem

\usepackage{listings}
%\usebackgroundtemplate{\includegraphics[width=\paperwidth]{logo}}
\definecolor{mygreen}{rgb}{0,0.6,0}
\definecolor{mygray}{rgb}{0.5,0.5,0.5}
\definecolor{mymauve}{rgb}{0.58,0,0.82}

\lstset{ %
  backgroundcolor=\color{white},   % choose the background color
  basicstyle=\small,        % size of fonts used for the code
  breaklines=true,                 % automatic line breaking only at whitespace
  captionpos=b,                    % sets the caption-position to bottom
  commentstyle=\color{mygreen},    % comment style
  %escapeinside={\%*}{*)},          % if you want to add LaTeX within your code
  keywordstyle=\color{blue},       % keyword style
  stringstyle=\color{mymauve},     % string literal style
  showstringspaces=false,
}
   
\begin{document}
    \begin{frame}
        \titlepage
    \end{frame}
    \begin{frame}
        \frametitle{Topics for the Session}
        \begin{itemize}
            \item Odometer Problem
            \item Functional and Generic or Object Oriented Way
        \end{itemize}
    \end{frame}

    \begin{frame}[containsverbatim]
        \frametitle{What is our odometer problem?}
        \begin{itemize}
        \item This odometer can only show digits 1 to 9
        \item Only the reading that are in ascending order are valid
        \item The odometer rolls over to the first valid reading after the last
        \end{itemize}
    \end{frame}

    \begin{frame}[containsverbatim]
        \frametitle{Examples}
        \begin{itemize}
        \item Valid 4 digit readings, 
        \item First is 1234 and last is 6789
        \item Not valid reading example 2314
        \end{itemize}
    \end{frame}
 
    \begin{frame}[containsverbatim]
        \frametitle{What would be the functionality needed?}
        \begin{itemize}
        \item If you know the size of the odometer, we would know everything about odometer
        \item Start reading is 123
        \item End reading is 789
        \item Next reading to 789 is 123
        \item Previous reading of 123 is 789
        \item Next reading of 149 is 156
        \item Previous reading of 149 is 148
        \end{itemize}
    \end{frame}

    \begin{frame}[containsverbatim]
        \frametitle{Set of functions needed}
        \begin{itemize}
        \item Prerequisite - Knowing size of odometer
        \item \alert {First Reading} 
        \item Function name as first\_reading()
        \item \alert {Last Reading}
        \item Function name as last\_reading()
        \end{itemize}
    \end{frame}

    \begin{frame}[containsverbatim]
        \frametitle{Functionality needed - Continued}
        \begin{itemize}
        \item \alert {Next Reading}
        \item[] next\_reading()
        \item \alert {Previous Reading}
        \item[] prev\_reading()
        \end{itemize}
    \end{frame}

    \begin{frame}[containsverbatim]
        \frametitle{Functionality needed - Continued}
        \begin{itemize}
        \item \alert {Next kth Reading}
        \item[] next\_kth\_reading()
        \item \alert {Previous kth Reading}
        \item[] prev\_kth\_reading()
        \end{itemize}
    \end{frame}

    \begin{frame}[containsverbatim]
        \frametitle{Functionality needed - Continued}
        \begin{itemize}
        \item \alert {Is particular reading valid?}
        \item[] is\_reading()
        \item \alert {Difference between two given readings} 
        \item[] diff()
        \end{itemize}
    \end{frame}

    \begin{frame}[containsverbatim]
        \frametitle{Functionalities Implemented}
        \begin{itemize}
        \item Ideally this odometer problem is nothing but implementation of following:
        \item Partition of numbers or 
        \item Combinations of probability.
        \end{itemize}
    \end{frame}

    \begin{frame}[containsverbatim]
        \frametitle{Sample - 2 Digits Odometer}
        \begin{itemize}
        \item Two digits odometer valid readings are 
        \item[] 12, 13, 14, 15, 16, 17, 18, 19
        \item[]     23, 24, 25, 26, 27, 28, 29
        \item[]         34, 35, 36, 37, 38, 39
        \item[]             45, 46, 47, 48, 49
        \item[]                 56, 57, 58, 59
        \item[]                     67, 68, 69
        \item[]                         78, 79
        \item[]                             89
        \item So, total 8 + 7 + ... + 1 = 36 or through combinations 9c2, which is 36.
        \end{itemize}
    \end{frame}

    \begin{frame}[containsverbatim]
        \frametitle{What if we need to implement multiple odometers?}
        \begin{itemize}
        \item Say, we need to implement 3 odometers each maintaining its own readings. 
        \item Each odometer can be any digit odometers
        \end{itemize}
    \end{frame}

    \begin{frame}[containsverbatim]
        \frametitle{How to do this?}
        \begin{itemize}
        \item Use class or object oriented methodology
        \item Defining behaviour and state for the functionality
        \item Defining our own data type called odometer
        \end{itemize}
    \end{frame}

    \begin{frame}[containsverbatim]
        \frametitle{What are the advantages?}
        \begin{itemize}
        \item Storing first and last reading eliminates use of 2 functions for first reading and last reading
        \item Maintaining first, current and last reading for every required odometer
        \end{itemize}
    \end{frame}

    \begin{frame}[containsverbatim]
        \frametitle{What are the advantages?}
        \begin{itemize}
        \item Next reading and next pth reading can be clubbed by passing default value (1)
        \item Using built-in functions can maintain required functionalities say size of odometer (using len method)
        \item Adding operations such as +1 or -1 or any number for next or previous reading gives valid reading
        \end{itemize}
    \end{frame}

    \begin{frame}[containsverbatim]
        \frametitle{What are the advantages?}
        \begin{itemize}
        \item Restrict wrong usage say adding value to the reading or
        \item Assigning certain values to odometer reading
        \end{itemize}
    \end{frame}

\end{document}  
