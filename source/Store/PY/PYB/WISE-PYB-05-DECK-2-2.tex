\documentclass[14pt]{beamer}
\title{Python 101 :: Week 5}
\date{}
\author[TS]{TalentSprint}
\usefonttheme{serif}
\usepackage{bookman}
\usepackage{hyperref}
\usepackage[T1]{fontenc}
\usepackage{graphicx}
\usecolortheme{orchid}
\beamertemplateballitem

\usepackage{listings}
%\usebackgroundtemplate{\includegraphics[width=\paperwidth]{logo}}
\definecolor{mygreen}{rgb}{0,0.6,0}
\definecolor{mygray}{rgb}{0.5,0.5,0.5}
\definecolor{mymauve}{rgb}{0.58,0,0.82}

\lstset{ %
  backgroundcolor=\color{white},   % choose the background color
  basicstyle=\small,        % size of fonts used for the code
  breaklines=true,                 % automatic line breaking only at whitespace
  captionpos=b,                    % sets the caption-position to bottom
  commentstyle=\color{mygreen},    % comment style
  %escapeinside={\%*}{*)},          % if you want to add LaTeX within your code
  keywordstyle=\color{blue},       % keyword style
  stringstyle=\color{mymauve},     % string literal style
  showstringspaces=false,
}
   
\begin{document}
    \begin{frame}
        \titlepage
    \end{frame}
    \begin{frame}
        \frametitle{Topics for the Session}
        \begin{itemize}
            \item Higher Order Functions
            \item Comprehensions
        \end{itemize}
    \end{frame}

    \begin{frame}[containsverbatim]
        \frametitle{What does this function do?}
        \begin{itemize}
        \item Function's code
        \end{itemize}
        \begin{lstlisting}[language=Python]
        r = 0
        while r * r < n:
                r += 1
        return r * r == n
        \end{lstlisting}
        \begin{itemize}
        \item \alert {What does this function do?}
        \item is\_perfect\_square
        \end{itemize}
    \end{frame}
 
    \begin{frame}[containsverbatim]
        \frametitle{What does this function do?}
        \begin{itemize}
        \item Function's code
        \end{itemize}
        \begin{lstlisting}[language=Python]
        r = 0
        while r * r * r < n:
                r += 1
        return r * r * r == n
        \end{lstlisting}
        \begin{itemize}
        \item \alert {How would you name the function?}
        \item is\_perfect\_cube
        \end{itemize}
    \end{frame}

    \begin{frame}[containsverbatim]
        \frametitle{What does this function do?}
        \begin{itemize}
        \item Function's code
        \end{itemize}
        \begin{lstlisting}[language=Python]
        r = 0
        while 3 ** r < n:
                r += 1
        return 3 ** r == n
        \end{lstlisting}
        \begin{itemize}
        \item \alert {How would you name the function?}
        \item is\_power\_of3
        \end{itemize}
    \end{frame}

    \begin{frame}[containsverbatim]
        \frametitle{What does this function do?}
        \begin{itemize}
        \item Function's code
        \end{itemize}
        \begin{lstlisting}[language=Python]
        r = 0
        while r * (r + 1) < n * 2:
                r += 1
        return r * (r + 1) == n * 2
        \end{lstlisting}
        \begin{itemize}
        \item \alert {How would you name the function?}
        \item is\_triangular
        \end{itemize}
    \end{frame}

    \begin{frame}[containsverbatim]
        \frametitle{What does this function do?}
        \begin{itemize}
        \item Function's code
        \end{itemize}
        \begin{lstlisting}[language=Python]
        r = 0
        while r * (3 * r - 1) < n * 2:
                r += 1
        return r * (3 * r - 1) == n * 2
        \end{lstlisting}
        \begin{itemize}
        \item \alert {How would you name the function?}
        \item is\_pentagon
        \end{itemize}
    \end{frame}
    
    \begin{frame}[containsverbatim]
        \frametitle{What are the observations?}
        \begin{itemize}
            \item \alert{In general:}
            \item Duplication in structure of code
            \item \alert{How do we fix this?}
            \item Move the duplication code to function
            \item Call this as function to required mathematical function      
        \end{itemize}
    \end{frame}
	
    \begin{frame}[containsverbatim]
        \frametitle{Higher Order Function}
        \begin{itemize}
            \item \alert{Call as}
            \item isTriangular(15) for True
            \item isTriangular(12) for False
        \begin{lstlisting}[language=Python]
def triangular(r: int):
        return (r * (r + 1)) // 2

def is_f(f, n: int) -> bool:
        r = 0
        while f(r) < n:
                r += 1
        return f(r) == n

def isTriangular(n: int) -> bool:
        return is_f(triangular, n)
        \end{lstlisting}
        \end{itemize}
    \end{frame}
    
    \begin{frame}[containsverbatim]
        \frametitle{Higher Order Functions}
        \begin{itemize}
            \item Passing function as argument to another function or
            \item Return function as the result
        \end{itemize}
	\end{frame}
	
    \begin{frame}[containsverbatim]
        \frametitle{How to add certain value to each item in any iterator?}
        \begin{itemize}
            \item Say marks of students in an iterator, say list and want to add 3, say grace marks to all
            \item \alert{Syntax: Usual way}
        \end{itemize}
        \begin{lstlisting}[language=Python]
>>> marks = [70, 60, 65, 80, 55]
>>> newmarks = []
>>> for mark in marks:
...     newmarks.append(mark + 3)
>>> newmarks
[73, 63, 68, 83, 58]
>>>
        \end{lstlisting}
    \end{frame}

    \begin{frame}[containsverbatim]
        \frametitle{How to do this in simpler way?}
        \begin{itemize}
            \item \alert{Use Higher Order Function: map}
            \item map(f, l) -> f(l[0]), f(l[1]), .... f(l[-1])
            \item Applying the same rule for all elements 
            \item Returns same size as input size
            \item Return data type is same as return type of function 
            \item Map returns map object, need to type cast to required data type (say list) to view
        \end{itemize}
    \end{frame}

    \begin{frame}[containsverbatim]
        \frametitle{Let us check}
		\begin{itemize}
		\item Using Higher order function - Map
		\end{itemize}
        \begin{lstlisting}[language=Python]
def grace_marks(x: int) -> int:
        return x + 3
newmarks = list(map(grace_marks, marks))
print (newmarks)
        \end{lstlisting}
    \end{frame}

    \begin{frame}[containsverbatim]
        \frametitle{How to remove certain items?}
        \begin{itemize}
            \item How to remove items from an iterator which does not satisfy given condition?
            \item Say we have list of marks in a list and want to have only more than pass marks, say 40
            \item \alert{Syntax: Usual way}
        \end{itemize}
        \begin{lstlisting}[language=Python]
>>> marks = [50, 60, 35, 39, 70, 75, 85, 30]
>>> newmarks = []
>>> for mark in marks:
...     if mark > 40:
...             newmarks.append(mark)
>>> newmarks
[50, 60, 70, 75, 85]
        \end{lstlisting}
    \end{frame}

    \begin{frame}[containsverbatim]
        \frametitle{How to do this in simpler way?}
        \begin{itemize}
            \item \alert{Use Higher Order Function: filter}
            \item filter(f, l) -> f(l[x]), say, item x, which satisfies given condition 
            \item Returns subset or part of provided iterator or container
            \item Filter returns map object, need to type cast to required data type (say list) to view
        \end{itemize}
    \end{frame}

    \begin{frame}[containsverbatim]
        \frametitle{Let us check this out}
		\begin{itemize}
		\item Using Higher order function - Filter
		\end{itemize}
        \begin{lstlisting}[language=Python]
def pass_marks(x: int) -> bool:
        return x > 40

marks = [50, 60, 35, 39, 70, 75, 85, 30]
newmarks = list(filter(pass_marks, marks))
print (newmarks)
        \end{lstlisting}
    \end{frame}

    \begin{frame}[containsverbatim]
        \frametitle{Armstrong Number check and range}
		\begin{itemize}
		\item Using Higher Order Functions
		\end{itemize}
        \begin{lstlisting}[language=Python]
def cube(x):
        return x ** 3
def armstrong(n: int) -> int:
        return sum(map(cube, map(int, str(n))))
def is_armstrong(n: int) -> bool:
    return n == armstrong(n)
def make_armstrong(start: int, limit: int) -> [int]:
        return filter(is_armstrong, range(start, limit))

print(make_armstrong(100, 999))
        \end{lstlisting}
    \end{frame}

    \begin{frame}[containsverbatim]
        \frametitle{Continued}
		\begin{itemize}
		\item Clubbing armstrong to is\_armstrong function
		\end{itemize}
        \begin{lstlisting}[language=Python]
def cube(x):
        return x ** 3
def is_armstrong(n: int) -> bool:
    return n == sum(map(cube, map(int, str(n))))
def make_armstrong(start: int, limit: int) -> [int]:
        return filter(is_armstrong, range(start, limit))

print(make_armstrong(100, 999))
        \end{lstlisting}
    \end{frame}

    \begin{frame}[containsverbatim]
        \frametitle{Generalized and Optimized}
		\begin{itemize}
		\item Let us generalize armstrong check for any numbers and avoiding use cube function
		\end{itemize}
        \begin{lstlisting}[language=Python]
def is_armstrong(n: int) -> bool:
        digits = list(map(int, str(n)))
        return n == sum(map(lambda r: r ** len(digits), map(int, str(n))))

def make_armstrong(start: int, limit: int) -> [int]:
        return filter(is_armstrong, range(start, limit))

print(make_armstrong(1, 100000))
        \end{lstlisting}
    \end{frame}

    \begin{frame}[containsverbatim]
        \frametitle{How it is done?}
		\begin{itemize}
		\item Removing need of function cube to include in higher order function itself as inline function using lambda
		\item Using lambda which is anonymuous function, which only purpose here is to perform simple arithmetic
		\item Lambda can be used for use of say expressions and can't be used for complex functionality
		\end{itemize}
    \end{frame}

    \begin{frame}[containsverbatim]
        \frametitle{Let us check this out}
		\begin{itemize}
		\item Take 2 iterators
		\item How to take even numbers from f and add 1 to each element
		\item \alert {Club map and filter}
		\end{itemize}
        \begin{lstlisting}[language=Python]
>>> f = [0, 1, 1, 2, 3, 5, 8]
>>> p = [2, 3, 5, 7, 11, 13, 17]
>>> list(map(lambda x: x + 1, filter(lambda x: x % 2 == 0, f)))
[1, 3, 9]
>>> list(filter(lambda x: x % 2 == 0, f))
[0, 2, 8]
>>> 
        \end{lstlisting}
    \end{frame}

    \begin{frame}[containsverbatim]
        \frametitle{How to do this in any other way?}
		\begin{itemize}
		\item \alert {Comprehensions}
		\end{itemize}
        \begin{lstlisting}[language=Python]
>>> [x + 1 for x in f if x % 2 == 0]
[1, 3, 9]
        \end{lstlisting}
    \end{frame}

    \begin{frame}[containsverbatim]
        \frametitle{What is comprehension?}
		\begin{itemize}
		\item Comprehension or List Comprehension
		\item Generalizes use of map and filter to use either or one of them
		\item In order words, it is set builder
		\item You can use \_ instead of variables as variables are not performing any other purpose than storing temporary values
		\end{itemize}
        \begin{lstlisting}[language=Python]
[f(x) for x in l]
[ x for x in l if P(x)]

>>> [_ + 1 for _ in f if _ % 2 == 0]
[1, 3, 9]
>>> 
        \end{lstlisting}
    \end{frame}

    \begin{frame}[containsverbatim]
        \frametitle{Armstrong using Comprehensions}
		\begin{itemize}
		\item Code
		\end{itemize}
        \begin{lstlisting}[language=Python]
def is_armstrong(n: int) -> bool:
        digits = [int(x) for x in str(n)]
        return n == sum([x ** len(digits) for x in digits])

def make_armstrong(start: int, limit: int) -> [int]:
        return [x for x in range(start, limit) if is_armstrong(x)]

print(make_armstrong(1, 100000))
        \end{lstlisting}
    \end{frame}

    \begin{frame}[containsverbatim]
        \frametitle{Adding two lists}
		\begin{itemize}
		\item Say we have 2 lists and how to add each element of list 1 to list 2
		\end{itemize}
        \begin{lstlisting}[language=Python]
>>> zip(f, p)
<zip object at 0x7f6c372194c8>
>>> list(zip(f, p))
[(0, 2), (1, 3), (1, 5), (2, 7), (3, 11), (5, 13), (8, 17)]
>>> list(map(sum, zip(f, p)))
[2, 4, 6, 9, 14, 18, 25]
>>> 
        \end{lstlisting}
    \end{frame}

    \begin{frame}[containsverbatim]
        \frametitle{Using zip}
		\begin{itemize}
		\item Clubs values in order for 2 or more lists
		\end{itemize}
        \begin{lstlisting}[language=Python]
>>> list(zip(f, p, range(100, 120)))
[(0, 2, 100), (1, 3, 101), (1, 5, 102), (2, 7, 103), (3, 11, 104), (5, 13, 105), (8, 17, 106)]
>>> 
        \end{lstlisting}
    \end{frame}

    \begin{frame}[containsverbatim]
        \frametitle{How to add resultant list}
		\begin{itemize}
		\item How to add the elements of resultant list
		\item \alert {Use Reduce functionality}
		\end{itemize}
        \begin{lstlisting}[language=Python]
>>> import functools
>>> functools.reduce(lambda x, y: x + y, list(map(sum, zip(f, p))))
78
>>>
        \end{lstlisting}
    \end{frame}

    \begin{frame}[containsverbatim]
        \frametitle{What is reduce functionality?}
        \begin{itemize}
		\item Processes elements in order to make single value result
		\end{itemize}
        \begin{lstlisting}[language=Python]
reduce(f, [a0, a1, a2, a3....])
reduce(f, [f(a0, a1), a2, ....])
reduce(f, [f(a01), a3, ....])
        \end{lstlisting}
    \end{frame}

    \begin{frame}[containsverbatim]
        \frametitle{Making into dictionary}
        \begin{itemize}
		\item Using f and p lists
		\end{itemize}
        \begin{lstlisting}[language=Python]
>>> {x:y for x, y in zip(f, p)}
{0: 2, 1: 5, 2: 7, 3: 11, 5: 13, 8: 17}
>>> {x[0]:x[1] for x in zip(f, p)}
{0: 2, 1: 5, 2: 7, 3: 11, 5: 13, 8: 17}
>>> dict(zip(f, p))
{0: 2, 1: 5, 2: 7, 3: 11, 5: 13, 8: 17}
>>>
   \end{lstlisting}
    \end{frame}

\end{document}  
