\documentclass[14pt]{beamer}
\title{Python 101 :: Week 3}
\date{}
\author[TS]{TalentSprint}
\usefonttheme{serif}
\usepackage{bookman}
\usepackage{hyperref}
\usepackage[T1]{fontenc}
\usepackage{graphicx}
\usecolortheme{orchid}
\beamertemplateballitem

\usepackage{listings}
%\usebackgroundtemplate{\includegraphics[width=\paperwidth]{logo}}
\definecolor{mygreen}{rgb}{0,0.6,0}
\definecolor{mygray}{rgb}{0.5,0.5,0.5}
\definecolor{mymauve}{rgb}{0.58,0,0.82}

\lstset{ %
  backgroundcolor=\color{white},   % choose the background color
  basicstyle=\small,        % size of fonts used for the code
  breaklines=true,                 % automatic line breaking only at whitespace
  captionpos=b,                    % sets the caption-position to bottom
  commentstyle=\color{mygreen},    % comment style
  %escapeinside={\%*}{*)},          % if you want to add LaTeX within your code
  keywordstyle=\color{blue},       % keyword style
  stringstyle=\color{mymauve},     % string literal style
  showstringspaces=false,
}
   
\begin{document}
    \begin{frame}
        \titlepage
    \end{frame}
    \begin{frame}
	\frametitle{Topics for the Session}
	\begin{itemize}
	    \item Sets
	\end{itemize}
    \end{frame}
    \begin{frame}
	\frametitle{Introduction}
	\begin{itemize}
	    \item \alert{How do you remove duplicates from a list? }
	    \pause
	    \item Solution : One way is to use Sets
	\end{itemize}
    \end{frame}
    \begin{frame}[containsverbatim]
	\frametitle{What is a Set?}
	\begin{itemize}
	    \item Set is a unordered collection of elements
            \item Set can't have duplicate elements 
	    \item Set is Immutable
	    \item \alert{Example:}
	\end{itemize}
	\begin{lstlisting}[language=python]
	    primes = {2, 3, 5, 7, 11, 13}
            print(type(primes))
	\end{lstlisting}
	\textbf{Note:} Set is represented using \{ \}	
    \end{frame}
    \begin{frame}[containsverbatim]
	\frametitle{Accessing elements from a Set}
	\begin{lstlisting}[language=python]
	    print(primes[0])
	\end{lstlisting}
	\begin{itemize}
	    \item What gets printed?
	    \item Can we perform slicing on the sets?
	\end{itemize}
    \end{frame}
    \begin{frame}[containsverbatim]
	\frametitle{Accessing elements from a Set}
	\begin{lstlisting}[language=python]
	    for prime in primes:
	        print(prime)
	\end{lstlisting}
	\begin{itemize}
	    \item What gets printed?
	\end{itemize}
    \end{frame}
    \begin{frame}[containsverbatim]
	\frametitle{Set Methods}
	\begin{lstlisting}[language=python]
	    primes.add(17)
	    print(primes)
	\end{lstlisting}
	\begin{itemize}
	   \item What gets printed?
	   \item How many elements are present in \emph{primes} set?
	\end{itemize}
    \end{frame}
    \begin{frame}[containsverbatim]
        \frametitle{Set Methods}
	\begin{lstlisting}[language=python]
	    primes.update(17)
 	    print(primes)
	\end{lstlisting}
	\begin{itemize}
	   \item What gets printed?
	   \item How many elements are present in \emph{primes} set?
	\end{itemize}
    \end{frame}
    \begin{frame}[containsverbatim]
	\frametitle{Set Methods}
	\begin{lstlisting}[language=python]
	    primes.pop()
	    print(primes)
	\end{lstlisting}
	\begin{itemize}
	   \item What gets printed?
	   \item What gets printed if you use remove() instead of pop()?
	   \item What gets printed if you use discard() instead of pop()?
	   \item What gets printed if you use clear() instead of pop()?
	\end{itemize}
    \end{frame}
    \begin{frame}[containsverbatim]
	\frametitle{Set Methods}
	\begin{lstlisting}[language=python]
	    A = {2, 4, 6}
            B = {6, 8, 10, 12, 2}
            print(A.union(B))
	\end{lstlisting}
	\begin{itemize}
	    \item What gets printed?
	    \item What gets printed if you use intersection() instead of union()?
	    \item What gets printed if you use difference() instead of union()?
	    \item What gets printed if you use symmetric\_difference() instead of union()?
	\end{itemize}
    \end{frame}
    \begin{frame}
	\frametitle{Set Methods}
	\begin{itemize}
	    \item What gets printed if you use issubset() instead of union()?
	    \item What gets printed if you use issuperset() instead of union()?
	    \item What gets printed if you use isdisjoint() instead of union()?
	\end{itemize}
    \end{frame}
    \begin{frame}
	\frametitle{Let's Solve}
	\begin{description}
	    If we list all the natural numbers below 10 that are multiples of 3 or 5, we get 3, 5, 6 and 9. The sum of these multiples is 23. Find the sum of all the multiples of 3 or 5 below 1000.
	\end{description}
    \end{frame}
\end{document}
