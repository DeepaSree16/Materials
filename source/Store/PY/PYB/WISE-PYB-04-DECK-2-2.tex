\documentclass[14pt]{beamer}
\title{Python 102 :: Week 4}
\date{}
\author[TS]{TalentSprint}
\usefonttheme{serif}
\usepackage{bookman}
\usepackage{hyperref}
\usepackage[T1]{fontenc}
\usepackage{graphicx}
\usecolortheme{orchid}
\beamertemplateballitem

\usepackage{listings}
%\usebackgroundtemplate{\includegraphics[width=\paperwidth]{logo}}
\definecolor{mygreen}{rgb}{0,0.6,0}
\definecolor{mygray}{rgb}{0.5,0.5,0.5}
\definecolor{mymauve}{rgb}{0.58,0,0.82}

\lstset{ %
  backgroundcolor=\color{white},   % choose the background color
  basicstyle=\small,        % size of fonts used for the code
  breaklines=true,                 % automatic line breaking only at whitespace
  captionpos=b,                    % sets the caption-position to bottom
  commentstyle=\color{mygreen},    % comment style
  %escapeinside={\%*}{*)},          % if you want to add LaTeX within your code
  keywordstyle=\color{blue},       % keyword style
  stringstyle=\color{mymauve},     % string literal style
  showstringspaces=false,
}
   
\begin{document}
    \begin{frame}
        \titlepage
    \end{frame}
    \begin{frame}
        \frametitle{Topics for the Session}
        \begin{itemize}
            \item Higher Order Functions
            \item Comprehensions
        \end{itemize}
    \end{frame}
    \begin{frame}
        \frametitle{Higher Order Functions}
        \begin{itemize}
            \item Takes function as argument or returns function as a result
            \item They are often used to abstract common iteration operations.
        \end{itemize}
    \end{frame}

    \begin{frame}[containsverbatim]
        \frametitle{Taking function as argument}
        \begin{itemize}
            \item How to pass function name as argument?
            \item How to call the passed function? 
            \item \alert {Example - Calling function square with argument 9}
        \end{itemize}
        \begin{lstlisting}[language=Python]
def call_square_9(square):
        return (square(9))
        \end{lstlisting}
	\end{frame}
	
    \begin{frame}[containsverbatim]
        \frametitle{Taking function as argument - Continued}
        \begin{itemize}
		\item Returns square of given number		
        \item \alert {Example}
        \end{itemize}
        \begin{lstlisting}[language=Python]
def square(num):
        return (num * num)
        \end{lstlisting}
    \end{frame}

    \begin{frame}[containsverbatim]
        \frametitle{Taking function as argument - Continued}
        \begin{itemize}
		\item How you would call square function as function argument?
        \item \alert {Calling square() through call\_square\_9()}
        \end{itemize}
        \begin{lstlisting}[language=Python]
	print ("Square of number {} is {}".format(9, call_square_9(square)))
        \end{lstlisting}
    \end{frame}

    \begin{frame}[containsverbatim]
        \frametitle{Taking function as argument - Generalized}
        \begin{itemize}
		\item How to make call\_square\_9() a generalized function?
        \item \alert {Pass Argument}
        \end{itemize}
        \begin{lstlisting}[language=Python]
def call_square(square, num):
        return(square(num))
        \end{lstlisting}
    \end{frame}

    \begin{frame}[containsverbatim]
        \frametitle{Taking function as argument - Generalized}
        \begin{itemize}
		\item How to call the generalized function?
        \item \alert {Pass Argument}
        \end{itemize}
        \begin{lstlisting}[language=Python]
	print (call_square(square, 6))
	print (call_square(square, 7))
	print (call_square(square, 8))
        \end{lstlisting}
    \end{frame}

    \begin{frame}[containsverbatim]
        \frametitle{Taking function as argument - Generalized}
        \begin{itemize}
		\item How to call the function with an iterable?
        \item \alert {Use List}
        \end{itemize}
        \begin{lstlisting}[language=Python]
numbers = [6, 7, 8, 9]
for num in numbers:
        print (call_square(square, num))
        \end{lstlisting}
    \end{frame}

    \begin{frame}[containsverbatim]
        \frametitle{Make it simple - How?}
        \begin{itemize}
        \item Isn't the process is complicated?
        \item How to make it simple?
		\item Use Higher Order Functions 
        \item \alert {Such as map, filter, reduce}
        \end{itemize}
    \end{frame}
    
    \begin{frame}[containsverbatim]
        \frametitle{What is map function?}
        \begin{itemize}
            \item Takes an input iterable of values (like list or tuple) and return a list of results. 
            \item Same order, same length, but mapped via a function.
            \item \alert{Syntax:}
        \end{itemize}
        \begin{lstlisting}[language=Python]
	map(function, iterable)    
        \end{lstlisting}
    \end{frame}
    
    \begin{frame}[containsverbatim]
        \frametitle{map - Continued}
        \begin{itemize}
            \item Define a function, say square()
            \item \alert{Example:}
        \end{itemize}
        \begin{lstlisting}[language=Python]
	def square(number):
		return number ** 2
        \end{lstlisting}

        \begin{itemize}        
        \item \alert {Take a list of numbers}
        \end{itemize}        
        \begin{lstlisting}[language=Python]
	numbers = [1, 2, 3, 4]
        \end{lstlisting}
    \end{frame}

    \begin{frame}[containsverbatim]
        \frametitle{map - Continued}
        \begin{itemize}
            \item How to calculate square of numbers in list using map?
            \item \alert{Example:}
        \end{itemize}
        \begin{lstlisting}[language=Python]
	print(map(square, numbers))
        \end{lstlisting}

        \begin{itemize}
            \item What is the output from above statement?
            \item \alert{Map Object}
        \end{itemize}
    \end{frame}

    \begin{frame}[containsverbatim]
        \frametitle{map - Continued}
        \begin{itemize}
            \item How to get the output instead of object address?
            \item \alert{Get Result:}
        \end{itemize}
        \begin{lstlisting}[language=Python]
	print(list(map(calculate_square, numbers)))
        \end{lstlisting}
    \end{frame}

    \begin{frame}[containsverbatim]
        \frametitle{What is filter function?}
        \begin{itemize}
            \item Filter takes a function and an iterable to produce an output list of every item on the input list that passes a test.
            \item \alert{Syntax:}
        \end{itemize}
        \begin{lstlisting}[language=Python]
            filter(function, iterable)
        \end{lstlisting}
    \end{frame}
    
    \begin{frame}[containsverbatim]
        \frametitle{filter - Continued}
		\begin{itemize}
		\item Take a function that checks whether passed number is even or not.
        \item \alert{Example:}
        \end{itemize}
        \begin{lstlisting}[language=Python]
	def is_even(number):
		return (number % 2 == 0)
        \end{lstlisting}
		\begin{itemize}
		\item Print results
        \item \alert{Example:}
        \end{itemize}
        \begin{lstlisting}[language=Python]
print(list(filter(is_even, [2,3,4,5,6])))
        \end{lstlisting}

    \end{frame}
    
    \begin{frame}[containsverbatim]
        \frametitle{What is reduce function?}
        \begin{itemize}
            \item Reduce takes an iterable of input data and to produce a single value output.
            \item \alert{Syntax:}
        \end{itemize}
        \begin{lstlisting}[language=Python]
	reduce(function, iterable)
        \end{lstlisting}
    \end{frame}
    \begin{frame}[containsverbatim]
        \frametitle{reduce - Continued}
		\begin{itemize}
		\item Take a function that takes 2 inputs, returns the result after subtraction.
        \item \alert{Example:}
        \end{itemize}
        \begin{lstlisting}[language=Python]
	def subtract(num1, num2):
		return num1 - num2
        \end{lstlisting}
    \end{frame}
    \begin{frame}[containsverbatim]
        \frametitle{reduce - Continued}
		\begin{itemize}
		\item Let us call with reduce
        \item \alert{Example:}
        \end{itemize}
        \begin{lstlisting}[language=Python]
print(reduce(subtract,[1, -23, 3, 45, 5]))
        \end{lstlisting}
		\begin{itemize}
		\item What happens?
        \item \alert{Error: reduce not defined}
        \item \alert{How to fix this?}
        \item \alert {Include reduce from functools module}
        \end{itemize}
        \begin{lstlisting}[language=Python]
from functools import reduce
        \end{lstlisting}
    \end{frame}
    \begin{frame}[containsverbatim]
        \frametitle{reduce - Continued}
		\begin{itemize}
		\item Why is no mention of iterable list with reduce?
        \item \alert{Reason:}
        \item Reduce - Single value output 
        \end{itemize}
    \end{frame}

    \begin{frame}[containsverbatim]
        \frametitle{What is Comprehension?}
		\begin{itemize}
		\item Comprehension or list Comprehension?
        \item \alert{Meaning:}
        \item Used to construct lists in a very
natural and easy way
		\item List Comprehensions have the form
		
		[ expression for expr in sequence1 \\
		  for expr2 in sequence2 ... \\
		  for exprN in sequenceN \\
		  if condition ] \\
        \end{itemize}
    \end{frame}

    \begin{frame}[containsverbatim]
        \frametitle{Comprehension - Continued}
		\begin{itemize}
		\item Define an iterable, 
		\item Say, nums = [5, 6, 7, 8, 9]
		\item Write a function to calculate squares using comprehension
        \item \alert{Example}
        \item nums = [ num * num for num in nums]
		\item print (nums)
        \end{itemize}
    \end{frame}

    \begin{frame}[containsverbatim]
        \frametitle{Comprehension - Continued}
		\begin{itemize}
		\item Print even numbers for given iterable
		\item Say, [5, 6, 7, 8, 9, 10, 20, 25]
        \item \alert{Example}
        \item print ([ num for num in nums if num \% 2 == 0 ])
        \end{itemize}
    \end{frame}
    
\end{document}  
