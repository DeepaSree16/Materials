\documentclass[14pt]{beamer}
\title{Python 101 :: Week 1}
\date{}
\author[TS]{TalentSprint}
\usefonttheme{serif}
\usepackage{bookman}
\usepackage{hyperref}
\usepackage[T1]{fontenc}
\usepackage{graphicx}
\usecolortheme{orchid}
\beamertemplateballitem

\usepackage{listings}
%\usebackgroundtemplate{\includegraphics[width=\paperwidth]{logo}}
\definecolor{mygreen}{rgb}{0,0.6,0}
\definecolor{mygray}{rgb}{0.5,0.5,0.5}
\definecolor{mymauve}{rgb}{0.58,0,0.82}

\lstset{ %
  backgroundcolor=\color{white},   % choose the background color
  basicstyle=\small,        % size of fonts used for the code
  breaklines=true,                 % automatic line breaking only at whitespace
  captionpos=b,                    % sets the caption-position to bottom
  commentstyle=\color{mygreen},    % comment style
  %escapeinside={\%*}{*)},          % if you want to add LaTeX within your code
  keywordstyle=\color{blue},       % keyword style
  stringstyle=\color{mymauve},     % string literal style
  showstringspaces=false,
}
   
\begin{document}
\begin{frame}
  \titlepage
\end{frame}
    
\begin{frame}
  \frametitle{Topics for the Session}
  \begin{itemize}
  \item Set up your environment
  \item Review: Lists and Strings
  \item Review: Loops
  \item Review: Functions
  \end{itemize}
\end{frame}
\begin{frame}{Environment}
  \begin{itemize}
      \item Terminal set up
      \pause
      \item What is ssh?
      \pause
      \item Login into Rollno@wise.talentsprint.com
  \end{itemize}
\end{frame}
\begin{frame}{Environment}
  \begin{itemize}
  \item what version of python?
	  \pause
  \item Run the interpreter
  \item keep the docs open in the browser
  \item CTRL + SHIFT + T
  \item ALT + 1
  \end{itemize}
\end{frame}

\begin{frame}{Relook at Strings}
  \begin{itemize}
  \item What is a string?
    \pause
  \item What are string delimiters?
    \pause
  \item What is the difference?
    \pause
  \item How to combine two strings?
    \pause
  \item How to repeat a string?
    \pause
  \item How to get a specific character? 
	  \pause
  \item How to get part of the string?
  \end{itemize}
\end{frame}
    
\begin{frame}
  \frametitle{String Methods}
  \begin{itemize}
  \item Case conversion: \textbf{\texttt{capitalize, title, upper, lower, swapcase}}
    \pause
  \item Search: \textbf{\texttt{index, find, startswith, endswith}}
    \pause
  \item Format: \textbf{\texttt{center, strip, just, zfill, format}} 
    \pause
  \item Type: \textbf{\texttt{isalpha, isnum, \ldots}}
    \pause
  \item Tokenize: \textbf{\texttt{split, partition, join}}
  \end{itemize}
  And many more \ldots 
  
  https://docs.python.org/3/library/
\end{frame}

\begin{frame}
  \frametitle{Relook at Lists}
  \begin{itemize}
  \item What is a list?
    \pause
  \item What can it contain?
    \pause
  \item How to combine two lists?
    \pause
  \item How to select an item? a sub-list?
    \pause
  \item How to check the number of elements? if an element is present? 
  \end{itemize}
\end{frame}

\begin{frame}
  \frametitle{List Methods}
  \begin{itemize}
  \item Stats: \textbf{\texttt{max, min, sum, count}}
    \pause
  \item Mutators: \textbf{\texttt{append, insert, extend, sort, reverse, remove, clear}}
    \pause
  \item Slice: Access as well as modify
  \end{itemize}
  More in the docs.
\end{frame}

\begin{frame}{Common Sequence Types}
  Strings, Lists, Tuples are all common sequence types.

 More on slicing and striding
\end{frame}

\begin{frame}{Relook at functions}
    \begin{block}{Classical Syntax}
      \texttt{\textbf{def} square(p):}
      
      \texttt{~~~~\textbf{return} p * p}
    \end{block}
    \pause
    \begin{block}{Modern syntax: 3.5+}
      \texttt{\textbf{def} square(p:int) -> int:}

      \texttt{~~~~\textbf{return} p * p}
    \end{block}
  \end{frame}
  
\begin{frame}{return}    
  \begin{itemize}
  \item	\textbf{\texttt{return}} statement: What can be returned?
    \pause
  \item Why functions?
  \end{itemize}
\end{frame}

\end{document}
