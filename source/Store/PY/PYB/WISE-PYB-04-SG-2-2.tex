\documentclass{book}
\usepackage{listings}
\usepackage{color}
\usepackage{graphicx}
\usepackage{booktabs}
\usepackage{fancyhdr}
\usepackage[english]{babel}
\pagestyle{fancy}
\fancyhf{}
\rhead{\includegraphics[width=2cm, height=0.5cm]{../images/logo}}
\lhead{Python Programming}
\lfoot{COPYRIGHT ©TALENTSPRINT, 2020. ALL RIGHTS RESERVED.}
\rfoot{\thepage}

\begin{document}


\section*{List Comprehension}
List comprehension can be used to construct lists in a very natural, easy way, like a mathematician is used to do.

List comprehensions provide a concise way to create lists. Common applications are to make new lists where each element is the result of some operations applied to each member of another sequence or iterable, or to create a subsequence of those elements that satisfy a certain condition.

\subsection*{How to use List Comprehension}

\begin{description}

\item [Example-01]

Generate the squares for all the numbers from 1 to 10?

\begin{figure}[ht]
\begin{center}
\includegraphics[scale=0.6]{../images/listcomprehension-13-1.png}
\caption{List Comprehension}
\label{List Comprehension}
\end{center}
\end{figure}

The same can be re-writen in one line using list comprehension as shown below:
\begin{figure}[ht]
\begin{center}
\includegraphics[scale=0.4]{../images/listcomprehension-13-2.png}
\caption{List Comprehension}
\label{List Comprehension}
\end{center}
\end{figure}

\textbf{Explanation}

for the above statement [n ** 2 for n in range(1,11)], \texttt{n ** 2} is commonly referred to as the output function , \texttt{n} is the element from the list generated by \texttt{range(1,11)} function. This statement says to do \texttt{n ** 2} on each \texttt{n} in LIST geneted by \texttt{range(1,11)}.

\item [Example-02]

Generate the squares for all the even numbers from 1 to 20?

\begin{figure}[ht]
\begin{center}
\includegraphics[scale=0.6]{../images/listcomprehension-13-3.png}
\caption{List Comprehension}
\label{List Comprehension}
\end{center}
\end{figure}

The same can be re-writen in one line using list comprehension as shown below:
\begin{figure}[ht]
\begin{center}
\includegraphics[scale=0.4]{../images/listcomprehension-13-4.png}
\caption{List Comprehension}
\label{List Comprehension}
\end{center}
\end{figure}

A list comprehension consists of an expression followed by a \texttt{for} clause, then \textbf{zero} or \textbf{more} \texttt{for} or \texttt{if} clauses. The result will be a new list resulting from evaluating the expression in the context of the \texttt{for} and \texttt{if} clauses.
\end{description}
\end{document}
