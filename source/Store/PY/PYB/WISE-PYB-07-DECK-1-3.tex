\documentclass[14pt]{beamer}
\title{Python 102 :: Week 7}
\date{}
\author[TS]{TalentSprint}
\usefonttheme{serif}
\usepackage{bookman}
\usepackage{hyperref}
\usepackage[T1]{fontenc}
\usepackage{graphicx}
\usecolortheme{orchid}
\beamertemplateballitem

\usepackage{listings}
%\usebackgroundtemplate{\includegraphics[width=\paperwidth]{logo}}
\definecolor{mygreen}{rgb}{0,0.6,0}
\definecolor{mygray}{rgb}{0.5,0.5,0.5}
\definecolor{mymauve}{rgb}{0.58,0,0.82}

\lstset{ %
  backgroundcolor=\color{white},   % choose the background color
  basicstyle=\small,        % size of fonts used for the code
  breaklines=true,                 % automatic line breaking only at whitespace
  captionpos=b,                    % sets the caption-position to bottom
  commentstyle=\color{mygreen},    % comment style
  %escapeinside={\%*}{*)},          % if you want to add LaTeX within your code
  keywordstyle=\color{blue},       % keyword style
  stringstyle=\color{mymauve},     % string literal style
  showstringspaces=false,
}
   
\begin{document}
    \begin{frame}
        \titlepage
    \end{frame}
    \begin{frame}
        \frametitle{Topics for the Session}
        \begin{itemize}
            \item File Handling
        \end{itemize}
    \end{frame}
    \begin{frame}
        \frametitle{Introduction to Files}
        \begin{itemize}
            \item A file is an object on a computer that stores data, information, settings, or instructions used inside the computer programs.
            \item Two types of files:
                \begin{itemize}
                    \item Text - Stores data in the form of strings
                    \item Binary - Stores data in the form of bytes 
                \end{itemize}
        \end{itemize}
    \end{frame}
    \begin{frame}[containsverbatim]
        \frametitle{File Operations :: Reading}
        \alert{Opening an existing file}
        \begin{lstlisting}[language=Python]
        fileExample = open("FileExample.txt", "r")
        print(fileExample)
        \end{lstlisting}
    \end{frame}
    \begin{frame}[containsverbatim]
        \frametitle{File Operations :: Reading}
        \alert{Reading it}
        \begin{lstlisting}[language=Python]
        fileExample.read()
        \end{lstlisting}
    \end{frame}
    \begin{frame}[containsverbatim]
        \frametitle{File Operations :: Reading}
        \alert{Closing a file}
        \begin{lstlisting}[language=Python]
            fileExample.close()
        \end{lstlisting}
    \end{frame}
    \begin{frame}[containsverbatim]
        \frametitle{File Operations :: Writing}
        \alert{Opening a new file}
        \begin{lstlisting}[language=Python]
            fileExample = open("FileExample.txt", "w")
            print(fileExample)
        \end{lstlisting}
    \end{frame}
    \begin{frame}[containsverbatim]
        \frametitle{File Operations :: Writing}
        \alert{Writing to it}
        \begin{lstlisting}[language=Python]
            fileExample.write("Hello, message is written into the file")
            fileExample.close()
        \end{lstlisting}
        \textbf{Note:} Only after calling close() the changes appears in file.
    \end{frame}
    \begin{frame}[containsverbatim]
        \frametitle{File Operations :: Appending}
        \begin{lstlisting}[language=Python]
            fileExample = open("FileExample.txt", "ab")
            print(fileExample)
        \end{lstlisting}
    \end{frame}
    \begin{frame}[containsverbatim]
        \frametitle{File Operations :: Appending}
        \begin{lstlisting}[language=Python]
            fileExample.write("Hello, I am back again!!")
            fileExample.close()
        \end{lstlisting}
        \textbf{Note:} In append mode the file pointer is set to the end of the opened file
    \end{frame}
    \begin{frame}[containsverbatim]
        \frametitle{Files :: with statement}
        \begin{itemize}
            \item Can be used while opening a file
            \item It will take of closing a file without using file.close()
        \end{itemize}
        \begin{lstlisting}[language=Python]
            with open("FileExample.txt", "r") as f:
                f.read()
        \end{lstlisting}
    \end{frame}
\end{document}
