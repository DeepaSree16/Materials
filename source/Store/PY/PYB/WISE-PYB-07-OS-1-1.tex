\documentclass[11pt,a4paper]{article}
\usepackage{graphicx}
\usepackage{listings}
\lstset{language=python,numbers=left,numberstyle=\tiny,numbersep=10pt,showstringspaces=false}
\usepackage{array}
\usepackage{enumitem}

\def\AnswerBox{\fbox{\begin{minipage}{4in}\hfill\vspace{0.5in}\end{minipage}}}
\usepackage{fancyhdr}
 
\pagestyle{fancy}
\renewcommand\headrule{}
\rhead {\includegraphics[scale=.5]{../images/logo}}
\begin{document}
\section*{Workbook 5 - Files}
%\section*{Workbook - 08}\
\subsection*{Review Questions}

\begin{enumerate}\itemsep10pt
    \item The \underline{\hspace{3cm}} function is used to open the text file. 
    \item Write a statement to open \texttt{data.txt} file in read mode \underline{\hspace{3cm}}.      
    \item The function used to read data from a file line by line is \underline{\hspace{4cm}}.
    \item The function used to write data into the file is \underline{\hspace{3cm}}.
    \item The \underline{\hspace{3cm}} mode opens the file for both reading and writing.  
\end{enumerate}

\subsection*{Code Reading Exercises}
\begin{itemize}
    \item Write the expected output, or errors if any, for each of the following programs in the box provided below each program.
    \item Then execute the programs and check your answers.
    \item Also answer the questions given below.
\end{itemize}

\begin{enumerate}[label=\bfseries Program \arabic*:]

    \item ~
    \begin{lstlisting}
    def isPrime(num):
        limit = num // 2;
        for i in range(2, limit + 1):
            if num % i == 0:
                return False
        return True

    def getPrimes(n1, n2):
        fp = open(``primes.txt'', ``w'')
        for i in range(n1, n2 + 1):
            if isPrime(i):
                fp.writelines(str(i) + ``\n'' )

    getPrimes(10, 50)

    \end{lstlisting}
    \AnswerBox
    \begin{enumerate}[label=\bfseries Q\arabic*:]\itemsep10pt
         \item Write your understanding.\vspace{1cm}
    \end{enumerate}

    \item ~
    \begin{lstlisting}
    def Count():
        clines = 0
        for x in open(``primes.txt''):
            clines += 1
        return clines

    print(Count())
    \end{lstlisting}
    \AnswerBox 
    \begin{enumerate}[label=\bfseries Q\arabic*:]\itemsep10pt
         \item Write your understanding.\vspace{1cm}
    \end{enumerate}
    \item ~
    \begin{lstlisting}
    def fun2():
        l_count = 0
        w_count = 0 
        for line in open(``Data.txt''):
            l_count += 1
            l_words = line.split()
            w_count += len(l_words)
        return l_count, w_count

    print(fun2())
    \end{lstlisting}
    \AnswerBox
    \begin{enumerate}[label=\bfseries Q\arabic*:]\itemsep10pt
         \item Write your understanding.\vspace{2cm}
    \end{enumerate} 


   \item ~
    \begin{lstlisting}
    str1 = ``1, 34, 2, 23, 8, 74, 34, 45, 23, 2, 74, 8, 9''
    str1 = str1.replace(`,',` ')
    lst1 = str1.split()
    print(set(lst1))
    \end{lstlisting}
    \AnswerBox

    \item ~
    \begin{lstlisting}
     def XYZ(string):
         return `'.join(c for c in sorted(set(string)))

     def ABC1():
         str1 = ``Integer,Header,Delete,Camera,
                  Boolean,Namespace,Online,Partition''
         f = open(``File1.txt'',``w'')
         for word in str1.split(``,''):
             f.writelines(word + `` --> '' + XYZ(word) + ``\n'')
    
     ABC1()
    \end{lstlisting}
    \AnswerBox
    \item ~
    \begin{lstlisting}
     def cv(word):
         vowels = ``aeiou''
         word = word.lower()
         count = 0
         for ch in vowels:
             count += word.count(ch)
         return count

     str1 = ``education,grandchildren,regulation,rhythm,
              strength,government,authorize,instruction''
     f = open(``vowels.txt``, ``w'')
     for word in str1.split(``,''):
         f.writelines(word + `` : '' + str(cv(word)) + ``\n'');  

    \end{lstlisting}
    \AnswerBox
\end{enumerate}
\subsection*{Coding Exercises}
\begin{enumerate}
    \item Write a program to store all three digit amicable pair numbers  in to a file \texttt{A\_pairs.txt}.
    \item Write a program to store armstrong numbers with in the given range in to a file \texttt{A\_nums.txt}.
    \item Write a program which reads the words from a file \texttt{names.txt} and prints all anagrams.
    \begin{enumerate}
     \item Change the program such that it prints all anagrams which have more than 5 occurrences.
    \end{enumerate}
    \emph{Note: Copy words into the file ``names.txt'', from\\ https://code.google.com/p/wfuzz/source/browse/trunk/wordlist/others/names.txt?r=2}
\end{enumerate}
\end{document}
