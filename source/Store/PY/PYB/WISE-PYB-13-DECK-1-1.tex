\documentclass[14pt]{beamer}
\title{Python 102 :: Week 13}
\date{Asokan - 07-04-2020}
\author[TS]{TalentSprint}
\usefonttheme{serif}
\usepackage{bookman}
\usepackage{hyperref}
\usepackage[T1]{fontenc}
\usepackage{graphicx}
\usecolortheme{orchid}
\beamertemplateballitem

\usepackage{listings}
%\usebackgroundtemplate{\includegraphics[width=\paperwidth]{logo}}
\definecolor{mygreen}{rgb}{0,0.6,0}
\definecolor{mygray}{rgb}{0.5,0.5,0.5}
\definecolor{mymauve}{rgb}{0.58,0,0.82}

\lstset{ %
  backgroundcolor=\color{white},   % choose the background color
  basicstyle=\small,        % size of fonts used for the code
  breaklines=true,                 % automatic line breaking only at whitespace
  captionpos=b,                    % sets the caption-position to bottom
  commentstyle=\color{mygreen},    % comment style
  %escapeinside={\%*}{*)},          % if you want to add LaTeX within your code
  keywordstyle=\color{blue},       % keyword style
  stringstyle=\color{mymauve},     % string literal style
  showstringspaces=false,
}
   
\begin{document}
 
    \begin{frame}
        \titlepage
    \end{frame}
    \begin{frame}
        \frametitle{Topics for the Session}
        \begin{itemize}
            \item GIT
        \end{itemize}
    \end{frame}

    \begin{frame}
        \frametitle{Local Repository}
        \begin{itemize}
        	\item Creating user credentials in GITLAB
            \item Understanding of basic GIT commands such as status, init, config, add, log, checkout, commit 
        \end{itemize}
    \end{frame}

    \begin{frame}
        \frametitle{Remote Repository}
        \begin{itemize}
        	\item Creating sample project using gitlab web-site
            \item Additional GIT commands apart from Local Repository such as clone, push and pull commands.
        \end{itemize}
    \end{frame}
\end{document}  
