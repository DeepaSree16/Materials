\documentclass{book}
\usepackage{listings}
\usepackage{color}
\usepackage{graphicx}
\usepackage{booktabs}
\usepackage{fancyhdr}
\usepackage[english]{babel}
\pagestyle{fancy}
\fancyhf{}
\rhead{\includegraphics[width=2cm, height=0.5cm]{../images/logo}}
\lhead{Python Programming}
\lfoot{COPYRIGHT ©TALENTSPRINT, 2020. ALL RIGHTS RESERVED.}
\rfoot{\thepage}

\begin{document}
    \section*{Higher Order Functions}
    According to Wikipedia, a higher-order function is a function that does at least one of the following:
    \begin{itemize}
        \item takes one or more functions as arguments
        \item returns a function as its result.
    \end{itemize}

    \subsection*{Map}
    Map applies a function over an iterable to produce a new iterable.
    \paragraph{Example:} ~
    \begin{verbatim}
        def square (number):
            return number ** 2
        numbers = [1, 2, 3, 4]
        print(list(map(square, numbers)))
    \end{verbatim}

    \subsection*{Reduce}
    Reduce applies a function of two arguments cumulatively to the elements of an iterable, optionally starting with an initial argument.
    \paragraph{Example:} ~

    \begin{verbatim}
        from functools import reduce
        import operator
        numbers = [4, 77, 888, 9]
        reduce(operator.add, numbers)
    \end{verbatim}

    \subsection*{Filter}
    Filter is a higher-order function that processes a data structure (usually a list) in some order to produce a new data structure containing exactly those elements of the original data structure for which a given predicate returns the boolean value true.
    
    \paragraph{Example:} ~

    \begin{verbatim}
        def is_odd (number):
            return number % 2 != 0
        numbers = [1, 44, 2, 77, 3, 55, 8, 20]
        print(list(filter(is_odd, numbers)))
    \end{verbatim}

\end{document}
