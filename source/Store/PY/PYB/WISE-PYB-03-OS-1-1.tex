\documentclass[11pt,a4paper]{article}
\usepackage{graphicx}
\usepackage{listings}
\lstset{language=python,numbers=left,numberstyle=\tiny,numbersep=10pt,showstringspaces=false}
\usepackage{array}
\usepackage{enumitem}

\def\AnswerBox{\fbox{\begin{minipage}{4in}\hfill\vspace{0.5in}\end{minipage}}}
\usepackage{fancyhdr}
 
\pagestyle{fancy}
\renewcommand\headrule{}
\rhead {\includegraphics[scale=.5]{../images/logo.png}}
\begin{document}
\section*{Dictionary}
%\section*{WorkBook - 07}\
\subsection*{Review Questions}
\begin{enumerate}\itemsep10pt
    \item Dictionary is a mapping between a \underline{\hspace{3cm}} and \underline{\hspace{3cm}}.
    \item \underline{\hspace{3cm}} function is used to create a new dictionary with no items.     
    \item Dictionaries define \underline{\hspace{3cm}} relationship between keys and values.
    \item Dictionaries do not normally have any notion of order. (True or False)
    \item Write a statement to create an empty dictionary \underline{\hspace{3cm}}. 
    \item The symbol used to separate the \texttt{key} and \texttt{value} in a dictionary is \underline{\hspace{3cm}}.    
    \item The \texttt{values} of a dictionary can be of any data type. (True / False)
    \item \underline{\hspace{3cm}} are unique within a dictionary.
    \item The function used to find the size of a dictionary is \underline{\hspace{3cm}}.
    \item The function used to clear all the items in a dictionary is \underline{\hspace{3cm}}.
    \item The items in a dictionary can be accessed with an index. State True / False.
    \item The \underline{\hspace{3cm}} function returns a list of all the values available in a dictionary.
    \item Write a statement to delete a dictionary \texttt{``d1''} \underline{\hspace{3cm}}.
    \item The operator used to check whether a  particular \texttt{key} exist in a dictionary is \underline{\hspace{3cm}}.
\end{enumerate}
\subsection*{Code Reading Exercises}
\begin{itemize}
    \item Write the expected output, or errors if any, for each of the following programs in the box provided below each program.
    \item Then execute the programs and check your answers.
    \item Also answer the questions given below.
\end{itemize}

\begin{enumerate}[label=\bfseries Program \arabic*:]

    \item ~
    \begin{lstlisting}
     def fun1():
         d1 = {}
         for i in range(1, n + 1):
             d1[i] = i * i
         return d1

     print(fun1(5))
    \end{lstlisting}
    \AnswerBox
    \begin{enumerate}[label=\bfseries Q\arabic*:]\itemsep10pt
         \item What will be the output if line 4 is replaced with \texttt{d1[i] = i ** i}?
    \end{enumerate}

    \item ~
    \begin{lstlisting}
    d1 = {``1'': ``One'', ``2'': ``Two'', ``3'': ``Three'',
          ``4'': ``Four'', ``5'': ``Five'', ``6'': ``Six'',
          ``7'': ``Seven'', ``8'': ``Eight'',
          ``9'': ``Nine'', ``0'': ``Zero''}
    x = input(``Enter The Number'')# Let x be 2576 
    for let in x:
        print(d1[let], end = `');
    \end{lstlisting}
    \AnswerBox 
    \begin{enumerate}[label=\bfseries Q\arabic*:]\itemsep10pt
         \item What will be the output if the value of \texttt{x} is 23234?
         \item What will be the output if the value of \texttt{x} is 304?
    \end{enumerate}
    \item ~
    \begin{lstlisting}
    def fun2(val):
        d1 = {1: ``One'', 2: ``Two'', 3: ``Three'',
              4: ``Four'', 5: ``Five'', 6: ``Six'', 
              7: ``Seven'', 8: ``Eight'', 9: ``Nine''}
        return d1[len(val)] + `` Digit Number''
    
    print(fun2(``45254'')) 
    \end{lstlisting}
    \AnswerBox
    \begin{enumerate}[label=\bfseries Q\arabic*:]\itemsep10pt
         \item What will be the output if argument value is 352?
         \item What will be the output if argument value is 5427?
         \item What will be the output if argument value is 5637835?   
    \end{enumerate}
    \item ~
    \begin{lstlisting}
    def XYZ(mon):
        d1 = {``JAN'' : 31, ``FEB'' : 28, ``MAR'' : 31, 
              ``APR'' : 30, ``MAY'' : 31, ``JUN'' : 30, 
              ``JUL'' : 31, ``AUG'' : 31, ``SEP'' : 30, 
              ``OCT'' : 31, ``NOV'' : 30, ``DEC'' : 31}
        if mon in d1:
            return d1[mon]
        return -1

    print(XYZ(``FEB''))
    \end{lstlisting}
    \AnswerBox

    \begin{enumerate}[label=\bfseries Q\arabic*:]\itemsep10pt
         \item What will be the output if the argument value is ``MAY''?
         \item What will be the output if the argument value is ``APRIL''?
         \item What will be the output if the argument value is ``January''?
         \item What will be the output if the argument value is ``may''?
    \end{enumerate}
    \item ~
    \begin{lstlisting}
    def char_frequency(string):
        dcount = {}
        for ch in string:
            if ch is not ` ':
                if ch in dcount.keys():
                    dcount[ch] += 1
                else:
                    dcount[ch] = 1 
        return dcount

    print(char_frequency(``TRY IT BY YOURSELF''))
    \end{lstlisting}
    \AnswerBox

    \item ~
    \begin{lstlisting}
     def count(s):
         dcount = {}
         s = s.lower()
         for ch in s:
             if ch in 'aeiou':
                 if ch in dcount.keys():
                     dcount[ch] += 1
                 else:
                     dcount[ch] = 1
         return len(dcount)

    print(count(``EDUCATION''))    
    \end{lstlisting}
    \AnswerBox
    \begin{enumerate}[label=\bfseries Q\arabic*:]\itemsep10pt
         \item What will be the output if the argument value is ``REVOLUTIONARIES''?
         \item What will be the output if the argument value is ``CRYPT''?
    \end{enumerate}
    \end{enumerate}
    
\subsection*{Coding Exercises}
\begin{enumerate}
    \item Write a program to display all anagram strings.

    \emph{Note: An anagram of a word, is a new word got by rearranging the letters of a given word. SILENT and LISTEN are anagrams.}
     \begin{description}
     \item [Input :]  ``pest step ball palm heat listen post enlist silent stop lamp''
     \item [Output :]\

           pest step\\
           listen enlist silent\\
           post stop
     \end{description} 
     \item Write a program to implement Rock Paper Scissors game.
      The winner of Rock Paper Scissors is determined as follows:

       Rock defeats (breaks) Scissors.

       Scissors defeats (cuts) Paper.

       Paper defeats (covers) Rock.
    
       If both players choose the same item, the game is a tie and they play again.

       \emph{Note: For further information please visit: http://en.wikipedia.org/wiki/Rock-paper-scissors}
\end{enumerate}
\end{document}
