\documentclass[14pt]{beamer}
\title{Python 101 :: Week 2}
\date{}
\author[TS]{TalentSprint}
\usefonttheme{serif}
\usepackage{bookman}
\usepackage{hyperref}
\usepackage[T1]{fontenc}
\usepackage{graphicx}
\usecolortheme{orchid}
\beamertemplateballitem

\usepackage{listings}
%\usebackgroundtemplate{\includegraphics[width=\paperwidth]{logo}}
\definecolor{mygreen}{rgb}{0,0.6,0}
\definecolor{mygray}{rgb}{0.5,0.5,0.5}
\definecolor{mymauve}{rgb}{0.58,0,0.82}

\lstset{ %
  backgroundcolor=\color{white},   % choose the background color
  basicstyle=\small,        % size of fonts used for the code
  breaklines=true,                 % automatic line breaking only at whitespace
  captionpos=b,                    % sets the caption-position to bottom
  commentstyle=\color{mygreen},    % comment style
  %escapeinside={\%*}{*)},          % if you want to add LaTeX within your code
  keywordstyle=\color{blue},       % keyword style
  stringstyle=\color{mymauve},     % string literal style
  showstringspaces=false,
}
   
\begin{document}

    \begin{frame}
        \titlepage
    \end{frame}
    \begin{frame}
	\frametitle{Topic for the Session}
	\begin{itemize}
		\item Dictionaries
	\end{itemize}
    \end{frame}
    \begin{frame}
        \frametitle{Introduction}
	\begin{itemize}
	\item Capital
		\begin{itemize}
			\item State is associated with it is capital
		\end{itemize}
	\item Student
		\begin{itemize}
			\item Student is associated with an id
			\item Student can take up number of courses
		\end{itemize}
	\pause
	\item \alert{How to capture such relationships?}
	\end{itemize}
    \end{frame}
    \begin{frame}[containsverbatim]
	\frametitle{What is a dictionary?}
	\begin{itemize}
		\item Connects pieces of related information
		\item Related information is stored as a key-value pair
	\end{itemize}
	\begin{lstlisting}[language=python]
	capitals = {"Bihar": "Patna", "Goa": "Panaji", "Haryana": "chandigarh"}
	print(capitals)
	\end{lstlisting}
    \end{frame}
    \begin{frame}[containsverbatim]
	\frametitle{Accessing values of a dictionary}
	\begin{lstlisting}[language=python]
	print(capitals["Goa"])
	\end{lstlisting}
	\begin{itemize} \item What gets printed? \end{itemize}
     \end{frame}
     \begin{frame}[containsverbatim]
	\frametitle{Accessing values of a dictionary}
	\begin{lstlisting}[language=python]
	print(capitals.get("Goa"))
	\end{lstlisting}
	\begin{itemize} \item What gets printed? \end{itemize}
    \end{frame} 
    \begin{frame}[containsverbatim]
	\frametitle{Modifying values in an dictionary}
	\begin{lstlisting}[language=Python]
	capitals["Assam"] = "Dispur"
	print(capitals)
	\end{lstlisting}
	\begin{itemize}
		\item What gets printed?
		\item How many capitals are present in \emph{capitals} dictionary?
	\end{itemize}
    \end{frame}
    \begin{frame}[containsverbatim]
	\frametitle{Modifying values in an dictionary}
        \begin{lstlisting}[language=Python]
	capitals["Haryana"] = "Chandigarh"
	\end{lstlisting}
	\begin{itemize}
		\item  How many capitals are present in \emph{capitals} dictionary?
	\end{itemize}
    \end{frame}
    \begin{frame}[containsverbatim]
	\frametitle{Dictionary Methods}
	\begin{lstlisting}[language=python]
	for key in capitals.keys():
            print(key)
	\end{lstlisting}
	\begin{itemize}
		\item What gets printed?
		\item What gets printed if you use capitals.values () instead of capitals.keys()? 
	\end{itemize}
    \end{frame}
    \begin{frame}[containsverbatim]
	\frametitle{Dictionary Methods}
	\begin{lstlisting}[language=python]
	for key, value in capitals.items():
            f"{value} is capital of {key} "
	\end{lstlisting}
	\begin{itemize}
		\item What gets printed?
		\item What is difference between items(), values() and keys()?
	\end{itemize}
    \end{frame}
    \begin{frame}[containsverbatim]
	\frametitle{Dictionary Methods}
	\begin{lstlisting}[language=python]
	capitals.popitem()
	\end{lstlisting}
	\begin{itemize}
		\item What will be the output of above code snippet?
		\item What happens if you use the \emph{pop()} instead of \emph{popitem()}?
		\item What happens if you use the \emph{clear()} instead of \emph{popitem()}?

	\end{itemize}
    \end{frame}
    \begin{frame}
	\frametitle{Overview of Dictionary Methods}
	\begin{itemize}
	\item \textbf{Accessing key and values} - items(), keys(), values(), get(), fromkeys()
	\item \textbf{Removing elements} - pop(), popitem(), clear()
	\item \textbf{Updating values} - update()
	\end{itemize}
    \end{frame}
    \begin{frame}
	\frametitle{Interview Question}
	\begin{itemize}
		\item Can we use list as key in the dictionary?
	\end{itemize}
    \end{frame}
    \begin{frame}
	\frametitle{Let's Solve}
	\begin{description}
		\item [Problem Statement:] If the numbers 1 to 5 are written out in words: one, two, three, four, five, then there are 3 + 3 + 5 + 4 + 4 = 19 letters used in total.

		If all the numbers from 1 to 1000 (one thousand) inclusive were written out in words, how many letters would be used?
	\end{description}
    \end{frame}
    \begin{frame}
	\frametitle{Let's Solve}
	\begin{description}
		\item {Note:} Do not count spaces or hyphens. For example, 342 (three hundred and forty-two) contains 23 letters and 115 (one hundred and fifteen) contains 20 letters. The use of "and" when writing out numbers is in compliance with British usage.
	\end{description}
    \end{frame}
\end{document}
