\documentclass[14pt]{beamer}
\title{Python 102 :: Week 5}
\date{}
\author[TS]{TalentSprint}
\usefonttheme{serif}
\usepackage{bookman}
\usepackage{hyperref}
\usepackage[T1]{fontenc}
\usepackage{graphicx}
\usecolortheme{orchid}
\beamertemplateballitem

\usepackage{listings}
%\usebackgroundtemplate{\includegraphics[width=\paperwidth]{logo}}
\definecolor{mygreen}{rgb}{0,0.6,0}
\definecolor{mygray}{rgb}{0.5,0.5,0.5}
\definecolor{mymauve}{rgb}{0.58,0,0.82}

\lstset{ %
  backgroundcolor=\color{white},   % choose the background color
  basicstyle=\small,        % size of fonts used for the code
  breaklines=true,                 % automatic line breaking only at whitespace
  captionpos=b,                    % sets the caption-position to bottom
  commentstyle=\color{mygreen},    % comment style
  %escapeinside={\%*}{*)},          % if you want to add LaTeX within your code
  keywordstyle=\color{blue},       % keyword style
  stringstyle=\color{mymauve},     % string literal style
  showstringspaces=false,
}
   
\begin{document}
    \begin{frame}
        \titlepage
    \end{frame}
    \begin{frame}
        \frametitle{Topics for the Session}
        \begin{itemize}
            \item Lambda Function
            \item Itertools
            \item Collections
            \item Generators and Coroutines
        \end{itemize}
    \end{frame}
    \begin{frame}
        \frametitle{Lambda Function}
        \begin{itemize}
            \item Also known as anonymous functions
            \item Every anonymous function in Python has three parts:
                \begin{itemize}
                    \item lambda keyword
                    \item parameters, and
                    \item body
                \end{itemize}
        \end{itemize}
    \end{frame}
    \begin{frame}[containsverbatim]
        \frametitle{Lambda Function}
        \alert{Syntax:}
        \begin{lstlisting}[language=Python]
            lambda p1, p2 : Expression
        \end{lstlisting}
    \end{frame}
    \begin{frame}[containsverbatim]
        \frametitle{Lambda Function}
        \alert{Example:}
        \begin{lstlisting}[language=Python]
            multiply = lambda x, y : x * y
            print(multiply (3, 4))
        \end{lstlisting}
    \end{frame}
    \begin{frame}
        \frametitle{Itertools}
        \begin{itemize}
            \item collection of tools for handling iterators
        \end{itemize}
    \end{frame}
    \begin{frame}[containsverbatim]
        \frametitle{Itertools}
        \alert{combinations}
        \begin{lstlisting}[language=Python]
            import itertools
            numbers = [1, 3, 66, 4]
            result = itertools.combinations(numbers, 2)
            for ele in result:
                print(ele)
        \end{lstlisting}
    \end{frame}
    \begin{frame}[containsverbatim]
        \frametitle{Itertools}
        \alert{permutations}
        \begin{lstlisting}[language=Python]
            import itertools
            numbers = [1, 3, 66, 4]
            result = itertools.permutations(numbers, 2)
            for ele in result:
                print(ele)
        \end{lstlisting}
    \end{frame}
    \begin{frame}
        \frametitle{Itertools}
        \alert{More functions from Itertools}
        \begin{itemize}
            \item \textbf{chain} - takes several iterators as arguments and returns a single iterator that produces the contents of all of them as though they came from one sequence.
            \item \textbf{islice} - returns an iterator which returns selected items from the input iterator, by index. 
            \item \textbf{izip()} - returns an iterator that combines the elements of several iterators into tuples
        \end{itemize}
    \end{frame}
    \begin{frame}
        \frametitle{Itertools}
        \alert{More functions from Itertools}
        \begin{itemize}
            \item \textbf{cycle} - This function cycles through an iterator endlessly.
            \item \textbf{compress} - This function filters one iterable with another.
            \item \textbf{dropwhile} - Make an iterator that drops elements from the iterable as long as the predicate is true; afterwards, returns every element.
        \end{itemize}
    \end{frame}
    \begin{frame}
        \frametitle{Collections}
        Collections module in python implements specialized data structures which provide alternative to python’s built-in container data types.
    \end{frame}
    \begin{frame}[containsverbatim]
        \frametitle{Collections}
        \alert{namedtuple} - It returns a tuple with a named entry, which means there will be a name assigned to each value in the tuple
        \begin{lstlisting}[language=Python]
            a = namedtuple('fruits' , 'apple , orange')
            fruitsQuantity = a(1 , 2)
            fruitsQuantity
        \end{lstlisting}
    \end{frame}
    \begin{frame}[containsverbatim]
        \frametitle{Collections}
        \alert{deque} pronounced as ‘deck’ is an optimized list to perform insertion and deletion easily.
        \begin{lstlisting}[language=Python]
            from collections import deque
            numbers = [1, 2, 3, 4]
            deque(numbers)
        \end{lstlisting}
    \end{frame}
    \begin{frame}[containsverbatim]
        \frametitle{Collections}
        \alert{Counter} is a dictionary subclass which is used to count hashable objects.
        \begin{lstlisting}[language=Python]
            from collections import Counter
            a = [1,1,1,1,2,3,3,4,3,3,4]
            Counter(a)
        \end{lstlisting}
    \end{frame}
    \begin{frame}
        \frametitle{Collections}
        \begin{itemize}
            \item \textbf{defaultdict} - It is a dictionary subclass which calls a factory function to supply missing values.
            \item \textbf{UserDict} - This class acts as a wrapper around dictionary objects. 
            \item \textbf{UserList} - This class acts like a wrapper around the list objects
        \end{itemize}
    \end{frame}
    \begin{frame}
        \frametitle{Generators}
        \begin{itemize}
            \item function that produces a sequence of results instead of a single value
            \item instead of returning a value, you generate a series of values
        \end{itemize}
    \end{frame}
    \begin{frame}[containsverbatim]
        \frametitle{Generators}
        \alert{Example:}
        \begin{lstlisting}[language=Python]
           def cube_numbers(nums):
               for i in nums:
                       yield(i ** 3)
           cubes = cube_numbers([1, 2, 3, 4, 5])
           print(cubes)
        \end{lstlisting}
    \end{frame}
    \begin{frame}
        \frametitle{Corountines}
        \begin{itemize}
            \item Coroutines are computer program components that generalize subroutines for non-preemptive multitasking, by allowing execution to be suspended and resumed.
        \end{itemize}
    \end{frame}
\end{document}
