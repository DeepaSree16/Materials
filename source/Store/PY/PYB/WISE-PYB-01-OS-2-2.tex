\documentclass[11pt,a4paper]{article}
\usepackage{graphicx}
\usepackage{listings}
\lstset{language=python,numbers=left,numberstyle=\tiny,numbersep=10pt,showstringspaces=false}
\usepackage{array}
\usepackage{enumitem}

\def\AnswerBox{\fbox{\begin{minipage}{4in}\hfill\vspace{0.5in}\end{minipage}}}
\usepackage{fancyhdr}
 
\pagestyle{fancy}
\renewcommand\headrule{}
\rhead {\includegraphics[scale=.5]{../images/logo.png}}
\begin{document}
\section*{Strings}

\subsection*{Code Reading}

\begin{itemize}
    \item Write the expected output, or errors if any, for each of the following programs in the box provided below each program.
    \item Then execute the programs and check your answers.
    \item Also answer the questions given below.
\end{itemize}

\begin{enumerate}[label=\bfseries Program \arabic*:]

    \item ~
    \begin{lstlisting}
        def maps(x):
            k = []
            for i in x:
                k.append(len(i))
            print(max(k))
    \end{lstlisting}
    \AnswerBox
    \item ~
    \begin{lstlisting}
        def song():
        print('99 bottles of beer on the wall, 99 bottles of beer')
        for i in range(99)[::-1]:
            print('Take one down, pass it around,' + str(i) +   ' bottles of beer on the wall.')
    \end{lstlisting}
    \AnswerBox
    \item ~
    \begin{lstlisting}
        n = int(input())
        total = 0
        for i in range(1, n + 1):
            total += i / (i + 1)
        print(round(total, 2))
    \end{lstlisting}
    \AnswerBox
    \item ~
    \begin{lstlisting}
        def translate(x):
            s = ''
            for i in x:
                if i not in ('aeiou'):
                    s += i + "o" + i
                else:
                    s += i 
        print(s) 
    \end{lstlisting}
    \AnswerBox
\end{enumerate}
\subsection*{Additional Exercises}
\begin{itemize}
    \item Create a function that calculates what percentage of the box is filled in. Give your answer as a string percentage rounded to the nearest integer.
    \begin{lstlisting}
        percent_filled([
            "####",
            "#  #",
            "#o #",
            "####"
        ]) ➞ "25%"
    \end{lstlisting}
    \item Write a function to create a Christmas tree based on height h.
    \begin{lstlisting}
        tree(2) ➞ [
                " # ",
                "###"
        ]
    \end{lstlisting}
\end{itemize}
\end{document}
