\documentclass[14pt]{beamer}
\title{Python 102 :: Week 10}
\date{26-03-2020}
\author[TS]{TalentSprint}
\usefonttheme{serif}
\usepackage{bookman}
\usepackage{hyperref}
\usepackage[T1]{fontenc}
\usepackage{graphicx}
\usecolortheme{orchid}
\beamertemplateballitem

\usepackage{listings}
%\usebackgroundtemplate{\includegraphics[width=\paperwidth]{logo}}
\definecolor{mygreen}{rgb}{0,0.6,0}
\definecolor{mygray}{rgb}{0.5,0.5,0.5}
\definecolor{mymauve}{rgb}{0.58,0,0.82}

\lstset{ %
  backgroundcolor=\color{white},   % choose the background color
  basicstyle=\small,        % size of fonts used for the code
  breaklines=true,                 % automatic line breaking only at whitespace
  captionpos=b,                    % sets the caption-position to bottom
  commentstyle=\color{mygreen},    % comment style
  %escapeinside={\%*}{*)},          % if you want to add LaTeX within your code
  keywordstyle=\color{blue},       % keyword style
  stringstyle=\color{mymauve},     % string literal style
  showstringspaces=false,
}
   
\begin{document}
    \begin{frame}
        \titlepage
    \end{frame}
    \begin{frame}
        \frametitle{Topics for the Session}
        \begin{itemize}
            \item Assignments Verification/Inputs
        \end{itemize}
    \end{frame}

    \begin{frame}[containsverbatim]
        \frametitle{Problem Statements (use OOP)}
        \begin{itemize}
        \item[1.] Even Fibonacci numbers (Project Euler - Problem Statement 2)
        \item[2.] Return Collatz sequence for a given input value.
        \end{itemize}
    \end{frame}

    \begin{frame}[containsverbatim]
        \frametitle{Even Fibonacci Numbers}
        \begin{itemize}
        \item Each new term in the Fibonacci sequence is generated by adding the previous two terms. 
        \item By starting with 1 and 2, the first 10 terms will be:        
1, 2, 3, 5, 8, 13, 21, 34, 55, 89, ...
        \end{itemize}
    \end{frame}

    \begin{frame}[containsverbatim]
        \frametitle{Even Fibonacci Numbers - Continued}
        \begin{itemize}
		\item By considering the terms in the Fibonacci sequence whose values do not exceed four million, find the sum of the even-valued terms.
        \end{itemize}
    \end{frame}

    \begin{frame}[containsverbatim]
        \frametitle{Collatz Sequence}
        \begin{itemize}
        \item To get the Collatz sequence from a number, the respective number needs to be divided by 2, if it is even. 
        \item In case the given number is odd it needs to be multiplied by 3 and increased by 1.
        \item Same operation needs to be continued on the result of the previous operation until the number becomes 1.
        \end{itemize}
    \end{frame}

    \begin{frame}[containsverbatim]
        \frametitle{Collatz Sequence - Continued}
        \begin{itemize}
        \item For any given number the series will end with 4 2 1.
        \item Ex: For number 3 collatz sequence is [3, 10, 5, 16, 8, 4, 2, 1]
        \item Ex: For number 9 collatz sequence is [9, 28, 14, 7, 22, 11, 34, 17, 52, 26, 13, 40, 20, 10, 5, 16, 8, 4, 2, 1]

        \end{itemize}
    \end{frame}

\end{document}  
