\documentclass[14pt]{beamer}
\title{Python 102 :: Week 11}
\date{}
\author[TS]{TalentSprint}
\usefonttheme{serif}
\usepackage{bookman}
\usepackage{hyperref}
\usepackage[T1]{fontenc}
\usepackage{graphicx}
\usecolortheme{orchid}
\beamertemplateballitem

\usepackage{listings}
%\usebackgroundtemplate{\includegraphics[width=\paperwidth]{logo}}
\definecolor{mygreen}{rgb}{0,0.6,0}
\definecolor{mygray}{rgb}{0.5,0.5,0.5}
\definecolor{mymauve}{rgb}{0.58,0,0.82}

\lstset{ %
  backgroundcolor=\color{white},   % choose the background color
  basicstyle=\small,        % size of fonts used for the code
  breaklines=true,                 % automatic line breaking only at whitespace
  captionpos=b,                    % sets the caption-position to bottom
  commentstyle=\color{mygreen},    % comment style
  %escapeinside={\%*}{*)},          % if you want to add LaTeX within your code
  keywordstyle=\color{blue},       % keyword style
  stringstyle=\color{mymauve},     % string literal style
  showstringspaces=false,
}
   
\begin{document}
    \begin{frame}
        \titlepage
    \end{frame}
    \begin{frame}   
        \frametitle{Topics for the Session}
        \begin{itemize}
            \item Python UI Libraries - Curses
        \end{itemize}
    \end{frame}
    \begin{frame}
        \frametitle{What is a Curses?}
        \begin{itemize}
            \item The curses library supplies a terminal-independent screen-painting and keyboard-handling facility for text-based terminals
            \item The curses library was originally written for BSD Unix; the later System V versions of Unix from AT\&T added many enhancements and new functions. 
            \item This is a python port of a popular C library \emph{ncurses}.
        \end{itemize}
    \end{frame}
    \begin{frame}[containsverbatim]
        \frametitle{Example :: Hello World}
        \begin{itemize}
            \item Open the terminal
            \item Create a \emph{CursesHelloWorld.py} file and open it using text editor in insert mode
            \item Import curses into the program by writing the following code:
        \end{itemize}
        \begin{lstlisting}[language=Python]
            # Importing curses
            import curses
            # Initialization of curses
            stdscr = curses.initscr() 
        \end{lstlisting}
    \end{frame}
    \begin{frame}
        \frametitle{Example :: Hello World}
        \begin{itemize}
            \item Now save the file and exit from the text editor by pressing ``esc" button. Then write ``:wq" and press enter.
            \item Virtually, whenever you initialize Curses, a default window called ``stdscr" is created, which has the same height and width as the terminal.
            \item Write the below code in ``CursesHelloWorld.py"
        \end{itemize}
    \end{frame}
    \begin{frame}[containsverbatim]
        \frametitle{Example :: Hello World}
        \begin{lstlisting}[language=Python]
        # enable or disable special Key values such as curses.KEY\_LEFT etc
        scr.keypad(0)
        # turn off auto echoing of keypress on to screen
        curses.noecho() 
        # takes a Python string or bytestring as the value to be displayed.
        # Performing an action with screen
        scr.addstr("hello world") 
        \end{lstlisting}
    \end{frame}
    \begin{frame}[containsverbatim]
        \frametitle{Example :: Hello World}
        \begin{lstlisting}[language=Python]
        #  to update the physical screen display.
        scr.refresh() 
        # To get input from the window
        scr.getch()  
        # Terminating a curses
        curses.endwin()
        \end{lstlisting}
    \end{frame}
\end{document}
