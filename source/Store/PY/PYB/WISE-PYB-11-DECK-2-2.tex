\documentclass[14pt]{beamer}
\title{Python 102 :: Week 11}
\date{31-03-2020}
\author[TS]{TalentSprint}
\usefonttheme{serif}
\usepackage{bookman}
\usepackage{hyperref}
\usepackage[T1]{fontenc}
\usepackage{graphicx}
\usecolortheme{orchid}
\beamertemplateballitem

\usepackage{listings}
%\usebackgroundtemplate{\includegraphics[width=\paperwidth]{logo}}
\definecolor{mygreen}{rgb}{0,0.6,0}
\definecolor{mygray}{rgb}{0.5,0.5,0.5}
\definecolor{mymauve}{rgb}{0.58,0,0.82}

\lstset{ %
  backgroundcolor=\color{white},   % choose the background color
  basicstyle=\small,        % size of fonts used for the code
  breaklines=true,                 % automatic line breaking only at whitespace
  captionpos=b,                    % sets the caption-position to bottom
  commentstyle=\color{mygreen},    % comment style
  %escapeinside={\%*}{*)},          % if you want to add LaTeX within your code
  keywordstyle=\color{blue},       % keyword style
  stringstyle=\color{mymauve},     % string literal style
  showstringspaces=false,
}
   
\begin{document}
    \begin{frame}
        \titlepage
    \end{frame}
    \begin{frame}
        \frametitle{Topics for the Session}
        \begin{itemize}
            \item Problem Statement - Cows and Bulls Game
        \end{itemize}
    \end{frame}

    \begin{frame}[containsverbatim]
        \frametitle{Cows and Bulls}
        \begin{itemize}
		\item Create a program that will play the "cows and bulls" game with the user. 
		\item The game works like this:
		\item Randomly generate a 4-digit number.
		\item Ask the user to guess a 4-digit number. 
        \end{itemize}
    \end{frame}

    \begin{frame}[containsverbatim]
        \frametitle{Cows and Bulls - Continued}
        \begin{itemize}
		\item For every digit that the user guessed correctly in the correct place, they have a "cow". 
		\item For every digit the user guessed correctly in the wrong place is a "bull". 
        \end{itemize}
    \end{frame}

    \begin{frame}[containsverbatim]
        \frametitle{Cows and Bulls - Continued}
        \begin{itemize}
		\item Every time the user makes a guess, tell them how many "cows" and "bulls" they have.
		\item Once the user guesses the correct number, the game is over. 
		\item Keep track of the number of guesses the user makes throughout the game and tell the user at the end.
        \end{itemize}
    \end{frame}

    \begin{frame}[containsverbatim]
        \frametitle{Cows and Bulls - Continued}
        \begin{itemize}
		\item Say the number generated by the computer is 1038. 
		\item An example interaction could look like this:
  Welcome to the Cows and Bulls Game! 
  Enter a number: 
  >>> 1234
  2 cows, 0 bulls
  >>> 1256
  1 cow, 1 bull
  ...
Until the user guesses the number.
        \end{itemize}
    \end{frame}

    \begin{frame}[containsverbatim]
        \frametitle{Cows and Bulls Game -  Instructions}
        \begin{itemize}
		\item Ensure the game would work for any number of digits
		\item Game is treated as complete if cows count is equal to number of digits
		\item After a game is complete, repeat the next game as long as user wants to play
        \end{itemize}
    \end{frame}

    \begin{frame}[containsverbatim]
        \frametitle{Secret Number}
        \begin{itemize}
		\item Generate Secret Number using random number generator module random
		\item Ensure secret number has no duplicate digits
        \end{itemize}
    \end{frame}

    \begin{frame}[containsverbatim]
        \frametitle{Possible Methods for Cows and Bulls Class}
        \begin{itemize}
        \item Initialization (\_\_init\_\_()) to maintain properties or variables required for the cows and bulls object
        \item \_count\_bulls(), to count number of bulls (ensure to subtract number of cows from the bulls count)
        \item \_count\_cows(), to count number of cows
        \end{itemize}
    \end{frame}

    \begin{frame}[containsverbatim]
        \frametitle{Possible Methods for Cows and Bulls Class - Continued}
        \begin{itemize}
        \item check\_cows\_bulls(), to call bull and cow count methods (ensure to call cows count before bull count)
        \item finished() to check whether game is complete
        \item score() to know the current score i.e., number of cows and number of bulls
        \end{itemize}
    \end{frame}

\end{document}  
