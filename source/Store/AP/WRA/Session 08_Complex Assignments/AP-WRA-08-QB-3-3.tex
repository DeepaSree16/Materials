
% Options for packages loaded elsewhere
\PassOptionsToPackage{unicode}{hyperref}
\PassOptionsToPackage{hyphens}{url}
%
\documentclass[
]{article}
\usepackage{amsmath,amssymb}
\usepackage{lmodern}
\usepackage{iftex}
\ifPDFTeX
\usepackage[T1]{fontenc}
\usepackage[utf8]{inputenc}
\usepackage{textcomp} % provide euro and other symbols
\else % if luatex or xetex
\usepackage{unicode-math}
\defaultfontfeatures{Scale=MatchLowercase}
\defaultfontfeatures[\rmfamily]{Ligatures=TeX,Scale=1}
\fi
% Use upquote if available, for straight quotes in verbatim environments
\IfFileExists{upquote.sty}{\usepackage{upquote}}{}
\IfFileExists{microtype.sty}{% use microtype if available
	\usepackage[]{microtype}
	\UseMicrotypeSet[protrusion]{basicmath} % disable protrusion for tt fonts
}{}
\makeatletter
\@ifundefined{KOMAClassName}{% if non-KOMA class
	\IfFileExists{parskip.sty}{%
		\usepackage{parskip}
	}{% else
		\setlength{\parindent}{0pt}
		\setlength{\parskip}{6pt plus 2pt minus 1pt}}
}{% if KOMA class
	\KOMAoptions{parskip=half}}
\makeatother
\usepackage{xcolor}
\IfFileExists{xurl.sty}{\usepackage{xurl}}{} % add URL line breaks if available
\IfFileExists{bookmark.sty}{\usepackage{bookmark}}{\usepackage{hyperref}}
\hypersetup{
	hidelinks,
	pdfcreator={LaTeX via pandoc}}
\urlstyle{same} % disable monospaced font for URLs
\usepackage{longtable,booktabs,array}
\usepackage{calc} % for calculating minipage widths
% Correct order of tables after \paragraph or \subparagraph
\usepackage{etoolbox}
\makeatletter
\patchcmd\longtable{\par}{\if@noskipsec\mbox{}\fi\par}{}{}
\makeatother
% Allow footnotes in longtable head/foot
\IfFileExists{footnotehyper.sty}{\usepackage{footnotehyper}}{\usepackage{footnote}}
\makesavenoteenv{longtable}
\usepackage{graphicx}
\makeatletter
\def\maxwidth{\ifdim\Gin@nat@width>\linewidth\linewidth\else\Gin@nat@width\fi}
\def\maxheight{\ifdim\Gin@nat@height>\textheight\textheight\else\Gin@nat@height\fi}
\makeatother
% Scale images if necessary, so that they will not overflow the page
% margins by default, and it is still possible to overwrite the defaults
% using explicit options in \includegraphics[width, height, ...]{}
\setkeys{Gin}{width=\maxwidth,height=\maxheight,keepaspectratio}
% Set default figure placement to htbp
\makeatletter
\def\fps@figure{htbp}
\makeatother
\setlength{\emergencystretch}{3em} % prevent overfull lines
\providecommand{\tightlist}{%
	\setlength{\itemsep}{0pt}\setlength{\parskip}{0pt}}
\setcounter{secnumdepth}{-\maxdimen} % remove section numbering
\ifLuaTeX
\usepackage{selnolig}  % disable illegal ligatures
\fi

\author{}
\date{}
\usepackage{multirow}
\usepackage[inline]{enumitem}
\usepackage[margin=1.0in]{geometry}
\usepackage[english]{babel}
\usepackage[utf8]{inputenc}
\usepackage{fancyhdr}

\pagestyle{fancy}
\fancyhf{}
\rhead{\includegraphics[width=5.21667in, height=0.38819in]{image1.png}}
\lhead{ Reasoning: Complex Assignments }
\lfoot{www.talentsprint.com }
\rfoot{\thepage}
\begin{document}
	
 

\begin{center}
	{\Large \textbf{Complex Assignments \\}}
\end{center}

{\large \textbf{ Additional Examples \\}}

1. \textbf{Directions (I-III):} Study the following information carefully and answer the questions given
below:\\
\includegraphics[width=0.60555in,height=0.32083in]{image2.png}
Five persons V, W, X, Y and Z are participating in a cycling race. They ride on five tracks.
Track 1 is at the extreme left and Track 5 is at the extreme right. W and Z are not cycling
adjacent to each other. Y is not on the tracks on the extreme ends. W is on track three. V is to
the left of X.\\

I. If X is on Track 4 then Z is on which of the following tracks?\\
1) Track 1 \hspace{2mm}2) Track 5 \hspace{2mm}3) Track 2
\hspace{2mm}4) Cannot be determined \hspace{2mm}5) None of these\\

II. X rides definitely on which of the following tracks?\\
1) Track 2 \hspace{2mm}2) Track 4 \hspace{2mm}3) Either 1) or 2)
\hspace{2mm}4) Track 1 \hspace{2mm}5) None of these\\

III. If Y is to the left of V then X is on which of the following tracks?\\
1) Track 4 \hspace{2mm}2) Track 2 \hspace{2mm}3) Track 1 \hspace{2mm}4) Track 5 \hspace{2mm}5) None of these\\

2. There are eight friends – Prema, Tina, Ruchi, Bhumu, Vani, Kriti, Radha and Shreya. They
have \includegraphics[width=0.60555in,height=0.32083in]{image2.png}been given 8 different letters token – A, B, C, D, E, F, G and H, but not necessarily in
the same order. They like different fruits – apple, grapes, banana, lichi, papaya, mango,
orange and melon but not necessarily in the same order. They all are sitting around a
circular table with equal people facing inside and outside. The one having token D is sitting
third to left of one having token A whose name is not Vani and does not like banana. The
one having token C is facing outside and sitting between the ones having tokens F and G
who like mango and papaya respectively. Tine is sitting to the immediate left of the one
having token A who does not like lichi and melon. Shreya is sitting second to right of the
one who likes mango. The one who likes melon is facing inside and sitting second to left of
the one who likes lichi and he is also sitting third to right of Prema. The one having token C
is sitting second to right of the one having token D whose name is ruche and likes apple.
Kriti who likes grapes is neither A nor H. Prema is not sitting to the immediate left of
Ruchi. Both the neighbors of the one having token B are facing inside. The one having token
F is facing inside. The ones having tokens B and C face same direction. The one having
token E is to the immediate right of Radha and he is also sitting second to right of the one
having token H. The one who likes orange is sitting second to left of the one having token G.\\

3. Use the information given below to answer.\\
\includegraphics[width=0.60555in,height=0.32083in]{image2.png}
i) There is a group of 5 persons A, B, C, D and E\\
ii) In the group, there is one badminton player, one chess player and one tennis player\\
iii) A and D are unmarried ladies and do not play any games\\
iv) No lady is a chess player or a badminton player\\
v) There is a married couple in the group of which E is the husband\\
vi) B is the brother of C and is neither a chess player not a tennis player\\

I. Which of the group has only ladies?\\
a) ABC \hspace{2mm}b) BCD \hspace{2mm}c) CDE \hspace{2mm}d) ACD \hspace{2mm}e) None of these\\

II. Who is the tennis player?\\
a) B \hspace{2mm}b) C \hspace{2mm}c) D \hspace{2mm}d) E \hspace{2mm}e) A\\

III. Who is the wife of E?\\
a) A \hspace{2mm}b) B \hspace{2mm}c) D \hspace{2mm}d) C \hspace{2mm}e) None of these\\

4. Study the given information and answer the following questions.\\
\includegraphics[width=0.60555in,height=0.32083in]{image2.png}
A, B, C, D, E, F, G and H are the eight family members. These members are four married
couples and are sitting in a circle facing the centre. Each male member likes different games
football, cricket, tennis and baseball. D and H are sitting together. D takes cricket and H
likes baseball. The wife of each man is seated beside her husband. G, the wife of the person
who likes football is seated second to the right of H. F is seated between G and H. B is the
wife of the person who likes tennis. C does not like tennis and E is male.\\

I. What is F’s position with respect to C?\\
1) Immediate right \hspace{2mm}2) Second to the left \hspace{2mm}3) Immediate left
\hspace{2mm}4) Third to the left \hspace{2mm}5) None of these\\

II. Who is E's wife?\\
1) B \hspace{2mm}2) C \hspace{2mm}3) G \hspace{2mm}4) F \hspace{2mm}5) A\\

III. Which of the following is true?\\
1) G is D's wife \hspace{2mm}2) The husband of A is E \hspace{2mm}3) F is H's wife
\hspace{2mm}4) B is the wife of C \hspace{2mm}5) None of these\\

IV. Whose wives are seated together?\\
1) C and H \hspace{2mm}2) E and C \hspace{2mm}3) D and H \hspace{2mm}4) D and C \hspace{2mm}5) Cannot be determined\\

5. A, B, C, D, E, F, G are seven friends from seven different cities namely Ranchi, Mumbai,
Delhi, Jaipur, \includegraphics[width=0.60555in,height=0.32083in]{image2.png}Patna, Chennai, Kolkata but not necessarily in the same order working in
three departments of a company in Production, HR and Marketing with at least two in each
department and following information is provided\\
1 - F is working in the HR department.\\
2 - The one who is from Mumbai is working in production with E, E is not from Delhi. The
one who is from Delhi is working with B but not with F.\\
3 - G is working only with the one who is from Kolkata but not in HR department.\\
4 - F is working with the one who is from Jaipur.\\
5 - The person who is from Ranchi is neither F nor C.\\
6 - D is working only with the one who is from Patna.\\
7 - A is neither working with B nor with G.\\

6. There are 12 friends A, B, C, D, E, F, G, H, I, J, K, L in a party sitting in two rows facing each
other \includegraphics[width=0.60555in,height=0.32083in]{image2.png}such that the person in 1st row faces south and person in 2nd row faces north and vice
versa. Each person likes one of the color among green, red, yellow, blue, brown, black,
orange, violet, white, pink. Further information is as follows:\\
I. Violet color is liked by 3 persons and all the remaining colors are liked by one person
each.\\
II. J does not like violet.\\
III. F likes black and is 3rd to the right of E.\\
IV. B does not like pink or yellow and sits opposite to the person who likes black.\\
V. A likes white and is 3rd from left end in the 2nd row, opposite to G.\\
VI. E sits between H and A.\\
VII. H is on the left side of A and likes red.\\
VIII. The person who likes pink does not sit at corners.\\
IX. L likes green and sits in the 1st row opposite to the person who likes violet.\\
X. C and E like violet and sit opposite to each other.\\
XI. K likes brown and sits opposite to the person who likes red.\\
XII. I likes orange and sits at one of the corners of 1st row.\\

I. Who is sitting on the immediate right of F?\\
1) D \hspace{2mm}2) I \hspace{2mm}3) J \hspace{2mm}4) B \hspace{2mm}5) None of these\\

II. Who amongst the following is not sitting in the 2nd row?\\
1) A \hspace{2mm}2) E \hspace{2mm}3) J \hspace{2mm}4) B \hspace{2mm}5) F\\

III. Who is sitting opposite to the person who likes black?\\
1) L \hspace{2mm}2) J \hspace{2mm}3) I \hspace{2mm}4) G \hspace{2mm}5) None of these\\

IV. Which of the following statements is false?\\
1) G likes pink \hspace{2mm}2) K is sitting in the 1st row
\hspace{2mm}3) D likes violet \hspace{2mm}4) L is on the immediate right of B
\hspace{2mm}5) J is sitting opposite to the person who likes black\\

V. Who among the following is not sitting at the corner?\\
1) J \hspace{2mm}2) D \hspace{2mm}3) I \hspace{2mm}4) H \hspace{2mm}5) K\\

7. Eight children a, b, c, d, e, f, g, h are going to enact as eight different planets, Mercury,
Venus, Earth, \includegraphics[width=0.60555in,height=0.32083in]{image2.png}Mars, Jupiter, Saturn, Uranus and Neptune in a school play. They are
standing in straight column and are facing the north direction. The position are numbered 1-
8 from front end. The teacher has made the following rules for the play.\\
The one who acts Saturn stands just behind G.
One who acts as Mars is standing 6 places behind the who acts as Earth.\\
The one who plays the role of Neptune is standing at an even number position and is an
immediate neighbour of A and F.\\
A and B are at a gap of 4 positions.\\
B acts as Saturn or Uranus.\\
One who acts as Mercury stands 3 places behind G.\\
One who acts as Mercury stand behind one who acts as Uranus.\\
D acts as Venus.\\
H stands just behind E.\\
One who acts as Jupiter is not the neighbour of one who acts as Venus.\\
E does not act as the Saturn.\\

8. Study the following information carefully and answer the questions given below.\\
\includegraphics[width=0.60555in,height=0.32083in]{image2.png}
Seven MR boys T, U, V, W, X, Y and Z go to seven different hospitals, viz RML, Ganga Ram,
BLK, Guru Nanak, Patel, Apollo and AIIMS on four different days - Tuesday, Thursday,
Saturday and Sunday – of a week but not necessarily in the same order. At least one boy but
not more than two boys go to hospital on each day. W goes to Guru Nanak on Sunday. Y
goes to AIIMS alone. The one who goes to Patel does not go with X or U. U goes on
Thursday and he does not go to BLK. Z does not go to Apollo. V goes on Thursday. The
person who goes to BLK goes with the person who goes to RML. The boy who goes to
Apollo goes on Sunday. T goes on Saturday and goes to Patel.\\

9. Eight cans of cold drinks are arranged in a stack one above another. They are given a serial
number 1-8 \includegraphics[width=0.60555in,height=0.32083in]{image2.png}from bottom to top. The cold drinks are Pepsi, Cococola, 7-Up, Sprite, Thums-
Up, Dew, Maaza and Limca. The cans are to different colours red, black, yellow, blue,
purple, green, pink and white not necessarily in the same order.\\
1) The pink coloured can is 3 cans below Limca.\\
2) 4 drinks are kept between the black coloured can and Dew.\\
3) The serial number of Maaza is twice the serial number of the red can which is kept just
above Pepsi.\\
4) Thums–Up is kept exactly between the blue can and green can.\\
5) Purple can is kept 4 cans above the white can\\
6) The serial number of 7-Up is thrice of the white can\\
7) Sprite is kept just above the blue can\\

10. Read the following information carefully and answer the questions given below:\\
\includegraphics[width=0.60555in,height=0.32083in]{image2.png}
Four sisters Pawell, Sinjo, Jayawardane and Ronaldo go to singing classes. Each of the
sisters take away watch belonging to other sister and a mobile belonging to another sister.
Pawell takes away the watch belonging to sister whose mobile is taken by Sinjo while Sinjo
watch is taken by sister who takes Pawell’s mobile. Ronaldo takes mobile of Jayawardane.\\

11. Directions (I-VI): Study the following information carefully and answer the questions given
below:\\
\includegraphics[width=0.60555in,height=0.32083in]{image2.png}
Six Indian professors from Six different institutions J, M, X, N, P and U went to Abroad to
attend an international conference. They stay in Six successive rooms on the second floor of
a hotel (201-206). Each of them has published papers in a number of Journals and has
donated to a number of institutions last year.\\
\begin{itemize}
\item The professor in room no. 202 has published twice as many journals as the professor
who donated to 8 institutions last year.
\item The professor from U and the professor in Room no. 206 together published a total
number of 40 journals.
\item The professor from J published in 8 journals lesser than the professor from P but
donated to 10 more institutions last year.
\item Four times the number of the journals published by the professor in room number 202 is
equal to the number of institutions to which he donated last year.
\item The professor in room no. 203 published in the number of 12 journals and donated to 8
institutions last year.
\item The professor who published in 16 journals donated to 24 institutions last year.
\item The professor in room no. 205 published in 8 journals and donated to 2 institutions
lesser than the professor from X last year. The professor from X is staying on odd
numbered room.
\item The professor of M is staying two rooms ahead of the professor from P who is staying
two rooms ahead of the professor from X in ascending order of room number.
\item The professor from X and J do not stay in room no. 206.
\item The professor of X did not publish highest number of journals.
\end{itemize}


I. In which of the following rooms is the professor from M staying?\\
a) 201 \hspace{2mm}b) 202 \hspace{2mm}c) 203 \hspace{2mm}d) 204 \hspace{2mm}e) 205\\

II. How many institutions did the professor from J donate to the last year?\\
a) 8 \hspace{2mm}b) 3 \hspace{2mm}c) 18 \hspace{2mm}d) 24 \hspace{2mm}e) None of these\\

III. The professor of which institute is staying in Room no. 206?\\
a) J \hspace{2mm}b) U \hspace{2mm}c) X \hspace{2mm}d) N \hspace{2mm}e) None of these\\

IV. The professor of which institute donated to 24 institutions last year?\\
a) J \hspace{2mm}b) X \hspace{2mm}c) U \hspace{2mm}d) N \hspace{2mm}e) None of these\\

V. The professor of which institute published in the maximum number of journals?\\
a) N \hspace{2mm}b) J \hspace{2mm}c) M \hspace{2mm}d) U \hspace{2mm}e) None of these\\

VI. In how many journals did the J professor publish?\\
a) 8 \hspace{2mm}b) 4 \hspace{2mm}c) 12 \hspace{2mm}d) 20 \hspace{2mm}e) None of these\\

12. \textbf{Directions (I-IV):} Study the following information carefully and answer the questions given
below:\\
\includegraphics[width=0.60555in,height=0.32083in]{image2.png}
There are seven persons in a family, namely, J, K, L, M, N, O and P. All of them are related
to each other in an order. Also each person has a different age
(Note : Assume that wife is younger than husband but older than his younger brother)
J is older than K but younger than N. The third oldest person in the family is 36 years old, K
is sister of J. L is father of J. The third youngest person of the family is 33 years old. P is the
oldest person of the family. K is the niece of O. N is older than K.O is husband of N. P and
M are a married couple. L’s mother is 65 years old. L is younger then O. The oldest of the
family is a male member.\\

I. If the total age of M and K is 80 years, then what is the age of K?\\
a) 17 years \hspace{2mm}b) 20 years \hspace{2mm}c) 15 years \hspace{2mm}d) 19 years \hspace{2mm}e) None of these\\

II. How is M related to N?\\
a) Mother \hspace{2mm}b) Mother in law \hspace{2mm}c) Sister
\hspace{2mm}d) None of these \hspace{2mm}e) Can’t be determined\\

III. What is the possible age of the oldest person of the family?\\
a) 64 years \hspace{2mm}b) 60 years \hspace{2mm}c) 62 years \hspace{2mm}d) 78 years \hspace{2mm}e) 58 years\\

IV. Who among the following is 33 years old?\\
a) O \hspace{2mm}b) N \hspace{2mm}c) L
\hspace{2mm}d) Either b or c \hspace{2mm}e) None of these\\

13. \textbf{Directions (I-VI):} Study the following information carefully and answer the questions given
below:\\
\includegraphics[width=0.60555in,height=0.32083in]{image2.png}
There are seven family members A, B, D, E, F, H and K sitting in a row facing east. There are
two couples in the family. There are three generations in the family. The grandson of family,
who is the only person of third generation sits exactly between grandfather and
grandmother. A sits on the immediate left of H, who is sister of K. D sits at the extreme
north end of the row. D is daughter-in-law of E, who is on the immediate right of F and she
is in first generation only with one person. Only one person sits between the son of K and
maternal uncle of F. the number of persons between B and D and B and E is equal, which is
not more than two. B is not of the second generation and does sit exactly between D and E. E
doesn’t sit at any extreme end. E has only one son and one daughter and one of them is
unmarried.\\

I. Who is the grandson of B?\\
1) A \hspace{2mm}2) E \hspace{2mm}3) F \hspace{2mm}4) H \hspace{2mm}5) Cannot be determined\\

II. How many females are there in the family?\\
1) One \hspace{2mm}2) Two \hspace{2mm}3) Three \hspace{2mm}4) Five \hspace{2mm}5) None\\

III. Who is father-in-law of H’s sister –in-law?\\
1) A \hspace{2mm}2) B \hspace{2mm}3) E \hspace{2mm}4) F \hspace{2mm}5) Cannot be determined\\

IV. Who among the following sits fourth to the right of F’s grandfather?\\
1) H \hspace{2mm}2) E \hspace{2mm}3) K \hspace{2mm}4) A \hspace{2mm}5) F\\

V. How is A related to K?\\
1) Brother \hspace{2mm}2) Brother – in-law \hspace{2mm}3) Sister-in-law
\hspace{2mm}4) Cannot be determined \hspace{2mm}5) Sister\\

VI. Who among the following are immediate neighbours of those at the extreme ends?\\
1) K, E \hspace{2mm}2) B, A \hspace{2mm}3) E, D \hspace{2mm}4) A, K \hspace{2mm}5) D, H\\

14. \textbf{Direction:} Study the following information carefully and answer the questions given below:\\
\includegraphics[width=0.60555in,height=0.32083in]{image2.png}
7 friends - A, B, C, D, E, F and G likes different colors - Red, Pink, Brown, Blue, Black, Green
and Orange but not necessarily in the same order. They go to picnic on three different days
of the week - Tuesday, Wednesday and Sunday. At least two and not more than three
persons go to picnic on the same day.
D likes Green and he does not go to picnic on Sunday. The one who likes Black goes to
picnic on Wednesday. A goes to picnic on Tuesday only with E and he likes Red. B and C do
not go to picnic on the same day. Those who go to picnic on Tuesday do not like Brown
color. F likes Blue but does not go to picnic on Wednesday. G goes to picnic on the same day
as F. C likes Pink color.\\
%\includegraphics[width=0.60555in,height=0.32083in]{image2.png}
%\hspace{2mm}

\end{document}