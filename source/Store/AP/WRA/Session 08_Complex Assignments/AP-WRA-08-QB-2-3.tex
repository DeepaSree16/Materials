
% Options for packages loaded elsewhere
\PassOptionsToPackage{unicode}{hyperref}
\PassOptionsToPackage{hyphens}{url}
%
\documentclass[
]{article}
\usepackage{amsmath,amssymb}
\usepackage{lmodern}
\usepackage{iftex}
\ifPDFTeX
\usepackage[T1]{fontenc}
\usepackage[utf8]{inputenc}
\usepackage{textcomp} % provide euro and other symbols
\else % if luatex or xetex
\usepackage{unicode-math}
\defaultfontfeatures{Scale=MatchLowercase}
\defaultfontfeatures[\rmfamily]{Ligatures=TeX,Scale=1}
\fi
% Use upquote if available, for straight quotes in verbatim environments
\IfFileExists{upquote.sty}{\usepackage{upquote}}{}
\IfFileExists{microtype.sty}{% use microtype if available
	\usepackage[]{microtype}
	\UseMicrotypeSet[protrusion]{basicmath} % disable protrusion for tt fonts
}{}
\makeatletter
\@ifundefined{KOMAClassName}{% if non-KOMA class
	\IfFileExists{parskip.sty}{%
		\usepackage{parskip}
	}{% else
		\setlength{\parindent}{0pt}
		\setlength{\parskip}{6pt plus 2pt minus 1pt}}
}{% if KOMA class
	\KOMAoptions{parskip=half}}
\makeatother
\usepackage{xcolor}
\IfFileExists{xurl.sty}{\usepackage{xurl}}{} % add URL line breaks if available
\IfFileExists{bookmark.sty}{\usepackage{bookmark}}{\usepackage{hyperref}}
\hypersetup{
	hidelinks,
	pdfcreator={LaTeX via pandoc}}
\urlstyle{same} % disable monospaced font for URLs
\usepackage{longtable,booktabs,array}
\usepackage{calc} % for calculating minipage widths
% Correct order of tables after \paragraph or \subparagraph
\usepackage{etoolbox}
\makeatletter
\patchcmd\longtable{\par}{\if@noskipsec\mbox{}\fi\par}{}{}
\makeatother
% Allow footnotes in longtable head/foot
\IfFileExists{footnotehyper.sty}{\usepackage{footnotehyper}}{\usepackage{footnote}}
\makesavenoteenv{longtable}
\usepackage{graphicx}
\makeatletter
\def\maxwidth{\ifdim\Gin@nat@width>\linewidth\linewidth\else\Gin@nat@width\fi}
\def\maxheight{\ifdim\Gin@nat@height>\textheight\textheight\else\Gin@nat@height\fi}
\makeatother
% Scale images if necessary, so that they will not overflow the page
% margins by default, and it is still possible to overwrite the defaults
% using explicit options in \includegraphics[width, height, ...]{}
\setkeys{Gin}{width=\maxwidth,height=\maxheight,keepaspectratio}
% Set default figure placement to htbp
\makeatletter
\def\fps@figure{htbp}
\makeatother
\setlength{\emergencystretch}{3em} % prevent overfull lines
\providecommand{\tightlist}{%
	\setlength{\itemsep}{0pt}\setlength{\parskip}{0pt}}
\setcounter{secnumdepth}{-\maxdimen} % remove section numbering
\ifLuaTeX
\usepackage{selnolig}  % disable illegal ligatures
\fi

\author{}
\date{}
\usepackage{multirow}
\usepackage[inline]{enumitem}
\usepackage[margin=1.0in]{geometry}
\usepackage[english]{babel}
\usepackage[utf8]{inputenc}
\usepackage{fancyhdr}

\pagestyle{fancy}
\fancyhf{}
\rhead{\includegraphics[width=5.21667in, height=0.38819in]{image1.png}}
\lhead{ Reasoning: Complex Assignments }
\lfoot{www.talentsprint.com }
\rfoot{\thepage}
\begin{document}
	
 

\begin{center}
	{\Large \textbf{Complex Assignments \\}}
\end{center}

{\large \textbf{ Part 2 - Advanced \\}}

1. \textbf{Directions (I-V):} Study the following information carefully and answer the questions.\\
\includegraphics[width=0.60555in,height=0.32083in]{image2.png}
Seven people, P, Q, R, S, T, U and V have a seminar, but not necessarily in the same order, in
seven different months (of the same year), namely January, February, March, June, August,
October, December. Each of them also likes a different fruit, namely Banana, Grapes,
Papaya, Orange, Mango, Litchi and Apple, but not necessarily in the same order.\\
R has a seminar in a month which has less than 31 days. Only two people have a seminar
between R and S. The one who likes Banana has a seminar immediately before S. Only one
person has a seminar before the one who likes Papaya. Q has a seminar immediately after
the one who likes Papaya. Only three people have a seminar between Q and the one who
likes Mango. T likes neither Mango nor Papaya. P has a seminar immediately before T. V
likes Apple. The one who likes Grapes has a seminar in the month which has less than 31
days. The one who has a seminar in March does not like Orange.\\

I. In which of the following months does S have a seminar?\\
1) January \hspace{2mm}2) Cannot be determined \hspace{2mm}3) October
\hspace{2mm}4) December \hspace{2mm}5) June\\

II. Who among the following have seminars in January and June respectively?\\
1) V,S \hspace{2mm}2) U,S \hspace{2mm}3) Q,T \hspace{2mm}4) V,R \hspace{2mm}5) U,R\\

III. How many people has/have a seminar between the months in which V and R have
seminars?\\
1) None \hspace{2mm}2) Two \hspace{2mm}3) Three \hspace{2mm}4) One \hspace{2mm}5) More than three\\

IV. As per the given arrangement, R is related to Banana and Q is related to Orange
following a certain pattern. Which of the following is V related to following the same
pattern?\\
1) Mango \hspace{2mm}2) Litchi \hspace{2mm}3) Apple \hspace{2mm}4) Papaya \hspace{2mm}5) Grapes\\

V. Which of the following fruits does U like?\\
1) Orange \hspace{2mm}2) Papaya \hspace{2mm}3) Mango \hspace{2mm}4) Banana \hspace{2mm}5) Grapes\\

2. \textbf{Directions:} Study the following information carefully and answer the questions given
below it.\\
\includegraphics[width=0.60555in,height=0.32083in]{image2.png}
Nine boxes namely – A, B, C, D, E, P, Q, R and S are kept one above other. Each box contains
different articles viz. Pen, Ring, Cup, Ball, Book, Laptop, Toy, Watch and Calculator. Each
box is wrapped with different colored paper viz. Black, Blue, Green, Yellow, Brown, White,
Red, Pink and Orange. All the given information is not necessary in same order. Box Q
which contains Watch is kept at a gap of three boxes from one which is wrapped with Blue
paper. Only one box is kept between box that contains Watch and box C. Only three boxes
are kept between box C and the one that contains Ring, which is kept at top. Box which
contains Toy is neither kept adjacent to the one which contains Laptop nor kept at bottom.\\
The box which contains Book and box C, which doesn’t contain Pen are kept adjacent to
each other. Box A is wrapped with Yellow paper is kept at a gap of three boxes from the one
which is wrapped with Green paper. Box S, which is neither kept adjacent to box Q nor with
the one that contains Book, but it is kept at a gap of three boxes from the one which contains
Pen. Box that contains Pen and Calculator are kept adjacent to each other. Only four boxes
are kept between the one which contains Pen and the one which is wrapped with Red
paper. Only two boxes are kept between the one that contains Ball and the box which is
wrapped with Black paper. The box wrapped with Green paper is kept just above box D
which contains Laptop. Box that wrapped with Brown paper is not kept at bottom. Box that
contains Cup kept just below one wrapped with White paper. Box E, which is neither
wrapped with Pink nor white paper is kept at any place above box R, which is kept at a gap
of one box from box B. Number of boxes between E and R is same as number of boxes
between box P and one which wrapped with Orange paper. Box P neither wrapped with
Brown paper nor contains Books. Box which wrapped with Pink paper is kept at any place
above box wrapped with Orange paper. Box which contains Toy and one wrapped with
Brown paper is kept adjacent to each other.\\

3. \textbf{Directions (I-V):} Six candidates who were interviewed for faculty positions in a reputed
man- \includegraphics[width=0.60555in,height=0.32083in]{image2.png}agement institute are ranked according to their performances in six parameters –
Educational qualification (E), Analytical Ability (A), Logical Ability (L), Communication
Skills (V), Teaching Skills (T), and Creativity (C). No two candidates got the same rank in
any single parameter and the rank of a candidate in no two parameters are the same. The six
candidates considered for selection were Vishal, Payal, Rabi, Mohit, Akash and Sindhu and
the rank obtained by them in some of the parameters are given below. A candidate get 12
points for each parameter in which he/she gets the first rank, 8 points for rank 2, 5 points for
rank 3, 3 points for rank 4, 2 points for rank 5 and 1 point for rank 6.\\

\begin{tabular}{|c|c|c|c|c|c|c|c|c|c|c|c|c|}
\hline
Person/Parameters &Vishal &Payal &Rabi &Mohit &Akash &Sindhu\\ 
\hline
Educational Qualification(E) &3 & & 1 & & & \\
\hline
Analytical Ability (A) &2 & & & & & 4\\
\hline
Logical Ability (L) &1 &2 &3 &4 &5 &6\\
\hline
Communication Skills (V) & & & 4 & & & \\
\hline
Teaching Skills (T) &6 & & & 3 & & \\
\hline
Creativity (C) & & & 1 & 3 & & 2\\
\hline
\end{tabular}

I. Which of the following candidates obtained first rank in teaching skills (T)?\\
a) Rabi \hspace{2mm}b) Akash \hspace{2mm}c) Sindhu \hspace{2mm}d) Visha \hspace{2mm}e) None of these\\

II. If only parameters A, C, V and T are considered, then which candidate got the third
highest total score?\\
a) Vishal \hspace{2mm}b) Payal \hspace{2mm}c) Rabi \hspace{2mm}d) Mohit \hspace{2mm}e) None of these\\

III. If only parameters E, A, L and T are considered, then which candidate got the highest
total score?\\
a) Mohit \hspace{2mm}b) Rabi \hspace{2mm}c) Vishal \hspace{2mm}d) Akash \hspace{2mm}e) None of these\\

IV. If only parameters C, L and T are considered, then the difference between the scores
of the candidates getting the highest score and least score is\\
a) 6 \hspace{2mm}b) 8 \hspace{2mm}c) 13 \hspace{2mm}d) 14 \hspace{2mm}e) None of these\\

V. If only parameters E, A and V are considered and only the candidates getting the top
four scores are selected, then which of the candidates are not selected?\\
a) Mohit and Rabi \hspace{2mm}b) Rabi and Sindhu \hspace{2mm}c) Sindhu and Mohit
\hspace{2mm}d) Rabi and Vishal \hspace{2mm}e) None of these\\

4. Study the following information carefully and complete the arrangement:\\
\includegraphics[width=0.60555in,height=0.32083in]{image2.png}
Ajay, Kamal, Sandhya, Rajkumar, Sharda are five aspirants for SBI PO 2017. To Pursue their
aim they joined classroom coaching of TalentSprint. They joined the classes at different
centres. These centres Located in Delhi, Lucknow, Patna, Bhubaneshwar, Hyderabad. No
two centres of TalentSprint are in a single town. Owing to the different problems they were
confronting they joined different types of courses. Names of the courses are BFC, MFC,
DCC, PPC and TS. Each person has different surnames viz, DIXIT, TRIPATHI, BAJPAI,
SAHOO, AND SINGH. One who joined BFC course is associated with TalentSprint centre at
Hyderabad. Three people among them, i.e. Ajay Dixit, Ms Sahoo, and the one who is
associated with Delhi centre, are good at Nonverbal reasoning. But the other two Mr.Singh
and the one who has joined PPC are good at Verbal reasoning. No person is good at both
Nonverbal and verbal Reasoning. One who has joined PPC is not Kamal. Sharda and the
person who has joined DCC have been friends since school. Rajkumar has joined TS and is
Good at Verbal Reasoning. Sharda is not associated with Delhi, but she is good at Nonverbal
Reasoning. Ms. Sahoo and the person who is associated with Patna centre are unfamiliar
with each other but good at Nonverbal Reasoning. Ms.Triphati has joined PPC.\\

5. Study the following information carefully and complete the arrangement:\\
\includegraphics[width=0.60555in,height=0.32083in]{image2.png}
Seven people namely, A, B, C, D, E, F and G like different brands of jeans and footwear.\\
They are also wearing different brands of watches among Hublot, Rolex, Fastrack, Fossil,
Omega, IWC and Armani. Different brands of jeans are Peter England, US Polo, Jack &
Jones and Diesel. Not more than two people like the same brand of jeans but at least one
person like the above-mentioned brand. Different brands of Footwear are Clarks, Puma and
Timberland. Not more than three people like the same brand of footwear at least one person
like the above-mentioned brand. The following information is known about them:\\
i. C doesn’t wear Rolex and D doesn’t wear Armani watch.\\
ii. G wears Hublot watch and D likes Clarks footwear.\\
iii. E and G like same footwear brand and are the only two who like that brand.\\
iv. C and A don’t like same footwear brand.\\
v. E doesn’t like Peter England jeans brand but wears either Rolex or Fossil watch.\\
vi. G likes a jeans brand, which is liked by none other than him.\\
vii. The persons wearing Armani and IWC watches have same liking for jeans and footwear
brands and they are the only ones who like same jean and footwear brands as well. None of
them is C.\\
viii. Two people like Peter England jeans brand.
ix. One of the persons who likes Jack & Jones jeans wears Fastrack watch and in any case is
not E and likes either Puma or Timberland footwear.\\
x. A likes US Polo jeans.\\
xi. The one who wears Omega watch likes Puma footwear but doesn’t like Jack & Jones
jeans.\\
xii. C and F like the same jeans brand.\\

6. \textbf{Directions (I-VII):} Study the given information and answer the following questions.\\
\includegraphics[width=0.60555in,height=0.32083in]{image2.png}
There are five units i.e. 1, 2, 3, 4 and 5. Each unit has a different height. Also each unit
contains books and boxes. Unit 2 is above unit 1 and unit 3 is above unit 2 and so on. Every
unit belongs to different country i.e. Beijing, Paris, London, Sydney and Zurich. The total
height of all five units is equal to 252 ft. Total height of a unit is equal to the height of books
plus height of boxes in each unit. Height of books is not equal to the height of boxes, unless
specified so. The books that belong to London is in an even unit. The total height of unit 1 is
75 ft. Sydney does not belong to unit 1. The total height of unit, which belongs to Sydney is
55 ft. There is only one gap between London and Paris. The height of books and height of
boxes in unit 3 are equal. The height of books in unit 2 is not less than 30 ft. The height of
books in unit 4 is 4 ft more than the books, which are in unit 3. The total height of the unit
for London is not 37 ft. The total height of unit, which is 37 ft high is not taking the place,
which is immediate above the unit, which contains 20 ft more than the unit, which belongs
to Sydney. The height of boxes in unit 2 is 23 ft. Unit belonging to Zurich does not contain
equal height of books and boxes. The total height of unit 2 is an odd number and is more
than 50 ft and less than 55 ft. Height of boxes in Unit 1 is 23 ft more than the height of boxes
in unit 4 and height of books in unit 5 is 7 ft less than that in unit 1.\\

I. What is the total height in unit 3?\\
a) 37 ft \hspace{2mm}b) 32 ft \hspace{2mm}c) 53 ft \hspace{2mm}d) 75 ft \hspace{2mm}e) None of these\\

II. Unit 3 belongs to which country?\\
a) Paris \hspace{2mm}b) Zurich \hspace{2mm}c) Sydney \hspace{2mm}d) Beijing \hspace{2mm}e) London\\

III. If ‘Sydney’ is related to 37 ft in the same way as ‘Beijing’ is related to 53 ft.Which of the
following is ‘Paris’ related to, following the same pattern?\\
a) 53 ft \hspace{2mm}b) 37 ft \hspace{2mm}c) 75 ft \hspace{2mm}d) 32 ft \hspace{2mm}e) None of these\\

IV. Four of the following five are alike in certain way and hence they form group. Which
one of the following does not belong to that group?\\
a) Sydney \hspace{2mm}b) 32 ft \hspace{2mm}c) 75 ft \hspace{2mm}d) Paris \hspace{2mm}e) Beijing\\

V. What is the height of box in unit 4?\\
a) 23 ft \hspace{2mm}b) 17 ft \hspace{2mm}c) 27 ft \hspace{2mm}d) 40 ft \hspace{2mm}e) 20 ft\\

VI. Which unit contains 30 ft book?\\
a) None of these \hspace{2mm}b) Unit 2 \hspace{2mm}c) Unit 3 \hspace{2mm}d) Unit 1 \hspace{2mm}e) Unit 5\\

7. \textbf{Directions(I-IV):} These questions are based on the following data\\
\includegraphics[width=0.60555in,height=0.32083in]{image2.png}
Exactly six persons from amongst five boys A, B, C, D, E and four girls P, Q, R, S are to sit in
six chairs, which are arranged in a row from left to right, and the others must stand. The
following conditions are to be adhered to while making the arrangement\\
i) No two girls sit in adjacent seats\\
ii) Exactly three boys should be among those who are seated in these six chairs\\
iii) A and P are seated next to each other\\
iv) If E sits, then R also sits and vice-versa, but they do not sit next to each other\\
v) If P or R sits, then Q will stand\\

I. If C sits at the extreme left end and A is sitting in the second seat from the extreme right
end, then who sits to the immediate right of A?\\
1) P \hspace{2mm}2) R \hspace{2mm}3) Q
\hspace{2mm}4) Cannot be determined \hspace{2mm}5) None of these\\

II. Which of the following is not a valid arrangement of persons sitting from left to right?\\
1) P, A, E, S, C, R \hspace{2mm}2) A, P, E, S, C, R \hspace{2mm}3) S, E, P, A, R, D
\hspace{2mm}4) B, R, E, A, P, Q \hspace{2mm}5) None of these\\

III. If P is sitting at the extreme left end, then who could be sitting at the extreme right end?\\
1) R or E \hspace{2mm}2) E or S \hspace{2mm}3) S or R
\hspace{2mm}4) R or E or S \hspace{2mm}5) None of these\\

IV. If it is known that S sits in the third seat from the extreme left end and B sits in the
second seat from the extreme right, then who sits exactly next to two girls?\\
1) A \hspace{2mm}2) E \hspace{2mm}3) B
\hspace{2mm}4) Data Inadequate \hspace{2mm}5) None of these\\

8. Solve the below Arrangement\\
\includegraphics[width=0.60555in,height=0.32083in]{image2.png}
Nine friends A , B, C, D, E, F, G, H, and I are living in a same building of nine floor counting
from one to nine. Ground floor is numbered 1 and so on. They all like different colours i.e
plum, cyan, magenta, maroon, azure, crimson, chartreuse, pink, and grey but not
necessarily in the same order. There are three floors between A and I, who likes azure
colour. H likes magenta colour and lives immediately above the floor on which A lives.\\
There is only one floor between H and G, who likes maroon colour. F likes cyan colour and
lives below the floor on which G lives. F does not live on even number floor. There is only
two floors between F and E, who likes plum colour. B likes pink colour and lives below the
floor on which F lives. B lives on even number floor. There is as many as floor between B
and one, who like crimson colour and between A and C. A does not like chartreuse colour.\\

9. \textbf{Directions (I-V):} Study the information below and answer questions\\
\includegraphics[width=0.60555in,height=0.32083in]{image2.png}
Eight persons A, B, C, D E, F , G and H all of them face north direction but not necessarily in
same order. All of them stay in different floors viz. 2nd, 6th, 10th, 15th, 20th, 30th, 31st, and
37th but not necessarily in same order. D sits fifth to the left of the person who stays on 20th
floor. The person on the 20th floor does not sit at extreme ends. There are two persons
sitting between the one who stays on 20th floor and F. The person staying on 6th floor sits
second to the right of G. G is not an immediate neighbor of D. G does not sit on 20th floor.\\
Difference between the numerical values for floor numbers of immediate neighbours of G is
10. The person staying on 10th floor is not an immediate neighbor of F. Two persons are
sitting between A and the person staying on 10th floor. Neither F nor G is staying on 10th
floor. More than one person sits between B and the person staying on 2nd floor. The floors
on which C and H are staying are higher than the floor upon which A is staying. C is not an
immediate neighbor of D. One person is sitting between B and the person staying on 31st
floor.\\

I. H lives on which floor?\\
1) 10th \hspace{2mm}2) 37th \hspace{2mm}3) 2nd \hspace{2mm}4) 30th

II. How many persons sit between A and E?\\
1) Two \hspace{2mm}2) One \hspace{2mm}3) Three \hspace{2mm}4) Four\\

III. Who among the following sits immediate left of the person one who lives on 15th floor?\\
1) A \hspace{2mm}2) F \hspace{2mm}3) D \hspace{2mm}4) B\\

IV. Who among the following sits third to right of F?\\
1) D \hspace{2mm}2) C \hspace{2mm}3) A \hspace{2mm}4) B\\

10. Seven people M, N, O, P, Q, R and S stay in an eight storied building with the lowermost
floor numbered \includegraphics[width=0.60555in,height=0.32083in]{image2.png}as 1 and so on. One of the floors is vacant. Each of them like a different
fruits viz, Banana, Apple, Plum, Orange, Papaya, Pear and Mango not necessarily in the
same order. The arrangement is based on the following rules:\\
\begin{itemize}

\item S lives on an even numbered floor.
\item The floor immediately above or below the one in which S resides is vacant.
\item There is a gap of four floors between the one who likes Mango and the floor that is vacant.
\item Q likes Apple and lives immediately below the one who likes Papaya.
\item There is a gap of one floor between the one who likes Papaya and the floor that is vacant.
\item R's immediate neighbours like Pear and Papaya.
\item P lives above M and M likes neither Papaya nor Banana.
\item There are four floors between the ones who like Orange and Banana. One who likes
Orange stays on a floor higher than the one who likes Banana.
\item N does not like Papaya.
\end{itemize}

11. \textbf{Directions(I-IV):} These questions are based on the following information.\\
\includegraphics[width=0.60555in,height=0.32083in]{image2.png}
Eight students – A, B, C, D, E, F, G and H – went to four different places among Resort,
Beach, Hotel and Cinema, such that each place was visited by two students each.
Each student visited exactly one place. After their return, their teacher asked them about the
place visited by each of them. Following were their answers:\\
i) A said “I did not go with C or D and went to the Resort or the Cinema”.\\
ii) B said “ I did not go with E or G and went to the Hotel or the Cinema.\\
iii) C said “ I did not go with D or F and went to the Beach or the Resort.\\
iv) D said “ I did not go with B or H and went to the Beach or the Hotel”.\\
v) E said “ I went with B or C or D or F or H and went to the Cinema or the Beach”.\\
vi) F said “ I did not go with A or G and went to the Resort or the Cinema”.\\
vii) G said “ I went with B or D or E or F or H and went to the Beach or the Hotel”.\\
viii) H said “ I did not go with C or A and went to the Resort or the Beach”.\\

I. Who among the following went with A?\\
a) E \hspace{2mm}b) B \hspace{2mm}c) G \hspace{2mm}d) H \hspace{2mm}e) None of these\\

II. E went with \_\_\_\_ and visited the \_\_\_\\
a) C, Beach \hspace{2mm}b) F, Cinema \hspace{2mm}c) D, Beach \hspace{2mm}d) G, Beach \hspace{2mm}e) None of these\\

III. If only D and H lied about the places visited by them, then with whom did D visit the place
of his choice?\\
a) H \hspace{2mm}b) F \hspace{2mm}c) G \hspace{2mm}d) None of these \hspace{2mm}e) Cannot be determined\\

IV. If A and G exchange their company with each other then which pair is travelling to Hotel?\\
a) A,B \hspace{2mm}b) B,D \hspace{2mm}c) G,D \hspace{2mm}d) A,D \hspace{2mm}e)None of these\\

12. \textbf{Directions(I-V):} Study the information carefully and answer the questions given below.\\
\includegraphics[width=0.60555in,height=0.32083in]{image2.png}
Anbu, Babu, Chandu, Dena, Elango, Ferros, Gnanam, Hari, Ibrahim, John, Karthi and Latha
are 12 sports persons sitting in two rows, among them Anbu, Chandu, Dena, Ferros,
Ibrahim and Karthi are facing north while the remaining are facing south. Each person faces
exactly one person in the other row. Each one of them plays a specific sport from among
Cricket, Hockey, Rugby, Tennis, Football, Basketball, Swimming, Wrestling, Boxing,
Archery, Athletics and Chess (not necessarily in the same order) and belongs to one city
each from among Bangalore, Indore, Coimbatore, Gwalior, Darjeeling, Nagpur, Thane,
Mangalore, Vijayawada, Kozhikode, Jaipur and Surat (again not necessarily in the same
order).\\
\begin{itemize}
\item The football player is from Nagpur and is facing Ferros. None of Babu, Latha and
Elango play football or cricket. The players from Coimbatore and Surat are the
immediate neighbours of John.
\item Babu is second to the right of Gnanam, who is from Thane. Babu is a swimmer and
faces the player from Kozhikode, who is third to the right of Dena. Neither Anbu nor
Dena are hockey players or from Indore.
\item Karthi, the rugby player, sits second to the right of the archery player, who faces the
player from Surat, none of them sits at an extreme end.
\item The hockey player, who is from Jaipur, sits as far as possible from Dena.
\item Latha, the basketball player, sits opposite the athlete, who is an immediate neighbor
of the person sitting at an extreme end.
\item The swimmer and tennis player are from Coimbatore and Bangalore respectively.
The basketball player and tennis player face the same direction.
\item Ferros, the wrestler from Gwalior, sits equidistant in the same row from the boxer
and the player from Indore.
\item The player from Vijayawada is opposite the player from Bangalore and second to the
left of the player from Gwalior.
\item Anbu, who is not the athlete, is third to the left of Ibrahim; and neither of them is a
hockey player.
\item Hari is the chess player and is third to the right of basketball player from Darjeeling.
\end{itemize}


I. How many persons are there in between the wrestler and the person from Jaipur?\\
1) Three \hspace{2mm}2) Two \hspace{2mm}3) Four \hspace{2mm}4) Either a) or b) \hspace{2mm}5) None of these\\

II. The archer is belongs to which city?\\
1) Gwalior \hspace{2mm}2) Indore \hspace{2mm}3) Mangalore \hspace{2mm}4) Jaipur \hspace{2mm}5) Nagpur

III. Who is boxer?\\
1) Anbu \hspace{2mm}2) Karthi \hspace{2mm}3) Chandu \hspace{2mm}4) Dena \hspace{2mm}5) Babu\\

IV. Who among the following is not sitting opposite to each other?\\
1) Dena, Elango \hspace{2mm}2) John, Ferros \hspace{2mm}3) Latha, Ibrahim
\hspace{2mm}4) Gnanam, Chandu \hspace{2mm}5) John, Karthi\\

V. Who among the following is not seated at the end of the row?\\
1) Elango \hspace{2mm}2) Karthi \hspace{2mm}3) Gnanam \hspace{2mm}4) Dena \hspace{2mm}5) None of these\\

13. \textbf{Directions:} Study the following information to answer the given questions.\\
\includegraphics[width=0.60555in,height=0.32083in]{image2.png}
Rohan, Bina, Aarti, Vishal, Jay, Kavita, Arun and Savita all have laptops of different brands,
that are HP, Asus, Dell, Apple, Sony, Samsung, Acer and Lenovo not necessarily in that
order All the laptops have different operating systems, such as Vista, W-8, MacOs, Xp, W-7,
Linux, Unix, and Chrome OS installed on each one of them All laptops are either white,
black, blue, or grey in colour
Following is the additional information
Vishal has a grey laptop but not Asus There is only one more person with grey laptop
Lenovo has W-8 But is not blue in colour
Savita has a blue laptop with W-7 installed Dell is not black or blue in colour
Apple has MacOS and is white in colour, there is only one white laptop Asus is grey in
colour
Samsung laptop with Chrome OS is blue in colour but is not owned by Savita
A black laptop has Vista installed Linux is installed in a blue laptop Number of black and
grey laptops is equal
Sony and Acer do not have vista installed Acer does not have W-7 installed
Rohan and aarti have black coloured laptops Arun and Kavita have same coloured laptops
Jay doesn’t have an apple laptop He has Unix installed in his laptop\\

14. Eight persons A, B, C, D, E, F, G, G, H are sitting around a circular table and they all are of
 \includegraphics[width=0.60555in,height=0.32083in]{image2.png}different ages. People whose age is an even number is facing inside the circle and other
facing outside the circle of the table. Each of them likes different food and plays a different
game. All the given information is not necessarily in same order. The one who likes Italian
food sits to the second right of youngest person in this group. The oldest person in this
group likes Chinese food and plays Polo game. The one who likes mexico food also plays
Wrestling sits to second of E. The one whose age is 17 years old sits second to left of G. Two
persons sit between G and H. The one who sits immediate right of H is 41 years of old. One
of the immediate neighbours of H is facing inside. The one who likes Italian food faces the
one who likes Indian food. The youngest person in this group likes Taiwan food who sits to
the second left of the one who plays Hockey. A sits second to the right of H. The one who
likes Punjabi food sits to the third right of G who sits immediate right of the one who likes
Gujarati food who face outside. The one who plays Football sits to the second left of D who
sits to the immediate left of the one who plays Tennis. Three persons sits between the one
whose age is 33 years old and the one whose age is 19 years. H's age is neither 33 nor 19. B is
an immediate neighbour of H, but not an immediate neighbour of A. C is 22 years old. The
one who sits second to the left of A is 25 years old. The one whose age is 33 years old plays
baseball and likes French food. The one who plays cricket is not an immediate neighbour of
the one who plays Baseball. G is 7 years younger than H. Neither C nor B is an immediate
neighbour of D. E is 3 years younger than B, but not the youngest person in the group. Two
persons sit between the one who is the youngest and A. One person in this group plays
kabaddi. B's age is an even number.\\

\textbf{Answers}\\
1. I. December II. V,R III. Two IV. Litchi V. Papaya\\

2. -----\\

3. I. Sindhu II. Payal III. Vishal IV. 13 V. Rabi and Sindhu\\

4. ---\\

5. -\\

6. I. 32 ft \hspace{2mm}II. Beijing \hspace{2mm}III. 32 ft \hspace{2mm}IV. Paris \hspace{2mm}V. 17 ft \hspace{2mm}VI. Unit 2\\

7. I. R \hspace{2mm}II. B, R, E, A, P, Q \hspace{2mm}III. R or E or S \hspace{2mm}IV. A\\

8. ---\\

9. ----\\

10. ----\\

11. I. B \hspace{2mm}II. C, Beach \hspace{2mm}III. F \hspace{2mm}IV. None of these\\

12. I) Two \hspace{2mm}II) Mangalore \hspace{2mm}III) Dena \hspace{2mm}IV) John and Karthi \hspace{2mm}V) Karthi\\

13. ----\\

14. -----\\
%\includegraphics[width=0.60555in,height=0.32083in]{image2.png}
%\hspace{2mm}

\end{document}