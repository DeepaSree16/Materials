
% Options for packages loaded elsewhere
\PassOptionsToPackage{unicode}{hyperref}
\PassOptionsToPackage{hyphens}{url}
%
\documentclass[
]{article}
\usepackage{amsmath,amssymb}
\usepackage{lmodern}
\usepackage{iftex}
\ifPDFTeX
\usepackage[T1]{fontenc}
\usepackage[utf8]{inputenc}
\usepackage{textcomp} % provide euro and other symbols
\else % if luatex or xetex
\usepackage{unicode-math}
\defaultfontfeatures{Scale=MatchLowercase}
\defaultfontfeatures[\rmfamily]{Ligatures=TeX,Scale=1}
\fi
% Use upquote if available, for straight quotes in verbatim environments
\IfFileExists{upquote.sty}{\usepackage{upquote}}{}
\IfFileExists{microtype.sty}{% use microtype if available
	\usepackage[]{microtype}
	\UseMicrotypeSet[protrusion]{basicmath} % disable protrusion for tt fonts
}{}
\makeatletter
\@ifundefined{KOMAClassName}{% if non-KOMA class
	\IfFileExists{parskip.sty}{%
		\usepackage{parskip}
	}{% else
		\setlength{\parindent}{0pt}
		\setlength{\parskip}{6pt plus 2pt minus 1pt}}
}{% if KOMA class
	\KOMAoptions{parskip=half}}
\makeatother
\usepackage{xcolor}
\IfFileExists{xurl.sty}{\usepackage{xurl}}{} % add URL line breaks if available
\IfFileExists{bookmark.sty}{\usepackage{bookmark}}{\usepackage{hyperref}}
\hypersetup{
	hidelinks,
	pdfcreator={LaTeX via pandoc}}
\urlstyle{same} % disable monospaced font for URLs
\usepackage{longtable,booktabs,array}
\usepackage{calc} % for calculating minipage widths
% Correct order of tables after \paragraph or \subparagraph
\usepackage{etoolbox}
\makeatletter
\patchcmd\longtable{\par}{\if@noskipsec\mbox{}\fi\par}{}{}
\makeatother
% Allow footnotes in longtable head/foot
\IfFileExists{footnotehyper.sty}{\usepackage{footnotehyper}}{\usepackage{footnote}}
\makesavenoteenv{longtable}
\usepackage{graphicx}
\makeatletter
\def\maxwidth{\ifdim\Gin@nat@width>\linewidth\linewidth\else\Gin@nat@width\fi}
\def\maxheight{\ifdim\Gin@nat@height>\textheight\textheight\else\Gin@nat@height\fi}
\makeatother
% Scale images if necessary, so that they will not overflow the page
% margins by default, and it is still possible to overwrite the defaults
% using explicit options in \includegraphics[width, height, ...]{}
\setkeys{Gin}{width=\maxwidth,height=\maxheight,keepaspectratio}
% Set default figure placement to htbp
\makeatletter
\def\fps@figure{htbp}
\makeatother
\setlength{\emergencystretch}{3em} % prevent overfull lines
\providecommand{\tightlist}{%
	\setlength{\itemsep}{0pt}\setlength{\parskip}{0pt}}
\setcounter{secnumdepth}{-\maxdimen} % remove section numbering
\ifLuaTeX
\usepackage{selnolig}  % disable illegal ligatures
\fi

\author{}
\date{}
\usepackage{multirow}
\usepackage[inline]{enumitem}
\usepackage[margin=1.0in]{geometry}
\usepackage[english]{babel}
\usepackage[utf8]{inputenc}
\usepackage{fancyhdr}

\pagestyle{fancy}
\fancyhf{}
\rhead{\includegraphics[width=5.21667in, height=0.38819in]{image1.png}}
\lhead{ Reasoning: Complex Assignments }
\lfoot{www.talentsprint.com }
\rfoot{\thepage}
\begin{document}
	
 

\begin{center}
	{\Large \textbf{Complex Assignments \\}}
\end{center}

\textbf{Part 1 - Basic}\\

1. \textbf{Directions (I-V):} Study the following information carefully and answer the questions given
below:\\
\includegraphics[width=0.60555in,height=0.32083in]{image2.png}
A, B, C, D, E, F, G and H are employees of a company who go to station to catch different
trains for different cities. There were different trains on the station, viz Garib Rath, Swarn
Jayanti, Rajdhani, Shatabdi, Magadh Express, Premium, Duranto and Chennai Express. All
trains go to different cities, viz, Patna, Ranchi, Kolkata, Jaipur, Lucknow, Kanpur, Bengaluru
and Pune, but not necessarily in the same order.
A goes to Lucknow. Swarn Jayanti goes to Kolkata. H travels in Shatabdi Express and E goes
to Patna. C travels in that train which goes to Kolkata. Rajdhani goes to Ranchi. B goes to
Pune by Chennai Express. G goes to Kanpur by Duranto Express. D travels by Premium
train. A and D do not go to Jaipur and they do not go by Magadh Express.\\

I. Who among the following goes to Bengaluru?\\
a) H \hspace{2mm}b) D \hspace{2mm}c) C \hspace{2mm}d) B \hspace{2mm}e) None of these\\

II. Garib Rath goes to which of the following cities?\\
a) Lucknow \hspace{2mm}b) Jaipur \hspace{2mm}c) Bengalure
\hspace{2mm}d) Can’t be determined \hspace{2mm}e) None of these\\

III. H goes to which of the following cities?\\
a) Kolkata \hspace{2mm}b) Bengaluru \hspace{2mm}c) Jaipur \hspace{2mm}d) Ranchi \hspace{2mm}e)None of these\\

IV. C goes by which of the following trains?\\
a) Swam Jayanti \hspace{2mm}b) Premium \hspace{2mm}c) Rajdhani
\hspace{2mm}d) Garib Rath \hspace{2mm}e) None of these\\

V. Which of the following combinations is true?\\
a) H – Jaipur – Shatabdi Express \hspace{2mm}b) D – Bengaluru – Premium
\hspace{2mm}c) E – Patna – Magadh Express \hspace{2mm}d) All are true
\hspace{2mm}e) None of these

2. \textbf{Directions (I-III):} Study the following information carefully and answer the questions given
below:\\
\includegraphics[width=0.60555in,height=0.32083in]{image2.png}
Five persons V, W, X, Y and Z are participating in a cycling race. They ride on five tracks.
Track 1 is at the extreme left and Track 5 is at the extreme right. W and Z are not cycling
adjacent to each other. Y is not on the tracks on the extreme ends. W is on track three. V is to
the left of X.\\

I. If X is on Track 4 then Z is on which of the following tracks?\\
1) Track 1 \hspace{2mm}2) Track 5 \hspace{2mm}3) Track 2
\hspace{2mm}4) Cannot be determined \hspace{2mm}5) None of these\\

II. X rides definitely on which of the following tracks?\\
1) Track 2 \hspace{2mm}2) Track 4 \hspace{2mm}3) Either 1) or 2)
\hspace{2mm}4) Track 1 \hspace{2mm}5) None of these\\

III. If Y is to the left of V then X is on which of the following tracks?\\
1) Track 4 \hspace{2mm}2) Track 2 \hspace{2mm}3) Track 1 \hspace{2mm}4) Track 5 \hspace{2mm}5) None of these\\

3. \textbf{Directions (I-V):} Study the given information and answer the following questions.\\
\includegraphics[width=0.60555in,height=0.32083in]{image2.png}
Krishna Jewellers held a contest in seven different cities: Delhi, Haridwar, Aligarh, Agra,
Sonipat, Ghaziabad and Faridabad. It was organized for seven days, starting from Sunday
and ending on Saturday in July 2013. The contest was held only in one city in a day. It was
held in only four cities between the duration of Faridabad and Aligarh contests. The
Faridabad contest was not held on Friday. Only one contest was held between the
Faridabad contest and the Haridwar contest. The Delhi contest was held just after the
Faridabad contest. The Sonipat contest was held just before the Ghaziabad contest. The
Ghazidabad contest was not held on Thursday.\\

I. How many contests were held between the contests of Faridabad and Sonipat?\\
1) Two \textbf{}2) Three \hspace{2mm}3) Four
\hspace{2mm}4) More than four \hspace{2mm}5) None of these\\

II. Which of the following statements is true?\\
1) The Agra contest was held on Saturday.\\
2) The Haridwar contest was held on Sunday.\\
3) The Ghaziabad contest was held on Friday.\\
4) The Delhi contest was held on Monday.\\
5) None of these\\

III. Four of the following five options are similar in a certain way, which is based on the given
order and hence form a group. Which one does not belong to the group?\\
1) Sunday-Agra \hspace{2mm}2) Monday- Faridabad
\hspace{2mm}3) Wednesday- Haridwar \hspace{2mm}4) Friday- Delhi \hspace{2mm}5) Saturday – Aligarh\\

IV. Which of the following cities held the contest on Wednesday?\\
1) Agra \hspace{2mm}2) Aligarh \hspace{2mm}3) Delhi \hspace{2mm}4) Haridwar \hspace{2mm}5) Ghaziabad\\

V. The Ghaziabad contest was held on which day?\\
1) Saturday \hspace{2mm}2) Sunday \hspace{2mm}3) Thursday \hspace{2mm}4) Monday \hspace{2mm}5) Friday\\

4. \textbf{Directions (I-V):} Study the following information carefully and answer the questions given
below:\\\includegraphics[width=0.60555in,height=0.32083in]{image2.png}
Eight friends A, B, C, D, E, F, G and H work in different companies, viz Puma, Nike, IBM,
Adidas, Sony, Titan, Walmart and Motorola. They belong to different cities, viz Patna,
Ranchi, Kolkata, Jaipur, Lucknow, Kanpur, Bengaluru and Pune, but not necessarily in the
same order.
A belongs to Lucknow. The one who works in Nike belongs to Kolkata. H works in Adidas
and E belongs to Patna. The one who works in IBM belongs to Ranchi. B belongs to Pune
and works in Motorola. G belongs to Kanpur and works in Walmart. D works in Titan. A
and D do not belong to Jaipur and they do not work in Sony. C does not belong to Ranchi.\\

I. D belongs to which of the following cities?\\
1) Bengaluru \hspace{2mm}2) Ranchi \hspace{2mm}3) Jaipur
\hspace{2mm}4) Pune \hspace{2mm}5) None of these\\

II. The one who works in Puma belongs to which of the following cities?\\
1) Ranchi \hspace{2mm}2) Lucknow \hspace{2mm}3) Pune \hspace{2mm}4) Jaipur \hspace{2mm}5) None of these\\

III. Who among the following belongs to Jaipur?\\
1) C \hspace{2mm}2) D \hspace{2mm}3) H \hspace{2mm}4) B \hspace{2mm}5) None of these\\

IV. F works in which of the following companies?\\
1) Sony \hspace{2mm}2) Nike \hspace{2mm}3) Puma \hspace{2mm}4) IBM \hspace{2mm}5) None of these\\

\textbf{Answers}\\

1. I. D \hspace{2mm}II. Lucknow \hspace{2mm}III. Jaipur \hspace{2mm}IV. Swam Jayanti \hspace{2mm}V. All are true\\
2.\\
3. I. Two \hspace{2mm}II. The Ghaziabad contest was held on Friday. \hspace{2mm}III. Friday- Delhi \hspace{2mm}IV. Haridwar \hspace{2mm}V. Friday\\
4. I. Bengaluru \hspace{2mm}II. Lucknow \hspace{2mm}III. H \hspace{2mm}IV. IBM\\
%\includegraphics[width=0.60555in,height=0.32083in]{image2.png}
%\hspace{2mm}

\end{document}