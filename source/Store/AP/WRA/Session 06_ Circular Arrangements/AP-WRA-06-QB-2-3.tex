
% Options for packages loaded elsewhere
\PassOptionsToPackage{unicode}{hyperref}
\PassOptionsToPackage{hyphens}{url}
%
\documentclass[
]{article}
\usepackage{amsmath,amssymb}
\usepackage{lmodern}
\usepackage{iftex}
\ifPDFTeX
\usepackage[T1]{fontenc}
\usepackage[utf8]{inputenc}
\usepackage{textcomp} % provide euro and other symbols
\else % if luatex or xetex
\usepackage{unicode-math}
\defaultfontfeatures{Scale=MatchLowercase}
\defaultfontfeatures[\rmfamily]{Ligatures=TeX,Scale=1}
\fi
% Use upquote if available, for straight quotes in verbatim environments
\IfFileExists{upquote.sty}{\usepackage{upquote}}{}
\IfFileExists{microtype.sty}{% use microtype if available
	\usepackage[]{microtype}
	\UseMicrotypeSet[protrusion]{basicmath} % disable protrusion for tt fonts
}{}
\makeatletter
\@ifundefined{KOMAClassName}{% if non-KOMA class
	\IfFileExists{parskip.sty}{%
		\usepackage{parskip}
	}{% else
		\setlength{\parindent}{0pt}
		\setlength{\parskip}{6pt plus 2pt minus 1pt}}
}{% if KOMA class
	\KOMAoptions{parskip=half}}
\makeatother
\usepackage{xcolor}
\IfFileExists{xurl.sty}{\usepackage{xurl}}{} % add URL line breaks if available
\IfFileExists{bookmark.sty}{\usepackage{bookmark}}{\usepackage{hyperref}}
\hypersetup{
	hidelinks,
	pdfcreator={LaTeX via pandoc}}
\urlstyle{same} % disable monospaced font for URLs
\usepackage{longtable,booktabs,array}
\usepackage{calc} % for calculating minipage widths
% Correct order of tables after \paragraph or \subparagraph
\usepackage{etoolbox}
\makeatletter
\patchcmd\longtable{\par}{\if@noskipsec\mbox{}\fi\par}{}{}
\makeatother
% Allow footnotes in longtable head/foot
\IfFileExists{footnotehyper.sty}{\usepackage{footnotehyper}}{\usepackage{footnote}}
\makesavenoteenv{longtable}
\usepackage{graphicx}
\makeatletter
\def\maxwidth{\ifdim\Gin@nat@width>\linewidth\linewidth\else\Gin@nat@width\fi}
\def\maxheight{\ifdim\Gin@nat@height>\textheight\textheight\else\Gin@nat@height\fi}
\makeatother
% Scale images if necessary, so that they will not overflow the page
% margins by default, and it is still possible to overwrite the defaults
% using explicit options in \includegraphics[width, height, ...]{}
\setkeys{Gin}{width=\maxwidth,height=\maxheight,keepaspectratio}
% Set default figure placement to htbp
\makeatletter
\def\fps@figure{htbp}
\makeatother
\setlength{\emergencystretch}{3em} % prevent overfull lines
\providecommand{\tightlist}{%
	\setlength{\itemsep}{0pt}\setlength{\parskip}{0pt}}
\setcounter{secnumdepth}{-\maxdimen} % remove section numbering
\ifLuaTeX
\usepackage{selnolig}  % disable illegal ligatures
\fi

\author{}
\date{}
\usepackage{multirow}
\usepackage[inline]{enumitem}
\usepackage[margin=1.0in]{geometry}
\usepackage[english]{babel}
\usepackage[utf8]{inputenc}
\usepackage{fancyhdr}

\pagestyle{fancy}
\fancyhf{}
\rhead{\includegraphics[width=5.21667in, height=0.38819in]{image1.png}}
\lhead{ Reasoning: Circular Arrangements }
\lfoot{www.talentsprint.com }
\rfoot{\thepage}
\begin{document}
	
 

\begin{center}
	{\Large \textbf{Circular Arrangements \\}}
\end{center}

{\large \textbf{ Part 2 - Advanced \\}}

1. \textbf{Directions (I-VII):} Study the following information carefully and answer the given
questions:\\
\includegraphics[width=0.60555in,height=0.32083in]{image2.png}
A, B, C, D, E, F, G and H are sitting around a circle around a circle facing the centre but not
necessarily in the same order.\\
1) B sits second to left of H’s husband. No female is an immediate neighbor of B.\\
2) D’s daughter sits second to right of F. F is the sister of G. F is not an immediate neighbor
of H’s Husband.\\
3) Only one person sits between A and F. A is the father of G. H’s brother D sits to the
immediate left of H’s mother. Only one person sits between H’s mother and E.\\
4) Only one person sits between H and G. G is the mother of C. G is not an immediate
neighbour of E.\\

I. What is the position of A with respect to his mother in-law?\\
1) Immediate left \hspace{2mm}2) Third to the right \hspace{2mm}3) Third to the left
\hspace{2mm}4) Second to the right \hspace{2mm}5) Fourth to the left\\

II. Who amongst the following is D’s daughter?\\
1) B \hspace{2mm}2) C \hspace{2mm}3) E \hspace{2mm}4) G \hspace{2mm}5) H\\

III. What is the position of A with respect to his grandchild?\\
1) Immediate right \hspace{2mm}2) Third to the right \hspace{2mm}3) Third to the left
\hspace{2mm}4) Second to the left \hspace{2mm}5) Fourth to the left\\

IV. How many people sit between G and her uncle?\\
1) One \hspace{2mm}2) Two \hspace{2mm}3) Three \hspace{2mm}4) Four \hspace{2mm}5) More than four\\

V. Four of the following five are alive in a certain way based on the given information and
so form a group. Which is the one that does not belong to that group?\\
1) F \hspace{2mm}2) C \hspace{2mm}3) E \hspace{2mm}4) H \hspace{2mm}5) G\\

VI. Which of the following is true with respect to the given seating arrangement?\\
1) C is the cousin of E\\
2) H and H’s husband are immediate neighbours of each other.\\
3) No female is an immediate neighbour of C\\
4) H sits third to left of her daughter\\
5) B is the mother of H\\

VII. Who sits to the immediate left of C?\\
1) F’s grandmother \hspace{2mm}2) G’s son \hspace{2mm}3) D’s mother-in-law
\hspace{2mm}4) A \hspace{2mm}5) G\\

2. \textbf{Directions (I-V):} Study the following information carefully and answer the given
questions.\\
\includegraphics[width=0.60555in,height=0.32083in]{image2.png}
P, Q, R, S, T, V, W and X are captains of eight different cricket teams, namely Australia, New
Zealand, India, Pakistan, Sri Lanka, England, West Indies and South Africa, but not
necessarily in the same order. All of them are seated around a circular table and are facing
centre.\\
P sits third to the left of the Sri Lanka captain. Only two people sit between T and W.
Neither T nor W is immediate neighbour of P. Neither T nor W is the captain of Sri Lanka.
The captain of South Africa sits second to the right of S. S is not an immediate neighbour of
P. S is not the Sri Lankan captain and P is not the captain of South Africa. The Australian
captain sits third to the left of V. The Australian and Sri Lanka captains are not immediately
neighbours. Only one person sits between S and the Indian captain. Captains of Pakistan
and New Zealand are immediate neighbours. S is not the captain of New Zealand’s team.\\
Only one person sits between Q and the captain of England. The captain of England is an
immediate neighbour of X. W and Q are not immediate neighbours.\\

I. How many people sit between T and the captain of England when counted in clockwise
direction form T?\\
1) None \hspace{2mm}2) One \hspace{2mm}3) Two \hspace{2mm}4) Four \hspace{2mm}5) Five\\

II. Who is the captain of the Australian team?\\
1) P \hspace{2mm}) V \hspace{2mm}3) W \hspace{2mm}4) T \hspace{2mm}5) Q\\

III. Which of the following is true with respect to the given seating arrangement?\\
1) R is the captain of South Africa. 2) W is an immediate neighbour of V.\\
3) The captains of Australia and England are immediate neighbours.\\
4) Four people sit between W and Q.\\
5) X sits second to the left of S.\\

IV. Who is the Indian captain?\\
1) Q \hspace{2mm}2) V \hspace{2mm}3) X \hspace{2mm}4) T \hspace{2mm}5) Cannot be determined\\

V. What is the position of the captain of West Indies with respect to R?\\
1) Immediately left \hspace{2mm}2) Second to the left \hspace{2mm}3) Third to the right
\hspace{2mm}4) Second to the right \hspace{2mm}5) Third to the left\\

3. \textbf{Directions (I-V):} Study the following information carefully and answer the questions given
below:\\
\includegraphics[width=0.60555in,height=0.32083in]{image2.png}
Eight friends E, F, G, H, S, T, U and V are sitting around a square table in such a way that
four of them at four corners of the square table while four sit in middle of each of the four
sides. The ones who sit at the four corners face the centre while those who sit in the middle
of the sides face outside.\\
\begin{itemize}
    \item S is an immediate neighbour of both E and V. S sits in the middle of one of the sides of
the table.
    \item Only one person sits between E and H.
    \item T sits second to the right of U. U is not an immediate neighbour of E but V.
    \item F is not an immediate neighbour of G.
    \item G faces a direction opposite to that of U. (i.e. If U faces the centre then G faces outside
and vice-versa)
\end{itemize}

I. How many people sit between F and G when counted from the right side of F?\\
1) Three \hspace{2mm}2) Two \hspace{2mm}3) Four \hspace{2mm}4) None \hspace{2mm}5) One\\

II. Which of the following is true regarding F?\\
1) Only three people sit between F and H\\
2) F sits in middle of one of the sides\\
3) F sits second to right of U\\
4) None of the given options is true.\\
5) Both S and G are the immediate neighbours of F.\\

III. Who sits to the immediate right of H?\\
1) U \hspace{2mm}2) S \hspace{2mm}3) G \hspace{2mm}4) F \hspace{2mm}5) T\\

IV. Four of the following five are alike in a certain way based on the given arrangement and
so form a group. Which is the one that does not belong to that group?\\
1) T \hspace{2mm}2) V \hspace{2mm}3) G \hspace{2mm}4) H \hspace{2mm}5) E\\

V. What is the position of U with respect to E?\\
1) Third to the right \hspace{2mm}2) Fifth to the left \hspace{2mm}3) Second to the left
\hspace{2mm}4) Third to the left \hspace{2mm}5) Second to the right\\

4. \textbf{Directions (I-III):} Study the following information carefully and answer the questions given
below:\\
\includegraphics[width=0.60555in,height=0.32083in]{image2.png}
Seven friends A, B, C, D, E, F and G are sitting around a circular table facing either the
centre or outside. Each one of them belongs to different department, Finance, Marketing,
Sales, HR, Corporate Finance, Investment Banking and Operations but not necessarily in the
same order. C sits third to the right of G. G faces the centre. Only one person sits between C
and the person working in the HR department. Immediate neighbours of C face outside.
Only one person sits between F and D. Both F and D are facing the centre. D does not work
in HR Department. A works in Investment Banking Department. A face the centre. Two
person sit between the person who works in Investment Banking and Marketing
departments. The person who works in the Corporate Finance sits immediate left of E. C
faces same direction as E. The person who works in the Corporate Finance sits to the
immediate left of the person who work for Operations department.\\

I. For which of the following Department Does B works?\\
1) Finance \hspace{2mm}2) HR \hspace{2mm}3) Operation \hspace{2mm}4) Marketing \hspace{2mm}5) Corporate Finance\\

II. Who sits to the immediate right of E?\\
1) The person works for Marketing Department\\
2) The person works for HR Department\\
3) C \hspace{2mm}4) B \hspace{2mm}5) A\\

III. Who sits exactly between C \& the person who works for HR department?\\
1) The person works for Marketing Department\\
2) The person works for Operation Department\\
3) B \hspace{2mm}4) D \hspace{2mm}5) G\\

5. \textbf{Directions (I-V):} Study the following information carefully and answer the questions below:\\
\includegraphics[width=0.60555in,height=0.32083in]{image2.png}
Ravi, Minal, Himanshu, Amit, Santosh, Ankit, Usha and Savita are eight persons sitting
around a circle facing center in one arrangement and in a straight line facing north in
another arrangement. Usha sits on the immediate left of Savita in the circle, but both are not
immediate neighbours of each other in the straight line. Santosh sits third to the right of
Minal in the circle, while fourth to his left in the straight line. The one who sits on the
immediate left of Minal in straight line is sitting on the immediate right of Minal in the
circle. Ankit and Himanshu are the immediate neighbour of Minal in both the
arrangements, but Himanshu is not at the extreme ends of the row. One of the immediate
neighbour of Savita in straight line sits opposite Savita in the circle. The one who sits on the
extreme left end sits second to the right of Santosh in the circle. Savita is not on the
immediate left of Ankit in both the arrangements. Amit sits third to right of Ankit in the
straight line.\\

I. Who sits third to left of Ravi in the circle?\\
a) Santosh \hspace{2mm}b) Minal \hspace{2mm}c) Himanshu \hspace{2mm}d) Savita \hspace{2mm}e) None of these\\

II. Which of the following pair sits at extreme ends of the row?\\
a) Savita, Minal \hspace{2mm}b) Usha, Ravi \hspace{2mm}c) Amit, Ankit
\hspace{2mm}d) Amit, Usha \hspace{2mm}e) None of these\\

III. Who among the following sits on the immediate left of Amit in the circle?\\
a) Ravi \hspace{2mm}b) Minal \hspace{2mm}c) Himanshu \hspace{2mm}d) Santosh e) None of these\\

IV. The person sitting exactly between Usha and Ankit in the straight line is sitting at what
position in the circle?\\
a) Second to right of Minal \hspace{2mm}b) Immediate right of Ravi
\hspace{2mm}c) Between Himanshu and Usha \hspace{2mm}d) Third to left of Santosh
\hspace{2mm}e) None of these\\

V. The one who sitting in the left end of the row is sitting at what position in the circle?\\
a) Between Minal and Ankit \hspace{2mm}b) Immediate right of Savita
c) Between Santosh and Usha \hspace{2mm}d) Second to the right of Santosh
e) None of these\\

6. Study the following information carefully and complete the arrangement:\\
\includegraphics[width=0.60555in,height=0.32083in]{image2.png}
Eight members A, B, C, D, E, F, G and H of a family are sitting around a circular table. Four
among them are facing the center of a table whereas rest four members are facing outward.
There are three married couples among them. Each family member is born in a different
year. Among A and B, one is facing the center and the other is facing outward. D is facing
opposite side of that of A and is sitting beside A. G is the oldest member and is beside the
one who is facing A. A’s husband is facing inward. B is brother of G and they have 8 years
of gap between them. C is the son of F who is the only sister-in-law of B. C was born when
his mother was 20 years old. G’s wife was born in 1964 and is younger to G by 3 years. D is
younger to his sister by 5 years. C is second to the left of B’s wife and is to the immediate left
of C’s father. C is the nephew of A’s husband. F is three places from C and is sitting beside
C’s wife who is 3 years younger to C. D is the maternal uncle of H and is third to the left of
his brother-in-law. H is the youngest among them and was born in 2010. H is third to right
of his Grandfather.\\

7. \textbf{Directions:} Study the following information carefully to answer the given questions.\\
\includegraphics[width=0.60555in,height=0.32083in]{image2.png}
Ten members of a family are sitting in a restaurant in two parallel rows of chairs containing
five people each, in such a way that there is equal distance between adjacent persons.
In row 1: M, N , O , P and Q are seated and all of them are facing south. In row 2: A, B, C, D,
and E are seated and all of them are facing north. Each of them likes different flavor of ice
cream, viz Butterscotch, Vanilla, Strawberry, Black Cherry, Chocobar, Mango Bar, Butter
Cluster, Tutti Frutti, Orange Sorbet and Kurly Wurly but not necessarily in the same order.
In the given seating arrangement, each member seated in a row faces another member of the
other row. D sits third to the left of the person who likes Orange Sorbet. M, who likes Black
Cherry faces the immediate neighbour of D. O, who likes Strawberry, sits second to the
right of M. Only one person sits between N and P, who like Vanilla and Mango Bar
respectively. B and E are immediate neighbours of each other. E who does not face M and N,
likes Butterscotch. B does not like Orange Sorbet. A sits second to the right of the person
who likes Choco Bar. C likes neither Black Cherry nor Butter Cluster. The one who likes
Vanilla faces the one who likes Kurly Wurly. Q does not like Black Cherry.\\

I. Who faces the one who likes Butter Cluster?\\
a) E \hspace{2mm}b) A \hspace{2mm}c) B \hspace{2mm}d) D \hspace{2mm}e) None of these\\

II. Which of the following combinations is false in respect of the given information?\\
a) D- Kurly Wurly \hspace{2mm}b) M – Black Cherry \hspace{2mm}c) Q- Orange Sorbet
\hspace{2mm}d) Data inadequate \hspace{2mm}e) None of these\\

III. Which of the following information is true in respect of the given information?\\
a) D likes Tutti Frutti \hspace{2mm}b) P likes Mango Bar and sits on the immediate left of N
\hspace{2mm}c) A likes Black Cherry \hspace{2mm}d) E is the immediate neighbour of B and D
\hspace{2mm}e) None of these\\

IV. Who likes Black Cherry?\\
a) Q \hspace{2mm}b) D \hspace{2mm}c) C \hspace{2mm}d) M \hspace{2mm}e) None of these\\

V. Who sits third to the left of N?\\
a) P \hspace{2mm}b) Q \hspace{2mm}c) M \hspace{2mm}d) O \hspace{2mm}e) None of these\\

8. Study the following information carefully and complete the arrangement:\\
\includegraphics[width=0.60555in,height=0.32083in]{image2.png}
There are eight friends – Prema, Tina, Ruchi, Bhumu, Vani, Kriti, Radha and Shreya. They
have been given 8 different letters token – A, B, C, D, E, F, G and H, but not necessarily in
the same order. They like different fruits – apple, grapes, banana, lichi, papaya, mango,
orange and melon but not necessarily in the same order. They all are sitting around a
circular table with equal people facing inside and outside. The one having token D is sitting
third to left of one having token A whose name is not Vani and does not like banana. The
one having token C is facing outside and sitting between the ones having tokens F and G
who like mango and papaya respectively. Tine is sitting to the immediate left of the one
having token A who does not like lichi and melon. Shreya is sitting second to right of the
one who likes mango. The one who likes melon is facing inside and sitting second to left of
the one who likes lichi and he is also sitting third to right of Prema. The one having token C
is sitting second to right of the one having token D whose name is ruche and likes apple.
Kriti who likes grapes is neither A nor H. Prema is not sitting to the immediate left of Ruchi.
Both the neighbors of the one having token B are facing inside. The one having token F is
facing inside. The ones having tokens B and C face same direction. The one having token E
is to the immediate right of radha and he is also sitting second to right of the one having
token H. The one who likes orange is sitting second to left of the one having token G.\\

9. \textbf{Directions (I-IV):} Study the given information and answer the following questions.\\
\includegraphics[width=0.60555in,height=0.32083in]{image2.png}
A, B, C, D, E, F, G and H are the eight family members. These members are four married
couples and are sitting in a circle facing the centre. Each male member likes different games
football, cricket, tennis and baseball. D and H are sitting together. D takes cricket and H
likes baseball. The wife of each man is seated beside her husband. G, the wife of the person
who likes football is seated second to the right of H. F is seated between G and H. B is the
wife of the person who likes tennis. C does not like tennis and E is male.\\

I. What is F’s position with respect to C?\\
1) Immediate right \hspace{2mm}2) Second to the left \hspace{2mm}3) Immediate left
\hspace{2mm}4) Third to the left \hspace{2mm}5) None of these\\

II. Who is E's wife?\\
1) B \hspace{2mm}2) C \hspace{2mm}3) G \hspace{2mm}4) F \hspace{2mm}5) A\\

III. Which of the following is true?\\
1) G is D's wife \hspace{2mm}2) The husband of A is E \hspace{2mm}3) F is H's wife
\hspace{2mm}4) B is the wife of C \hspace{2mm}5) None of these\\

IV. Whose wives are seated together?\\
1) C and H \hspace{2mm}2) E and C \hspace{2mm}3) D and H \hspace{2mm}4) D and C \hspace{2mm}5) Cannot be determined\\

10. Six couples have been invited to a dinner party. They are Nitika, Geetika, Lajwanti, Rekha,
Savitri, \includegraphics[width=0.60555in,height=0.32083in]{image2.png}Chameli and Faizal, Harbhajan, Akshay, Tirlochan, Ranveer, Aamir. They are seated
on a circular table facing each other.\\
(i) Geetika refuses to sit next to Aamir\\
(ii) Lajwanti wants to be between Akshay and Harbhajan\\
(iii) Chameli refuses to sit next to Faizal\\
(iv) Nitika is seated on Aamir's right hand side\\
(v) Faizal and Tirlochan are seated exactly opposite to each other\\
(vi) Ranveer and Savitri are seated to the left of Chameli\\
(vii) Akshay and Rekha want to enjoy the company of Lajwanti and Tirlochan respectively
and are seated closest to them\\
(viii) The seating arrangement is such that minimum one woman is always between two
men\\

11. There are 16 persons – B,C,D,E,F,G,H,I,P,Q,R,S,T,U,V and W standing in a Circular plot.\\
\includegraphics[width=0.60555in,height=0.32083in]{image2.png}Inside a circular plot, a circularly shaped garden is developed. The persons who are
standing inside the garden facing outside. The persons who are standing outside the garden
facing inside the centre and lives in a different number of floors. So all the persons standing
in the inner circle faces the persons standing in the outer circle and hold a different number
of chocolates.\\
G faces outside and S faces G. D sits immediate right of G. There are four persons sits
between G and E. H is not an immediate neighbour of E. There are two persons standing
between D and H. H faces R. There are three persons standing between R and U. U stands
exactly between the B and F. B faces D. There are two persons standing between P and C.
Neither S nor R is an immediate neighbour of P. I stands to the immediate left of H. I faces
T. The one who faces F stands exactly between the person's Q and W. W faces P. H stands
second to the left of G. B lives on the second floor and sits exactly opposite to the person
who lives on the floor which is the square number of the floor of B. F lives on the third floor
and stands exactly opposite to the person who lives on the floor which is the square number
of the floor of F. P lives on 6th floor and S lives immediately above P. U lives immediately
below B. R lives immediately above T. The one who faces P holds chocolates two less than
the number of the floor occupied by P. The one who faces U holds chocolates six more than
the number of the floor occupied by U. Number of chocolates hold by E is the difference
between the number of chocolates hold by D and W. Number of chocolates hold by G is the
sum of the number of chocolates hold by D and E also equals to number of chocolates hold
by V and H. Number of chocolates hold by I is the square of the number of chocolates hold
by H.\\

12. \textbf{Directions (I-II):} Study the information carefully and answer the questions given below.\\
\includegraphics[width=0.60555in,height=0.32083in]{image2.png}
There are two square fields of different size such that the larger one is surrounding smaller
field. Four gates are there for each field in the middle of the sides. Eight people A, B, C, D, E,
F, G and H are standing at different gates but not necessary in the same sequence. The
persons who are on the sides of larger park facing center and The persons who are at side of
smaller park facing outside such that inner sides persons and outer sides persons are facing
each other.\\
There is one person standing between B and D. C faces B. A is to the immediate right of C. G
is not the immediate neighbor of D. G faces neither D nor F. One person is standing between
H and F. E is facing the center.\\

I. Which of the following persons are facing to each other?\\
1) BD \hspace{2mm}2) EB \hspace{2mm}3) FH \hspace{2mm}4) DE \hspace{2mm}5) AH\\

II. Four of the following five are alike in a certain way based from a group which one of the
following does not belong to that group?\\
1) EF \hspace{2mm}2) CH \hspace{2mm}3) DA \hspace{2mm}4) FC \hspace{2mm}5) BH\\

13. \textbf{Directions (I-V):} Study the following information and answer the questions below\\
\includegraphics[width=0.60555in,height=0.32083in]{image2.png}
Five girls N, K, J, M and P are sitting in a circle and facing towards the center. They like
different colors Green, Blue, Yellow, Black and Red and study in different classes 12th, 8th,
7th, 6th and 3rd. One who studies in odd number class does not like Red and the oldest sit
second to the left of N. J is neither oldest nor youngest and does not like Yellow. M sits
between P and J.K is not studying in even number class. One who likes Black is in 8th class.
Three people sit between the one who likes Blue and N. Youngest like Green, and Blue sits
second to the right of Yellow. N studying in 8th class. The girl who likes Red sits immediate
left of Yellow.\\
I. Who is in 7th class?\\
1) One who likes Red \hspace{2mm}2) One who is immediate right of N
\hspace{2mm}3) One who does not like Yellow \hspace{2mm}4) who sits second to the left of J
\hspace{2mm}5) One who likes Green\\

II. Who sits exactly between M and N?\\
1) One who likes Blue \hspace{2mm}2) One who studies in 7th
\hspace{2mm}3) One who likes Yellow \hspace{2mm}4) One who studies in 6th
\hspace{2mm}5) 1 and 4 both\\

III. What is true about M?\\
1) Does not like Black colour \hspace{2mm}2) Studies in 12th class
\hspace{2mm}3) Second to the left of K \hspace{2mm}4) Likes Green colour
\hspace{2mm}5) 1, 3 and 4\\

IV. If another girl sits between K and P, then which statement is correct?\\
1) She is immediate right of K \hspace{2mm}2) She second to the right of J
\hspace{2mm}3) She is third to the left of M \hspace{2mm}4) None of these
5) A and C\\

V. What is the correct sequence of colors if we check anti-clockwise from P?\\
1) Red, Blue, Green, Yellow, Black \hspace{2mm}2) Red, Yellow, Blue, Black, Green
\hspace{2mm}3) Yellow, Black, Blue, Green, Red \hspace{2mm}4) Red, Yellow, Black, Blue, Green
\hspace{2mm}5) Green, Blue, Black, Yellow, Red\\

14. \textbf{Directions (I-IV):} Study the information carefully and answer the questions\\
\includegraphics[width=0.60555in,height=0.32083in]{image2.png}
9 friends J, K, L, M, N, O, P, Q and R were seated around a circular table facing the center.
They were playing the game of cards and everyone was holding a single card bearing a rank
from 2nd - 10th but not necessarily in the same order. It was further known that:\\
\begin{itemize}
    \item N and K had cards whose product was equal to the product of M and L 's cards.
    \item M and L had even ranked cards.
    \item K had a rank that was twice of L's card rank. K and L sat together.
    \item The person with the lowest card rank was 2nd to the left of Q
    \item M sat 4th to the left of L and the sum of their card ranks was equal to the card rank of O
    \item P got a smaller card rank than N
    \item 3 persons were seated between the persons having the card ranks 7 and 5 when counted in a clockwise manner from the person who had 5 rank card.
    \item R was seated 3rd to the left of K. R had a card rank greater than K
    \item Q was to the immediate right of O.
\end{itemize}

I. What was the position of J with respect to the one who have card rank 6?\\
1) Immediate left \hspace{2mm}2) Immediate right
\hspace{2mm}3) Second to the right \hspace{2mm}4) Second to the left
\hspace{2mm}5) Third to the right\\

II. How many persons had a card rank greater than the card rank of the one who was to
the immediate right of N?\\
1) None \hspace{2mm}2) One \hspace{2mm}3) Two \hspace{2mm}4) Three \hspace{2mm}5) More than three\\

III. What was the difference between J and M’s card ranks?\\
1) 1 \hspace{2mm}2) 2 \hspace{2mm}3) 3 \hspace{2mm}4) 4 \hspace{2mm}5) 5\\

IV. In which of the following groups was the 3rd person seated exactly in between the
1st and the 2nd persons?\\
1) K L J \hspace{2mm}2) J M N \hspace{2mm}3) M Q N \hspace{2mm}4) P L K \hspace{2mm}5) K P O\\

15. Study the following information carefully and answer the questions given below.\\
\includegraphics[width=0.60555in,height=0.32083in]{image2.png}
Eight friends G, H, I, J, K, L, A and B are sitting around a circle in one arrangement, facing
centre, and in a straight line in another arrangement where they face south but not
necessarily in the same order. The one who sits on the immediate right of H in the straight
line is sitting on the immediate left of H in the circle. K is not on any of the extreme ends. J
sits third to the left of L in the straight line. A sits on the immediate right of B in the circle,
but they are not immediate neighbours of each other in straight line. The one who sits on the
extreme left end sits to the immediate left of K in the circle. B is not the immediate right of L
in both arrangements. I and L are the immediate neighbours of H in both arrangements but I
is not at the extreme ends of the line. K sits third to the left of H in the circle, while fourth to
his right in the straight line. One of the immediate neighbours of B in the straight line sits
opposite of B in the circle.\\

16. Study the following information to answer the given questions.\\
\includegraphics[width=0.60555in,height=0.32083in]{image2.png}
Eight friends A B C D P Q R S are sitting in two circles in such a manner that each member
of inner circle sits exactly opposite to the member of the outer circle. The members sitting in
the outer circle are P Q R S and all of them are facing towards the centre while the members
of the inner circle A B C D and they are facing away from the centre. Each of them likes
different colours red, yellow, blue, green, white, black, orange and purple but not
necessarily in the same order. D like neither yellow nor white and faces R, who likes neither
black nor purple. The person who likes orange faces the person who likes red. A, who likes
green faces the immediate neighbour of the one who likes blue. R sits second to the left of S.
The person one who likes white and red are in the separate circles. The person who likes
black sits on the immediate left of S. Q, who does not like Blue does not face A. The person
who likes black and purple are immediate neighbours and one of them faces B, the one who
likes yellow colour. The person who likes orange and green sits in the same circle but they
are not immediate neighbours.\\

17. Study the following information carefully and answer the questions given below:\\
\includegraphics[width=0.60555in,height=0.32083in]{image2.png}
A, B, C, D, E, F, H and K are sitting around a circular table facing away from the centre, but
not necessarily in the same order. Each one of them is also related to B in some way. B has
siblings, E sits third to the left of C. B sits on the immediate right of C. Only one person sits
between E and necessarily in the same order. D like neither yellow nor white and faces R,
who likes neither black nor purple. The person who likes orange faces the person B’s wife
and A. D sits on the immediate left of F. Only three people sit between B’s mother and B’s
brother. B’s son sits second to the left of K. B’s brother sits third to the left of B’s father. B’s
mother sits second to the left of D.\\

18. \textbf{Directions (I-VI):} Study the following information carefully to answer the given questions\\
\includegraphics[width=0.60555in,height=0.32083in]{image2.png}
Eight friends A, B, C, D, E, F, G and H are sitting around a circular table but not necessarily
in the same order. They are working at four different shops, viz Cloth, Food, Sweets and
Stationary. Two friends work at the same shops. Some of them are facing outwards
Note: Facing the same direction means if one person faces the centre then the other person
also faces the centre and if one person faces outward then the other person also faces
outward. Facing opposite directions means if one person faces the centre then the other
person faces outward and vice-versa.\\
The one who works at the Sweet shop is an immediate neighbour of both the persons who
work at Stationery. Those who work at Food shop sit opposite to each other. The one who is
on the immediate left of F is not facing the centre. F sits second to the right of D. F and D are
immediate neighbours of H. C sits third to the left of H, who works at Sweet Shop, and both
are facing the same direction. Those who work at Cloth shop sit adjacent to each other but
face opposite directions. G sits on the immediate right of B, who works at cloth Shop. The
immediate neighbours of E do not face outward. D does not face the centre and works at
Stationary shop. B and C do not face the same direction but C is an immediate neighbour of
E, who is the fourth to the left of G. E and G face opposite directions but both work at the
same shop.\\
I. How many friends are facing outward?\\
a) Two \hspace{2mm}b) Three \hspace{2mm}c) Four \hspace{2mm}d) Five \hspace{2mm}e) None of these\\

II. F works at which of the following shops?\\
a) Sweets \hspace{2mm}b) Food \hspace{2mm}c) Stationary \hspace{2mm}d) Cloth \hspace{2mm}e) Either food or Cloth\\

III. Who sits on the immediate right of the one who works at Food shop?\\
a) B \hspace{2mm}b) C \hspace{2mm}c) F
\hspace{2mm}d) Either B or F \hspace{2mm}e) Either B or C\\

IV. Who amongst the following works at Sweet Shop who is also the immediate neighbour
of the one who works at the Stationary Shop?\\
a) C \hspace{2mm}b) D \hspace{2mm}c) H
\hspace{2mm}d) Either C or H \hspace{2mm}e) Either C or D\\

V. If D and F interchnage their places, who amongst the following is on the immediate left
of G?\\
a) F \hspace{2mm}b) D \hspace{2mm}c) H
\hspace{2mm}d) Either D or H \hspace{2mm}e) None of these\\

VI. Who amongst the following friends face inside who also sits fourth to the left of the one
who works at Cloth Shop?\\
a) B \hspace{2mm}b) C \hspace{2mm}c) E \hspace{2mm}d) F \hspace{2mm}e) None of these\\

19. \textbf{Directions:} Study the following information carefully to answer the given questions\\
\includegraphics[width=0.60555in,height=0.32083in]{image2.png}
Eight persons A, B, C, D, E, F, G and H were seated around a circular table facing the centre.\\
They enacted different roles in a play: Gabbar, Mausi, Jai, Veeru, basanti, Sambha, Thakur
and Ramlal not necessarily in the same order:\\
\begin{itemize}

\item D was seated 3rd to the right of E’s neighbour who enacted Thakur’s role.
\item The ones who acted as Thakur and Ramlal were seated at a gap of 1.
\item The one who acted as Ramlal was opposite to A whose neighbour acted as Basanti.
\item G was exactly between B and the one who acted as Jai.
\item F acted as Sambha and his neighbour acted as Veeru.
\item The one who acted as Veeru was 2nd to the right of the one who acted as Gabbar.
\item H and the one who acted as Sambha were seated at a gap of 1.
\item G’s neighbour was opposite to D’s neighbour.
\item A and C did not sit together.\\
\end{itemize}

20. Study the following information to answer the given questions.\\
\includegraphics[width=0.60555in,height=0.32083in]{image2.png}
D, E, F, G, H, I, J and K are sitting around a rectangular table. One person each is sitting at
the shorter side and three people each on the longer side. Every person is sitting exactly in
front of the person on the opposite side. Three of them are facing away from the table while
the rest are facing towards the table. People sitting at the shorter sides are facing towards
the same direction i.e., if one person sitting at one of the shorter side is facing towards the
table then person sitting at the other shorter side must be facing away from the table or vice
versa.\\
\begin{itemize}

\item I is sitting second to the right of E, who is facing the table.
\item D is facing J and also facing towards the table.
\item Neither I nor F is sitting at any of the shorter sides.
\item K is sitting third to the right of H. Both H and K are facing towards the table and are
on the longer sides.
\item J doesn't sit to the immediate left of G, who is facing away from the table.
\item F is not sitting next to K.\\
\end{itemize}

21. \textbf{Directions (I-IV):} Study the following information carefully and answer the questions
given below\\
\includegraphics[width=0.60555in,height=0.32083in]{image2.png}
Eight family members M, N, O, P, Q, R, S and T are sitting around a circular table facing the
centre but not necessarily in the same order. Each of them likes different flowers viz - Rose,
Lotus, Marigold, Hibiscus, Jasmine, Daisy, Daffodil and Magnolia but not necessarily in the
same order.\\
P is the head of family and he sits third to the left of his wife Q. T has two sisters and he sits
second to the right of his grandfather, who likes Rose. S sits on the immediate left of her
brother, who sits opposite to his father. N sits exactly in the middle of the one who likes
Lotus and the one who likes Daisy. N likes Jasmine. Q has two sons and one of the Q’s sons
like Daisy. M sits opposite to her mother-in-law, who likes Hibiscus. O’s son likes Magnolia.
R likes Daisy while his sister-in-law likes Marigold.\\

I. What is the position of N’s brother with respect to the one who likes Marigold?\\
a) Third to the right \hspace{2mm}b) Second to the left \hspace{2mm}c) Third to the left
\hspace{2mm}d) Second to the right \hspace{2mm}e) None of these\\

II. How many persons sit between S and O’s brother, if counted anti clockwise direction
from S?\\
a) Three \hspace{2mm}b) One \hspace{2mm}c) None \hspace{2mm}d) Four \hspace{2mm}e) None of these\\

III. Who likes Jasmine?\\
a) R \hspace{2mm}b) Q \hspace{2mm}c) T \hspace{2mm}d) N \hspace{2mm}e) None of these\\

IV. Who sits immediate left of the one who likes Lotus?\\
a) The one who likes Marigold \hspace{2mm}b) The one who likes Daisy
c) The one who likes Hibiscus \hspace{2mm}d) The one who likes Jasmine
e) None of these\\

22. \textbf{Directions (I-V):} Study the information given below and answer the questions based on it.\\
\includegraphics[width=0.60555in,height=0.32083in]{image2.png}
Eight colleagues Satya, Abhi, Rajesh, Rinku, Venka, Mohit, Ronu and Somu are sitting
around a circle but not necessarily in the same order. Some of them are facing towards the
centre while some others are facing outside. All of them like different brands of bags
namely: Puma, Nike, Adidas, Samsonite, Skybags, American Tourister, VIP and Wildcraft
and all of them are working in different companies such as Wipro, TCS, Tech Mahindra,
Infosys, IBM, HCL, Mphasis and Oracle but not necessarily in the same order.
Venka sits third to the left of Satya and working in Wipro. Both Satya and Venka are not
facing inside the centre. Mohit is not an immediate neighbour of Satya and Venka and
working in Oracle and likes Adidas bag. Rinku does not face the same direction of Venka
and likes Samsonite bag and is sitting opposite to Mohit. Abhi sits to the immediate right of
Rinku. Somu sits second to the right of Mohit and working in HCL. Rajesh does not face
inside and sits second to the right of Ronu who does not face outside. Rajesh likes Puma
bag. Abhi and Somu are facing the same direction as Ronu. The one, who is working in Tech
Mahindra is sitting second to the right of that person who is working in Infosys and likes
VIP bag. Rajesh is not working in IBM. The one who is working in TCS likes Wildcraft bag
and is sitting opposite to the person who is working in Tech Mahindra. Venka does not like
American Tourister bag and Abhi likes Nike bag. The one who is working in IBM is not an
immediate neighbour of the person who is working in Mphasis.\\

I. Who of the following likes Wildcraft bag?\\
1) Rajesh \hspace{2mm}2) Mohit \hspace{2mm}3) Ronu \hspace{2mm}4) Venka \hspace{2mm}5) satya\\

II. Who among the following are neighbours of Venka?\\
1) Abhi, ronu \hspace{2mm}2) Ronu, mohit \hspace{2mm}3) Satya, ronu
4) Abhi, rinku \hspace{2mm}5) Mohit, Rajesh\\

III. What is the position of Abhi with respect to Rinku?\\
1) Immediate right \hspace{2mm}2) None of these \hspace{2mm}3) Third to right
\hspace{2mm}4) Second to the left \hspace{2mm}5) Immediate left\\

IV. How many person are facing outside?\\
1) 2 \hspace{2mm}2) 3 \hspace{2mm}3) 4 \hspace{2mm}4) 5 \hspace{2mm}5) More than 5\\

V. The person sitting to the immediate right of Mohit working in which company\\
1) TCS \hspace{2mm}2) Mahindra \hspace{2mm}3) Wipro \hspace{2mm}4) Mphasis \hspace{2mm}5) Infosys\\

23. \textbf{Directions (I-VI):} Study the following information carefully and answer the questions given
below:\\
\includegraphics[width=0.60555in,height=0.32083in]{image2.png}
There are eight family members I, J, K, L, M, N, O and P sitting around a circular table
facing the centre but not necessarily in the same order. There are equal number of males and
females in the family. I and L are males. J is fourth to the right of his father I, who is son of
K, who has two children. P is not the wife of I. No male is a neighbour of J. Grandfather is
an immediate neighbour of his son. Grandson is on immediate left of his wife N, and sits on
the immediate right of his grandfather. The wife of I sits fourth to the right of her father-in-
law. There is only one person between the wife of I and O, who is wife of J.\\

I. How is P related to I?\\
a) Sister \hspace{2mm}b) Daughter \hspace{2mm}c) Wife \hspace{2mm}d) Mother \hspace{2mm}e) None of these\\

II. Who among the following sits third to the left of K?\\
a) Son of K \hspace{2mm}b) Daughter of K
\hspace{2mm}c) Daughter –in – law of K \hspace{2mm}d) Grandson of K \hspace{2mm}e) None of these\\

III. Who among the following is mother of L and J?\\
a) P \hspace{2mm}b) O \hspace{2mm}c) M
\hspace{2mm}d) Can’t be determined \hspace{2mm}e) None of these\\

IV. How many persons are there between I and his wife? (if counted from the left of I)\\
a) One \hspace{2mm}b) Two \hspace{2mm}c) Three \hspace{2mm}d) Four \hspace{2mm}e) None of these\\

V. How is N related to O?\\
a) Sister \hspace{2mm}b) Daughter \hspace{2mm}c) Mother
\hspace{2mm}d) Sister-in-law \hspace{2mm}e) Can’t be determined\\

VI. Which of the following statements is true?\\
a) J sits on the immediate left of his wife \hspace{2mm}b) K sits opposite his daughter
\hspace{2mm}c) Both the neighbours of P are females \hspace{2mm}d) All are true
\hspace{2mm}e) None of these\\

24. \textbf{Directions (I-IV):} Study the following information carefully and answer the questions given
below.\\
\includegraphics[width=0.60555in,height=0.32083in]{image2.png}
P, Q, R, S, T, U, V and W are sitting around a circular table but not necessarily in the same
order. Four of them are facing the centre, while the remaining four are not facing the centre.
They all have obtained different marks in an examination. The marks of all the four persons
who are facing the centre are multiples of 3, while the marks of the other four who are
facing away from the centre are multiples of 2.\\
T sits third to the left of P, who is not facing the centre. Only two persons bit between T and
the one whose marks is 52. W sits second to the right of the one whose marks is 52. Only
three persons sit between W and Q, whose marks is 74. One of W’s immediate neighbor‘s
marks is 63. R sits second to the right of the one whose marks is 63. Only one person sits
between R and U. The marks of the one who sits second to the left of U is one mark more
than Q. The marks of the one who is an immediate neighbor of the one whose marks is 75 is
equal to difference between the marks of Q and P. S is not an immediate neighbor of T. The
marks of one of them is 21 but this is not the immediate neighbor of Q. The marks of W is a
perfect square and his marks lies between the marks of Q and P. The marks of S is 5 more
than of W.\\
I. What is the mark of S?\\
a) 64 \hspace{2mm}b) 69 \hspace{2mm}c) 36 \hspace{2mm}d) 22 \hspace{2mm}e) None of these\\

II. How many persons are there between P and S, when counted from P in anti – clockwise
direction?\\
a) Four \hspace{2mm}b) Two \hspace{2mm}c) Three \hspace{2mm}d) None \hspace{2mm}e) More than Five\\

III. Who among the following sits opposite the one whose mark is 52?\\
a) U \hspace{2mm}b) V \hspace{2mm}c) T \hspace{2mm}d) S \hspace{2mm}e) None of these\\

IV. What is the difference between the marks of U and that of R?\\
a) 69 \hspace{2mm}b) 22 \hspace{2mm}c) 42 \hspace{2mm}d) 21 \hspace{2mm}e) None of these\\

25. \textbf{Directions (I-V):} Study the following information carefully and answer the questions given
below:\\
\includegraphics[width=0.60555in,height=0.32083in]{image2.png}
Eight chemicals A, B, C, D, E, F, G and H, contained in eight different bottles, are placed
around a circular table in such a manner that the tap fixed to each bottle is directed outward
from the centre of the table. Each chemical is of a different colour, viz Blue, Yellow, Orange,
White, Green, Violet, Brown and Black, but not necessarily in the same order.\\
\begin{itemize}

\item Chemical B is placed third to the right of Chemical D.
\item The Yellow chemical is placed on the immediate left of Chemical B.
\item The colour of Chemical B is not White.
\item Chemical F is placed fourth to the left of Chemical A.
\item Neither Chemical F nor Chemical A is an immediate neighbour of Chemical B.
\item The colour of Chemical C is Violet and is placed third to the left of the Yellow chemical.
\item The Green chemical is placed second to the right of the Violet chemical.
\item The Brown chemical is placed second to the right of Chemical B.
\item H is the Orange chemical and is placed exactly between Chemical C and Chemical A.
\item The Blue chemical is placed second to the left of the Orange chemical.
\item Chemical G is placed third to the right of Chemical C.
\end{itemize}

I. Which of the following chemicals is of White colour?\\
a) D \hspace{2mm}b) E \hspace{2mm}c) G \hspace{2mm}d) F \hspace{2mm}e) None of these\\

II. What is H’s position with respect to F?\\
a) Third to the left \hspace{2mm}b) Second to the left \hspace{2mm}c) Third to the right
\hspace{2mm}d) Second to the right \hspace{2mm}e) None of these\\

III. How many chemical bottles are placed between Chemical A and G (counted from G
clockwise)?\\
a) One \hspace{2mm}b) Two \hspace{2mm}c) Three \hspace{2mm}d) Four \hspace{2mm}e) None of these\\

IV. Four of the following five are alike in a certain way and hence form a group. Which of
the following does not belong to that group?\\
a) Violet – Brown \hspace{2mm}b) Black – Green \hspace{2mm}c) Orange – White \hspace{2mm}d) Yellow – Blue \hspace{2mm}e) Green – Yellow\\

V. What is the colour of Chemical B?\\
a) Blue \hspace{2mm}b) Black \hspace{2mm}c) Yellow \hspace{2mm}d) Can’t say \hspace{2mm}e) None of these\\

26. Study the following information to answer the given questions:\\
\includegraphics[width=0.60555in,height=0.32083in]{image2.png}
Ten friends A,B,C,D,E,F,G,H,I and J are sitting around a rectangular table in such a way that
four of them sit at the corners, two each on the longer sides, and one each on the smaller
sides but not necessarily in the same order.some face the centre and some are not facing the
centre.\\

\textbf{NOTE:} Not more than two friends sitting together face the same direction.
Each friend likes a different game, viz. Football, Volleyball, Cricket, Badminton, Lawn
tennis, Table tennis, Carrom, Ludo, Chess and Rugby.\\

E sits on the immediate left of D and is not an immediate neighbour of C.A and E face the
same direction.The one who likes football sits on the immediate left of the one who likes
table tennis.D and G sit diagonally and face opposite directions.Five of them face the same
direction.H doesn't like table tennis or chess.the one who likes ludo sits on the immediate
right of I,who likes chess.The one who likes Table tennis sits second to the left of C.Only two
among the four sitting on the centres are not facing the centre.H and D are sitting on the
immediate left and third to the left of J respectively.A sits on one of the smaller sides and
third to the right of F.D likes carrom and sits third to the left of the one who likes lawn tennis.J sits on the immediate right of the one who likes rugby. E likes Volleyball and sits
second to the left of the one who likes badminton. I is not not an immediate neighbour of
A,B or F, but sits on the immediate right of C,who is not facing the centre. one of the four
friends sitting on the corner is I.\\

27. \textbf{Directions:} Study the following information carefully and answer the questions given
below:\\
\includegraphics[width=0.60555in,height=0.32083in]{image2.png}
Eight persons A, B, C, D, E, F, G and H are sitting around a circular table but not necessarily
in the same order. Four of them are facing the centre while four of them are facing outside
from the centre.\\

\textbf{Note :} Facing the same direction means if one person faces the centre then the other person
also faces the centre and if one person faces the outside from centre then the other person
also faces outside from centre. Facing the opposite direction means if one person faces the
centre then the other person faces outside from centre and vice versa.\\

C sits on the immediate right of B. Both the immediate neighbours of B face opposite
directions. B sits third to the left of H. G sits second to the right of A. B sits third to the right
of E. H sits second to the left of E. Both the immediate neighbours of E face outside from the
centre. E faces the centre. F is an immediate neighbour of D. C is an immediate neighbour of
G. B and D are not immediate neighbours. D faces outside from the centre and both the
immediate neighbours of D face the centre.\\

I. What is the position of E with respect to G?\\
a) Third to the right \hspace{2mm}b) Immediate left \hspace{2mm}c) Second to the right \hspace{2mm}d) Third to the left \hspace{2mm}e) None of these\\

II. Which of the following statements is/are true?\\
a) E is an immediate neighbour of F. \hspace{2mm}b) Only two persons sit between B and F.
\hspace{2mm}c) H sits second to the left of B. \hspace{2mm}d) A faces outside from the centre.
\hspace{2mm}e) None of these\\

28. \textbf{Directions:} Study the following information carefully to answer the given questions.\\
\includegraphics[width=0.60555in,height=0.32083in]{image2.png}
Eight people viz, P, Q, R, S, T, U, V, W are sitting around a circular table in such a way that no two
successive people are sitting together in an alphabetical order (for ex-P can’t sit with Q, Q
can’t sit with R etc.) .They all work in NASA and all of them are doing research in current
year 2017 on different bodies of Solar System viz, Asteroids, Comets, Dwarf Planets, Jupiter,
Meteors, Pluto, Uranus, and Venus. All of them have a target year viz, 2019, 2021, 2022,
2023, 2024, 2025, 2027, and 2030 to complete the research work on concerned planet. Note:
Some of them are facing north while some are facing south. Not more than two people
sitting together are facing same direction. P is sitting third to the left of the one, who is
doing research on Comets. W has a target year which is three years after U and more than
ten years from the current year. The one who is doing research on Meteors has a target year
which is two year after the current year. R is sitting opposite to the one who is doing
research on Dwarf Planets and both of them are facing opposite direction to each other.
Only one person sits between P and R. Q is doing research on Comets. The difference
between the target year of the one who is doing research on Pluto and dwarf planet in same
as the difference between Q and the one who is doing research work on Meteors. W is an
immediate neighbour of R and sits fourth to the right of the one who is doing research on
Pluto. Q is not facing towards facing the centre. U sits second to the left of the one who is
doing research on Pluto. No person has a target year before T. S has a target year two years
prior to U. S sits third of the right of the one who has a target year 2024. The one who is
doing research on asteroids sits third to the left of the one who is doing research on Uranus.
P is not doing research on Uranus and its target year is an even number. The one, who is
doing research on Venus is facing same direction to that of V, whose target year is after P. T
is facing opposite direction of the both of the one who is doing research on Jupiter and the
one who is doing on asteroids.\\

29. \textbf{Directions:} Eight friends-Romil, Ramesh, Rakesh, Rohit, Rahul Abhijeet, Abhishek and Anil
\includegraphics[width=0.60555in,height=0.32083in]{image2.png}
are sitting around a circular table, but not necessarily in the same order. Four of them are
facing inside and the other four are facing outside. All eight friends belong to eight different
cities - Bhopal, Patna, Kolkata, Delhi, Gwalior, Bengaluru, Chennai and Rajkot - but not
necessarily in the same order. Abhijeet faces the centre and sits third to the right of Rakesh.
Rohit belongs to Kolkata and faces the person who belongs to Bengaluru. Abhishek sits
third to the right of Ramesh, who stays in Bhopal. The person who belongs to Delhi is facing
the same direction as the person who belongs to Gwalior. Rahul is sitting between the
person who belongs to Kolkata and the one from Rajkot respectively. Romil belongs to
Gwalior and Rakesh belongs to Patna. The person who belongs to Chennai is facing
outward and is immediate neighbour of the person who belongs to Rajkot. Anil is the
immediate neighbour of the persons who belong to Gwalior and Chennai. Rahul is on the
immediate left of Rohit.\\

30. \textbf{Directions (I-V):} Study the following information carefully and answer the questions below:\\
\includegraphics[width=0.60555in,height=0.32083in]{image2.png}
There are eight teachers sitting around a rectangular table. All of them are facing towards
the centre except three persons. They are namely. K, W, E, P, V, N, R and T but not
necessarily in the same order. They teach different subjects viz, Physics, polity, Hindi,
Economics, English, Maths, Biology and Spanish, but not necessarily in the same order. Four
of them sit at four corners while the rest sit in the middle of the four sides. R and K are
neighbours of each other while neither they nor their neighbours teach English or Hindi. T
and K are sitting facing in opposite direction of each other (i.e., if one is facing west other is
facing east and vice-versa). E sits between T and the one who teaches Economics. The one
who teaches Maths sit on the immediate left of both P and the one who teaches biology. The
one who teaches Physics faces E. N sits on the immediate left of the one who teaches
physics. P is not an immediate neighbour of either K or T. Neither R nor V teaches Maths.
The one who teaches Polity sits on the immediate left of T. The one who teaches Spanish is
not the immediate neighbour of those who teach either Hindi or English. The one who
teaches Hindi sits third to the left of W. The one who teaches Physics does not sit at middle
of the sides.\\

I. Who among the following teaches ‘Biology’?\\
a) R \hspace{2mm}b) K \hspace{2mm}c) T \hspace{2mm}d) W \hspace{2mm}e) V\\

II. How many person(s) sit(s) between V and W?\\
a) One \hspace{2mm}b) Two \hspace{2mm}c) Three \hspace{2mm}d) None \hspace{2mm}e) More than three\\

III. Which of the following group of persons are facing away from the centre?\\
a) K, T, N \hspace{2mm}b) V, K, W \hspace{2mm}c) W, R, T \hspace{2mm}d) K, T, W \hspace{2mm}e) N, K, W\\

IV. N teaches which of the following subjects?\\
a) Physics \hspace{2mm}b) Biology \hspace{2mm}c) Polity \hspace{2mm}d) Hindi \hspace{2mm}e) Spanish\\

V. Who among the following sits third to the left of the one who teaches Polity?\\
a) P \hspace{2mm}b) E \hspace{2mm}c) K \hspace{2mm}d) R \hspace{2mm}e) N\\

31. Study the following information carefully and answer the questions given below:\\
\includegraphics[width=0.60555in,height=0.32083in]{image2.png}
Eight students Arpit, Bhavesh, Chinmay, Dinesh, Eknath, Faruk, Girish and Hiten in a
college competition participated in a game in which they were sitting around a circular ring.
The seats of the ring are not directed towards the centre. All the eight students are in four
groups named- Crazy gang, Master Minds, The Mafia and Rockstar, i.e. two students in
each group, but not necessarily in the same order. These students are from different states-
Maharashtra, Bihar, Assam, MP, Telangana, Rajasthan, Haryana and Karnataka. No two
students of the same group are sitting adjacent to each other except those of group The
Mafia. Students from group Rockstar are sitting opposite each other. Dinesh is neither from
Karnataka nor from Assam state. The student from Haryana is sitting on the immediate
right of the student from Karnataka. Chinmay, who is from MP, is in group Crazy gang. He
is sitting on the immediate right of Faruk, who is in group The Mafia. Faruk is not from
Assam and he has also participated in other game. Bhavesh is from Bihar is neither in group
Rockstar nor in group Crazy gang or Master Minds. Bhavesh is sitting opposite Eknath.
Only Rajasthan state participant Arpit is sitting between Haryana participant Eknath and
the Telangana participant. Both the students of group Master Minds are sitting adjacent to
students of group Rockstar.\\

32. \textbf{Directions(I-IV):} Study the following information carefully and answer the questions given
below:\\
\includegraphics[width=0.60555in,height=0.32083in]{image2.png}
Eight women Priti, Bhawana, Amrita, Juhi, Madhuri, Sanjana, Ragini and Madhu are sitting
round a rectangular table. Four of them sit at middle of the side of the table and face the
centre while four of them sit at the corner of the table and they face outside from the centre.
The age of all the four women sit at the corner of table is multiple of 3 and the age of all four
women who sit at the middle of the table is multiple of 4. Madhuri sits third to the right of
Priti. Priti faces the centre. There are two ladies sit between Madhuri and the one whose age
is 52 years. There are three ladies sit between Madhu and Bhawana whose age is 76 years.
One of the immediate neighbour of Madhu is 69 years old. Amrita sits second to the left of
the one whose age is 69 years old. One lady sits between Amrita and Sanjana. The age of one
who sits second to the right of Sanjana is one year less than Bhawana. The age of one of the
immediate neighbour of the one whose age is 75 years is 4 years less than to the difference
between the age of Bhawana and Priti. Juhi is not an immediate neighbour of Madhuri. The
age of one of them is 21 years but she is not an immediate neighbour of Bhawana. The age of
Madhu is perfect square and her age lies between Priti and Bhawana. The age of Juhi is 1
year less than the age of Madhu.\\

I. What is the position of Juhi with respect to Ragini?\\
a) Third to the right \hspace{2mm}b) Third to the left \hspace{2mm}c) Second to the left \hspace{2mm}d) Immediate right \hspace{2mm}e) None of these\\

II. Four of the following five are alike in a certain way and so form a group. Which one does
not belong to that group?\\
a) Ragini – 63 \hspace{2mm}b) Madhu – 76 \hspace{2mm}c) Priti – 20 \hspace{2mm}d) Amrita – 75 \hspace{2mm}e) Juhi – 69\\

III. Who is the youngest?\\
a) Sanjana \hspace{2mm}b) Juhi \hspace{2mm}c) Amrita \hspace{2mm}d) Madhuri \hspace{2mm}e) None of these\\

IV. Who sits third to the left of Sanjana?\\
a) Juhi \hspace{2mm}b) Amrita \hspace{2mm}c) Ragini \hspace{2mm}d) Priti \hspace{2mm}e) Bhawana\\

33. \textbf{Directions (I-V):} Study the information given below and answer the questions based on it.\\
\includegraphics[width=0.60555in,height=0.32083in]{image2.png}
Ten friends A, B, C, D, E, F, G, H, I and J are sitting at a rectangular table and likes a
different chocolate among Fruit \& Nut Chocolate, Bournville, ChocOn, Coconut, Snickers,
Galaxy, Dairy milk, Ferrero Rocher, Nutellia, Kinder and Twix but not necessarily in the
same order. All of them are sitting at a rectangular table in such a way that four of them sit
at the corners two each on the longer sides and one each on the smaller sides, but not
necessarily in the same order. Five of them face the same direction. Not more than two
friends sitting together face the same direction. E sits on the immediate left of D and is not
an immeidate neighbour of C. A and E face the same direction. The one who likes Fruit\&Nut Chocolate sits immediate left of the one who likes Dairy milk. D and G sit diagonally
and face the oppposite. H does not like Dairy milk and Kinder. The one who likes Nutella
sits immedaite right of I, who likes Kinder. The one who likes Dairy milk sits second to the
left of C. Only two among four sitting on the corners face outward. H and D are sitting on
the immediate left and third to the left of J respectivley. A sits on one of the smaller sides
and third to the right of F. D likes Fferrero Rocher and sits third to the left of the one who
likes Galaxy. J sits immediate right of the one who likes Twix. E likes Bournville and sits
second to the left of the one who likes Snickers. I is not an immediate neighbours of A, B or
F but sits on the immediate right of C, who si not facing the centre. One of the four friends
sitting on the corner is I.\\

I. Who among the following is second to the right of A?\\
a) D \hspace{2mm}b) J \hspace{2mm}c) H \hspace{2mm}d) F \hspace{2mm}e) I\\

II. Four of the folloing five are alike in some way, find out the odd one?\\
a) BA \hspace{2mm}b) AH \hspace{2mm}c) DF \hspace{2mm}d) HD \hspace{2mm}e) DI\\

III. Who among the following sits fourth to the left of the one who likes Snicker chocolate?\\
a) G \hspace{2mm}b) A \hspace{2mm}c) one who likes Kinder \hspace{2mm}d) One who likes Nutella
\hspace{2mm}e) Both 1 and 3\\

IV. What is the position of J with respect to the one who likes Dairy milk chocolate?\\
a) Second to the right \hspace{2mm}b) Second to the left\hspace{2mm} c) Fifth to the left
\hspace{2mm}d) Immediate neighbour \hspace{2mm}e) Other than the given options\\

V. Which of the following statements is/are true\\
a) The one who likes Nutella sits second to the right of the one who likes Galaxy
\hspace{2mm}b) B is an immediate neighbour of A and J
\hspace{2mm}c) Both 1 and 2
\hspace{2mm}d) One of the neighbour of I likes Bournville
\hspace{2mm}e) All are true\\

34. Ten persons A, B, C, D, E, F, G, H, I and J are sitting around a rectangular table such that
two person \includegraphics[width=0.60555in,height=0.32083in]{image2.png}are sitting on the shorter side and three persons are sitting on longer side each.
No two persons sit adjacent and opposite to each other according to the English alphabet
(i.e. A does not sit adjacent to B, B does not sit adjacent to A and C and so on). A faces C. J
sits 3rd right to D. Not more than five persons sit between D and F, when counted from the
right of D. A faces C. I sits on the shorter side of the table and is immediate left to A. E sits
3rd left to B.\\

\textbf{Answers}\\

1. I. Second to the right \hspace{2mm}II. E \hspace{2mm}III. Immediate right \hspace{2mm}IV. Three \hspace{2mm}V. C \hspace{2mm}VI. B is the mother of H \hspace{2mm}VII. F’s grandmother\\
2. I. Two \hspace{2mm}II. P \hspace{2mm}III. The captains of Australia and England are immediate neighbours. \hspace{2mm}IV. T \hspace{2mm}V. Second to the right\\
3. I. Four \hspace{2mm}II. F sits in middle of one of the sides \hspace{2mm}III. T \hspace{2mm}IV. T \hspace{2mm}V. Third to the left\\
4. I. Operation \hspace{2mm}II. The person works for HR Department. \hspace{2mm}III. The person works for Marketing Department\\
5. I. Himanshu \hspace{2mm}II. Amit, Usha \hspace{2mm}III. Santosh \hspace{2mm}IV. Between Himanshu and Usha \hspace{2mm}V. Second to the right of Santosh\\
6. –\\
7. I. Two \hspace{2mm}II. T \hspace{2mm}III. The captains of Australia and England are immediate neighbours \hspace{2mm}IV. T \hspace{2mm}V. Second to the right\\
8. -\\
9. I. Second to the left \hspace{2mm}II. B \hspace{2mm}III. F is H's wife\hspace{2mm} IV. C and H\\
10. –\\
11. –\\
12. I. DE \hspace{2mm}II. BH\\
13. I. who sits second to the left of J \hspace{2mm}II. 1 and 4 both \hspace{2mm}III. 1, 3 and 4 \hspace{2mm}IV. None of these \hspace{2mm}V. Red, Yellow, Black, Blue, Green\\
14. I. Third to the right \hspace{2mm}II. None \hspace{2mm}III. 1 \hspace{2mm}IV. P L K\\
15. –\\
16. –\\
17. –\\
18. -\\
19. –\\
20. –\\
21. I. Third to the right \hspace{2mm}II. None of these \hspace{2mm}III. N \hspace{2mm}IV. The one who likes Jasmine\\
22. I. satya \hspace{2mm}II. Abhi, ronu \hspace{2mm}III. Immediate right \hspace{2mm}IV. 3 \hspace{2mm}V. Mphasis\\
23. I. Sister \hspace{2mm}II. Daughter of K \hspace{2mm}III. M \hspace{2mm}IV. Two \hspace{2mm}IV. Sister-in-law \hspace{2mm}V. Both the neighbours of P are females\\
24. I. 69 \hspace{2mm}II. None \hspace{2mm}III. V \hspace{2mm}IV. 42\\
25. I. D \hspace{2mm}II. Third to the left \hspace{2mm}III. Two \hspace{2mm}IV. Green-Yellow \hspace{2mm}V. Black\\
26. –\\
27. (1) D; \hspace{2mm}(2) D.\\
28. –\\
29. –\\
%\includegraphics[width=0.60555in,height=0.32083in]{image2.png}
%\hspace{2mm}

\end{document}