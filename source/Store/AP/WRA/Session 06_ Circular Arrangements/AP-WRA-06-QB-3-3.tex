
% Options for packages loaded elsewhere
\PassOptionsToPackage{unicode}{hyperref}
\PassOptionsToPackage{hyphens}{url}
%
\documentclass[
]{article}
\usepackage{amsmath,amssymb}
\usepackage{lmodern}
\usepackage{iftex}
\ifPDFTeX
\usepackage[T1]{fontenc}
\usepackage[utf8]{inputenc}
\usepackage{textcomp} % provide euro and other symbols
\else % if luatex or xetex
\usepackage{unicode-math}
\defaultfontfeatures{Scale=MatchLowercase}
\defaultfontfeatures[\rmfamily]{Ligatures=TeX,Scale=1}
\fi
% Use upquote if available, for straight quotes in verbatim environments
\IfFileExists{upquote.sty}{\usepackage{upquote}}{}
\IfFileExists{microtype.sty}{% use microtype if available
	\usepackage[]{microtype}
	\UseMicrotypeSet[protrusion]{basicmath} % disable protrusion for tt fonts
}{}
\makeatletter
\@ifundefined{KOMAClassName}{% if non-KOMA class
	\IfFileExists{parskip.sty}{%
		\usepackage{parskip}
	}{% else
		\setlength{\parindent}{0pt}
		\setlength{\parskip}{6pt plus 2pt minus 1pt}}
}{% if KOMA class
	\KOMAoptions{parskip=half}}
\makeatother
\usepackage{xcolor}
\IfFileExists{xurl.sty}{\usepackage{xurl}}{} % add URL line breaks if available
\IfFileExists{bookmark.sty}{\usepackage{bookmark}}{\usepackage{hyperref}}
\hypersetup{
	hidelinks,
	pdfcreator={LaTeX via pandoc}}
\urlstyle{same} % disable monospaced font for URLs
\usepackage{longtable,booktabs,array}
\usepackage{calc} % for calculating minipage widths
% Correct order of tables after \paragraph or \subparagraph
\usepackage{etoolbox}
\makeatletter
\patchcmd\longtable{\par}{\if@noskipsec\mbox{}\fi\par}{}{}
\makeatother
% Allow footnotes in longtable head/foot
\IfFileExists{footnotehyper.sty}{\usepackage{footnotehyper}}{\usepackage{footnote}}
\makesavenoteenv{longtable}
\usepackage{graphicx}
\makeatletter
\def\maxwidth{\ifdim\Gin@nat@width>\linewidth\linewidth\else\Gin@nat@width\fi}
\def\maxheight{\ifdim\Gin@nat@height>\textheight\textheight\else\Gin@nat@height\fi}
\makeatother
% Scale images if necessary, so that they will not overflow the page
% margins by default, and it is still possible to overwrite the defaults
% using explicit options in \includegraphics[width, height, ...]{}
\setkeys{Gin}{width=\maxwidth,height=\maxheight,keepaspectratio}
% Set default figure placement to htbp
\makeatletter
\def\fps@figure{htbp}
\makeatother
\setlength{\emergencystretch}{3em} % prevent overfull lines
\providecommand{\tightlist}{%
	\setlength{\itemsep}{0pt}\setlength{\parskip}{0pt}}
\setcounter{secnumdepth}{-\maxdimen} % remove section numbering
\ifLuaTeX
\usepackage{selnolig}  % disable illegal ligatures
\fi

\author{}
\date{}
\usepackage{multirow}
\usepackage[inline]{enumitem}
\usepackage[margin=1.0in]{geometry}
\usepackage[english]{babel}
\usepackage[utf8]{inputenc}
\usepackage{fancyhdr}

\pagestyle{fancy}
\fancyhf{}
\rhead{\includegraphics[width=5.21667in, height=0.38819in]{image1.png}}
\lhead{ Reasoning: Circular Arrangements }
\lfoot{www.talentsprint.com }
\rfoot{\thepage}
\begin{document}
	
 

\begin{center}
	{\Large \textbf{Circular Arrangements \\}}
\end{center}

{\large \textbf{ Additional Examples \\}}


1. Study the following information carefully and answer the given questions.\\
\includegraphics[width=0.60555in,height=0.32083in]{image2.png}Representatives from eight different Banks viz. Pankaj, Roshan, Dhiraj, Anvi, Anurag,
Sanjay, Swapnil and Rujuta are sitting around a circular table facing the centre but not
necessarily in the same order. Each one of them is from a different Bank viz. A, B, C, D, E, F,
G and H. Sanjay sits second to right of the representative from D. Representative from G is
an immediate neighbor of the representative from D. Two people sit between the
representative of G and Roshan. Dhiraj and Anurag are immediate neighbors of each other.
Neither Dhiraj nor Anurag is an immediate neighbor of either Roshan or the representative
from D. Representative from C sits second to right of Anvi. Anvi is neither the
representative of D nor G. Swapnil and the representative from A are immediate neighbors
of each other. Roshan is not the representative of A. Only one person sits between Dhiraj
and the representative from B. Rujuta sits third to left of the representative from H.
Representative from F sits second to left of the representative from E.\\

2. Directions (I-V): Study the below information and answer the questions below\\
\includegraphics[width=0.60555in,height=0.32083in]{image2.png}
Six friends E, F, G, H, J, and K are sitting around a circular table facing the centre not
necessarily in the same order. They play different sports Cricket, Hockey, Badminton,
Chess, Football and Tennis. Each has different professions Lawyer, Banker, Engineer,
Professor, Doctor and Businessman. The persons who are Doctor, Engineer and Professor
play neither Tennis nor Cricket. The persons who play Chess and Hockey are neither Doctor
nor Businessman. E neither plays Tennis nor sits on the immediate left of the person who is
Banker. The only person who is between J and K is Lawyer. The person who is on the left
side of the person who likes Tennis is not Businessman. H is Banker and likes to play
Hockey. He is facing the person who is Lawyer. One who is a Doctor is sitting opposite the person who plays Football, while the person who plays Hockey is on the left of the person
who is Professor. One who is Businessman is on the immediate right of the person who likes
Tennis but on the left of the person who is Engineer. G is not Engineer and K is not Doctor.\\

I. Who is sitting between the one who is businessman and the one who is playing
Hockey?\\
1) E \hspace{2mm}2) G \hspace{2mm}3) F \hspace{2mm}4) K \hspace{2mm}5) J\\

II. Who is playing Badminton?\\
1) Businessman \hspace{2mm}2) Doctor \hspace{2mm}3) Engineer
\hspace{2mm}4) Professor \hspace{2mm}5) Banker\\

III. What is the profession of the person who is sitting on the left of the person who is
opposite to the person-playing cricket?\\
1) Engineer \hspace{2mm}2) Doctor \hspace{2mm}3) Professor \hspace{2mm}4) Banker \hspace{2mm}5) Lawyer\\

IV. Who is sitting opposite to the Engineer?\\
1) H \hspace{2mm}2) G \hspace{2mm}3) F \hspace{2mm}4) K \hspace{2mm}5) J\\

V. The one who is playing tennis is on the third right of \_\_\_\_\_\_?\\
1) The person who is playing football\\
2) The person who is playing Hockey\\
3) The person who is playing Badminton\\
4) The person who is playing cricket\\
5) None of the above\\

3. Study the below information and answer the questions below.\\
\includegraphics[width=0.60555in,height=0.32083in]{image2.png}Eight friends-Aman, Bharat, Rohit, Dinesh, Ishant, Farhan, Aditya and Hirender—are
sitting around a circular table, facing the centre but not necessarily in the same order. Each
of them has a different designation, viz Banker, Teacher, Company MD, Politician,
Corporate Leader, General Manager, Specialist Officer, and CEO. Aman sits third to the
right of the CEO. Only two people sit between the CEO and Hirender. The Specialist Officer
and the Company MD are immediate neighbours. Neither Aman nor Hirender is either a
Specialist Officer or a Company MD. The Specialist Officer is not an immediate neighbour of
the CEO. The Banker sits second to the left of Ishant. Ishant is not an immediate neighbour
of Hirender. The Banker is an immediate neighbour of both the Politician and the Corporate
Leader. The Corporate Leader sits third to the right of Bharat. Bharat is not a Specialist
Officer. Rohit sits on the immediate right of the Teacher. Aman is not a Teacher. Farhan is
not an immediate neighbour of Aman. Aditya is not an immediate neighbour of the Banker.

4. Eight persons A, B, C, D, E, F, G and H are sitting around a circular table with equal
distance between \includegraphics[width=0.60555in,height=0.32083in]{image2.png}each other but not necessarily in the same order. Some of them are facing
centre while some facing outside (i.e. away from the centre). Note- Facing the same direction
means if one faces the centre then the other also the centre and vice versa. Facing the opposite directions means if one faces the centre then the other faces the outside and vice-
versa. Immediate neighbours face the same direction means if one neighbour is facing the centre then other person also facing centre and vice-versa. E is third to the right of G. B sits
third to the left of E. E and B is facing same direction. Immediate neighbours of B faces the
centre. C is second to the right of B. H sits on the immediate left of C. D faces the same
direction as B. A sits second to the left of D. Immediate neighbours of E are facing opposite
directions. H and A are facing opposite directions.\\

5. Eight persons J, K, L, M, N, O, P and Q are sitting around a circular table. Each of them has a
laptop of \includegraphics[width=0.60555in,height=0.32083in]{image2.png}a different company, viz Dell, Lenovo, HP, Sony, Acer, Apple, Asus and Toshiba
but not necessarily in the same order. Four of them are not facing the centre. The one who
has Acer is on the immediate left of M, who does not have Asus. M is third to the right of Q.
P is fourth to the left of O. Neither P nor O is an immediate neighbor of M. L has HP and sits
exactly between J and O. L is facing the centre and is to the right of both J and O. N is third
to the left J, who has Toshiba. J sits third to the left of the one who has Acer. The one who
has Lenovo sits second to the left of the one who has Toshiba. The one who has Dell sits
second to the right of M. The one who has Apple sits second to the right of the person who
has HP. K does not like Acer. J faces the centre.\\

6. A, B, C, D, E, F, G, H and K are sitting around a circle facing the centre. B is fourth to the left
of G who \includegraphics[width=0.60555in,height=0.32083in]{image2.png}is second to the right of C. F is fourth to the right of C and second to the left of K.\\
A is fourth to the right of K. D is not an immediate neighbour of either K or B, H is third to
the right of E.\\

7. Eight people - L, M, N, O, P, Q, R and S are sitting around a circular table facing the centre.\\
\includegraphics[width=0.60555in,height=0.32083in]{image2.png}Each of them likes different colours viz., Red, Orange, Blue, Pink, Black, Purple, Brown and
Green, but not necessarily in same order. S is sitting second to the left of N. There are two
persons between S and the person who likes Orange colour. M sits second to the left of
person who likes Orange colour. L is the immediate neighbour of S. R is the third to right of
P. O likes purple colour. The person who likes pink colour is second to the right of P. The
person who likes Brown colour is the third to the left of the person who likes Blue colour.\\
Neither S nor P likes Brown colour. N likes neither Green nor Blue colour. L likes Red
colour.\\

8. Six persons are sitting around a circular table. Ajay is facing a person who is sitting to the
left of Arvind \includegraphics[width=0.60555in,height=0.32083in]{image2.png}and right of Sanjay. Suman is to the right of Aravind. Hemant is facing Sanjay
if Hemant and Manoj, Arvind and Sanjay mutually exchange their positions. Who is now
sitting to the right of Manoj?\\
1) Arvind \hspace{2mm}2) Ajay \hspace{2mm}3) Hemant \hspace{2mm}4) Sanjay\\

9. Study the following information to answer the given questions.\\
\includegraphics[width=0.60555in,height=0.32083in]{image2.png}Eight students J, K, L, M, N, X, Y and Z are sitting around a square table in such a way four
of them sit at four corners while four sit in the middle of each of the four sides. The one who
sits at the 4 corners faces the outside the center and others facing the center. J sits fourth to
the right of X. Neither M nor Z is an immediate neighbor of J. M sits third to the right of Z
who sits in front of K. L sits corner side of the table who is not the immediate neighbor of M.
Neither J nor X is neighbor of Y.

10. M, N, O, P, Q, R, S and T are sitting around a circle.Some of the persons are facing the center
and some \includegraphics[width=0.60555in,height=0.32083in]{image2.png}are facing outside.\\
O sits 2nd to the right of R. R is facing towards the center. There are 2 persons in between O
and N. S sits 2nd to the right of O. T sits on the immediate right of N. S and N are facing
opposite direction ( if S is facing center then N is facing outside and vice versa). There are 3
persons in between P and Q. Both P and M are not immediate neighbours of R. Q sits 2nd to
the right of M. Both T and Q faces opposite direction of O.Both the immediate persons of S
are facing same direction.\\

11. Study the following information and answer the questions given below: Eight friends A, B,
C, D, E, F, G \includegraphics[width=0.60555in,height=0.32083in]{image2.png}and H are sitting around a circular table not necessarily in the same order.
Three of them are facing outward while five are facing towards the centre. There are equal
number of males and females in the group. C is facing the centre. E is sitting third to the
right of C. F is sitting third to the left of E. Three persons are sitting between F and B. The immediate neighbours of B are females. G is sitting third to the right of F. D is sitting third
to the right of A. A is not an immediate neighbour of E. Neighbours of E are males .The
immediate neighbours of Dare are females and face outside. The one sitting third to the left
of B is a male. No female is an immediate neighbour of G.\\

12. Study the following information and answer the questions given below:\\
\includegraphics[width=0.60555in,height=0.32083in]{image2.png}Eight friends P, Q, R, S, T, U, V and W are sitting around a circular table. Some are facing
the centre while some are not facing the centre (i.e. in a direction opposite the centre). S sits
third to the right of Q. T sits second to the left of Q. The immediate neighbour of Q face the
same direction (if one person is facing the centre then the other also faces the centre and vice
versa). R sits second to the left of T. T faces the centre. U sits third to the right of R. V sits
second to the left of W. W is not an immediate neighbour of Q. V faces the same direction as
S. The immediate neighbour of T face opposite directions (that if one person is facing the
centre then the other facing outward and vice versa). The immediate neighbour of U face
opposite direction.

13. Study the following information and answer the questions given below:\\
\includegraphics[width=0.60555in,height=0.32083in]{image2.png}N, O, P, Q, R, S, T and U are sitting around a circular area at equal distance between each
other , but not necessarily in the same order. some of the people are facing the centre while
some face outside. R sits second to the right of T. T faces the centre. O sits third to the left of
R. R and O face opposite direction. Immediate neighbours of O face the centre. P sits second
to the right of O. U sits to the immediate left of P. N sits second to the left of Q. Q faces the
same as O. Q is not an immediate neighbour of T. Immediate neighbours of R face opposite
direction (if one neighbour faces centre then the other face outside)
Who sits second to the left of U?
1) T 2) O 3) S 4) Q 5) none of these

14. Study the following information carefully and answer the questions given below:\\
\includegraphics[width=0.60555in,height=0.32083in]{image2.png}Eight persons - A, B, C, D, E, F, G and H are sitting around a circular table facing the centre.\\
Each one of them has a different profession viz., Doctor, Lawyer, Painter, Librarian,
Architect, Engineer, Teacher and Accountant, but not necessarily in the same order.
A sits third to the right of F. Only one person sits between A and C. Accountant is sitting
third to the right of C. Accountant is sitting to the immediate left of Engineer. B is sitting to
the immediate left of H. Three persons sit between B and Architect. D is an immediate
neighbour of G. D is neither an Engineer nor an Architect. Only one person sits between
Librarian and Architect. Painter is to the immediate left of Teacher. D is not a Doctor. G is
neither a Librarian nor a Lawyer. Lawyer is an immediate neighbour of Architect. Painter is
sitting opposite to the Doctor.\\

15. \textbf{Direction:} Eight members of a club are sitting around a circular table, viz Z, Y, X, W, V, U, T
and S, \includegraphics[width=0.60555in,height=0.32083in]{image2.png}but not necessarily in the same order. Four of them are not facing the centre. W is
facing the centre and is third to the right of Z, who is third to the right of U, who is facing
the centre. X is fourth to the left of T. S is third to the right of V, who is not a neighbour of T.
V sits second to the right of X. Y is on the immediate right of T.\\

I. What is S’s position with respect to Y?\\
a) Third to the left \hspace{2mm}b) Second to the right \hspace{2mm}c) Second to the left \hspace{2mm}d) Either 2) or 3) \hspace{2mm}e) None of these\\

II. Who is second to the left of V?\\
a) W \hspace{2mm}b) X \hspace{2mm}c) Y \hspace{2mm}d) T \hspace{2mm}e) None of these\\

III. Four of the following five are alike in a certain way and so form a group. Which is the
one that does not belong to that group?\\
a) W \hspace{2mm}b) Y \hspace{2mm}c) V \hspace{2mm}d) Z \hspace{2mm}e) U\\

IV. Which of the following pairs are neighbours of T?\\
a) W, Y \hspace{2mm}b) U, S \hspace{2mm}c) Z, W \hspace{2mm}d) None of these \hspace{2mm}e) Can’t be determined\\

V. Which of the following statements is/are true?\\
a) Z is not facing the centre \hspace{2mm}b) Y is second to the right of W \hspace{2mm}c) T is facing the centre
\hspace{2mm}d) Both 1) and 2) are true \hspace{2mm}e) None of these\\

16. Eight persons P, Q, R, S, T, U, V, W are sitting around a circular table. Some of them are
facing \includegraphics[width=0.60555in,height=0.32083in]{image2.png}towards the centre while some are facing outside the centre of the table. Not more
than two persons sitting together are facing same direction. U sits third to the left of P. T sits
third to the right of S, who is an immediate neighbour of P. Both S and P are facing in same
direction to each other. As many as person sit between S and W, as between W and V. R sits
second to the left of U. T does not sit opposite to P. Both R and Q face same direction to each
other. T and W faces opposite direction to each other. W face outside the center.\\
%\includegraphics[width=0.60555in,height=0.32083in]{image2.png}
%\hspace{2mm}

\end{document}