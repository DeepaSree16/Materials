
% Options for packages loaded elsewhere
\PassOptionsToPackage{unicode}{hyperref}
\PassOptionsToPackage{hyphens}{url}
%
\documentclass[
]{article}
\usepackage{amsmath,amssymb}
\usepackage{lmodern}
\usepackage{iftex}
\ifPDFTeX
\usepackage[T1]{fontenc}
\usepackage[utf8]{inputenc}
\usepackage{textcomp} % provide euro and other symbols
\else % if luatex or xetex
\usepackage{unicode-math}
\defaultfontfeatures{Scale=MatchLowercase}
\defaultfontfeatures[\rmfamily]{Ligatures=TeX,Scale=1}
\fi
% Use upquote if available, for straight quotes in verbatim environments
\IfFileExists{upquote.sty}{\usepackage{upquote}}{}
\IfFileExists{microtype.sty}{% use microtype if available
	\usepackage[]{microtype}
	\UseMicrotypeSet[protrusion]{basicmath} % disable protrusion for tt fonts
}{}
\makeatletter
\@ifundefined{KOMAClassName}{% if non-KOMA class
	\IfFileExists{parskip.sty}{%
		\usepackage{parskip}
	}{% else
		\setlength{\parindent}{0pt}
		\setlength{\parskip}{6pt plus 2pt minus 1pt}}
}{% if KOMA class
	\KOMAoptions{parskip=half}}
\makeatother
\usepackage{xcolor}
\IfFileExists{xurl.sty}{\usepackage{xurl}}{} % add URL line breaks if available
\IfFileExists{bookmark.sty}{\usepackage{bookmark}}{\usepackage{hyperref}}
\hypersetup{
	hidelinks,
	pdfcreator={LaTeX via pandoc}}
\urlstyle{same} % disable monospaced font for URLs
\usepackage{longtable,booktabs,array}
\usepackage{calc} % for calculating minipage widths
% Correct order of tables after \paragraph or \subparagraph
\usepackage{etoolbox}
\makeatletter
\patchcmd\longtable{\par}{\if@noskipsec\mbox{}\fi\par}{}{}
\makeatother
% Allow footnotes in longtable head/foot
\IfFileExists{footnotehyper.sty}{\usepackage{footnotehyper}}{\usepackage{footnote}}
\makesavenoteenv{longtable}
\usepackage{graphicx}
\makeatletter
\def\maxwidth{\ifdim\Gin@nat@width>\linewidth\linewidth\else\Gin@nat@width\fi}
\def\maxheight{\ifdim\Gin@nat@height>\textheight\textheight\else\Gin@nat@height\fi}
\makeatother
% Scale images if necessary, so that they will not overflow the page
% margins by default, and it is still possible to overwrite the defaults
% using explicit options in \includegraphics[width, height, ...]{}
\setkeys{Gin}{width=\maxwidth,height=\maxheight,keepaspectratio}
% Set default figure placement to htbp
\makeatletter
\def\fps@figure{htbp}
\makeatother
\setlength{\emergencystretch}{3em} % prevent overfull lines
\providecommand{\tightlist}{%
	\setlength{\itemsep}{0pt}\setlength{\parskip}{0pt}}
\setcounter{secnumdepth}{-\maxdimen} % remove section numbering
\ifLuaTeX
\usepackage{selnolig}  % disable illegal ligatures
\fi

\author{}
\date{}
\usepackage{multirow}
\usepackage[inline]{enumitem}
\usepackage[margin=1.0in]{geometry}
\usepackage[english]{babel}
\usepackage[utf8]{inputenc}
\usepackage{fancyhdr}

\pagestyle{fancy}
\fancyhf{}
\rhead{\includegraphics[width=5.21667in, height=0.38819in]{image1.png}}
\lhead{ Reasoning: Date/Day/Month/Time-based }
\lfoot{www.talentsprint.com }
\rfoot{\thepage}
\begin{document}
	
 

\begin{center}
	{\Large \textbf{Date/Day/Month/Time-based \\}}
\end{center}

1. Four friends A, B, C, D were going to giving the different competitive exams viz: CAT, IBPS
PO, CDS \includegraphics[width=0.60555in,height=0.32083in]{image2.png}and CSAT (not necessarily in the same order). on the same day but in different
cities viz: Dehradun, Delhi, Dispur, Dhanbad but not necessarily in the same order, at
different time which were 9am, 12pm, 10am and 11am (not necessarily in the same order).\\
\begin{itemize}
    \item D's exam was just after the exam of CAT which was just after the exam which was given in Dehradun.
\item CSAT exam was given at least two hours later than B's exam.
\item CDS exam was given just after the exam which was given in Dhanbad, who had an
exam just after A.
\item The definite information which was given about three people among four one was C
and the one who has given the exam in Delhi and the one who has given the exam at
11am.
\end{itemize}


2. Four people were being interviewed for the same job, on the same day but in different
rooms (R1, R2, R3 \includegraphics[width=0.60555in,height=0.32083in]{image2.png}and R4), at different time and by different interviewers.\\
Determine which candidate was interviewed by whom, at which time and in which room\\

I. Teena’s appointment was just after Mr. Sharma’s, which was just after that of the person
in room R2.\\

II. Mr. Narurkar’s appointment was at least two hours later than Bimal’s.\\

III. Mr. Joshi’s appointment was just after the person who had an interview in room R4,
who had an appointment just after Chirag.\\

IV. Three of the four people were: (1) Deepak, (2) the one with interview in room R1, and
(3) the one who was interviewed at 1 p.m.\\

V. Interview times were – 11 a.m., 12 noon, 1 p.m., and 2 p.m.\\

VI. Sharma, Narurkar, Joshi and Zaidi were interviewers and Teena, Bimal, Chirag and Deepak were the interviewees.\\

3. Eight persons M, N, O, P, Q, R, S and T are attending the office meeting on four different
months like \includegraphics[width=0.60555in,height=0.32083in]{image2.png}January, May, June and August. The office meeting was conducted on 18th and
23rd of the month. All the given information is not necessarily in the same order. Two
persons attend the office meeting between T and Q. N and R are attending the office
meeting before P. As many persons attend the office meeting between M and N is same as
between R and S. Q attends the office meeting before T. Two persons attend the office
meeting between S and the one who attends the office meeting immediately after T. As
many persons attend the office meeting between R and Q is same as between N and P. R
and M do not attend the office meeting on same month. Q doesn’t attend the office meeting
on even numbered date. At least two persons attend the office meeting after M. R and T do
not attend the office meeting on same month. O doesn’t attend the office meeting on 23rd of
any month.\\

4. \textbf{Direction:} Read the following information carefully and answer the questions given below.\\
\includegraphics[width=0.60555in,height=0.32083in]{image2.png}
Seven people A, B, C, D, E, F and G are going to tour for seven different places in seven
different days in the week. The week days started from Monday to Sunday. The places are
namely Baku, Rabat,Nassau, Astana, Thimphu, Paris and Asmara. All the information is not
necessary to be in the same order.
A is going to Baku. Only two people are going between A and the person who is going to
Nassau. C is going to tour on Tuesday.Only one people is going to tour between B and A,
who is going to tour neither on Monday nor on Saturday. The person who is going to Rabat
is going to tour one of the day before the person who is going to Nassau.Only one person is
going to tour between the person who is going to Rabat and E, who is not going to Nassau.B
is going to either Paris or Asmara.E is not going to tour any day before the person who is
going to Rabat.Only one person is going to tour between G and D. G is neither going to
Nassau nor going to Rabat. F is not going to tour any day after G. Only two people are
going to tour between G and the person who is going to Thimphu, who is not B. Neither G
nor E is from Paris. Only two people are going to tour between the people who are going to
Paris and the person who is going to Astana.The person who is going to Asmara is going to
tour one of the day after the person who is going to Astana. The person who is going to
Asmara is not going to tour on Saturday.\\

5. \textbf{Directions (I-V):} Study the following information carefully and answer the questions given
below:\\
\includegraphics[width=0.60555in,height=0.32083in]{image2.png}
Aman, Bhavesh, Chandan, Diwakar, Vansh, Firoz and Gaurav are working in different
shops, viz Stationery, Grocery, Toys, Shoes, Clothes, Big Bazaar and Watch. They go to work
on a different days of a week from Monday to Sunday. Chandan works neither in Clothes
shop nor in the Watch shop but he goes on Wednesday. Vansh works in the Shoes shop but
not on Monday. The one who goes on Friday works in the Toys shop. The one who works in
Big Bazaar goes on Tuesday. Diwakar goes on Saturday, but does not work in the Clothes
shop. Gaurav works in the Grocery shop. Firoz goes to work on Thursday. Bhavesh does not
go to work in Big Bazaar.\\

I. The person who works in the Shoes shop goes on which of the following days?\\
a) Wednesday \hspace{2mm}b) Sunday \hspace{2mm}c) Tuesday \hspace{2mm}d) Monday \hspace{2mm}e) None of these\\

II. Firoz works in which of the following shops?\\
a) Clothes \hspace{2mm}b) Stationery \hspace{2mm}c) Grocery
\hspace{2mm}d) None of these \hspace{2mm}e) Can’t be determined\\

III. The person who goes on Wednesday works in which of the following shops?\\
a) Grocery \hspace{2mm}b) Watch \hspace{2mm}c) Shoes \hspace{2mm}d) Stationery \hspace{2mm}e) None of these\\

IV. Who among the following works in the Watch shop?\\
a) Firoz \hspace{2mm}b) Aman \hspace{2mm}c) Diwakar \hspace{2mm}d) Chandan \hspace{2mm}e) None of these\\

V. Who among the following works on Monday?\\
a) Gaurav \hspace{2mm}b) Diwakar \hspace{2mm}c) Vansh \hspace{2mm}d) Aman \hspace{2mm}e) None of these\\

6. \textbf{Directions (I-IV):} Study the following information carefully and answer the questions given
below:\\
\includegraphics[width=0.60555in,height=0.32083in]{image2.png}
There are seven persons A, B, C, D, E, F and G. Each of them in different professions viz—
Doctor, Engineer, Manager, Clerk, Professor, Judge and Businessman but not necessarily in
the same order. They are going to visit Taj Mahal on seven different days of a week. The
week starts from Monday and ends on Sunday. The Professor is going to visit on Thursday.
Only one person is going to visit between the Professor and E. The Doctor is going to visit
immediately after E. Only three persons are going to visit between C and the Doctor. Only
two people are going to visit between C and the Judge. G is going to visit before the Judge
but after Thursday. More than two persons are going to visit between G and the Manager. B
is going to visit immediately before F. D is going to visit immediately before the Engineer.
The Businessman is not going to visit on Tuesday.\\

I. Which of the following statement is/are true about D?\\
a) D is going to visit immediately after the Businessman\\
b) D is the Engineer\\
c) The Doctor is going to visit on Sunday\\
d) Both 1) and 3)\\
e) None of these\\

II. As per the given arrangement G is related to the Businessman and A is related to the
Engineer in a certain way. In the same way E is related to\\
a) Manager \hspace{2mm}b) Professor \hspace{2mm}c) Clerk
\hspace{2mm}d) Can’t be determined \hspace{2mm}e) None of these\\

III. Four of the following five are alike in a certain way and so form a group. Which one does
not belong to that group?\\
a) B – Tuesday \hspace{2mm}b) D – Monday \hspace{2mm}c) G – Sunday
\hspace{2mm}d) E – Friday \hspace{2mm}e) E – Saturday\\

IV. Who amongst the following is the Doctor?\\
a) E \hspace{2mm}b) D \hspace{2mm}c) B \hspace{2mm}d) G \hspace{2mm}e) None of these\\

7. \textbf{Directions (I-V):} Study the following information carefully and answer the questions given\\
\includegraphics[width=0.60555in,height=0.32083in]{image2.png}
below. Seven persons P, Q, R, S, T, U and V were born in seven different months (of the
same year), viz, January, February, March, April, June September and October but not
necessarily in the same order. Each of them likes a different subject, viz Physics, Chemistry,
Biology, Maths History, Civics, and Geography but not necessarily in the same order R was
born neither in the month which has 28 days nor in the month which has 30 days. Only
three persons were born between the months in which the one who likes Maths and R were
born respectively. T was born immediately before the month in which the person who likes
Maths was born. T was not born in the month which has less than 30 days. P was born
immediately before the month in which the one who likes Biology was born. P was not born
in the month which has less than 31 days. Only two persons were born between the months
in which the one who likes Biology and V were born respectively. The one who likes Physics
was born in the month which has 31 days but not in January. V likes neither Physics nor
Civics. Q was born in a month immediately before the month in which the one who likes
Civics was born. The one who likes Chemistry was born after the month in which the one
who likes Civics was born. U was born in February. The one who likes History was born
before U. Note: Assume the months to be in consecutive order.\\

I. Which of the following represents the persons who were born in March and September
respectively?\\
1) P, S \hspace{2mm}2) R, T \hspace{2mm}3) R, Q \hspace{2mm}4) T, Q \hspace{2mm}5) None of these\\

II. How many persons were born between the months in which U and T were born?\\
1) Two \hspace{2mm}2) More than three \hspace{2mm}3) Three \hspace{2mm}4) One \hspace{2mm}5) None\\

III. Which of the following statements is/are not true?\\
1) P was born immediately after the month in which the one who likes Geography was
born
\hspace{2mm}2) T was born in march \hspace{2mm}3) S likes Maths
\hspace{2mm}4) Both 2 and 3 \hspace{2mm}5) None of these\\

IV. Four of the following five are alike in a certain way and so form a group. Which is the one
that does not belong to that group?\\
1) V, T \hspace{2mm}2) R, T \hspace{2mm}3) P, S \hspace{2mm}4) P, U \hspace{2mm}5) Q, U\\

V. Which of the following statements is true?\\
1) Q was born in the month which has 31 days\\
2) S was born immediately before the month in which Q was born\\
3) The one who likes Biology was born immediately after the month in which P was born\\
4) V likes Biology\\
5) None of these\\

8. Study the following information carefully and answer the questions given below:\\
\includegraphics[width=0.60555in,height=0.32083in]{image2.png}
There are eight persons A, B, C, D, E, F, G and H, who are planning to take a leave in the
month of March, June, August and November . In each month, they will take a leave on 6th
or 11th of the given month. Only one person will take a leave on these given dates. E will
take a leave in the month which has less than 31 days. Three persons will take a leave
between E and C. Two persons will take a leave between C and G. Three persons will take a
leave between G and D. Two persons will take a leave between D and A. Three persons will
take a leave between A and F. Two persons will take a leave between F and B. H will not
take leave in the November month.\\

9. \textbf{Directions (I-IV):} Study the following information carefully and answer the questions given below:\\
\includegraphics[width=0.60555in,height=0.32083in]{image2.png}Seven persons A, B, C, D, E, F and G are going to visit seven different countries viz -
Malaysia, France, England, Australia, Canada, China, and Japan but not necessarily in the
same order. Each of them is going to visit on seven different days of a week. The week starts
from Monday and ends on Sunday. The one who is going to visit Canada, is not going to
visit on Monday. F is going to visit immediately before C. Only three persons are going to
visit between A and E. G is not going to visit France. A is going to visit immediately before
the one who is going to visit France. G is going to visit before the one who is going to visit
Japan. Only two persons are going to visit between G and the one who is going to visit
Australia. G is going to visit immediately before the one who is going to visit Malaysia.
Only four persons are going to visit between B and the one who is going to visit Malaysia.
Only two persons are going to visit between B and the one who is going to visit China. The
one who is going to visit China, is going to visit on Thursday.\\

I. Which of the following countries does G visit?\\
a) Japan \hspace{2mm}b) China \hspace{2mm}c) Canada \hspace{2mm}d) England \hspace{2mm}e) None of these\\

II. Which of the following statements is/are true?\\
a) G is going to visit on Thursday \hspace{2mm}b) D is going to visit China
\hspace{2mm}c) B is going to visit after C \hspace{2mm}d) All are true
\hspace{2mm}e) None of these\\

III. Four of the following five are alike in a certain way and so form a group. Which one
does not belong to that group?\\
a) B – Australia \hspace{2mm}b) A – China \hspace{2mm}c) E – Japan
\hspace{2mm}d) G – Malaysia \hspace{2mm}e) C – Canada\\

IV. How many persons is going to visit between F and the one who is going to visit Malaysia?\\
a) Two \hspace{2mm}b) One \hspace{2mm}c) None \hspace{2mm}d) Three \hspace{2mm}e) Three\\

10. Study the following information carefully and answer the questions given below:\\
\includegraphics[width=0.60555in,height=0.32083in]{image2.png}
Seven persons P, O, N, M, L, K and J were born in seven different months namely February,
March, April, June, September, October and November (of the same year) but not
necessarily in the same order. Each of them has different weight viz. 39, 49, 42, 33, 55, 17 and
23, but not necessarily in the same order. The one whose weight is 42 kg was not born in the
month which has less than thirty days. Both J and N were born in one of the months after
the month in which the person whose weight is 39 kg was born. The one who is the third
heaviest of them all was not born in the month which has 31 days. Only three persons were
born between the one whose weight is 42 kg and the one whose weight is 23 kg. Only two
persons were born between J and the one whose weight is 33 kg. N is not the lightest. K was
not born in the month which has thirty days. J was born immediately before N. M was born
immediately after the one whose weight is 33 kg. P was born immediately before the one
who is the second heaviest. Only one person was born between K and the person whose
weight is 39 kg. Only two persons were born between M and L.\\

11. Solve the below Arrangement There are ten persons namely J, K, L, M, N, O, P, Q, R and S.\\
\includegraphics[width=0.60555in,height=0.32083in]{image2.png}
They play jalikattu on five different days starting from Monday to Friday of the same week.
Each person plays jalikattu at two different time slots, i.e. 11:59 AM or 10:30 P.M. The
number of persons who plays jalikattu between P and M is same as the number of persons
who play jalikattu between L and Q. Only two people play jalikattu between O and S.
Neither N nor P plays jalikattu on Friday. R has plays jalikattu on Tuesday at 11:59 A.M. Q
does not plays jalikattu at 10:30 P.M. M does not plays jalikattu on any one of the days after
N. R does not plays jalikattu on any of the days before P. O does not plays jalikattu on any
of the days after Q. K plays jalikattu immediately before R. M plays jalikattu immediately
after the day of the one who plays jalikattu on Monday. O does not plays jalikattu at 10:30
P.M. Only three people play jalikattu between P and N.\\

12. \textbf{Directions (I-IV):} Study the information and answer the below questions\\
\includegraphics[width=0.60555in,height=0.32083in]{image2.png}
Seven people,
namely K, L, M, N, O, P \& Q have to attend a concert. But not necessarily in the same order
on seven different months namely January, February, April, May, June, September, and
November. Each of them also likes a different movie namely X-men, Transformers, Frozen,
Minions, Shrek, Tangled and Rio, but not necessarily in the same order. M will attend a
concert in a 31-day month. Only two people will attend a concert between M and the one
who likes Frozen. The one who likes Frozen will attend a concert on one of the months after
M. Only two people will attend a concert between the one who likes Frozen and the one
who likes Transformers. The one who like Transformers will attend the concert in a month
which has 31 days. K will attend a concert immediately after M. Only three person will
attend a concert between K and the one who likes tangled. Only one person will attend a
concert between the one who likes Tangled and L. Only two people will attend a concert
between L and the one who like Rio. The one who likes X-man will attend a concert
immediately before the one who likes Shrek. Only one person will attend a concert between
the one likes Shrek and P. Only three people will attend a concert between Q and O. Q will
not attend a concert in a month which has 30 days.\\

I. Who amongst the following likes X –men?\\
1) M \hspace{2mm}2) K \hspace{2mm}3) O \hspace{2mm}4) L \hspace{2mm}5) Q\\

II. How many people will attend a concert after M?\\
1) More than three \hspace{2mm}2) One \hspace{2mm}3) None \hspace{2mm}4) Two \hspace{2mm}5) Three\\

III. Who amongst the following likes Minions?\\
1) M \hspace{2mm}2) L \hspace{2mm}3) P \hspace{2mm}4) K \hspace{2mm}5) other those given options\\

IV. Which of the month in which K will attend a concert?\\
1) Sept \hspace{2mm}2) June \hspace{2mm}3) Nov \hspace{2mm}4) April \hspace{2mm}5) Can’t be determined\\

13. \textbf{Directions (I-V):} Study the information given below and answer the questions based on it.\\
\includegraphics[width=0.60555in,height=0.32083in]{image2.png}
There are 12 professors – A, B, C, D, E, F, G, H, I, J, K and L. They have conferences in
different months –January, March, April, June, July, and August. The conferences are on
dates 12th or 17th of a given month. Each conference is attended by some number of people.
If the month contains odd number of days then number of people attending the conference
is odd. Like, the conference in March (31 days) was attended by odd number of people and
conference in April (30 days) was attended by even number of people. A had conference on
12th April. I’s conference was after A. I’s conference was attended by 42 people and he had
conference on 17th of a month. There was one conference between the conference of I and
the conference of J. The number of people attending J’s conference was 25 less than the
number of people who attended I’s conference. There were 3 conferences between F’s
conference and the conference attended by 56 people. F’s conference is not in January. F’s
conference had before the conference attended by 56 people. There were 3 conferences
between E’s conference and the conference attended by 56 people. The number of people
who attended K’s conference was greater than the number of people who attended B’s
conference but less than the number of people who attended A’s conference. The number of
people who attended K’s conference was a multiple of 7. C and D had conference in same
month but not June. There was one conference between the conference of D and the
conference of B. B’s conference had odd number of people. The number of people of B’s
conference had more than J’s conference but less than 20. B’s conference is not in August.
The total number of people who attended the conferences of C and D was 58. Conference of
G and the conference attended by 28 people were in same month. K’s conference was before
L. The number of people who attended the conference of E was lesser than the number of
people who attended the conference of J by 4. The one who had conference in March had
less number of people attending it than E but more than 10. The total number of people who
attended the conferences of K and L was 24. The number of people attended D’s conference
is 9 times of the number of people attended L’s conference\\

I. What is the total number of people who attended the conferences of E and J?\\
A) 23 \hspace{2mm}B) 30 \hspace{2mm}C) 34 \hspace{2mm}D) 27 \hspace{2mm}E) 37\\

II. If G’s conference has more people than I’s conference but less than 50, then what can be
possible number of people who attended G’s conference?\\
A) 43 \hspace{2mm}B) 45 \hspace{2mm}C) 47 \hspace{2mm}D) 46 \hspace{2mm}E) 49\\

III. Which of the following is true with respect to given arrangement?\\
A) There are 2 conferences between I and L\\
B) K and J have conferences in the same month.\\
C) B’s conference attended by 21 people\\
D) J’s conference attended by more than 25 people\\
E) None is true\\

IV. Which of the following pair had conferences in January?\\
A) F and B \hspace{2mm}B) K and J \hspace{2mm}C) C and D \hspace{2mm}D) A and B \hspace{2mm}E) H and I\\

V. H has conference in which of the following month?\\
A) April \hspace{2mm}B) June \hspace{2mm}C) January \hspace{2mm}D) July \hspace{2mm}E) August\\

14. \textbf{Directions (I-V):} Study the following information carefully to answer the given questions\\
\includegraphics[width=0.60555in,height=0.32083in]{image2.png}
The festival of an institute was organized from 14th to 21st December. 14th December was
Wednesday. During that period six functions were organized, viz Swimming, Singing, Quiz,
Dancing, Sports and Drama. Only one function is organized on each day according to
further information\\
\begin{itemize}
    \item Drama was not organized on the 21st December.
\item Quiz was organized on the previous day of Singing.
\item Sports was organized neither on Wednesday nor on Saturday.
\item No function was organized on Thursday and Sunday. Dancing was organized on
Monday.
\item There was a gap of two day’s between Singing and Sports
\end{itemize}

I. The festival started with which of the following functions?\\
a) Quiz \hspace{2mm}b) Singing \hspace{2mm}c) Drama \hspace{2mm}d) Sports \hspace{2mm}e) None of these\\

II. Which of the following pairs of functions was organized on Wednesday?\\
a) Drama, Sports \hspace{2mm}b) Drama, Quiz \hspace{2mm}c) Can’t be determined
\hspace{2mm}d) Drama, Swimming \hspace{2mm}e) None of these\\

III. A gap of how many days was there between Quiz and Swimming?\\
a) Five \hspace{2mm}b) Four \hspace{2mm}c) Three \hspace{2mm}d) Two \hspace{2mm}e) None of these\\

IV. On which day Dancing was organized?\\
a) Monday \hspace{2mm}b) Thursday \hspace{2mm}c) Saturday
\hspace{2mm}d) Wednesday \hspace{2mm}e) None of these\\

V. Which of the following functions exactly precedes Dancing?\\
a) Sports \hspace{2mm}b) Drama \hspace{2mm}c) Quiz \hspace{2mm}d) Swimming \hspace{2mm}e) None of these\\

15. \textbf{Directions:} Eight members of a family were A, B, C, D, E, F, G and H. They attended their
office \includegraphics[width=0.60555in,height=0.32083in]{image2.png}conferences in the months of January, March, April and May on either 10th or 21st of
the month. Only one person attended the conference on each day and each of them attended
the conference only once. The male members attended the conference on the odd valued
date and the female members attended the conferences on even value date. Each person was
related to A in some manner wife, mother, father, son, daughter, brother and sister not
necessarily in the same order.\\
\begin{itemize}
    \item A’s sister attended the conference just after B
\item B and D attended conferences at a gap of 3 persons
\item A attended the conference just before his wife
\item G attended the conference in May
\item E was a male member of the family
\item A’s father and mother attended the conferences at a gap of 2 persons
\item A’s daughter attended the conference just after A’s son
\item H and A’s father attended conferences on consecutive turns but not in the same month
\item F was not the first to attend the conference and was not A’s daughter
\end{itemize}

16. Eight persons R S T U V W X Y are going to inaugurate different events on different dates
11th and \includegraphics[width=0.60555in,height=0.32083in]{image2.png}28th of different months viz September, October, November and December. Only
one person going for inauguration between S and U and both of them are going on odd date
of month. T goes in a month having 30 days but before U. W goes immediately before R but
in same month. Three persons goes in between R and V. More than 3 persons goes between
X and Y. X goes before Y. More than one person goes between S and Y.\\

17. Six lectures are scheduled in a week starting from Monday and ending on Sunday of the
same week. \includegraphics[width=0.60555in,height=0.32083in]{image2.png}Computer Science is not on Tuesday or Saturday. Psychology is immediately
after Organisational Behaviour. Statistics is not on Friday and there is one day gap between
Statistics and Research Methods. One day prior to the schedule of Economics there is no
lecture (as that day is the off day and Monday is not the ’off’ day).\\

18. Ten students namely viz Jaidev, Jaya, Jeeva, Janaki, Jyothi, Jaspal, Jamuna, Jagadish,
Jayanthi and \includegraphics[width=0.60555in,height=0.32083in]{image2.png}Jasmine of ten different colleges but not necessarily in the same order have
exam on five different days starting from Monday to Friday of the same week. Each student
have exam at two different time slots, i.e 09.00 AM or 11.00 A.M The number of people who
have exam between Jamuna and Janaki is same as the number of people who have exam
between Jeeva and Jagadish. Only two people have exam between Jaspal and Jasmine.
Neither Jyothi nor Jamuna does not have exam on Friday.Jayanthi has exam on Tuesday at
09.00 A.M. Jagadish does not have exam at 11.00 AM. Janaki does not have exam on any one
of the days after Jyothi. Jayanthi does not have exam on any of the days before
Jamuna.Jaspal does not have exam on any of the days after Jagadish. Jaya has exam
immediately before Jayanthi.The one who has exam at 09.00 A.M. immediately before
Jasmine. Janaki has exam immediately after the day of one who has exam on Monday.
Jaspal does not have exam at 11.00 A.M. Only three people have exam between Jamuna and
Jyothi.\\

19. Study the following information carefully and answer the given questions.\\
\includegraphics[width=0.60555in,height=0.32083in]{image2.png}
Six teachers A, B,
C, D, E and F takes a class for different subjects in their school. Each of them take class of
different subjects: Science, English, Math, Computer, History and Hindi, not in same order
starting from Monday up to Sunday. There is a holiday in between when no class is
scheduled.\\
\begin{itemize}
    \item Holiday is not on Monday or Saturday
\item D takes class on Sunday and Science is scheduled on Wednesday
\item There is two days gap between Holiday and E’s class,
\item A takes class after E and B takes class before Holiday
\item There is two days gap between Computer and Math class
\item Hindi is scheduled on the next day of C's class
\item D does not take English
\item History is not taken by E But it is scheduled before English class.
\item Computer is scheduled before holiday and E did not take class on Monday.
\item English is not taken by A or B.
\end{itemize}

20. \textbf{Directions:} Study the following information carefully and answer the questions given
below:\\
\includegraphics[width=0.60555in,height=0.32083in]{image2.png}
There are seven bollywood actors named Aamir Khan, Akshay Kumar, Ajay Devgan,
Shahid kapoor, Hrithik Roshan, Salman Khan, and Shah Rukh Khan. They are born in a
different month of the same year from January to July, but not necessarily in the same order.
No two people are born in the same month. They also acted in different movies among
Dilwale, Sultan, Dangal, Rustom, Drishyam, Krrish and Dhoom.
Akshay Kumar has his birthday in a month which has 30 days. The ones who have their
birthday in the month immediately preceding \& succeeding that of Akshay Kumar’s acted
in Drishyam and Krrish respectively. Ajay Devgan has his birthday in the month just before
that of Akshay Kumar. Salman Khan has his birthday in a month that has more than 30
days. Dangal is not acted by Salman Khan or Shahid Kapoor. Hrithik Roshan acted in
Sultan. There is only one person’s birthday between Hrithik roshan and the one who acted
in Dilwale. The persons who acted in Rustom or Dhoom do not have birthdays in successive
months. Akshay kumar not acted in Rustom or Dhoom. There are three person’s birthday
between Aamir Khan and the one who acted in Dangal. Aamir Khan is not born in
February. The one who acted in Dhoom has his birthday before the one who acted in
Rustom.\\

21. \textbf{Directions(I-V):} Study the following information carefully and answer the questions given
below:\\
\includegraphics[width=0.60555in,height=0.32083in]{image2.png}
Six plays A, B, C, D, E and F are to be staged starting from Monday and ending on Sunday
with one of the days being an off day, not necessarily in the same order, Each of the plays
has different time duration: 1⁄2 hour, 1 hour, 1 1⁄2 hours, 2 hours, 2 1⁄2 hours and 3 hours,
again not necessarily in the same order.
Sunday is not an off day and a Play of 1⁄2 hour duration is staged on that day. Play A is
staged immediately before Play E. There are two plays staged between Play F which is for 3
hours and Play C which is for 1 1⁄2 hours. The off day is after the staging of Play E and there
are two days between the off day and Play A. Play D which is for 2 hours is not staged on
Monday. The play staged immediately before the off day is of 3 hours. Play A is for less than
2 1⁄2 hours.\\

I. What is the time duration of Play B?\\
a) 2 1⁄2 hours \hspace{2mm}b) 2 hours \hspace{2mm}c) 1 hour
\hspace{2mm}d) 1⁄2 hour \hspace{2mm}e) None of these\\

II. Which day is the off day?\\
a) Tuesday \hspace{2mm}b) Monday \hspace{2mm}c) Friday
\hspace{2mm}d) Saturday \hspace{2mm}e) Cannot be determined\\

III. How many plays are staged before the off day?\\
a) Two \hspace{2mm}b) One \hspace{2mm}c) Five \hspace{2mm}d) Three \hspace{2mm}e) None of these\\

IV. On which day is Play D staged?\\
a) Wednesday \hspace{2mm}b) Saturday \hspace{2mm}c) Tuesday
\hspace{2mm}d) Friday \hspace{2mm}e) Cannot be determined\\

V. Which of the following combinations of Play - Day - Time Duration is correct ?\\
a) E - Wednesday - 2 hours \hspace{2mm}b) A - Tuesday – 1 hour
\hspace{2mm}c) C - Thursday - 1 1⁄2 hours \hspace{2mm}d) F - Tuesday - 3 hours
\hspace{2mm}e) None is correct\\
%\includegraphics[width=0.60555in,height=0.32083in]{image2.png}
%\hspace{2mm}

\end{document}