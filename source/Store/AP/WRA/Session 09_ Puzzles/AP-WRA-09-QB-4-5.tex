
% Options for packages loaded elsewhere
\PassOptionsToPackage{unicode}{hyperref}
\PassOptionsToPackage{hyphens}{url}
%
\documentclass[
]{article}
\usepackage{amsmath,amssymb}
\usepackage{lmodern}
\usepackage{iftex}
\ifPDFTeX
\usepackage[T1]{fontenc}
\usepackage[utf8]{inputenc}
\usepackage{textcomp} % provide euro and other symbols
\else % if luatex or xetex
\usepackage{unicode-math}
\defaultfontfeatures{Scale=MatchLowercase}
\defaultfontfeatures[\rmfamily]{Ligatures=TeX,Scale=1}
\fi
% Use upquote if available, for straight quotes in verbatim environments
\IfFileExists{upquote.sty}{\usepackage{upquote}}{}
\IfFileExists{microtype.sty}{% use microtype if available
	\usepackage[]{microtype}
	\UseMicrotypeSet[protrusion]{basicmath} % disable protrusion for tt fonts
}{}
\makeatletter
\@ifundefined{KOMAClassName}{% if non-KOMA class
	\IfFileExists{parskip.sty}{%
		\usepackage{parskip}
	}{% else
		\setlength{\parindent}{0pt}
		\setlength{\parskip}{6pt plus 2pt minus 1pt}}
}{% if KOMA class
	\KOMAoptions{parskip=half}}
\makeatother
\usepackage{xcolor}
\IfFileExists{xurl.sty}{\usepackage{xurl}}{} % add URL line breaks if available
\IfFileExists{bookmark.sty}{\usepackage{bookmark}}{\usepackage{hyperref}}
\hypersetup{
	hidelinks,
	pdfcreator={LaTeX via pandoc}}
\urlstyle{same} % disable monospaced font for URLs
\usepackage{longtable,booktabs,array}
\usepackage{calc} % for calculating minipage widths
% Correct order of tables after \paragraph or \subparagraph
\usepackage{etoolbox}
\makeatletter
\patchcmd\longtable{\par}{\if@noskipsec\mbox{}\fi\par}{}{}
\makeatother
% Allow footnotes in longtable head/foot
\IfFileExists{footnotehyper.sty}{\usepackage{footnotehyper}}{\usepackage{footnote}}
\makesavenoteenv{longtable}
\usepackage{graphicx}
\makeatletter
\def\maxwidth{\ifdim\Gin@nat@width>\linewidth\linewidth\else\Gin@nat@width\fi}
\def\maxheight{\ifdim\Gin@nat@height>\textheight\textheight\else\Gin@nat@height\fi}
\makeatother
% Scale images if necessary, so that they will not overflow the page
% margins by default, and it is still possible to overwrite the defaults
% using explicit options in \includegraphics[width, height, ...]{}
\setkeys{Gin}{width=\maxwidth,height=\maxheight,keepaspectratio}
% Set default figure placement to htbp
\makeatletter
\def\fps@figure{htbp}
\makeatother
\setlength{\emergencystretch}{3em} % prevent overfull lines
\providecommand{\tightlist}{%
	\setlength{\itemsep}{0pt}\setlength{\parskip}{0pt}}
\setcounter{secnumdepth}{-\maxdimen} % remove section numbering
\ifLuaTeX
\usepackage{selnolig}  % disable illegal ligatures
\fi

\author{}
\date{}
\usepackage{multirow}
\usepackage[inline]{enumitem}
\usepackage[margin=1.0in]{geometry}
\usepackage[english]{babel}
\usepackage[utf8]{inputenc}
\usepackage{fancyhdr}

\pagestyle{fancy}
\fancyhf{}
\rhead{\includegraphics[width=5.21667in, height=0.38819in]{image1.png}}
\lhead{ Reasoning: Ranking \& Ordering based Puzzles }
\lfoot{www.talentsprint.com }
\rfoot{\thepage}
\begin{document}
	
 

\begin{center}
	{\Large \textbf{Ranking \& Ordering based Puzzles \\}}
\end{center}

1. If ranks of five candidates Ajay, Jay, Sanjay, Akshay and Akshat are arranged in ascending
order of their \includegraphics[width=0.60555in,height=0.32083in]{image2.png}marks in Quantitative aptitude, Akshat is the fourth and Akshay is the first.
When they are arranged in the ascending order of marks in Logical Reasoning, Ajay takes the
place of Akshat and Akshat takes the place of Jay. Sanjay’s position remains the same in both
the arrangements. Jay’s marks are lowest in one test and highest in the other test. Sanjay has
more marks than Ajay in Quantitative aptitude.\\
Which of the following groups of candidates has improvement in rank in Logical Reasoning
as compared to that in Quantitative aptitude?\\
1) Akshay – Ajay – Akshat \hspace{2mm}2) Ajay – Sanjay – Akshay
\hspace{2mm}3) Jay – Sanjay – Akshat \hspace{2mm}4) Sanjay – Ajay – Jay
\hspace{2mm}5) None of these\\

2. Eight persons A, B, C, D, E, F, G and H buy different fruits Apple, Pear, Peach, Plum, Orange,
Mango, \includegraphics[width=0.60555in,height=0.32083in]{image2.png}Kiwi and Papaya not necessarily in the same order. The weights of the fruits were
1kg, 2kg, 3kg, 4kg, 5kg, 6kg, 7kg, 8kg not necessarily in the same order.\\
The sum of the weights of the fruits D and F together is equal to the weight of the Mangoes.
Plum and the fruit bought by C together weight equal to weight of the fruit bought by E.
B's weight of the fruit was twice the weight of the fruit bought by C.
Apple weighed 3 kg more than Plum.
E's weight of the fruit was thrice the weight of fruit bought by F
A's weight of the fruit weighed less than the weight of the fruit bought by C.\\
H bought Papaya and Mangoes did not weigh 3 kg.
Peach weighed twice the weight of Kiwi
Weight of Pear and Kiwi together was equal to the weight of Papaya and Orange.\\

3. \textbf{Directions:} Study the information given below and solve the puzzle\\
\includegraphics[width=0.60555in,height=0.32083in]{image2.png}
Ms Catherine Chandler buys jewellery on a monthly basis. This month she has bought 7 sets
of jewellery namely anklet, bangle, earring, necklace, nose-ring, pendant and ring all having
different weights of 10 gm, 15 gm, 20 gm, 25 gm, 30 gm, 35 gm and 40 gm not necessarily in
the same order. The jewelleries are hung over a hanger which is facing north.\\
\begin{itemize}
    \item The necklace is hanging 3rd to the right of the 30 gm jewelry.
\item Anklet is hanging next to the earring which is hanging 2nd to the right of the bangle.
\item Either the necklace or the earring is 15 gm in weight.
\item The nose-ring is hanging 3rd to the right of the 15 gm jewelry.
\item Two jewelry are hanging between the 25 gm and the 40 gm jewelry.
\item 3 jewelries are hanging between the ring and the 20 gm jewelry.
\item The 25 gm pendant is hanging at either of the extreme ends.
\item Necklace is not 35 gm weighted.
\end{itemize}

4. \textbf{Directions (I - V):} A, B, C, D, E, F, and G are seven cars which are running on tracks and
travelling \includegraphics[width=0.60555in,height=0.32083in]{image2.png}and covering different distance but not necessary in the same order. Each car has
different colours i.e. red, orange, green, black, yellow, white and blue but not in same order
Car A is in odd number ranking and is not 3rd lowest in covering distance. Car which is
yellow colour is immediate more than in covering distance than Car A. There are only two
cars between car A and the one which is of green colour in covering distance. The car which
is of orange colour is odd number in ranking but greater than car D in travelling. D is covering
third lowest distance. Third lowest car in travelling covers a distance of 1218 km. Only three
cars are between car C and the one which is orange colour. The one which is green colour is
immediate more than in covering distance than car C. E covers a distance of 1456 km. The car
which is of red colour is immediate more than in covering distance than car which is of blue
colour. Car G is odd number ranking. Only one car is in between car B and E. Car B covers
more distance than car E. Neither car C nor A is of black colour. Car E is not of green colour.\\

I. Car A is of which colour?\\
a) WHITE \hspace{2mm}b) BLUE \hspace{2mm}c) GREEN \hspace{2mm}d) ORANGE \hspace{2mm}e) RED\\

II. Which of the following combinations true with respect to the given arrangement?\\
a) WHITE - C \hspace{2mm}b) ORANGE – F \hspace{2mm}c) BLUE – G
\hspace{2mm}d) YELLOW - D \hspace{2mm}e) BLACK – B\\

III. Which car is second highest in covering distance?\\
a) B \hspace{2mm}b) D \hspace{2mm}c)F
\hspace{2mm}d) CAR WHICH IS RED COLOUR \hspace{2mm}e) CAR WHICH IS OF BLUE\\

IV. If B+E covers a distance of 3216 km, then what is the possible score of A?\\
a) MORE THAN G \hspace{2mm}b) 1520 KM \hspace{2mm}c) 1368 KM
\hspace{2mm}d) LESS THAN 1456 KM \hspace{2mm}e) NONE OF THESE\\

V. What is the colour of Car F?\\
a) WHITE \hspace{2mm}b) BLUE \hspace{2mm}c) GREEN \hspace{2mm}d) ORANGE \hspace{2mm}e) RED\\

5. \textbf{Directions (I-II):} Study the following information carefully and answer the questions given below:\\
\includegraphics[width=0.60555in,height=0.32083in]{image2.png}
Four friends B, C, D and E have different professions viz – Teacher, Lawyer, Doctor
and Engineer. They have different numbers of houses. The Teacher has more houses than
only Doctor and the Engineer. D’s houses are fewer than those of only one person. C is an
Engineer. E does not have more houses than D and does not have the least number of houses.\\

I. Who has the highest number of houses?\\
1) Lawyer \hspace{2mm}2) E \hspace{2mm}3) D \hspace{2mm}4) Doctor \hspace{2mm}5) Can’t be determined\\

II. Who among the following is the Teacher?\\
1) B \hspace{2mm}2) D \hspace{2mm}3) Either D or B \hspace{2mm}4) E \hspace{2mm}5) None of these\\

6. \textbf{Directions (I-V):} Study the information below and answer questions\\
\includegraphics[width=0.60555in,height=0.32083in]{image2.png}
A, B, C, D, E and F are six persons who have joined 6 different banks OBC Bank, Punjab
National Bank, Bank of Maharashtra, Bank of India, Bank of Baroda, and State Bank of India
by scoring different marks in the written exam of a maximum of 200 marks. (marks are an
integer value) They all are sitting around a circle facing the centre with equal distance. C is
second to the right of the person who joined Bank of Baroda whose score is 169 marks which
were the third lowest marks out of the 6 students. A is immediate to the left of the person who
is opposite to the person who joined State Bank of India, who is not near to F. B scored the
2nd highest marks and did not join Bank of India but is 2nd to the left of the person, who
scored 174 marks. D is seated opposite to one who joined Bank of India. D is not near to B. E
has not scored the lowest marks. C is immediate to the left of the one who joined State Bank
of India and C scored 170 marks. The person who joined Punjab National Bank is second to
the left of the one who scored 172 marks. The person who joined OBC did not score the highest
marks. One of the six students scored which was a prime number. The lowest scored mark is
164.\\

I. Who amongst the following is from OBC?\\
1) D \hspace{2mm}2) E \hspace{2mm}3) C \hspace{2mm}4) A \hspace{2mm}5) B\\

II. Four of the following five are alike in a certain way based on the given seating
arrangement and thus form a group, who is the one that does not belong to that group?\\
1) A \hspace{2mm}2) B \hspace{2mm}3) C \hspace{2mm}4) D \hspace{2mm}5) E\\

III. Who is seated between D and the person from Bank of Baroda?\\
1) E \hspace{2mm}2) A \hspace{2mm}3) C
\hspace{2mm}4) None \hspace{2mm}5) The person from Bank of Maharashtra\\

IV. Which of the following is true regarding the given information?\\
1) F is from Bank of Baroda and seated immediate right of the person who is opposite to
the person who joined bank of Maharashtra.\\
2) A is from Punjab National Bank and scored 174 marks and is opposite to C\\
3) E is from OBC and is to the immediate right of B\\
4) C scored 170 marks and is opposite to one who joined Bank of Maharashtra.\\
5) The person from Bank of Maharashtra is seated opposite to one who scored 164 marks.\\

V. Which of the following order of marks in descending order is true?\\
1) A > B > C > D > E > F \hspace{2mm}2) A > B > C > F > E > D
\hspace{2mm}3) A > B > E > C > F > D \hspace{2mm}4) A > B > C > E > F > D
\hspace{2mm}5) A > B > C > D > F > E\\

7. Seven persons P, Q, R, S,T, U and V were born in the year 1990, 1996, 1999, 2004, 2009, 2012
and \includegraphics[width=0.60555in,height=0.32083in]{image2.png}2016 not necessarily in the same order. Everyone shared the same birthdate i.e., 1st
January. All calculations are done with respect to their present ages as on 1st January 2017.\\
\begin{itemize}
\item Difference between T’s and U’s age is same as the difference between P’s and S’ age
\item T was born before the year 2000\\
\item Q was elder to T\\
\item R was 5 years elder to P\\
\item U is elder to R but younger to Q\\
\item At least one person was born between T and R.\\
\end{itemize}


8. \textbf{Directions (I-III):} Read the following comprehension carefully to answer the given question.\\
\includegraphics[width=0.60555in,height=0.32083in]{image2.png}
Eight friends A, B, C, D, E, F, G and H took part in a 500 meters race and completed the whole
race in 1, 3, 2.8, 1.7, 2.5, 3.3, 1.5 and 2 minutes but not necessary in the same order. Each of
them works in a sector from banking, IT, retail, infrastructure, agriculture, mechanical,
research and sales but not necessary in the same order. The person took less time ranked one
and same follows for others too.
Four people ranked between B and C, who is not from mechanical sector or banking. A is
from sales and took 2 minutes. Only F ran faster than the person who works in the research
sector. The number of people ranked between A and the one from mechanical is same as the
H and B. Person from infrastructure ranked an odd numbered position. D ran slower than H.
No person ranked between the banker and H. C took more time than A but is not the slowest
among all. Three people ranked between the Banker and the one from agriculture. G is from
IT and ran faster than the banker.\\

I. How many people ran slower than E?\\
1) one \hspace{2mm}2) two \hspace{2mm}3) three
\hspace{2mm}4) more than three \hspace{2mm}5) E is the slowest\\

II. How many people ran faster than the one who works in agriculture?\\
1) one \hspace{2mm}2) two
\hspace{2mm}3) more than three \hspace{2mm}4) three \hspace{2mm}5) none\\

III. What is number of people ranked between A and C?\\
1) same as the number between F and G\\
2) same as the number between D and E\\
3) same as the number between E and G\\
4) same as the number between B and A\\
5) none of these\\

9. \textbf{Directions (I-V):} Study the information given below and answer the questions based on it.\\
\includegraphics[width=0.60555in,height=0.32083in]{image2.png}
Eight boxes numbered 1 to 8 placed from top to bottom, where box number 1 is at the top and
box number 8 is at the bottom. They are from different colors, i.e. Red, Blue, Pink, Brown,
Orange, Black, White and Green, but not necessarily in the same order. Theirs weights are i.e.
8kg, 9kg, 11kg, 14kg, 15kg, 16kg, 17kg and 22 kg but not necessarily in the same order. Box
number 6 is Pink color and its weight is multiple of 3. Two boxes are between Pink box and
Blue box. Brown box weight is twice of Red box. Brown box is just above Orange box. Black
box is heavier than Brown box but lighter than Orange box. Red box placed at the top. Box 4
is Green box and its weight is 11kg. The sum of the box 5 and box 6 is 29kg.\\

I. Which of the following box is the heaviest?\\
1) Blue \hspace{2mm}2) Green \hspace{2mm}3) Orange \hspace{2mm}4) Black \hspace{2mm}5) White\\

II. Which of the following combination is correct?\\
1) Brown: 22 Kg \hspace{2mm}2) Black: 17 Kg \hspace{2mm}3) Pink: 9 Kg
\hspace{2mm}4) Box 3: Black \hspace{2mm}5) Red: 11 Kg\\

III. Which of the following box number is Orange box?\\
1) 5th \hspace{2mm}2) 4th \hspace{2mm}3) 7th \hspace{2mm}4) 8th \hspace{2mm}5) 2nd\\

IV. Which of the following is TRUE regarding this arrangement?\\
1) Blue box is heavier than black box \hspace{2mm}2) None is true
\hspace{2mm}3) Red box is the lightest box \hspace{2mm}4) White box is just below the pink box
\hspace{2mm}5) Pink box is lighter than green box\\

V. What is the sum of the weight of Box 1 and Box 3?\\
1) 15 kg \hspace{2mm}2) 17 kg \hspace{2mm}3) 21 kg \hspace{2mm}4) 31 kg \hspace{2mm}5) 38 kg\\
%\includegraphics[width=0.60555in,height=0.32083in]{image2.png}
%\hspace{2mm}

\end{document}