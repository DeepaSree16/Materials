
% Options for packages loaded elsewhere
\PassOptionsToPackage{unicode}{hyperref}
\PassOptionsToPackage{hyphens}{url}
%
\documentclass[
]{article}
\usepackage{amsmath,amssymb}
\usepackage{lmodern}
\usepackage{iftex}
\ifPDFTeX
\usepackage[T1]{fontenc}
\usepackage[utf8]{inputenc}
\usepackage{textcomp} % provide euro and other symbols
\else % if luatex or xetex
\usepackage{unicode-math}
\defaultfontfeatures{Scale=MatchLowercase}
\defaultfontfeatures[\rmfamily]{Ligatures=TeX,Scale=1}
\fi
% Use upquote if available, for straight quotes in verbatim environments
\IfFileExists{upquote.sty}{\usepackage{upquote}}{}
\IfFileExists{microtype.sty}{% use microtype if available
	\usepackage[]{microtype}
	\UseMicrotypeSet[protrusion]{basicmath} % disable protrusion for tt fonts
}{}
\makeatletter
\@ifundefined{KOMAClassName}{% if non-KOMA class
	\IfFileExists{parskip.sty}{%
		\usepackage{parskip}
	}{% else
		\setlength{\parindent}{0pt}
		\setlength{\parskip}{6pt plus 2pt minus 1pt}}
}{% if KOMA class
	\KOMAoptions{parskip=half}}
\makeatother
\usepackage{xcolor}
\IfFileExists{xurl.sty}{\usepackage{xurl}}{} % add URL line breaks if available
\IfFileExists{bookmark.sty}{\usepackage{bookmark}}{\usepackage{hyperref}}
\hypersetup{
	hidelinks,
	pdfcreator={LaTeX via pandoc}}
\urlstyle{same} % disable monospaced font for URLs
\usepackage{longtable,booktabs,array}
\usepackage{calc} % for calculating minipage widths
% Correct order of tables after \paragraph or \subparagraph
\usepackage{etoolbox}
\makeatletter
\patchcmd\longtable{\par}{\if@noskipsec\mbox{}\fi\par}{}{}
\makeatother
% Allow footnotes in longtable head/foot
\IfFileExists{footnotehyper.sty}{\usepackage{footnotehyper}}{\usepackage{footnote}}
\makesavenoteenv{longtable}
\usepackage{graphicx}
\makeatletter
\def\maxwidth{\ifdim\Gin@nat@width>\linewidth\linewidth\else\Gin@nat@width\fi}
\def\maxheight{\ifdim\Gin@nat@height>\textheight\textheight\else\Gin@nat@height\fi}
\makeatother
% Scale images if necessary, so that they will not overflow the page
% margins by default, and it is still possible to overwrite the defaults
% using explicit options in \includegraphics[width, height, ...]{}
\setkeys{Gin}{width=\maxwidth,height=\maxheight,keepaspectratio}
% Set default figure placement to htbp
\makeatletter
\def\fps@figure{htbp}
\makeatother
\setlength{\emergencystretch}{3em} % prevent overfull lines
\providecommand{\tightlist}{%
	\setlength{\itemsep}{0pt}\setlength{\parskip}{0pt}}
\setcounter{secnumdepth}{-\maxdimen} % remove section numbering
\ifLuaTeX
\usepackage{selnolig}  % disable illegal ligatures
\fi

\author{}
\date{}
\usepackage{multirow}
\usepackage[inline]{enumitem}
\usepackage[margin=1.0in]{geometry}
\usepackage[english]{babel}
\usepackage[utf8]{inputenc}
\usepackage{fancyhdr}

\pagestyle{fancy}
\fancyhf{}
\rhead{\includegraphics[width=5.21667in, height=0.38819in]{image1.png}}
\lhead{ Reasoning: Relations-based Puzzles }
\lfoot{www.talentsprint.com }
\rfoot{\thepage}
\begin{document}
	
 

\begin{center}
	{\Large \textbf{Relations-based Puzzles \\}}
\end{center}

1. There are five men A, B, C, D and E and their nicknames are P, Q, R, S and T but not
necessary in same \includegraphics[width=0.60555in,height=0.32083in]{image2.png}order. They are married to L, M, N, O and F again not necessary in same
order.\\
1- C, D and E are not married to L and M and their nicknames are not P and Q.\\
2- B is not married to M and his name is not Q.\\
3- N is not married to T and her husband’s name is not D.\\
4- O is married to S.\\
5- C is neither R nor S and not married to N\\

2. \textbf{Directions (I-V):} Study the following information carefully and answer the questions given
below:\\
\includegraphics[width=0.60555in,height=0.32083in]{image2.png}
M, K, J, T, R, D and W are seven members of a family. There are two married couples among
them belonging to two different generations. Each of them has a different choice of cuisine -
Chinese, Continental, Thai, Punjabi, South Indian, Gujarati and Malwani. The grandfather in
the family likes Gujarati food. None of the ladies likes Continental or Thai food. T is the son
of M, who likes Chinese food. W is J's daughter-in-law and she likes South Indian food. K is
grandfather of D, who likes Punjabi food. J is mother of R, who likes Continental food.\\

I. How is R related to D?\\
a) Father \hspace{2mm}b) Brother \hspace{2mm}c) Uncle
\hspace{2mm}d) Data inadequate \hspace{2mm}e) None of these\\

II. How many male members are there in the family?\\
a) 3 \hspace{2mm}b) 4 \hspace{2mm}c) 5
\hspace{2mm}d) Data inadequate \hspace{2mm}e) None of these\\

III. Which of the following group contains one each from the same generations?\\
a) JRT \hspace{2mm}b) JRW \hspace{2mm}c) MRD \hspace{2mm}d) MWT \hspace{2mm}e) None of these\\

IV. Which food does T like?\\
a) Gujarati \hspace{2mm}b) Thai \hspace{2mm}c) Malwani \hspace{2mm}d) Data inadequate \hspace{2mm}e) None of these\\

V. Which of the following combinations represents favourite food of the two married
ladies?\\
a) Malwani, South Indian \hspace{2mm}b) South Indian, Punjabi
\hspace{2mm}c) Punjabi, Malwani \hspace{2mm}d) Data inadequate
\hspace{2mm}e) none of these\\

3. Four young men Anil, Mukesh, Piyush and Yogesh are lovingly called Munna, Babboo,
Prince and Papoo \includegraphics[width=0.60555in,height=0.32083in]{image2.png}by everyone. They are married to Madhu, Sunanada, Jyothi and Arti.\\
1. Arti and Madhu are not married to Piyush or Anil nor is their husband called
Babboo.\\
2. Babboo is not married to Sunanada and his name is not Piyush.\\
3. Sunanada is not married to Munna.\\
4. Mukesh is neither Munna nor prince nor is married to Madhu.\\

4. T, U, V, W, X, Y, Z are seven members of family. Two of them lives in Lucknow, two in Delhi,
one in \includegraphics[width=0.60555in,height=0.32083in]{image2.png}Mumbai, and one each lives in Patiala and Gurgaon. There are two married couples in
the family. W lives in Gurgaon and is married to one who lives in Delhi. No female member
of the family lives in Gurgaon. X who lives in Lucknow, is married to Z, who lives in Delhi.
Both the persons who live in Lucknow are females. T is married to W, who is father of Z.
Neither Y nor U lives in Lucknow. U is the son of the one who lives in Delhi and has only one
sister. Y is paternal aunt of V, who is granddaughter of the one who lives in Gurgaon. The
one who lives in Mumbai is a male.\\
How is U related to T?\\
I. a) Son \hspace{2mm}b) Daughter \hspace{2mm}c) Grandson
\hspace{2mm}d) Granddaughter \hspace{2mm}e) None of these\\

5. Study the following information carefully and complete the arrangement\\
\includegraphics[width=0.60555in,height=0.32083in]{image2.png}
(i) In a family of six persons, there are people from three generations. Each person has
separate profession and also they like different colours. There are two couples in the family.\\
(ii) Rohan is a CA and his wife neither is a doctor nor likes green colour.\\
(iii) Engineer likes red colour and his wife is a teacher.\\
(iv) Mohini is mother-in-law of Sunita and she likes orange colour.\\
(v) Vinod is grandfather of Tanmay and Tanmay, who is a principal, likes black colour.\\
(vi) Nanu is grand-daughter of Mohini and she likes blue colour. Nanu’s mother likes white
colour.\\
1. How many ladies are there in the family?\\
2. Which colour is liked by CA?\\

6. \textbf{Directions (I-V):} Study the following information carefully and answer the questions given
below:\\
\includegraphics[width=0.60555in,height=0.32083in]{image2.png}
A family of seven members go for a picnic in Scorpio. There are two couples in the family.
Each member has a coded name B1, B2, B3, B4, B5, B6 and B7. Only Two members can sit in the front
row as well as in the back row of the car and rest can sit in the middle row. Not more than
two males or females can sit in the same row or column. The daughter-in-law is driving the
car. One couple is seated in the same column. The granddaughter sits on the immediate left
of her grandmother.
B2 is father of B5. B7 has two sons and a daughter. B6 sits in the back row. B2 and B4 don’t sit in the
front row.B3 is married to B2 . B1 is father of B6 . Both the brothers sit in the same column. B4 is
daughter of B1
son-in-law of B are from different generations and they are the only one who are facing the
North and sitting together as well. Only three persons sit between F and G and both have the
same gender. F and G face the same direction. Only one couple sits together but they face
opposite directions and also none of them sits at the extreme end of the row. A, who does not
sit with F sits exactly in the middle and facing the North. B faces North. One of the immediate
neighbour of A is female. D has no uncle or aunt and also D does not have mother-in-law or
sister-in-law. I is not the son-in-law of B. D has a brother-in-law. E is a female. F is not the son-
in-law of H. Both persons sitting at the extreme end of the row are of same gender. D and I
are not married couples. E is not the sister of D.\\

I. Which of the following statements is/are not true?\\
a) One of the immediate neighbours of D faces South
\hspace{2mm}b) H is the father of C \hspace{2mm}c) A’s daughter sits third to the right of her mother
\hspace{2mm}d) Both a) and c) \hspace{2mm}e) None of these\\

II. How many persons face the South directions?\\
a) Four \hspace{2mm}b) Five \hspace{2mm}c) Three \hspace{2mm}d) Two \hspace{2mm}e) None of these\\

III. Who sits second to the left of G’s father-in-law?\\
a) The brother of C \hspace{2mm}b) The mother of F \hspace{2mm}c) The father of E
\hspace{2mm}d) Can’t be determined \hspace{2mm}e) None of these\\

IV. What is the position of the son of I with respect to the daughter-in-law of B?\\
a) Second to the right \hspace{2mm}b) Third to the right \hspace{2mm}c) Second to the left
\hspace{2mm}d) Third to the left \hspace{2mm}e) None of these\\

V. Four of the following five are alike in a certain way and so form a group. Which one does
not belong to that group?\\
a) F \hspace{2mm}b) E \hspace{2mm}c) A \hspace{2mm}d) I \hspace{2mm}e) G\\

8. Study the given information and answer the following questions.\\
\includegraphics[width=0.60555in,height=0.32083in]{image2.png}
A, B, C, D, E, F, G and H are the eight family members. These members are four married
couples and are sitting in a circle facing the centre. Each male member likes different games
football, cricket, tennis and baseball. D and H are sitting together. D takes cricket and H likes
baseball. The wife of each man is seated beside her husband. G, the wife of the person who
likes football is seated second to the right of H. F is seated between G and H. B is the wife of
the person who likes tennis. C does not like tennis and E is male.\\

I. What is F’s position with respect to C?\\
1) Immediate right \hspace{2mm}2) Second to the left \hspace{2mm}3) Immediate left
\hspace{2mm}4) Third to the left \hspace{2mm}5) None of these\\

II. Who is E's wife?\\
1) B \hspace{2mm}2) C \hspace{2mm}3) G \hspace{2mm}4) F \hspace{2mm}5) A\\

III. Which of the following is true?\\
1) G is D's wife \hspace{2mm}2) The husband of A is E \hspace{2mm}3) F is H's wife
\hspace{2mm}4) B is the wife of C \hspace{2mm}5) None of these\\

IV. Whose wives are seated together?\\
1) C and H \hspace{2mm}2) E and C \hspace{2mm}3) D and H
\hspace{2mm}4) D and C \hspace{2mm}5) Cannot be determined\\

9. \textbf{Directions (I-IV):} Study the following information carefully and answer the questions given
below:\\
\includegraphics[width=0.60555in,height=0.32083in]{image2.png}
There are seven persons in a family, namely, J, K, L, M, N, O and P. All of them are related
to each other in an order. Also each person has a different age.\\
(Note : Assume that wife is younger than husband but older than his younger brother)
J is older than K but younger than N. The third oldest person in the family is 36 years old, K
is sister of J. L is father of J. The third youngest person of the family is 33 years old. P is the
oldest person of the family. K is the niece of O. N is older than K.O is husband of N. P and
M are a married couple. L’s mother is 65 years old. L is younger then O. The oldest of the
family is a male member\\

I. If the total age of M and K is 80 years, then what is the age of K?\\
a) 17 years \hspace{2mm}b) 20 years \hspace{2mm}c) 15 years \hspace{2mm}d) 19 years \hspace{2mm}e) None of these\\

II. How is M related to N?\\
a) Mother \hspace{2mm}b) Mother in law \hspace{2mm}c) Sister
\hspace{2mm}d) None of these \hspace{2mm}e) Can’t be determined\\

III. What is the possible age of the oldest person of the family?\\
a) 64 years \hspace{2mm}b) 60 years \hspace{2mm}c) 62 years \hspace{2mm}d) 78 years \hspace{2mm}e) 58 years\\

IV. Who among the following is 33 years old?\\
a) O \hspace{2mm}b) N \hspace{2mm}c) L
\hspace{2mm}d) Either b or c \hspace{2mm}e) None of these\\

10. \textbf{Directions (I-VI):} Study the following information carefully and answer the questions given\\
\includegraphics[width=0.60555in,height=0.32083in]{image2.png}
below: There are seven family members A, B, D, E, F, H and K sitting in a row facing east.
There are two couples in the family. There are three generations in the family. The grandson
of family, who is the only person of third generation sits exactly between grandfather and
grandmother. A sits on the immediate left of H, who is sister of K. D sits at the extreme north
end of the row. D is daughter-in-law of E, who is on the immediate right of F and she is in
first generation only with one person. Only one person sits between the son of K and maternal
uncle of F. the number of persons between B and D and B and E is equal, which is not more
than two. B is not of the second generation and does sit exactly between D and E. E doesn’t
sit at any extreme end. E has only one son and one daughter and one of them is unmarried.\\

I. Who is the grandson of B?\\
a) A \hspace{2mm}b) E \hspace{2mm}c) F \hspace{2mm}d) H \hspace{2mm}e) Cannot be determined\\

II. Who among the following sits fourth to the right of F’s grandfather?\\
a) H \hspace{2mm}b) E \hspace{2mm}c) K \hspace{2mm}d) A \hspace{2mm}e) F\\

III. Who is father-in-law of H’s sister - in - law?\\
a) A \hspace{2mm}b) B \hspace{2mm}c) E \hspace{2mm}d) F \hspace{2mm}e) Cannot be determined\\

IV. How many females are there in the family?\\
a) One \hspace{2mm}b) Two \hspace{2mm}c) Three \hspace{2mm}e) Five \hspace{2mm}f) None\\

V. How is A related to K?\\
a) Brother \hspace{2mm}b) Brother - in-law \hspace{2mm}c) Sister-in-law
\hspace{2mm}d) Cannot be determined \hspace{2mm}e) Sister\\

VI. Who among the following are immediate neighbours of those at the extreme ends?\\
a) K, E \hspace{2mm}b) B, A \hspace{2mm}c) E, D \hspace{2mm}d) A, K \hspace{2mm}e) D, H\\
%\includegraphics[width=0.60555in,height=0.32083in]{image2.png}
%\hspace{2mm}

\end{document}