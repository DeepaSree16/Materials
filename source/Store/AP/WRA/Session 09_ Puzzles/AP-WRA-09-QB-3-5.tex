
% Options for packages loaded elsewhere
\PassOptionsToPackage{unicode}{hyperref}
\PassOptionsToPackage{hyphens}{url}
%
\documentclass[
]{article}
\usepackage{amsmath,amssymb}
\usepackage{lmodern}
\usepackage{iftex}
\ifPDFTeX
\usepackage[T1]{fontenc}
\usepackage[utf8]{inputenc}
\usepackage{textcomp} % provide euro and other symbols
\else % if luatex or xetex
\usepackage{unicode-math}
\defaultfontfeatures{Scale=MatchLowercase}
\defaultfontfeatures[\rmfamily]{Ligatures=TeX,Scale=1}
\fi
% Use upquote if available, for straight quotes in verbatim environments
\IfFileExists{upquote.sty}{\usepackage{upquote}}{}
\IfFileExists{microtype.sty}{% use microtype if available
	\usepackage[]{microtype}
	\UseMicrotypeSet[protrusion]{basicmath} % disable protrusion for tt fonts
}{}
\makeatletter
\@ifundefined{KOMAClassName}{% if non-KOMA class
	\IfFileExists{parskip.sty}{%
		\usepackage{parskip}
	}{% else
		\setlength{\parindent}{0pt}
		\setlength{\parskip}{6pt plus 2pt minus 1pt}}
}{% if KOMA class
	\KOMAoptions{parskip=half}}
\makeatother
\usepackage{xcolor}
\IfFileExists{xurl.sty}{\usepackage{xurl}}{} % add URL line breaks if available
\IfFileExists{bookmark.sty}{\usepackage{bookmark}}{\usepackage{hyperref}}
\hypersetup{
	hidelinks,
	pdfcreator={LaTeX via pandoc}}
\urlstyle{same} % disable monospaced font for URLs
\usepackage{longtable,booktabs,array}
\usepackage{calc} % for calculating minipage widths
% Correct order of tables after \paragraph or \subparagraph
\usepackage{etoolbox}
\makeatletter
\patchcmd\longtable{\par}{\if@noskipsec\mbox{}\fi\par}{}{}
\makeatother
% Allow footnotes in longtable head/foot
\IfFileExists{footnotehyper.sty}{\usepackage{footnotehyper}}{\usepackage{footnote}}
\makesavenoteenv{longtable}
\usepackage{graphicx}
\makeatletter
\def\maxwidth{\ifdim\Gin@nat@width>\linewidth\linewidth\else\Gin@nat@width\fi}
\def\maxheight{\ifdim\Gin@nat@height>\textheight\textheight\else\Gin@nat@height\fi}
\makeatother
% Scale images if necessary, so that they will not overflow the page
% margins by default, and it is still possible to overwrite the defaults
% using explicit options in \includegraphics[width, height, ...]{}
\setkeys{Gin}{width=\maxwidth,height=\maxheight,keepaspectratio}
% Set default figure placement to htbp
\makeatletter
\def\fps@figure{htbp}
\makeatother
\setlength{\emergencystretch}{3em} % prevent overfull lines
\providecommand{\tightlist}{%
	\setlength{\itemsep}{0pt}\setlength{\parskip}{0pt}}
\setcounter{secnumdepth}{-\maxdimen} % remove section numbering
\ifLuaTeX
\usepackage{selnolig}  % disable illegal ligatures
\fi

\author{}
\date{}
\usepackage{multirow}
\usepackage[inline]{enumitem}
\usepackage[margin=1.0in]{geometry}
\usepackage[english]{babel}
\usepackage[utf8]{inputenc}
\usepackage{fancyhdr}

\pagestyle{fancy}
\fancyhf{}
\rhead{\includegraphics[width=5.21667in, height=0.38819in]{image1.png}}
\lhead{ Reasoning: Floor-Based Puzzles }
\lfoot{www.talentsprint.com }
\rfoot{\thepage}
\begin{document}
	
 

\begin{center}
	{\Large \textbf{Grouping-Based Puzzles \\}}
\end{center}

1. Study the following information and answer these questions:\\
\includegraphics[width=0.60555in,height=0.32083in]{image2.png}
P, Q, R, S, T, U and G are students of a class. Each of them has a different favourite subject,
viz. Reasoning, Mathematics, Banking, Computer, English, GK and Marketing but not
necessarily in the same order. There are two such students whose one brother each is there
in the group. There is no other relation among the students. No girl likes Mathematics or
GK. S, who does not like Computer and English, is the sister of that student who likes
Marketing. The student who likes Computer is the brother of that girl student who likes
Reasoning. U is a girl student. Q is brother of P.\\

2. From amongst six boys G, H, I, J, K and L and five girls U, V, W, X, and Y a team of six is to
be selected \includegraphics[width=0.60555in,height=0.32083in]{image2.png}under the following conditions:\\
\begin{itemize}
\item G and J have to be together.
\item I cannot go with X.
\item X and Y have to be together.
\item H cannot be teamed with K.
\item J cannot go with U.
\item H and W have to be together.
\item I and V have to be together.
\end{itemize}


I. If there be five boys in the team, the only girl member is\\
1) U \hspace{2mm}2) V \hspace{2mm}3) W \hspace{2mm}4) X \hspace{2mm}5) None of these\\

II. If, including U, the team has three girls, the members other than U are\\
1) HILVW \hspace{2mm}2) GJKXY \hspace{2mm}3) GJHXY \hspace{2mm}4) HLWXY \hspace{2mm}5) None of these\\

3. Four players A, B, C, D are holding 4 cards each. Each of them has an Ace, a King, a Queen
and a Jack. \includegraphics[width=0.60555in,height=0.32083in]{image2.png}All of them have all the suits (Spades, Hearts, Clubs and Diamond)\\
1. A has Ace of spades and Queen of diamonds.\\
2. B has Ace of clubs and King of diamonds.\\
3. C has Queen of clubs and King of diamonds.\\
4. D has Jack of clubs\\

4. Eight persons A, C, E, F, G, L, M and P have different sim cards, viz. Docomo, Airtel and
BSNL. There \includegraphics[width=0.60555in,height=0.32083in]{image2.png}are at least two persons at most three persons who have the same sim. They
have recharged their sim with different amounts viz Rs.20, Rs.30 and Rs. 40. No two
persons, who have the same sim, recharged with the same amount. At least two persons and
at most three persons recharged with the same amount. P recharged with Rs.40 and has
BSNL sim. F and P have the same sim but F did not recharge with Rs.20. F and L recharged
with the same amount. The persons with Docomo sim, did not recharge with Rs.30. G and E
recharged with the same amount. G and M have the same sim which is not the same sim of
L. A and C recharged with the same amount but not with Rs.40. C and E have different
sims.\\

5. Study the following information carefully and answer the questions given below.\\
\includegraphics[width=0.60555in,height=0.32083in]{image2.png}
Of the five boys A, B, C, D and E two are good, one is poor and two are average in studies.
Two of them study in post-graduate classes and three in undergraduate classes. One comes
from a rich family, two from middle – class families and two from poor families. One of
them is interested in music, two in acting and one in sports. Of those studying in
undergraduate classes, two are average and one is poor in studies. Of the two boys
interested in acting one of them is a postgraduate student. The one interested in music
comes from a middle-class family. Both the boys interested in acting are not industrious.
The two boys coming from middle-class families are average in studies and one of them is
interested in acting. The boy interested in sports comes from a poor family, while the one
who is interested in music is not interested in industrious. E is industrious good in studies,
comes from a poor family and is not interested in acting, music or sports. C is poor in
studies inspite of being industrious. A comes from a rich family and is not industrious but
good in studies. B is industrious and comes from a middle – class family. Two boys are not
industrious in a family.\\
Which of the following group of boys are average in studies?\\
a) B and D \hspace{2mm}b) C and E \hspace{2mm}c) D and E \hspace{2mm}d) A and B \hspace{2mm}e) None of these\\

6. \textbf{Directions (I-V):} Study the following information to answer the given questions:\\
\includegraphics[width=0.60555in,height=0.32083in]{image2.png}
P, Q, R, S, T, V, W and Z are eight friends studying in three different engineering colleges –
A, B and C in three disciplines – Mechanical, Electrical and Electronics with not less than
two and not more than three in any college. Not more than three of them study in any of the
three disciplines. W studies Electrical in college B with only T, who studies Mechanical, P
and Z do not study in college C and study in the same discipline but not Electrical. R studies
Mechanical in college C with V, who studies Electrical. S studies Mechanical and does not
study in the same college where R studies. Q does not study Electronics.\\

I. Which of the following combinations of college student specialization is correct?\\
1) C – R – Electronics \hspace{2mm}2) A- Z- Electrical \hspace{2mm}3) B-W-Electrical
\hspace{2mm}4) B-Z-Electronics \hspace{2mm}5)None of these\\

II. In which of the following colleges do two students study in Electrical discipline?\\
1) A only \hspace{2mm}2) B only \hspace{2mm}3) C only
\hspace{2mm}4) Cannot be determined \hspace{2mm}5) None of these\\

III. In which discipline does Q study?\\
1) Electrical \hspace{2mm}2) Mechanical
\hspace{2mm}3) Electrical or Mechanical \hspace{2mm}4) Data Inadequate
\hspace{2mm}5) None of these\\

IV. In which of the colleges at least one student studies in Mechanical discipline?\\
1) A only \hspace{2mm}2) B only \hspace{2mm}3) C only
\hspace{2mm}4) Both A and B \hspace{2mm}5) All A,B and C\\

V. S studies in which college?\\
1) A \hspace{2mm}2) B \hspace{2mm}3) A or B
\hspace{2mm}4) Data Inadequate \hspace{2mm}5) None of these\\

7. \textbf{Directions (I-V):} Study the following information to answer the given questions:\\
\includegraphics[width=0.60555in,height=0.32083in]{image2.png}
P, Q, R, S, T, V and W are seven students of a school. Each of them studies in a different
standard—from Standard IV to Standard X-not necessarily in the same order. Each of them
has a favorite subject from English, Science, History, Geography, Mathematics, Hindi and
Sanskrit, not necessarily in the same order.
Q studies in VII Standard and does not like either Mathematics or Geography. R likes
English and does not study either in V or in IX. T studies in VIII Standard and likes Hindi.
The one who likes Science studies in X Standard. S studies in IV Standard. W likes Sanskrit.
P does not study in X Standard. The one who likes Geography studies in V Standard.\\

I. In which standard does W study?\\
1) VII \hspace{2mm}2) IX \hspace{2mm}3) X
\hspace{2mm}4) Data inadequate \hspace{2mm}5) None of these\\

II. Which subject does P like?\\
1) Geography \hspace{2mm}2) Mathematics \hspace{2mm}3) English
\hspace{2mm}4) History \hspace{2mm}5) None of these\\

III. Which subject does S like?\\
1) History \hspace{2mm}2) Geography \hspace{2mm}3)Mathematics
\hspace{2mm}4) Data inadequate \hspace{2mm}5) None of these\\

IV. In which standard does P study?\\
1) IV \hspace{2mm}2) VII \hspace{2mm}3) IX \hspace{2mm}4) X \hspace{2mm}5) None of these\\

V. Which of the following combinations of student standard-subject is correct?\\
1) T - VIII – Mathematics \hspace{2mm}2) W - VII - Sanskrit \hspace{2mm}3) Q-VII-Geography
\hspace{2mm}4) V - X - Science \hspace{2mm}5) None of these\\

8. \textbf{Directions (I-V):} Read the information carefully and answer the questions given below.\\
\includegraphics[width=0.60555in,height=0.32083in]{image2.png}
There are eight persons J, K, L, M, N, O, P and R , they came from different cities like Patna,
Mumbai, Hyderabad, Chennai, Bangalore, Lucknow, Kolkata and Delhi to attend the
seminar in Kanpur. They likes three different cars namely Scorpio, Bolero and Fortuner and
there are minimum two and maximum three persons like one car. The given information are
not necessary in the same order. The person one who is from Patna does not like Scorpio. M
doesn’t like the car same as one who is from Delhi. L and the person who is from Kolkata
does not like same car. J likes the Bolero. R does not like Bolero but likes the car same as one
who is from Lucknow. The person who is from Hyderabad likes Scorpio with only another
one. Both L and O are not from Patna and they doesn't like Bolero. The person M is not from
Mumbai. K is from Kolkata and does not like the car Bolero. O does not like the car same as
both M or R, who is from Bangalore. N likes the car same as the one who is from Patna, who
is not M. The one who is from Delhi likes Fortuner and he is not L.\\

I. Who among the following likes Scorpio?\\
1) The person who is from Mumbai and J \hspace{2mm}2) The persons from Bangalore and Hyderabad
\hspace{2mm}3) The person who is from Kolkatta and O \hspace{2mm}4) L and the person from Mumbai
\hspace{2mm}5) None of these\\

II. Which of the following will be true as per the given data?\\
1) J – Bolero – Delhi \hspace{2mm}2) P – Scorpio – Kolkatta \hspace{2mm}3) M – Fortuner – Chennai
\hspace{2mm}4) R – Scorpio – Bangalore \hspace{2mm}5) None is true\\

III. Four of the following alike in a certain way based on their position. Which of the one that
does not belongs to the group?\\
1) P and Lucknow \hspace{2mm}2) J and Delhi \hspace{2mm}3) Patna and M
\hspace{2mm}4) O and Kolkatta \hspace{2mm}5) Delhi and L\\

IV. Which of the following represents the group of persons likes the car Fortuner?\\
1) NKO \hspace{2mm}2) PNM \hspace{2mm}3) MJL \hspace{2mm}4) LRP \hspace{2mm}5) None of these\\

V. Who among the following from Chennai?\\
1) M \hspace{2mm}2) K \hspace{2mm}3) P \hspace{2mm}4) O \hspace{2mm}5) None of these\\

9. Read the following information carefully and answer the questions given below-\\
\includegraphics[width=0.60555in,height=0.32083in]{image2.png}
11, 12, 13, 14, 15, 16, 17, 18, 19 and 20 are the only ten members in a department. There is a
proposal to form a team from within the members of the department, subject to the
following conditions.\\
\begin{itemize}
    \item A team must include exactly one among 15, 17 and 18.
\item A team must include either 13 or 16 but not both.
\item If a team includes 11 then it must also include 12.
\item If a team includes one among 18, 19 and 20, then it must also include the other two.
\item 12 and 14 cannot be members of the same team.
\item 12 and 19 cannot be members of the same team.
\item The size of a team is defined as the number of members in the team.
\end{itemize}

Question: In how many ways a team can be constituted so that the team includes 14?\\

10. \textbf{Directions:} Study the following information carefully to answer the given questions.\\
\includegraphics[width=0.60555in,height=0.32083in]{image2.png}
There are seven lecturers A, B, C, D, E, F, G teaching seven different subjects Physics,
Economics, Geography, Biology, Chemistry, Maths and History. Their annual salary is
divided into three salary Slabs. 3-5 lakh, 6-8 lakh \& 10–13 lakh.
There are three persons only in the salary slab of 10-13 lakhs. F earns 12 lakh. B and G earn
lower than F’s salary in the same slab. D earns more than only one person who teaches
Economics. The person who teaches Chemistry earns more than the one who teaches
Geography and their total annual income is 15 lakh per annum in the same slab. There are
two persons only in the salary slab of 3-5 lakhs. The person who teaches History earns lower
than F and higher than the one who teaches Biology in the same slab. F does not teach
Physics. The total annual income of G \& A and B \& E are 17 lakh and 19 lakh respectively.\\
The difference between the annual income of D and C is 2 lakh.\\
%\includegraphics[width=0.60555in,height=0.32083in]{image2.png}
%\hspace{2mm}

\end{document}