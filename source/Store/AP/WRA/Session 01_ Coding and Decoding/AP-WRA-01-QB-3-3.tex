
% Options for packages loaded elsewhere
\PassOptionsToPackage{unicode}{hyperref}
\PassOptionsToPackage{hyphens}{url}
%
\documentclass[
]{article}
\usepackage{amsmath,amssymb}
\usepackage{lmodern}
\usepackage{iftex}
\ifPDFTeX
\usepackage[T1]{fontenc}
\usepackage[utf8]{inputenc}
\usepackage{textcomp} % provide euro and other symbols
\else % if luatex or xetex
\usepackage{unicode-math}
\defaultfontfeatures{Scale=MatchLowercase}
\defaultfontfeatures[\rmfamily]{Ligatures=TeX,Scale=1}
\fi
% Use upquote if available, for straight quotes in verbatim environments
\IfFileExists{upquote.sty}{\usepackage{upquote}}{}
\IfFileExists{microtype.sty}{% use microtype if available
	\usepackage[]{microtype}
	\UseMicrotypeSet[protrusion]{basicmath} % disable protrusion for tt fonts
}{}
\makeatletter
\@ifundefined{KOMAClassName}{% if non-KOMA class
	\IfFileExists{parskip.sty}{%
		\usepackage{parskip}
	}{% else
		\setlength{\parindent}{0pt}
		\setlength{\parskip}{6pt plus 2pt minus 1pt}}
}{% if KOMA class
	\KOMAoptions{parskip=half}}
\makeatother
\usepackage{xcolor}
\IfFileExists{xurl.sty}{\usepackage{xurl}}{} % add URL line breaks if available
\IfFileExists{bookmark.sty}{\usepackage{bookmark}}{\usepackage{hyperref}}
\hypersetup{
	hidelinks,
	pdfcreator={LaTeX via pandoc}}
\urlstyle{same} % disable monospaced font for URLs
\usepackage{longtable,booktabs,array}
\usepackage{calc} % for calculating minipage widths
% Correct order of tables after \paragraph or \subparagraph
\usepackage{etoolbox}
\makeatletter
\patchcmd\longtable{\par}{\if@noskipsec\mbox{}\fi\par}{}{}
\makeatother
% Allow footnotes in longtable head/foot
\IfFileExists{footnotehyper.sty}{\usepackage{footnotehyper}}{\usepackage{footnote}}
\makesavenoteenv{longtable}
\usepackage{graphicx}
\makeatletter
\def\maxwidth{\ifdim\Gin@nat@width>\linewidth\linewidth\else\Gin@nat@width\fi}
\def\maxheight{\ifdim\Gin@nat@height>\textheight\textheight\else\Gin@nat@height\fi}
\makeatother
% Scale images if necessary, so that they will not overflow the page
% margins by default, and it is still possible to overwrite the defaults
% using explicit options in \includegraphics[width, height, ...]{}
\setkeys{Gin}{width=\maxwidth,height=\maxheight,keepaspectratio}
% Set default figure placement to htbp
\makeatletter
\def\fps@figure{htbp}
\makeatother
\setlength{\emergencystretch}{3em} % prevent overfull lines
\providecommand{\tightlist}{%
	\setlength{\itemsep}{0pt}\setlength{\parskip}{0pt}}
\setcounter{secnumdepth}{-\maxdimen} % remove section numbering
\ifLuaTeX
\usepackage{selnolig}  % disable illegal ligatures
\fi

\author{}
\date{}
\usepackage{multirow}
\usepackage[inline]{enumitem}
\usepackage[margin=1.0in]{geometry}
\usepackage[english]{babel}
\usepackage[utf8]{inputenc}
\usepackage{fancyhdr}

\pagestyle{fancy}
\fancyhf{}
\rhead{\includegraphics[width=5.21667in, height=0.38819in]{image1.png}}
\lhead{ Reasoning: Coding and Decoding }
\lfoot{www.talentsprint.com }
\rfoot{\thepage}
\begin{document}
	
 

\begin{center}
	{\Large \textbf{Coding and Decoding \\}}
\end{center}

{\large \textbf{Additional Examples  \\}}

1. In a certain code ESTABLISHMENT is written as EOTNBTIMHBETT, then how will
ACCOMMODATE be written in that code?\\
\includegraphics[width=0.60555in,height=0.32083in]{image2.png}
1) ACUMEANOPDE \hspace{1mm}2) AUCEMNOPADE \hspace{1mm}3) AMUCENAOPEI
\hspace{1mm}4) AMUCWENOEID \hspace{1mm}5) None of these

2. In a certain code PROMOTIONS is written as NLNQOTOPJU, then how will SEQUENTIAL be written in that code?\\
\includegraphics[width=0.60555in,height=0.32083in]{image2.png}
1) LAITNEUQES \hspace{1mm}2) LAITNESEQUE \hspace{1mm}3) DTPDRMBJUO \hspace{1mm}4) DRDTPMBJUO \hspace{1mm}5) None of these\\

3. In a certain code DOMAIN is written as NPEOJB. How is STREAM written in that code?\\
\includegraphics[width=0.60555in,height=0.32083in]{image2.png}


4. Directions (i-v): Study the information and answer the following questions:\\
\includegraphics[width=0.60555in,height=0.32083in]{image2.png}
"Few students are poor in numerics" = (5\$) (6\$) (1'\$) (5\%) (3\$\$) (7\%)\\
“Some numerics are tough to crack” = (6\$) (8\%) (8\$) (7\$) (3'\$\$) (6\%)\\
“Tough numerics come in SBI” = (8\$) (9\$) (5\%) (3'\%) (3'\$\$)\\
“SBI never select poor student” = (1'\$) (1'\$) (1'\$) (3\%) (4\$)\\
i. What is the code for "goal is to crack SBI'?\\
a) (5\$\$) (9\$) (6\%) (3'\%)(3\%)\\
b) (8\$) (8\%) (9\%) (3\%) (1'\%)\\
c) (8\$) (5\$\$) (9\$) (6\%) (3\%)\\
d) Cannot be determined\\
e) None of these\\

ii. What is the code for "poor" from the above coded sentences?\\
a) (3\%) \hspace{1mm} b) (1'\$) \hspace{1mm} c) (4\$) \hspace{1mm} d)Cannot be determined \hspace{1mm} e) None of these\\

iii. What would be the code for "for" in the sentence "Few students are poor for SBI" ?\\
a) (5\$\$) \hspace{1mm} b) (5'\$\$) \hspace{1mm} c) (3\%) \hspace{1mm} d)(8\$) \hspace{1mm} e) None of these\\

iv. What is the code for "Some numerics come tough" from the above coding?\\
a) (8\$) (5\$) (9\$) (8\$) \hspace{1mm} b) (8\$) (3'\$\$) (9\$) (7\$) \hspace{1mm} c) (8\$) (3'\$\$) (9\$) (6\%) \hspace{1mm} d) (6\%) (8\%) (8\$) (9\$) \hspace{1mm} e) None of these\\

v. How many even numbers will come if "few are enjoying Indian Summer" is coded as above?\\
a) 1 \hspace{1mm} b) 3 \hspace{1mm} c) 5 \hspace{1mm} d) 2 \hspace{1mm} e) None of these\\

5. In certain code, ‘acquisition or construction should be completed within three years’ is written as ‘three be or within should years construction completed acquisition’. How will “interest paid on loan will be allowed for deduction” be written in that code?\\
\includegraphics[width=0.60555in,height=0.32083in]{image2.png}


6. In a certain code language the word ‘HOARDING’ is written as 23191321 and ‘LIMERICK’is as 21182714. How will the word ‘SUITABLE’ is written in that language?\\
\includegraphics[width=0.60555in,height=0.32083in]{image2.png}


7. In a certain code language ‘you are very intelligent’ is written as ‘4@W 7\$E 3\#y 4\%H \\
\includegraphics[width=0.60555in,height=0.32083in]{image2.png}

‘they seem very intelligent’ is written as ‘8\*O 7\$E 3\#y 9\&U\\
‘how intelligent is she’ is written as ‘3\#y 10\%L 6!O 2\$R’\\
‘how can you say’ is written as ‘1\%Q 3\#E 4\%H 10\%L\\

8. Study the information given below and answer the questions based on it.\\
\includegraphics[width=0.60555in,height=0.32083in]{image2.png}
'A \^ B' means 'A is neither smaller nor equal to B'\\
'A * B' means 'A is neither greater nor equal to B'\\
'A @ B' means 'A is not smaller than B'\\
'A \$ B' means 'A is neither smaller nor greater than B'\\
'A \# B' means 'A is not greater than B'\\
Now in the following question, assuming the given statements to be true, find which of the four conclusions given below them is/are true\\
Statements: D \$ E, D * F, H \# D, D \$ Q\\
Conclusions: I. F \$ H II. H \# E III. E \# Q IV. E * F\\
1) If only conclusions I and III are true\\
2) If only conclusions II and IV are true\\
3) If either conclusion I or III is true\\
4) If only conclusions I, II and IV are true\\
5) None is true\\

9. Directions (9): Study the following information carefully and answer the questions which follow\\
\includegraphics[width=0.60555in,height=0.32083in]{image2.png}
In a certain code language, ‘small and medium groups’ is written as ‘du fla kha gi’ ‘medium enterprises are good’ is written as 'flu kli du fu’, ‘groups should do good’ is written as ‘dlu ka kli fla’, 'small enterprises earn more’ is written as ‘gi hla flu dfi’ and ‘do more’ is written as ‘ka hla’.\\

I. What does ‘gi kli’ stands for?\\
1) small do \hspace{1mm} 2) group small \hspace{1mm} 3) good group \hspace{1mm} 4) good small \hspace{1mm} 5) good medium\\

10. In a certain code language RUSTICATE is written as QTTUIDBSD, how would STATISTIC be written in that code?\\
\includegraphics[width=0.60555in,height=0.32083in]{image2.png}
a) RSBUJTUHB \hspace{1mm} b) RSBUITUHB \hspace{1mm} c) RSBUIRSJD \hspace{1mm} d) TUBUITUMB\\

11. If GO = 32, SHE = 42, then SOME will be equal to-\\
\includegraphics[width=0.60555in,height=0.32083in]{image2.png}
a) 60 \hspace{1mm} b) 62 \hspace{1mm} c) 64 \hspace{1mm} d) 58

12. Directions (I-II): In the following questions, letters are given in the first line and numbers are given in the second line. Numbers are the codes for letters, and letters are the codes for the numbers. Choose the correct code as your answer from amongst the suggested options 1, 2, 3 and 4.\\
\includegraphics[width=0.60555in,height=0.32083in]{image2.png}
I. JUPITER\\
a) 1209874 \hspace{1mm} b) 1928074 \hspace{1mm} c) 1927084 \hspace{1mm} d) 1928054\\

II. AMPUTE\\
a) 3 5 28 0 7 \hspace{1mm} b) 3 5 2 9 0 7 \hspace{1mm} c) 2 3 5 7 0 9 \hspace{1mm} d) 3 8 5 2 0 7\\

13. Directions (I-V): Study the following information carefully and answer the questions given below:\\
\includegraphics[width=0.60555in,height=0.32083in]{image2.png}
In a certain code language some statements are coded as follow:\\
‘Reserve Bank of India’ is coded as ‘A11 B21 C31 C41’\\
‘India is great country’ is coded as ‘C41 D51 E61 G71’\\
‘Bank rate and deposit’ is coded as ‘B21 H81 I91 J85’\\
‘Mahatma Gandhi was great’ K75 L65 M64 E61\\

I. Which amongst the following options can be used for getting the code of ‘country’?\\
I. M64 D51 H81 K75\\
II. L65 K75 J85 A11\\
III. B21 D51 G71 C31\\
IV. G71 I91 J85 L65\\
a) Only I and II \hspace{1mm} b) Only I and IV \hspace{1mm} c) Only III and IV \hspace{1mm} d) Only II and III \hspace{1mm} e) None of these

II. Which amongst the following options can not be used for getting the code of ‘deposit’?\\
a) H81 A11 M64 L65 \hspace{1mm} b) B21 C41 J85 M64 \hspace{1mm} c) D51 I91 H81 C31 \hspace{1mm} d) D51 C31 K75 I91 \hspace{1mm} e) None of these\\
III. If ‘Ghandi’ can be coded as ‘L65’ which amongst the following options can be used for getting the code of ‘Mahatma’?\\
a) H81 K75 D51 M64 \hspace{1mm} b) A11 B21 K75 M64 \hspace{1mm} c) K75 A11 C41 D51 \hspace{1mm} d) M64 I91 E61 K75 \hspace{1mm} e) None of these\\

IV. What is the code for 'Bank' in the given code language?\\
a) B21 \hspace{1mm} b) C31 \hspace{1mm} c) A11
\hspace{1mm} d) None of these \hspace{1mm} e) Can’t be determined\\

V. What does the code ‘A11’ stand for?\\
a) Reserve \hspace{1mm} b) of \hspace{1mm} c) India \hspace{1mm} d) None of these \hspace{1mm} e) Can’t be determined

14. In a certain code language ‘CHANGED’ is written as ‘DJFQGGJ’. How would ‘FEMALES’ be written in that code?\\
\includegraphics[width=0.60555in,height=0.32083in]{image2.png}
a) PGIVDGO \hspace{1mm} b) PGIGDVO \hspace{1mm} c) PGIDVGO \hspace{1mm} d) OGIDVGO \hspace{1mm} e) None of these

15. Directions: Study the following information carefully and answer the questions given below:\\
\includegraphics[width=0.60555in,height=0.32083in]{image2.png}
In a certain code language\\
‘Bitcoin is a cryptocurrency’ is written as ’O93 N13 U39 M31’,\\
‘worldwide payment system’ is written as ‘T38 Z83 Y41’,\\
‘it is first digital currency’ is written as ‘S72 W25 V52 X14 N13’,\\
‘Bitcoin was invented’ is written as ‘Q46 R27 M31’,\\
‘cryptocurrency payment wallet’ is written as ‘P64 Z83 U39’ and\\
‘first worldwide invented currency’ is written as ‘S72 Q46 T38 W25’.\\

I. What is the code for ‘digital’ in the given code language?\\
a) X14 \hspace{1mm} b) W25 \hspace{1mm} c) V52 \hspace{1mm} d) None of these \hspace{1mm} e) Can’t be determined\\

II. What does the code ‘S72 W25’ stand for in the given code language?\\
a) first digital \hspace{1mm} b) invented currency \hspace{1mm} c) first currency \hspace{1mm} d) Can’t be determined \hspace{1mm} e) None of these

III. What may be the possible code for ‘Indian payment system’ in the given code language?\\
a) Z83 B49 Y41 \hspace{1mm} b) B49 Q46 Z83 \hspace{1mm} c) Z83 Y41 S72
\hspace{1mm} d) W25 Y41 B49 \hspace{1mm} e) None of these

16. If MACHINE is coded as 19 – 7 – 9 – 14 – 15 – 20 – 11, then how will you code DANGER in the same code?\\
\includegraphics[width=0.60555in,height=0.32083in]{image2.png}
a) 11 – 7 – 20 – 16 - 11 – 24 \hspace{1mm} b) 13 – 7 – 20 – 9 – 11 – 25
c) 10 – 7 – 20 – 13 – 11 – 24 \hspace{1mm} d) 13 – 7 – 20 – 10 – 11 – 25

17. In a certain language ‘GUST’ is coded as ‘@7\$2’, ‘SNIP’ is coded as ‘957\#’ and GAPE’ is coded as ‘\β\$35’. How will ‘SING’ be coded in the same code?\\
\includegraphics[width=0.60555in,height=0.32083in]{image2.png}
a) 9\$7\# b) 59\#\$ c) 9\β7\$ d) 7\$59 e) \$27\#

18. Directions: Study the following carefully and answer the questions given below.\\
\includegraphics[width=0.60555in,height=0.32083in]{image2.png}
In a certain code language\\
‘fortune favours the brave’ is written as "al ti si ni",\\
‘he is the brave politician’ is written as "ni pe ro tin si",\\
‘politician favours the corruption’ is written as "ro ni jo ti",\\
and ‘corruption is rampant there‘ is written as "le tin jo que".\\
I. Which of the following can be the code for ‘corruption is politician rampant’?\\
a) jo le se ro \hspace{1mm} b) ro tin si le \hspace{1mm} c) Either 1) or 2)
\hspace{1mm} d) None of these \hspace{1mm} e) Can’t be determined\\

II. ‘tin pe ni jo’ is the code for which of the following?\\
a) favours he is the \hspace{1mm} b) is he the corruption \hspace{1mm} c) corruption is the rampant \hspace{1mm} d) politician is the corruption \hspace{1mm} e) None of these\\

III. What does the code ‘ti’ stands for?\\
a) politician \hspace{1mm} b) corruption \hspace{1mm} c) fortune \hspace{1mm} d) favours \hspace{1mm} e) None of these\\

\end{document}
