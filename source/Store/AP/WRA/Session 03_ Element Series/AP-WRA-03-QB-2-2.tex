
% Options for packages loaded elsewhere
\PassOptionsToPackage{unicode}{hyperref}
\PassOptionsToPackage{hyphens}{url}
%
\documentclass[
]{article}
\usepackage{amsmath,amssymb}
\usepackage{lmodern}
\usepackage{iftex}
\ifPDFTeX
\usepackage[T1]{fontenc}
\usepackage[utf8]{inputenc}
\usepackage{textcomp} % provide euro and other symbols
\else % if luatex or xetex
\usepackage{unicode-math}
\defaultfontfeatures{Scale=MatchLowercase}
\defaultfontfeatures[\rmfamily]{Ligatures=TeX,Scale=1}
\fi
% Use upquote if available, for straight quotes in verbatim environments
\IfFileExists{upquote.sty}{\usepackage{upquote}}{}
\IfFileExists{microtype.sty}{% use microtype if available
	\usepackage[]{microtype}
	\UseMicrotypeSet[protrusion]{basicmath} % disable protrusion for tt fonts
}{}
\makeatletter
\@ifundefined{KOMAClassName}{% if non-KOMA class
	\IfFileExists{parskip.sty}{%
		\usepackage{parskip}
	}{% else
		\setlength{\parindent}{0pt}
		\setlength{\parskip}{6pt plus 2pt minus 1pt}}
}{% if KOMA class
	\KOMAoptions{parskip=half}}
\makeatother
\usepackage{xcolor}
\IfFileExists{xurl.sty}{\usepackage{xurl}}{} % add URL line breaks if available
\IfFileExists{bookmark.sty}{\usepackage{bookmark}}{\usepackage{hyperref}}
\hypersetup{
	hidelinks,
	pdfcreator={LaTeX via pandoc}}
\urlstyle{same} % disable monospaced font for URLs
\usepackage{longtable,booktabs,array}
\usepackage{calc} % for calculating minipage widths
% Correct order of tables after \paragraph or \subparagraph
\usepackage{etoolbox}
\makeatletter
\patchcmd\longtable{\par}{\if@noskipsec\mbox{}\fi\par}{}{}
\makeatother
% Allow footnotes in longtable head/foot
\IfFileExists{footnotehyper.sty}{\usepackage{footnotehyper}}{\usepackage{footnote}}
\makesavenoteenv{longtable}
\usepackage{graphicx}
\makeatletter
\def\maxwidth{\ifdim\Gin@nat@width>\linewidth\linewidth\else\Gin@nat@width\fi}
\def\maxheight{\ifdim\Gin@nat@height>\textheight\textheight\else\Gin@nat@height\fi}
\makeatother
% Scale images if necessary, so that they will not overflow the page
% margins by default, and it is still possible to overwrite the defaults
% using explicit options in \includegraphics[width, height, ...]{}
\setkeys{Gin}{width=\maxwidth,height=\maxheight,keepaspectratio}
% Set default figure placement to htbp
\makeatletter
\def\fps@figure{htbp}
\makeatother
\setlength{\emergencystretch}{3em} % prevent overfull lines
\providecommand{\tightlist}{%
	\setlength{\itemsep}{0pt}\setlength{\parskip}{0pt}}
\setcounter{secnumdepth}{-\maxdimen} % remove section numbering
\ifLuaTeX
\usepackage{selnolig}  % disable illegal ligatures
\fi

\author{}
\date{}
\usepackage{multirow}
\usepackage[inline]{enumitem}
\usepackage[margin=1.0in]{geometry}
\usepackage[english]{babel}
\usepackage[utf8]{inputenc}
\usepackage{fancyhdr}

\pagestyle{fancy}
\fancyhf{}
\rhead{\includegraphics[width=5.21667in, height=0.38819in]{image1.png}}
\lhead{ Reasoning: Element Series }
\lfoot{www.talentsprint.com }
\rfoot{\thepage}
\begin{document}
	
 

\begin{center}
	{\Large \textbf{ Element Series\\}}
\end{center}

\textbf{Additional Examples}\\

1. Answer the following question on the basis of following number series:\\
\includegraphics[width=0.60555in,height=0.32083in]{image2.png}
3 2 5 1 3 6 4 2 1 5 3 4 2 1 9 8 4 6 5 4 2 3 9 5 9 4\\
Which number is 3rd to the right of 4th to the right of the number which is 18 from left in the series?\\

2. Following questions are based on six four digit numbers given below\\
\includegraphics[width=0.60555in,height=0.32083in]{image2.png}
1252 1042 1098 2610 3357\\
If fourth and third digit of each of the number will interchange each other than how many number will become perfect square?\\

3. T J L 2 \$ D = M \# 8 C \% B < K 1 \& A W ? P E $\pi$ Q @ 7 F 6\\
\includegraphics[width=0.60555in,height=0.32083in]{image2.png}
If the first fifteen elements in the above sequence are written in reverse order, then which of the following will be twenty-first from the right end?\\

4. Study the following arrangement and answer questions given\\
\includegraphics[width=0.60555in,height=0.32083in]{image2.png}
E 3 * L J 8 \# R 1 4 \% S A \delta \hspace{1mm} 6\hspace{1mm} K\hspace{1mm} 9\hspace{1mm} M\hspace{1mm} \$\hspace{1mm} X\hspace{1mm} 7\hspace{1mm} \beta \hspace{1mm}B\hspace{1mm} Z\hspace{1mm} @\hspace{1mm} 2\hspace{1mm} L\\
Complete the series: K9X\hspace{1mm} \delta6M\hspace{1mm} SAK ?\\
1) \%AX \hspace{2mm}2) 4\%A \hspace{2mm}3) SA\% \hspace{2mm}4) 4\%U \hspace{2mm}5) None of these\\

5. Directions (I-II): The questions are based on five words given below:\\
\includegraphics[width=0.60555in,height=0.32083in]{image2.png}
PRV CHT SEL NKB MIG\\


I. If all the letters are changed to its previous letter in all the words then how many words will not have any vowels?\\
1) One \hspace{2mm}2) Two \hspace{2mm}3) Three \hspace{2mm}4) Four \hspace{2mm}5) Five\\

II. If first and second letters are interchanged in all the words and then all the words are
arranged in reverse alphabetical order, then which of the following word is the second to the left of second from the right?\\

6. Directions (I-III): Study the following arrangement carefully and answer the questions
given below:\\
\includegraphics[width=0.60555in,height=0.32083in]{image2.png}
X 9 E \% J \# I @ F 7 2 T A * R \& 3 O 4 \$ M 5 K\\

I. If all the symbols are dropped from the given arrangement and all the consonants are
changed to its next letter, then what will be the middle element between I and L\\
1) T \hspace{2mm}2) B \hspace{2mm}3) S \hspace{2mm}4) Q \hspace{2mm}5) 2\\

II. If all the elements are arranged in a particular order such that all the elements preceded
by a symbol are moved one step to the right, now what will be the 7th element from the
right?\\
1) * \hspace{2mm}2) & \hspace{2mm}3) R \hspace{2mm}4) 3 \hspace{2mm}5) O\\

III. Four of the following five are alike in a certain way based on their positions in the above
arrangement and so form a group. Which is the one that does not belong to that group\\
1) X T 8 \hspace{2mm}2) * 5 O \hspace{2mm}3) I 3 T \hspace{2mm}4) J A @ \hspace{2mm}5) E @ \#\\

7. How many such pairs of Numbers are there in the 7693142, each of which has as many Numbers between \includegraphics[width=0.60555in,height=0.32083in]{image2.png}them in the number, as they have in the numeric series?\\
a) None \hspace{2mm}b) One \hspace{2mm}c) Two \hspace{2mm}d) Three \hspace{2mm}e) More than three\\

8. In each question below is given a group of letters followed by four combinations of digits/symbols numbered \includegraphics[width=0.60555in,height=0.32083in]{image2.png}(A), (B), (C) and (D). You have to find out which of the
combinations correctly represents the group of letters based on the coding system and mark
the number of that combination as your answer. If none of the combinations correctly
represents the group of letters, mark (E), i.e., ”None of these” as your answer.\\

\begin{tabular}{ c c c c c c c c c c c c c c c c}
Letter& W &R &A &P &G &B &M &U &S &E &F &T &N &D\\
Digital/symbol Code& \$& 8& !& 2& 7& \#& 9& @& ?& 5& \beta& 4& *& 6\\
\end{tabular}

\textbf{Conditions:}
(i) If the middle letter is a vowel, the codes for the first and fourth letters are to be interchanged.\\
(ii) If the first two letters are consonants, the first letter is to be coded, no code may be given to the second letter and the remaining three letters are to be coded.\\
(iii) If the first letter is a vowel and the last letter is a consonant both are to be coded as the code for the Consonant.\\
WPSEN\\
a) \$?5* \hspace{2mm}b) \$?9*\# \hspace{2mm}c) \$?6*\# \hspace{2mm}d) \#?5! \hspace{2mm}e) None of these\\

9. Directions: To answer these questions study carefully the following arrangement of letters, digits and symbols.\\
\includegraphics[width=0.60555in,height=0.32083in]{image2.png}
M 7 \sum \hspace{1mm}8 \hspace{1mm}L \hspace{1mm}P \hspace{1mm}@ \hspace{1mm}? \hspace{1mm}6 \hspace{1mm}N \hspace{1mm}B \hspace{1mm}T \hspace{1mm}? \hspace{1mm}Y \hspace{1mm}3 \hspace{1mm}2 \hspace{1mm}= \hspace{1mm}E \hspace{1mm}\$ \hspace{1mm}4\hspace{1mm} 9\hspace{1mm} ©\hspace{1mm} G\hspace{1mm} H\hspace{1mm} 5\\

1. How many such letters are there in the arrangement each of which is immediately followed by a number?\\
1) Three \hspace{2mm}2) Four \hspace{2mm}3) One \hspace{2mm}4) Two \hspace{2mm}5) None of these\\

2. How many such symbols are there in the arrangement each of which is immediately preceded by a number?\\
1) Two \hspace{2mm}2) Three \hspace{2mm}3) Four \hspace{2mm}4) Nil \hspace{2mm}5) None of these\\

3. If all the symbols are deleted from the arrangement then which of the following will be fourth to the left of the 17th element from the left end?\\
1) 9 \hspace{2mm}2) E \hspace{2mm}3) 2 \hspace{2mm}4) Y \hspace{2mm}5) None of these\\

10. Direction: In every question two rows are given and to find out the resultant of a particular row you need to follow the following steps:\\
\includegraphics[width=0.60555in,height=0.32083in]{image2.png}
Step 1: If an even number is followed by an odd (prime) number then the resultant will be the addition of both the numbers.\\
Step 2: If an odd number is followed by a perfect square then the resultant will be the difference between the numbers.\\
Step 3: If an odd number is followed by another odd number then the resultant will be the addition of both the numbers.\\
Step 4: If an even number is followed by an odd (non-prime) number then the resultant will be the difference between the odd number and the even number.\\
Step 5: If an odd number is followed by an even number then the resultant comes by multiplying the numbers.\\
If the sum of the resultants of two rows is 20. Then find the value of X.\\

18\hspace{5mm}15\hspace{5mm}6\\
8 \hspace{5mm}3 \hspace{5mm}X\\
a) 3 \hspace{2mm}b) 9 \hspace{2mm}c) 2 \hspace{2mm}d) 5 \hspace{2mm}e) None of the above\\

\end{document}
