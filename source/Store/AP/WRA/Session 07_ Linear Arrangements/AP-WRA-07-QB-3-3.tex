
% Options for packages loaded elsewhere
\PassOptionsToPackage{unicode}{hyperref}
\PassOptionsToPackage{hyphens}{url}
%
\documentclass[
]{article}
\usepackage{amsmath,amssymb}
\usepackage{lmodern}
\usepackage{iftex}
\ifPDFTeX
\usepackage[T1]{fontenc}
\usepackage[utf8]{inputenc}
\usepackage{textcomp} % provide euro and other symbols
\else % if luatex or xetex
\usepackage{unicode-math}
\defaultfontfeatures{Scale=MatchLowercase}
\defaultfontfeatures[\rmfamily]{Ligatures=TeX,Scale=1}
\fi
% Use upquote if available, for straight quotes in verbatim environments
\IfFileExists{upquote.sty}{\usepackage{upquote}}{}
\IfFileExists{microtype.sty}{% use microtype if available
	\usepackage[]{microtype}
	\UseMicrotypeSet[protrusion]{basicmath} % disable protrusion for tt fonts
}{}
\makeatletter
\@ifundefined{KOMAClassName}{% if non-KOMA class
	\IfFileExists{parskip.sty}{%
		\usepackage{parskip}
	}{% else
		\setlength{\parindent}{0pt}
		\setlength{\parskip}{6pt plus 2pt minus 1pt}}
}{% if KOMA class
	\KOMAoptions{parskip=half}}
\makeatother
\usepackage{xcolor}
\IfFileExists{xurl.sty}{\usepackage{xurl}}{} % add URL line breaks if available
\IfFileExists{bookmark.sty}{\usepackage{bookmark}}{\usepackage{hyperref}}
\hypersetup{
	hidelinks,
	pdfcreator={LaTeX via pandoc}}
\urlstyle{same} % disable monospaced font for URLs
\usepackage{longtable,booktabs,array}
\usepackage{calc} % for calculating minipage widths
% Correct order of tables after \paragraph or \subparagraph
\usepackage{etoolbox}
\makeatletter
\patchcmd\longtable{\par}{\if@noskipsec\mbox{}\fi\par}{}{}
\makeatother
% Allow footnotes in longtable head/foot
\IfFileExists{footnotehyper.sty}{\usepackage{footnotehyper}}{\usepackage{footnote}}
\makesavenoteenv{longtable}
\usepackage{graphicx}
\makeatletter
\def\maxwidth{\ifdim\Gin@nat@width>\linewidth\linewidth\else\Gin@nat@width\fi}
\def\maxheight{\ifdim\Gin@nat@height>\textheight\textheight\else\Gin@nat@height\fi}
\makeatother
% Scale images if necessary, so that they will not overflow the page
% margins by default, and it is still possible to overwrite the defaults
% using explicit options in \includegraphics[width, height, ...]{}
\setkeys{Gin}{width=\maxwidth,height=\maxheight,keepaspectratio}
% Set default figure placement to htbp
\makeatletter
\def\fps@figure{htbp}
\makeatother
\setlength{\emergencystretch}{3em} % prevent overfull lines
\providecommand{\tightlist}{%
	\setlength{\itemsep}{0pt}\setlength{\parskip}{0pt}}
\setcounter{secnumdepth}{-\maxdimen} % remove section numbering
\ifLuaTeX
\usepackage{selnolig}  % disable illegal ligatures
\fi

\author{}
\date{}
\usepackage{multirow}
\usepackage[inline]{enumitem}
\usepackage[margin=1.0in]{geometry}
\usepackage[english]{babel}
\usepackage[utf8]{inputenc}
\usepackage{fancyhdr}

\pagestyle{fancy}
\fancyhf{}
\rhead{\includegraphics[width=5.21667in, height=0.38819in]{image1.png}}
\lhead{ Reasoning: Linear Arrangements }
\lfoot{www.talentsprint.com }
\rfoot{\thepage}
\begin{document}
	
 

\begin{center}
	{\Large \textbf{Linear Arrangements \\}}
\end{center}

{\large \textbf{ Additional Examples \\}}

1. Six boys are sitting in a row. Jose and Manu are sitting adjacent to Raju. Uday has Gopi and
Ram as his \includegraphics[width=0.60555in,height=0.32083in]{image2.png}adjacent. Gopi is not next to either Jose or Manu. Ram is not sitting next to
Manu.\\
Who are/is sitting adjacent to Jose\\
1) Raju and Uday \hspace{2mm}2) Raju and Manu
\hspace{2mm}3) Raju and Ram \hspace{2mm}4) Only Raju\\

2. Twelve friends A, B, C, D, E, F, G, H, I, J, K and L are sitting in a row facing east. J sits third
to the left \includegraphics[width=0.60555in,height=0.32083in]{image2.png}of E, who sits sixth to the right of A. G sits fourth to the right of L, who is not an
immediate neighbor of C. B sits fifth from the right end and H sits second to the left of D. E
is not an immediate neighbor of I or F, but sits second to the right of B. There are two friends
between C and K. K is not an immediate neighbor of J.\\

3. Ten executives A, B, C, D, E, F, G, H, I and J stay in flats in two rows opposite each other.
One row has \includegraphics[width=0.60555in,height=0.32083in]{image2.png}5 flats facing North and the other row has 5 flats facing south. F’s flat is second
to the right of J’s flat, which is exactly opposite of C’s flat facing North. D’s flat is on the
immediate left of J’s flat and A’s flat on the immediate left of C’s flat. B’s flat is on the right
of C’s flat. I and E have flats at the two ends of the same row. Flats of E and H are opposite
each other.\\

I. Which of the following is correct position of G’s flat?\\
a) Facing north \hspace{2mm}b) left of F \hspace{2mm}c) left of J \hspace{2mm}d) opposite to C\\

II. Who stays opposite to B?\\
a) A \hspace{2mm}b) D \hspace{2mm}c) C \hspace{2mm}d) F\\

III. Which of the following statements is definitely true?\\
a) J’s Flat is facing North \hspace{2mm}b) G and A have flats opposite each other.
\hspace{2mm}c) B’s flat is tot the right of E’s flat. \hspace{2mm}D) G’s flat is to the left of J’s flat
\hspace{2mm}e) None of these\\

IV. Which of the following groups of persons have flats in the same row?\\
a) ACG \hspace{2mm}b) GIE \hspace{2mm}c) GJF \hspace{2mm}d) FAJ \hspace{2mm}e) None of these\\

V. Which of the following pairs of executives stay in the flats at the end of a row?\\
a) FD \hspace{2mm}b) HD \hspace{2mm}c) HG \hspace{2mm}d) GD \hspace{2mm}e) None\\

4. \textbf{Directions (I-II):} Study the following information & answer the questions.\\
\includegraphics[width=0.60555in,height=0.32083in]{image2.png}
7 persons S, T, U, V, W, X & Y are seated in a straight line but not necessarily in the same
order, some of them are facing north while some are facing south. Only 3 persons sits
between U & V. Neither U nor V is standing at the any extreme end of the line. X sits 2nd to
the right of U. Y sits 3rd to the right of X. Neither S nor Y is an immediate neighbour of V. T
sits 2nd to the right of S. Both immediate neighbour of X face opposite direction (ie, if one
neighbour faces north other will face south & vice versa) V faces a direction opposite to that
of T. Both persons standing at the extreme ends face the same directions as that of X.\\
I. What is the position of U with respect to T?\\

II. Four of the following five are alike in a certain way based on the given arrangement &
hence they form a group. Which of the following does not belong to the group?\\
a) W \hspace{2mm}b) T \hspace{2mm}c) V \hspace{2mm}d) X \hspace{2mm}e) Y\\

5. Seven different boxes L, M, N, O, P, Q and R of different colours viz., Green, Red, Blue,
Yellow, Purple, \includegraphics[width=0.60555in,height=0.32083in]{image2.png}Pink and Orange are arranged one above the other. The box at the bottom of
arrangement is numbered 1, the above box is numbered 2 and so on. M is immediately
above P. More than two boxes are above the Green box. The Yellow box is immediately
below L. Only one box is between the Green box and Q. R is immediately above the Pink
box. Only one box is between M and the Red box. Only two boxes are between the Red and
the Orange box. Only two boxes are between the Yellow box and the Green box. The blue
box is neither at the top nor at the bottom of the arrangement. M is above Red box. N is
immediately above Q. Neither N nor R is a Yellow box. R is not a Green box.\\

6. \textbf{Directions (I-IV):} Answer the questions on the basis of the information given below.\\
\includegraphics[width=0.60555in,height=0.32083in]{image2.png}
There are 8 children – A, B, C, D, E, F, G and H who live on different floors of a 8-floor
building numbered 1 to 8 not necessarily in the same order. They have different chocolates –
11, 16, 19, 25, 34, 41, 46 and 50 again not necessarily in the same order.\\
The one who lives on 6th floor has 25 chocolates. One child lives between F and the one
having 25 chocolates. G lives below F on an even numbered floor. G does not have 25
chocolates. The one having 46 chocolates lives just above G. Two children live between F
and H. H lives below F. The total of number of chocolates with D and H is a multiple of 4.
Two children live between A and the one having 41 chocolates. A lives above G. The one
having 41 chocolates live above A. B has 34 chocolates. The one having 11 chocolates lives
just above the one having 16 chocolates. 1 child lives between C and E. The difference
between the number of chocolates with E and G is 6\\

I. How many children live above E?\\
A) 1 \hspace{2mm}B) 2 \hspace{2mm}C) 4 \hspace{2mm}D) None \hspace{2mm}E) 5\\

II. Who has 41 chocolates?\\
A) C \hspace{2mm}B) E \hspace{2mm}C) F \hspace{2mm}D) H \hspace{2mm}E) B\\

III. Who lives on third floor?\\
A) A \hspace{2mm}B) B \hspace{2mm}C) C \hspace{2mm}D) D \hspace{2mm}E) E\\

IV. Which of the following is correct with respect to the arrangement?\\
\includegraphics[width=0.60555in,height=0.32083in]{image2.png}
A) 2 children live between B and one having 25 chocolates\\
B) F lives on floor above A\\
C) A lives just below D\\
D) H has maximum chocolates\\
E) None of these is correct\\

V. What is the sum of chocolates with B and D?\\
A) 80 \hspace{2mm}B) 83 \hspace{2mm}C) 76 \hspace{2mm}D) 96 \hspace{2mm}E) 84\\

7. \textbf{Directions:} Answer the questions on the basis of the information given below\\
\includegraphics[width=0.60555in,height=0.32083in]{image2.png}
Eight person P, Q, R, S,T, U, V, & W are sitting in a straight line facing north. their ages are
12, 14, 18, 26, 29, 35, 42, 67. The one who is 12 years old is 4th to the left of the eldest person
who is sitting at the end. The sum of ages of S and Q is equal to P. S is not the youngest
person. S and W are neighbor of Q. R's age is not an even number and he is older than W
and younger than U. only 3 person are sitting between S and U. Only 2 person are sitting
between Q and T, who is 29 year old. P is not the neighbor of U. R is not sitting to the left of
Q.\\

8. Answer the questions on the basis of the information given below\\
\includegraphics[width=0.60555in,height=0.32083in]{image2.png}
Seven friends H, I, J, L, Z, Y and A are born in seven different months of the same year. No
two persons born in a same month. The months are January, March, April, July, September,
October and December. All the information above is not necessarily in the same order A
was born in a month which has 30 days. Only one person born between A and Z. L was
born one of the month after September. Number of person born between L and Z is same as
Y and J. Z is elder than Y. I didn't born in a month which has even number days. H is not
younger than Y. L is not a youngest person in friends. H doesn't born in a month which has
odd number days.\\

9. Answer the questions on the basis of the information given below\\
\includegraphics[width=0.60555in,height=0.32083in]{image2.png}
A, B, C, D, E, F and H are sitting in a straight line but not necessarily in the same order. Two
of them are facing south. C sits second to the right of E. A sits second to the left of F, who
sits third to the left of H. H is facing south. B sits third to the right of C, who is third from
left. E and B face opposite directions.\\

10. \textbf{Directions (I-V):} Study the information carefully and answer the questions given below.\\
\includegraphics[width=0.60555in,height=0.32083in]{image2.png}
Eight persons from different cities, viz. Jaipur, Delhi, Goa, Kolkata, Indore, Jodhpur,
Mumbai and Bangalore, are sitting in two parallel rows containing four persons each, in
such a way that there is an equal distance between adjacent persons. In row 1, T, U, V and
W are sitting and all of them are facing north. In row 2, P, Q, R and S are sitting and all of
them are facing south. Therefore, in the given seating arrangement each member sitting in a
row faces another member of the other row. (All the information given above does not
necessarily represent the order of seating as in the final arrangement.)\\
\begin{itemize}
\item The person from Indore faces the one who is on the immediate left of V. V is neither
from Delhi nor from Goa.
\item An immediate neighbour of P faces the person from Jaipur. The person from Jodhpur
faces the person who is on the left of the person from Delhi.
\item There is only one person sitting between the persons from Delhi and Mumbai but that
person is not T. The persons from Delhi and Goa are not sitting at the extreme ends.
\item Q sits on the immediate left of the person from Indore. Persons from Kolkata and
Jodhpur are immediate neighbours. R and U are not sitting at any of the ends.
\item W faces the one who is sitting on the immediate right of the person from Jodhpur. P is
not from Jodhpur or Kolkata.
\end{itemize}

I. Who amongst the following is from Bangalore?\\
a) P \hspace{2mm}b) V \hspace{2mm}c) Q \hspace{2mm}d) W \hspace{2mm}e) cannot be determined\\

II. Which of the following statements is false regarding W?\\
a) W is from mumbai\\
b) W is sitting at one of the extreme end\\
c) W is on immediate left from person from goa\\
d) W is sitting opposite the person form jodhpur\\
e) All are true\\

III. Four of the following five are alike in a certain way based on the given seating
arrangement and thus form a group. Which is the one that does not belong to that
group?\\
a) R \hspace{2mm}b) T \hspace{2mm}c) P \hspace{2mm}d) U \hspace{2mm}e) W\\

IV. from which of the following City?\\
a) mumbai \hspace{2mm}b) Jaipur \hspace{2mm}c) kolkata \hspace{2mm}d) jodhpur \hspace{2mm}e) none of these\\

V. Who is sitting between S and the person from Indore?\\
a) the person who is from goa. \hspace{2mm}b) Q \hspace{2mm}c) W
\hspace{2mm}d) the person who faces the one who is from goa \hspace{2mm}e) none of these\\

11. Eight person from different cities, viz jaipur, Mumbai, Pune, delhi, Kanpur, Goa, Ranchi and
Indore, are \includegraphics[width=0.60555in,height=0.32083in]{image2.png}sitting in two parallel rows containing four person each in such a way that there
is an equal distance between adjacent persons. In row 1, P, Q, R and S are sitting and all of
them are facing north. In row 2 A,B,C,D are sitting and all of them are facing south. The
person From Kanpur faces the one who sits immediate left of R. r is neither from mumbai
nor from pune. An immediate neighbour of A faces the person who is from jaipur. The
person from Goa faces the person who sits on the left of the person from mumbai. There is
only one person sitting between the person from mumbai and ranchi But That person is not
R. the person lfrom mumbai and pune are not sitting at the extreme ends. B sits on the
immediate left of the person from kanpur. The person from delhi and Goa are immediate
neighbours. C and Q are not sitting at any of the ends. S faces the one who is sitting on the
immediate right of the person from goa. A is not from Goa or delhi. P does not face the one
who is from Goa?\\

12. Ten people are sitting in two parallel rows containing five people each, in such a way that
there is equal \includegraphics[width=0.60555in,height=0.32083in]{image2.png}distance b/w adjacent persons. in row-1 J,K,L,M and N are seated( not
necessarily in the same order) and all of them are facing north. In row-2 V,W,X,Y and Z are
seated ( not necessarily in the same order)and all of them are facing south. Therefore in the
given seating arrangement each member seated in a row faces another member of the other
row.\\
K sits exactly in middle of the row. The one who faces K is an immediate neighbor of Z. V
sits to the immediate left of Z. only one person sits b/w V and X. The one who faces X sits to
immediate right of M. only one person sits b/w M and J. W does not sit at an extreme end of
the line. The one who faces W sits to the immediate left of N?\\

13. \textbf{Directions:} Study the following information carefully and answer the questions given
below:\\
\includegraphics[width=0.60555in,height=0.32083in]{image2.png}
Eight persons Q, R, P, L, Z, H, J and B are sitting in a row facing north. Each one of them
students in different standard from 1 to 8. All the above information is not necessarily in the
same order. Their ages are directly proportional to their standards.
Note: The one who students in the standard 8 is the eldest, while the one who studies in the
standard 1 is the youngest in the group.
Q is elder than P and R. R is younger than P. Three persons sit between Q and the one who
studies in 6th standard. Less than two persons sits between Q and R. Number of persons sits
to the left of P is same as the number of persons sits right of the one who studies in 8th
standard. P does not sit at any of the extreme end. Q sits at one of the extreme end. The one
who studies in 1st standard sits fourth to the left of B. B studies in 3rd standard. Sum of the
standards of L and H will be equal to the standard of J. H sits immediate right of L. Z and J
is not an immediate neighbor of Q. The one who studies in 2nd standard sits third to the
right of R.\\

I. R studies in which of the following standard?\\
\includegraphics[width=0.60555in,height=0.32083in]{image2.png}
a) 5 \hspace{2mm}b) 7 \hspace{2mm}c) 4 \hspace{2mm}d) 1 \hspace{2mm}e) None of these\\
%\includegraphics[width=0.60555in,height=0.32083in]{image2.png}
%\hspace{2mm}

\end{document}