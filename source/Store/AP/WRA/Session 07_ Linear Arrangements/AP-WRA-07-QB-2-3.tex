% Options for packages loaded elsewhere
\PassOptionsToPackage{unicode}{hyperref}
\PassOptionsToPackage{hyphens}{url}
%
\documentclass[
]{article}
\usepackage{amsmath,amssymb}
\usepackage{lmodern}
\usepackage{iftex}
\ifPDFTeX
\usepackage[T1]{fontenc}
\usepackage[utf8]{inputenc}
\usepackage{textcomp} % provide euro and other symbols
\else % if luatex or xetex
\usepackage{unicode-math}
\defaultfontfeatures{Scale=MatchLowercase}
\defaultfontfeatures[\rmfamily]{Ligatures=TeX,Scale=1}
\fi
% Use upquote if available, for straight quotes in verbatim environments
\IfFileExists{upquote.sty}{\usepackage{upquote}}{}
\IfFileExists{microtype.sty}{% use microtype if available
	\usepackage[]{microtype}
	\UseMicrotypeSet[protrusion]{basicmath} % disable protrusion for tt fonts
}{}
\makeatletter
\@ifundefined{KOMAClassName}{% if non-KOMA class
	\IfFileExists{parskip.sty}{%
		\usepackage{parskip}
	}{% else
		\setlength{\parindent}{0pt}
		\setlength{\parskip}{6pt plus 2pt minus 1pt}}
}{% if KOMA class
	\KOMAoptions{parskip=half}}
\makeatother
\usepackage{xcolor}
\IfFileExists{xurl.sty}{\usepackage{xurl}}{} % add URL line breaks if available
\IfFileExists{bookmark.sty}{\usepackage{bookmark}}{\usepackage{hyperref}}
\hypersetup{
	hidelinks,
	pdfcreator={LaTeX via pandoc}}
\urlstyle{same} % disable monospaced font for URLs
\usepackage{longtable,booktabs,array}
\usepackage{calc} % for calculating minipage widths
% Correct order of tables after \paragraph or \subparagraph
\usepackage{etoolbox}
\makeatletter
\patchcmd\longtable{\par}{\if@noskipsec\mbox{}\fi\par}{}{}
\makeatother
% Allow footnotes in longtable head/foot
\IfFileExists{footnotehyper.sty}{\usepackage{footnotehyper}}{\usepackage{footnote}}
\makesavenoteenv{longtable}
\usepackage{graphicx}
\makeatletter
\def\maxwidth{\ifdim\Gin@nat@width>\linewidth\linewidth\else\Gin@nat@width\fi}
\def\maxheight{\ifdim\Gin@nat@height>\textheight\textheight\else\Gin@nat@height\fi}
\makeatother
% Scale images if necessary, so that they will not overflow the page
% margins by default, and it is still possible to overwrite the defaults
% using explicit options in \includegraphics[width, height, ...]{}
\setkeys{Gin}{width=\maxwidth,height=\maxheight,keepaspectratio}
% Set default figure placement to htbp
\makeatletter
\def\fps@figure{htbp}
\makeatother
\setlength{\emergencystretch}{3em} % prevent overfull lines
\providecommand{\tightlist}{%
	\setlength{\itemsep}{0pt}\setlength{\parskip}{0pt}}
\setcounter{secnumdepth}{-\maxdimen} % remove section numbering
\ifLuaTeX
\usepackage{selnolig}  % disable illegal ligatures
\fi

\author{}
\date{}
\usepackage{multirow}
\usepackage[inline]{enumitem}
\usepackage[margin=1.0in]{geometry}
\usepackage[english]{babel}
\usepackage[utf8]{inputenc}
\usepackage{fancyhdr}

\pagestyle{fancy}
\fancyhf{}
\rhead{\includegraphics[width=5.21667in, height=0.38819in]{image1.png}}
\lhead{ Reasoning: linear Arrangements }
\lfoot{www.talentsprint.com }
\rfoot{\thepage}
\begin{document}
	
 

\begin{center}
	{\Large \textbf{Linear Arrangements \\}}
\end{center}

{\large \textbf{ Part 2 - Advanced \\}}
1. Study the following information carefully and complete the arrangement
Six lectures are scheduled in \includegraphics[width=0.60555in,height=0.32083in]{image2.png}a week starting from Monday and ending on Sunday of the
same week. Computer Science is not on Tuesday or Saturday. Psychology is immediately
after Organisational Behaviour. Statistics is not on Friday and there is one day gap between
Statistics and Research Methods. One day prior to the schedule of Economics there is no
lecture (as that day is the off day and Monday is not the ’off’ day).\\

2. \textbf{Directions (I-V):} Study the given information and answer the following questions.\\
\includegraphics[width=0.60555in,height=0.32083in]{image2.png}
There are 12 friends A, B, C, D, E, F, G, H, I, J, K, L in a party sitting in two rows facing each
other such that the person in 1st row faces south and person in 2nd row faces north and vice
versa. Each person likes one of the color among green, red, yellow, blue, brown, black,
orange, violet, white, pink. Further information is as follows:\\
I. Violet color is liked by 3 persons and all the remaining colors are liked by one person
each.\\
II. J does not like violet.\\
III. F likes black and is 3rd to the right of E.\\
IV. B does not like pink or yellow and sits opposite to the person who likes black.\\
V. A likes white and is 3rd from left end in the 2nd row, opposite to G.
VI. E sits between H and A.\\
VII. H is on the left side of A and likes red.\\
VIII. The person who likes pink does not sit at corners.\\
IX. L likes green and sits in the 1st row opposite to the person who likes violet.\\
X. C and E like violet and sit opposite to each other.\\
XI. K likes brown and sits opposite to the person who likes red.\\
XII. I likes orange and sits at one of the corners of 1st row.\\

I. Who is sitting on the immediate right of F?\\
1) D \hspace{2mm}2) I \hspace{2mm}3) J \hspace{2mm}4) B \hspace{2mm}5) None of these\\

II. Who amongst the following is not sitting in the 2nd row?\\
1) A \hspace{2mm}2) E \hspace{2mm}3) J \hspace{2mm}4) B \hspace{2mm}5) F\\

III. Who is sitting opposite to the person who likes black?\\
1) L \hspace{2mm}2) J \hspace{2mm}3) I \hspace{2mm}4) G \hspace{2mm}5) None of these\\

IV. Which of the following statements is false?\\
1) G likes pink \hspace{2mm}2) K is sitting in the 1st row
\hspace{2mm}3) D likes violet \hspace{2mm}4) L is on the immediate right of B
\hspace{2mm}5) J is sitting opposite to the person who likes black\\

V. Who among the following is not sitting at the corner?\\
1) J \hspace{2mm}2) D \hspace{2mm}3) I \hspace{2mm}4) H \hspace{2mm}5) K\\

3. Study the following information carefully and complete the arrangement\\
\includegraphics[width=0.60555in,height=0.32083in]{image2.png}
Eight children a, b, c, d, e, f, g, h are going to enact as eight different planets, Mercury,
Venus, Earth, Mars, Jupiter, Saturn, Uranus and Neptune in a school play. They are
standing in straight column and are facing the north direction. The position are numbered 1-
8 from front end. The teacher has made the following rules for the play.
The one who acts Saturn stands just behind G.
One who acts as Mars is standing 6 places behind the who acts as Earth.
The one who plays the role of Neptune is standing at an even number position and is an
immediate neighbour of A and F.\\
A and B are at a gap of 4 positions.
B acts as Saturn or Uranus.
One who acts as Mercury stands 3 places behind G.
One who acts as Mercury stand behind one who acts as Uranus.
D acts as Venus.
H stands just behind E.
One who acts as Jupiter is not the neighbour of one who acts as Venus.
E does not act as the Saturn.\\

4. Study the following information carefully and complete the arrangement.\\
\includegraphics[width=0.60555in,height=0.32083in]{image2.png}
Ten students namely viz Jaidev, Jaya, Jeeva, Janaki, Jyothi, Jaspal, Jamuna, Jagadish,
Jayanthi and Jasmine of ten different colleges but not necessarily in the same order have
exam on five different days starting from Monday to Friday of the same week. Each student
have exam at two different time slots, i.e 09.00 AM or 11.00 A.M The number of people who
have exam between Jamuna and Janaki is same as the number of people who have exam
between Jeeva and Jagadish. Only two people have exam between Jaspal and Jasmine.
Neither Jyothi nor Jamuna does not have exam on Friday.Jayanthi has exam on Tuesday at
09.00 A.M. Jagadish does not have exam at 11.00 AM. Janaki does not have exam on any one
of the days after Jyothi. Jayanthi does not have exam on any of the days before
Jamuna.Jaspal does not have exam on any of the days after Jagadish. Jaya has exam
immediately before Jayanthi.The one who has exam at 09.00 A.M. immediately before
Jasmine. Janaki has exam immediately after the day of one who has exam on Monday.
Jaspal does not have exam at 11.00 A.M. Only three people have exam between Jamuna and
Jyothi.\\

5. Nilesh, Sneha, Mohan, Sushil, Teena, Aman, Varun and Vikas are sitting in a straight line
but not \includegraphics[width=0.60555in,height=0.32083in]{image2.png}necessarily in the same order. Some of them are facing south while the remaining
are facing north. Sneha and Aman face opposite directions and Aman sits fourth to the right
of Sneha. Varun sits second to the left of Teena. The immediate neighbor of Sushil face same
direction as Sushil. Nilesh sits second to the left of Aman. The Immediate neighbour of
Nilesh faces the same direction as Varun. Vikas is not an immediate neighbour of Varun.
Both are immediate neighbours of Sneha face opposite directions. One of the immediate
neighbour of Vikas faces north. Teena is not third from the right end if we face north.
Mohan is at the fourth position with respect to Nilesh. Vikas is not facing south. Both the
immediate neighbours of Varun face same direction. Varun is third to the right of Sneha.
Mohan is not facing north.\\

6. \textbf{Directions (I-II):} These questions are based on the following data.\\
\includegraphics[width=0.60555in,height=0.32083in]{image2.png}
Pranav is working in Maharashtra state electricity as an assistant engineer. He is going to
the inspection of the towers in those areas. Signals among eight towers- P, Q, R, S, T, U, V
and W are transmitted in the following manner. P, Q, R, S, T, U, V and W are arranged in a
straight line facing North but not necessarily in the same order. Q is placed third to the left
of T. V is placed fifth to the right of Q but neither placed at any of the extreme ends. Two
way transmission is possible between P and Q, Q and S, S and U, R and P, T and R, T and V.
U does not arrange at an extreme end.One way transmission is possible from R to Q, S to T,
V to U. R and S towers are immediately neighbors of each other but neither of them is an
immediate neighbor of tower V. Only one tower is placed between R and P, who is not an
immediate neighbor of T\\

I. For which of the following pairs of towers, signals can be transmitted between them, in
both ways i.e, from the first tower to the second tower and from second tower to the first
such that in each case the signals passes through all other towers exactly one?\\
1) V, P \hspace{2mm}2) P, Q \hspace{2mm}3) V, U \hspace{2mm}4) U, P \hspace{2mm}5) None of these\\

II. How many different routes can the signal be transmitted from S to R and R to S passing
through others tower exactly once?\\
1) 6 \hspace{2mm}2) 7 \hspace{2mm}3) 5 \hspace{2mm}4) 4 \hspace{2mm}5) None of these\\

7. Eight persons – A, B, C, D, E, F, F, G and H are sitting in a straight line facing North (not
necessarily \includegraphics[width=0.60555in,height=0.32083in]{image2.png}in the same order). They have different ages – 12, 18, 27, 32, 34, 49, 55 and 63 (not
necessarily in the same order).
B is sitting second to left of one having age 49 years. Two persons are sitting between B and
D. One who is 32 years old is sitting second to right of D. One person is sitting between the
persons having ages 32 and 18 years. A is sitting second to left of E. A is sitting somewhere
to the left of D. The one who is 63 years old is sitting to immediate left of B. Difference
between the ages of B and G is 7 years. Both are not sitting together. One who is 27 years old
is sitting somewhere left of A. C is 6 years younger to D. The one who is 55 years old and H
are immediate neighbors. Same number of persons are sitting between H and one having
age 34 years and between F and one having age 55 years\\

8. There are seven people A, B, C, D, E, F, and G. They all were born on different years viz.\\
\includegraphics[width=0.60555in,height=0.32083in]{image2.png}
1947, 1952, 1960, 1968, 1982, 1990 and 1997 but not necessarily in same order. But the date
and month of birth of all these persons are same. Calculation is done with respect to the
present year 2017 and assuming months and date to be same. G’s age is in multiple of 5 .
The difference of age between A and D is double the difference of age between B and C. C
was born in even number of year. D is younger than A. G does not the oldest person. The
age of F is exactly double the age of one of them\\

9. \textbf{Directions (I-V):} Study the following information carefully to answer the given question:\\
\includegraphics[width=0.60555in,height=0.32083in]{image2.png}
Eight persons A, B, C, D, E, F, G and H are sitting in a straight line and facing north
direction. They earn different salary, i.e. 7000, 8000, 10000, 12000, 15000, 25000, 30000 and
32000, but not necessarily in the same order. The person who earns the highest sits at one of
the ends. Three persons are sitting between the one who earns 32000 and F. The number of
persons to the left of B is same as to the right of C. B is an immediate neighbor of the one
who earns the highest. E’s salary is sum of the salary of G and D also E sits at the corner. E
doesn’t earn the highest. B earns the least. Both D’s neighbours salaries difference is at least
3000. D earns more than G. A, the one who earns 30000, is 2nd to the left of D. At least two
persons earn less than C.\\
I. Who among the following earn the highest?\\
1) C \hspace{2mm}2) H \hspace{2mm}3) D \hspace{2mm}4) A \hspace{2mm}5) F\\

II. How many persons are sitting to the right of F?\\
1) 0 \hspace{2mm}2) 1 \hspace{2mm}3) 2 \hspace{2mm}4) 3 \hspace{2mm}5) 4\\

III. Who among the following sits to the immediate left of the one who earns the least?\\
1) H 2) The one who earns 30000
\hspace{2mm}3) The one who earns 32000 \hspace{2mm}4) A
\hspace{2mm}5) Both (1) and (3)\\

IV. Which of the following combination is correct?\\
1) G-10000 \hspace{2mm}2) A-25000 \hspace{2mm}3) D-32000 \hspace{2mm}4) C-8000 \hspace{2mm}5) None is true\\

V. Which of the following is true regarding D?\\
1) D earns 12000
\hspace{2mm}2) The one who earns the least is an immediate neighbor of D
\hspace{2mm}3) F and C are neighbours of D
\hspace{2mm}4) D earns less than C
\hspace{2mm}5) None is true\\

10. \textbf{Directions (I-V):} Study the information given below and answer the questions based on it.\\
\includegraphics[width=0.60555in,height=0.32083in]{image2.png}
6 cars M, N, O, P, Q and R are parked in a straight line not necessarily in the same order.
Distance between each car is successive multiple of 3 but not necessarily in the order. The
distance between car N and car O is 36m and no car is parked between them. The distance
between car M and car O is 102m. Car R is 99m to the right of car N. Only one car is parked
between car O and car R. Car M is parked to the immediate left to car P. The distance
between car M and car P is 12m more than the distance between car P and car N. The
distance between R and M is more than 60m. The distance between car P and car Q is 93m.
If car Q moves 20m to the north then takes a left turn and moves 50m then again takes a left
turn and moves for 10m and stops at point Z. Car A is 16m to the west of point Z. Car A
moves 66m towards west and stops at point Y\\
I. In which of the following direction is point Z with respect to car N?\\
A) South-east \hspace{2mm}B) North-east \hspace{2mm}C) North-west
\hspace{2mm}D) South-west \hspace{2mm}E) North\\

II. What is the distance between car P and car R?\\
A) 132m \hspace{2mm}B) 112m \hspace{2mm}C) 126m \hspace{2mm}D) 144m \hspace{2mm}E) 99m\\

III. Point Y is in which of the following direction and distance with respect to car M?\\
A) 10m, South \hspace{2mm}B) 10m, North \hspace{2mm}C) 20m, South
\hspace{2mm}D) 20m, North \hspace{2mm}E) 10m, North west\\

IV. What is the distance between point Z and Point Y?\\
A) 82m \hspace{2mm}B) 88m \hspace{2mm}C) 84m \hspace{2mm}D) 76m \hspace{2mm}E) 66m\\

V. What is the maximum distance between two cars?\\
A) 171m \hspace{2mm}B) 175n \hspace{2mm}C) 165m \hspace{2mm}D) 163m \hspace{2mm}E) 172m\\

11. Eight persons belong to different cities viz— Pune, Haridwar, Kanpur, Jaipur, Patna,
Mumbai, \includegraphics[width=0.60555in,height=0.32083in]{image2.png} Rohtak and Hisar are sitting in two parallel rows containing four people each, in
such way that there is an equal distance between adjacent persons. In row- M, N, O and P
are seated and all of them are facing South. In row-2 A, B, C and D are seated and all of
them are facing North. Therefore in the given arrangement each member seated in a row
faces another member of the other row. (All the information given above does not
necessarily represent the order of seating as in the final arrangement.) B does not sit at any
of the extreme end of the line. The one who belongs to Hisar does not face the one who
belongs to Haridwar. N faces the one who belongs to Patna. The one who faces P sits to the
immediate right of A. The one who belongs to Pune faces the person who belongs to
Mumbai. O does not belong to Mumbai. M does not belong to Jaipur. M does not face the
one who belongs to Hisar. Only one person sits between O and the one who belongs to
Jaipur. One of the immediate neighbour of the one who belongs to Jaipur faces the one who
belongs to Kanpur. C sits second to the left of the one who belongs to Hisar. O is an
immediate neighbour of the one who faces the one who belongs to Hisar.\\

12. Study the following information to answer the given questions.\\
\includegraphics[width=0.60555in,height=0.32083in]{image2.png}
There are twelve seats in two parallel rows having five people each. There is an equal
distance between adjacent persons. In row 1, A, B, C, D and E facing south and in row 2, U,
V, W, X and Y are facing north. One seat is vacant in each row. Therefore, in the given
seating arrangement, each member seated in a row faces another member of the other row.
All of them like different colours like Black, White, Cyan, Indigo, Pink, Beige, Blue, Brown,
Red and Green.
Y likes neither Brown nor Red. W and V are not immediate neighbours. D is immediate
nieghbour of A and C. One of the vacant seats is fourth to the right of E. C does not like
Black colour. There is only one seat between the persons who like Green and Cyan. V either
likes Black or Blue. The one who likes Brown is sitting fourth to the left of U. B and E are not
immediate neighbours. The one who likes Cyan is sitting diagonally opposite to the person
who likes Indigo. There are only two persons sitting between those who like pink and
White. Either A or B likes Pink. D either likes Beige or Brown. The one who likes Red is
sitting opposite A. the person sitting opposite to the one who likes White is also sitting
immediately next to a vacant seat.\\

13. Study the following information to answer the given questions.\\
\includegraphics[width=0.60555in,height=0.32083in]{image2.png}
Ten persons Rajan, Amit, Ravi, Manish, Pankaj, Sudhir, Praveen, Sumit, Gaurav and Asish
are sitting in two parallel rows having five persons in each row in such a way that there is
an equal distance between the adjacent persons. The persons sitting in row 1 are facing
South and the persons sitting in row 2 are facing North. Each of them have a different
hobby, viz, Reading, Net Surfing, Cycling, Acting, Dancing, Cooking, Painting, Shopping,
Swimming and Walking but not necessarily in the same order. Only one person sits between
Amit and the one whose hobby is Swimming. Asish faces the one who sits third to the left of
the one whose hobby is Reading. Walking is not Ravi's hobby. Rajan faces Manish's
neighbor. Only two persons sit between Praveen and the one whose hobby is Reading. Only
two persons sit between Rajan and Sumit. Ravi faces the one whose hobby is Cooking. The
one whose hobby is netsurfing sits second to the left of the one whose hobby is Cooking.
Manish faces South and he is opposite the one who sits third to the left of the one whose
hobby is Dancing. The one whose hobby is Shopping faces Pankaj. The one whose hobby is
painting is sitting second to the immediate right of Gaurav. Amit's neighbor faces Praveen.
The one who faces the person whose hobby is Swimming sits third to the right of the one
whose hobby is Cycling.\\

14. Study the following information to answer the given questions.\\
\includegraphics[width=0.60555in,height=0.32083in]{image2.png}
Eight friends A, B, C, D, E, F, G and H went to a party and has different food items: chips,
chocolate, pasta, pizza, chowmein, fries, burger and cake not necessarily in the same order
They are seated in a linear fashion with four of them facing South and the other four facing
North The arrangement is based on the following rules:
One who had the burger and D are at the two ends of line They face the opposite
directions
There is a gap of 3 people between the one who had chocolate and the one who had pasta
F had cake and is fourth to the right of D Neighbours of F face the same direction
One who had chips is a neighbour of both C and the one who had burger
B is second to the left of the one who had chocolate
B did not have a burger
H and G are neighbours and they face the same direction
G neither had a chocolate nor chips
One who had fries is 2nd to the right of the one who had chowmein
One who had pizza is 2nd to the left of G
E faces south and did not have a burger
F and H face the same direction\\

15. Study the following information carefully to answer the given questions\\
\includegraphics[width=0.60555in,height=0.32083in]{image2.png}
Ten persons M, N, O, P, Q, R, S, T, U and V are seated in a row but not necessarily in the
same order. Some of them are facing North while some of them are facing South. Each of
them likes different colour viz— Red, Green, Yellow, Blue, Black, White, Pink, Brown,
Purple and Orange but not necessarily in the same order.
Only one person sits between the one who likes White colour and the one who likes Pink
colour. N sits third to the right of O, who likes White colour. Only one person sits between
O and the one who likes Brown colour. T sits third to the left of O and he likes Red colour. P
and U do not face same directions. T and U face same direction. V sits immediate left of T. N
and Q face same direction. V faces opposite directions to N. V and T face same direction. S is
not an immediate neighbour of O. The one who likes Yellow colour faces South. M sits
fourth to the right of the one who likes Brown colour. M is not an immediate neighbour of
the one who likes Black colour. The one who likes Green colour sits fourth to the left of M
and he is an immediate neighbour of P. N likes Black colour and faces North. N sits one of
the end of the row. R sits exactly in the middle of O and S. The one who likes Pink colour
sits second to the right of the one who likes Blue colour. U, who likes Orange colour sits
fourth to the left of P.\\

16. \textbf{Directions (I-V):} Answer the questions on the basis of the information given below.\\
\includegraphics[width=0.60555in,height=0.32083in]{image2.png}
There are 8 children – A, B, C, D, E, F, G and H who live on different floors of a 8-floor
building numbered 1 to 8 not necessarily in the same order. They are in different class – 3, 5,
6 and 10 such that 2 children in same class. Children who are in same class live on even-odd
floors. Example: If B is in class 10 with H, then if B lives on 6th floor then H lives on any odd
floors – 1/3/5/7 and not 2/4/8. B and G are in same class. One of the children in class 3 lives
on 5th floor. The one who lives on 3rd floor is in even numbered class. The one who lives on
7th floor and C are in same class. A is in class 6 and lives on 4th floor. 2 children live
between E and A. D is in 10th class. One child lives between E and G, both of which are in
odd numbered classes. 2 children live between one of the children in class 10 and F. F lives
below this child. Both children in class 10 live above C.\\

I. Which of the following pair is in same class?\\
A) B, F \hspace{2mm}B) D, F \hspace{2mm}C) E, H \hspace{2mm}D) A, D \hspace{2mm}E) None of these\\

II. Who lives on 8th floor?\\
A) E \hspace{2mm}B) One of the children in class 6
\hspace{2mm}C) One of the children in class 10 \hspace{2mm}D) H
\hspace{2mm}E) One of the children in class 3\\

III. How many children live between D and F?\\
A) 2 \hspace{2mm}B) 1 \hspace{2mm}C) None \hspace{2mm}D) 4 \hspace{2mm}E) Cannot be determined\\

IV. Who lives just above G?\\
A) C \hspace{2mm}B) A \hspace{2mm}C) D \hspace{2mm}D) H \hspace{2mm}E) B\\

V. Which of the following combination of floor number – child – class is correct?\\
A) 1 – C – 5 \hspace{2mm}B) 6 – B – 10 \hspace{2mm}C) 5 – D – 3
\hspace{2mm}D) 3 – F – 10 \hspace{2mm}E) 2 – C – 6\\

17. \textbf{Directions:} Study the following information to answer the given questions.\\
\includegraphics[width=0.60555in,height=0.32083in]{image2.png}
Eight persons BARRY ALLEN, KILLER FROST, FIRESTORM, BLACK CANARY, CAPTION
COLD, CISCO ROMAN, IRIS WEST and JOE WEST stay on an eight storey building such
that the lower most floor is numbered as 1 and so on. They own different company wrist
watch namely Sonata, Titan, Fastrack, Timex, Maxima, Rolax, Casio and Citizen not
necessarily in the same order.\\
\begin{itemize}
\item Sum of BARRY ALLEN's and CAPTION COLD's floor number is equal to CISCO
ROMAN's floor number
\item Persons liking Casio and Maxima stay at a gap of 4 floors.
\item FIRESTORM likes Casio or Citizen
\item JOE WEST does not like Maxima.
\item There is a gap of 3 floors between the persons who like Rolax and Titan
\item Not more than 2 persons stay below JOE WEST's floor
\item Persons liking Casio and Sonata stay at a gap of 1 floor.
\item The one who likes Timex stays below JOE WEST's floor.
\item CAPTION COLD stays at a gap of 1 floor from BLACK CANARY
\item There is a gap of 4 floors between the one who likes Timex and FIRESTORM
\item FIRESTORM and BLACK CANARY stay on consecutive floors
\item The one who likes Rolax stays above CISCO ROMAN.
\item KILLER FROST and the one who likes Casio stay on consecutive floors.\\
\end{itemize}

18. \textbf{Directions (I-V):} Study the following information carefully to answer the given questions:\\
\includegraphics[width=0.60555in,height=0.32083in]{image2.png}
Eight houses A1, A2, A3, A4, A5, A6, A7 and A8 are situated across a street not necessarily in
the same order. The houses are facing north. They have different house numbers viz - B-101,
B-102, B-103, B-104, B-105, B-106, B-107 and B-108 but not necessarily in the same order. The
arrangement is based on the following rules.\\
\begin{itemize}
\item A8 is second to the right of the house no. B-102.
\item There are five house between the house no B-103 and house no. B-108.
\item House no. B-102 is to the immediate left of the house no. B-103.

\item There are two houses between A7 and the house no. B-101. A7 is not house no. B-101, B-
102, or B-103.

\item House no B-107 and B-105 are immediate neighbour of A7.
\item A4 is third to the left of the house no. B-107.
\item A3 has the house no. B-104 to its immediate left.
\item There are two houses between the house no. B-108 and A1.
\item A5 is fourth to the right of A6.
\end{itemize}

I. Which of the following houses are situated in a consecutive manner?\\
a) A2, A4, A6 \hspace{2mm}b) A8, A6, A7 \hspace{2mm}c) A1, A7, A5 \hspace{2mm}d) A6, A1, A7 \hspace{2mm}e) A3, A5, A4\\

II. If A2 is related to house no. B-103 and A4 is related to house no. B-101 then which of the
following house no. is related to A1?\\
a) House no. B-102 \hspace{2mm}b) House no. B-106 \hspace{2mm}c) House no. B-107
\hspace{2mm}d) House no. B-104 \hspace{2mm}e) House no. B-105\\

III. Which of the two houses are exactly in the middle of the street?\\
a) A6, A7 \hspace{2mm}b) A1, A7 \hspace{2mm}c) A6, A1 \hspace{2mm}d) A8, A6 \hspace{2mm}e) A1, A3\\

IV. Which of the following is the immediate neighbour of the house no. B-104 and house no.
B-108?\\
a) A2 \hspace{2mm}b) A3 \hspace{2mm}c) A1 \hspace{2mm}d) A6 \hspace{2mm}e) A4\\

V. Which of the following is the correct combination of house: house no.?\\
a) A2 : B-106 \hspace{2mm}b) A8 : B-105 \hspace{2mm}c) A1 : B-101
\hspace{2mm}d) A3 : B-107 \hspace{2mm}e) A6 : B-106\\

19. Eight friends G, H, I, J, K, L, A and B live in a nine-floored apartment where the floors are
 \includegraphics[width=0.60555in,height=0.32083in]{image2.png}numbered 1 to 9 from ground to top. Each of them plays a different game from among
cricket, badminton, tennis, golf, shooting, kho-kho, football and rugby. Each floor has one
flat and one of the flats in the building is vacant. The person who plays badminton is the
only person who lives between A and G. Only one person lives between H and G. B plays
kho-kho. L lives immediately below the vacant flat. One of the two people who live between
A and B plays golf. H lives on the floor immediately below B. The person who lives
immediately above the vacant flat plays football. J lives on an odd numbered floor
immediately below L. The person who plays tennis stays on an even numbered floor and
the person who plays shooting lives immediately above J. A lives on an even numbered
floor immediately above I. The person who plays cricket lives on an odd numbered floor,
above the fourth floor.\\

20. \textbf{Directions:} Study the following information carefully and answer the questions given
below:\\
\includegraphics[width=0.60555in,height=0.32083in]{image2.png}
Ten persons viz, E, F, K, L, N, P, R, T, V and W are sitting in two rows with five persons in
each row. The persons in row one (R – 1) are facing north and the persons in row two (R -2)
are facing south. Each persons in row one faces a person from the other row. All of them like
a different books viz, Flood of fire, Family life, The marvels, Playing it my way,
Hangwoman, The lives of others, Carrer of Evil, Odysseus Abroad, Ancestral Affairs and
Seahorse, but not necessarily in the same order.
The people who like Hangwoman and The lives of others sits opposite each other. P sits
opposite E, who likes ‘Flood of fire’. W is not facing south but sits third to the left of R, who
likes’ Family life’. The one who likes ‘Family life’ sits opposite the one who likes ‘Odysseus
Abroad. There is only one person between F and K. N sits at one of the extreme ends of the
row and likes ‘The lives of others’. The one who likes ‘Odysseus Abroad’ is on the
immediate right of L, who does not like ‘Seahorse’ but faces the one who likes Seahorse.
The persons who like ‘The Marvels’ and ‘Playing it my way’ respectively are not facing
south. K likes ‘Carrer of Evil’. The one who likes ‘The Marvels’ sits opposite the one who is
on the immediate right of F. V does not like ‘Seahorse’. N sits opposite the one who sits
second to the left of the one who likes ‘The Marvels’.\\

i. Who among the following likes ‘Seahorse’?\\
1) P \hspace{2mm}2) R \hspace{2mm}3) F \hspace{2mm}4) W \hspace{2mm}5) T\\

ii. F likes which of the following books?\\
1) V – Family life \hspace{2mm}2) The Marvels \hspace{2mm}3) Seahorse
\hspace{2mm}4) Odysseus Abroad \hspace{2mm}5) Carrer of Evil\\

iii. Which of the following is matched correctly?\\
1) V –Playing it my way \hspace{2mm}2) P – Family life \hspace{2mm}3) T – Ancerstral
Affairs
\hspace{2mm}4) K – Flood of fire \hspace{2mm}5) N – Hang woman\\

iv. Four of the following five are alike in some way, find out the odd one?\\
1) K \hspace{2mm}2) F \hspace{2mm}3) T \hspace{2mm}4) R \hspace{2mm}5) V\\

21. \textbf{Direction:}
There are nine people namely, A,B,C,D,E,F,G,H and I who sit in a row facing in
North \includegraphics[width=0.60555in,height=0.32083in]{image2.png} and South directions. They like different colors among Orange, Red, Pink and Blue.
At least two people like the same color and not more than three people like the same color.
All the people who like Pink have one of the neighbours as the one who likes Red. A sits
second to the right of F and they both face in the same direction. F and G face in the same
direction but opposite to that of H. E’s immediate neighbours are I and C. The two people
who like Orange is the only pair sitting together liking the same color and none of them is F.
H sits third to the left of the one who likes Pink. D sits somewhere between B, who is at one
of the ends, and A. C sits at 3rd position from right end and likes Blue which is same as that
of the person sitting at extreme right. H and G sit together and none of them likes Pink. D
faces in a direction opposite to that of A. B and E like the same color, are the only ones who
like this color, and also face in opposite directions. C’s neighbours face in the same direction
but opposite to that of C. The one who likes Pink sits at the middle facing in the South
direction.\\

22. \textbf{Directions (i - v):} A certain number of people sitting in the linear row facing north direction,
only 3 \includegraphics[width=0.60555in,height=0.32083in]{image2.png}people sit between A \& R. Only 4 people sit between K \& W. Only 5 people sit
between R \& K. T sits 3rd to the right of W. Only 6 people sit between R \& Y. Not more than
3 people sit between K \& Y. More than 4 people sit between T \& Y. Q sits 3rd to the right of
Y. No one sits between Q \& W. J sits eighth to the left of K. Not more than 3 persons sit
between A \& J.\\

I. How many people are sitting in the linear row?\\
a) 19 \hspace{2mm}b) 20 \hspace{2mm}c) 21 \hspace{2mm}d) 22 \hspace{2mm}e) none of these\\

II. How many people sit betweeen A and J?\\
a) 7 \hspace{2mm}b) 3 \hspace{2mm}c) 10 \hspace{2mm}d) 1 \hspace{2mm}e) none of these\\

III. If three people sit between W \& H, then which of the following statement is
definitely true?\\
a) 3 people sit between T and H. \hspace{2mm}b) More than 6 people sit between Q \& H.
\hspace{2mm}c) W sits 4th to the right of H. d) More than 5 people sit between Y and H.
\hspace{2mm}e)None of these\\

IV. How many people sit between Y and W?\\
a) 16 \hspace{2mm}b) 3 \hspace{2mm}c) 10 \hspace{2mm}d) 8 \hspace{2mm}e) none of these\\

V. How many people sit to the left of K?\\
a) 10 \hspace{2mm}b) 8 \hspace{2mm}c) 16 \hspace{2mm}d) 13 \hspace{2mm}e) none of these\\

23. There are some persons who are sitting in a row and facing North. The distance between
each of them \includegraphics[width=0.60555in,height=0.32083in]{image2.png}is equal. The number of persons who sit on the left side of E is one less than the
number of persons who sit on the right side of the E. There are more than two persons
between Q and T who sits to the left of W. Q sits on the left side of the X. F does not sit on
the extreme ends. There are four persons between P and V. There are two persons between
R and W. Neither N nor M is the neighbour of X. There are four persons between W and Z
who do not sit to the left of E. There are only two persons who sit to the left of X. There is
one person between E and P. E is not the immediate neighbour of X. P is the immediate
neighbour of X.\\

I. How many seats are there between W and T?\\
1) One \hspace{2mm}2) Two \hspace{2mm}3) Three \hspace{2mm}4) None \hspace{2mm}5) Can’t say\\

II. If the distance between V and E is 12 meters, find out the distance between F and W?\\
1) 16 meters \hspace{2mm}2) 18 meters \hspace{2mm}3) 20 meters \hspace{2mm}4) 15 meters \hspace{2mm}5) None of these\\

III. How many persons are there in that row?\\
1) Ten \hspace{2mm}2) Twelve \hspace{2mm}3) Fourteen \hspace{2mm}4) More than Fourteen \hspace{2mm}5) Thirteen\\

IV. Who among the following sits third to the right of T?\\
1) N \hspace{2mm}2) M \hspace{2mm}3) F \hspace{2mm}4) Either 1) or 2) \hspace{2mm}5) None of these\\

24. Directions(I-IV): Study the following information carefully to answer the given questions.\\
\includegraphics[width=0.60555in,height=0.32083in]{image2.png}
Complex Assignment or Linear Arrangement.\\
\begin{itemize}
\item Ten Stationery items Pencil, Books, Notes, Pens, Sharpeners, Chalk, Stickers, Gum, Scale,
Covers are placed in 10 boxes numbered 1, 2, 3, 4, 5, 6, 7, 8, 9 and 10 are placed adjacent to
one another in two different rows of 5 boxes each.\\
\item Boxes with odd numbers are situated opposite to the boxes with even numbers and no
two boxes with even number and no two boxes with odd numbers are adjacent to each
other. No two boxes have consecutive numbers.\\
\item Box no 8 is occupied by books and it is to the extreme left of one end of the row.
\item Box 4 and 6 are not in the row of box no 8 and any of the boxes numbered 4, 6, and 3 are
not in the middle of the row.
\item Gums are placed in Box no 5.and that box is not situated in the row where box no 6 is
situated.
\item Notes are placed in an odd numbered box and placed in a row where Books are situated
but both are not adjacent to each other.\\
\item Sharpener's and Chalk’s box are adjacent to box no 6 and Chalk’s box is not adjacent to
box no 4.\\
\item Pencil’s \& Cover’s box are placed in the same row. Pen’s box is even numbered but not 10 and diagonally opposite to box no 1 which contains stickers.\\
\item Box no 6 is second from the one of the end of the row. Stickers’ box is neither opposite to
Sharpness's box nor adjacent to it.\\
\end{itemize}

I. Covers are kept in which number box?\\
a) 4 \hspace{2mm}b) 6 \hspace{2mm}c) 2 \hspace{2mm}d) 3 \hspace{2mm}e) Can’t be determined\\

II. The number 9 box is placed opposite to which box and contains which stationery?\\
a) 5, Chalk \hspace{2mm}b) 6, Covers \hspace{2mm}c) 2, Pens \hspace{2mm}d) 10, scale \hspace{2mm}e) None of these\\

III. If pencil is placed in the box numbered 4 then which box is placed opposite to the box which
contains Covers?\hspace{2mm}
a) Pencils \hspace{2mm}b) Covers \hspace{2mm}c) Stickers \hspace{2mm}d) Note \hspace{2mm}e) None of these\\

IV. which of the following is exactly opposite of the box which contains Pen?\\
a) Scale \hspace{2mm}b) Books \hspace{2mm}\hspace{2mm}c) chalk \hspace{2mm}d) Notes \hspace{2mm}e) None of these\\

25. 8 persons A, C, D, K, P, Q, R and S are sitting in a straight line facing north direction. The
ages (in \includegraphics[width=0.60555in,height=0.32083in]{image2.png}years) of these persons are 20, 25, 28, 30, 34 and 42 and few of them having the same
age. Distance between each of the adjacent persons is equal. No two adjacent persons are of
same age.\\
\begin{itemize}
\item A sits second to the left of one of the persons whose age is a multiple of 5.\\
\item Ages of A and R are equal. The difference between the ages of S and R is at most 13
years.\\
\item P sits third to the right of A and P is the only one who is 25 years old.\\
\item At least 3 persons sit between D and K and difference in their ages is 12 years.\\
\item R sits to the immediate right of the person, who is 34 years old.\\
\item Neither Q nor S is 34 years old. Four persons sit between P and D.\\
\item At least 3 persons sit between C and Q.\\
\item Sum of the ages of D and Q is 70 years. K doesn’t sit to the right of P. R sits to the
right of K\\
\end{itemize}

Answers\\
1. -\\
2. I. J \hspace{2mm}II. B \hspace{2mm}III. None of these
\hspace{2mm}IV. J is sitting opposite to the person who likes black \hspace{2mm}V. D\\
3. -\\
4. -\\
5. -\\
6. I. None of these \hspace{2mm}II. 5\\
7. –\\
8. –\\
9. I. H \hspace{2mm}II. 3 \hspace{2mm}III. Both (1) and (3) \hspace{2mm}IV. G-10000
\hspace{2mm}V. F and C are neighbours of D\\
10. I. North-east \hspace{2mm}II. 126m \hspace{2mm}III. 10m, North \hspace{2mm}IV. 82m \hspace{2mm}V. 165m\\
11. –\\
12. –\\
13. –\\
14. –\\
15. -\\
16. I. D, F \hspace{2mm}II. One of the children in class 3 \hspace{2mm}III. 2 \hspace{2mm}IV. D \hspace{2mm}V. 3 – F – 10\\
17. -\\
18. I. A6, A1, A7 \hspace{2mm}II. House no. B-104 \hspace{2mm}III. A6, A1 \hspace{2mm}IV. A3 \hspace{2mm}V. A6 : B-106\\
19. –\\
20. i. 5 \hspace{2mm}ii. 4 \hspace{2mm}iii. 1 \hspace{2mm}iv. 3\\
21. –\\
22. –\\
23. I) 33m \hspace{2mm}II) 88m \hspace{2mm}III) C \hspace{2mm}IV) B,D\\
24. I) Can’t be determined \hspace{2mm}II) 10, Scale \hspace{2mm}III) Note \hspace{2mm}IV) Chalk\\
25. --\\
%\includegraphics[width=0.60555in,height=0.32083in]{image2.png}
%\hspace{2mm}

\end{document}