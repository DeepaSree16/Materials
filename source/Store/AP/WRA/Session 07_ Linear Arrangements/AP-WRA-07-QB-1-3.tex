
% Options for packages loaded elsewhere
\PassOptionsToPackage{unicode}{hyperref}
\PassOptionsToPackage{hyphens}{url}
%
\documentclass[
]{article}
\usepackage{amsmath,amssymb}
\usepackage{lmodern}
\usepackage{iftex}
\ifPDFTeX
\usepackage[T1]{fontenc}
\usepackage[utf8]{inputenc}
\usepackage{textcomp} % provide euro and other symbols
\else % if luatex or xetex
\usepackage{unicode-math}
\defaultfontfeatures{Scale=MatchLowercase}
\defaultfontfeatures[\rmfamily]{Ligatures=TeX,Scale=1}
\fi
% Use upquote if available, for straight quotes in verbatim environments
\IfFileExists{upquote.sty}{\usepackage{upquote}}{}
\IfFileExists{microtype.sty}{% use microtype if available
	\usepackage[]{microtype}
	\UseMicrotypeSet[protrusion]{basicmath} % disable protrusion for tt fonts
}{}
\makeatletter
\@ifundefined{KOMAClassName}{% if non-KOMA class
	\IfFileExists{parskip.sty}{%
		\usepackage{parskip}
	}{% else
		\setlength{\parindent}{0pt}
		\setlength{\parskip}{6pt plus 2pt minus 1pt}}
}{% if KOMA class
	\KOMAoptions{parskip=half}}
\makeatother
\usepackage{xcolor}
\IfFileExists{xurl.sty}{\usepackage{xurl}}{} % add URL line breaks if available
\IfFileExists{bookmark.sty}{\usepackage{bookmark}}{\usepackage{hyperref}}
\hypersetup{
	hidelinks,
	pdfcreator={LaTeX via pandoc}}
\urlstyle{same} % disable monospaced font for URLs
\usepackage{longtable,booktabs,array}
\usepackage{calc} % for calculating minipage widths
% Correct order of tables after \paragraph or \subparagraph
\usepackage{etoolbox}
\makeatletter
\patchcmd\longtable{\par}{\if@noskipsec\mbox{}\fi\par}{}{}
\makeatother
% Allow footnotes in longtable head/foot
\IfFileExists{footnotehyper.sty}{\usepackage{footnotehyper}}{\usepackage{footnote}}
\makesavenoteenv{longtable}
\usepackage{graphicx}
\makeatletter
\def\maxwidth{\ifdim\Gin@nat@width>\linewidth\linewidth\else\Gin@nat@width\fi}
\def\maxheight{\ifdim\Gin@nat@height>\textheight\textheight\else\Gin@nat@height\fi}
\makeatother
% Scale images if necessary, so that they will not overflow the page
% margins by default, and it is still possible to overwrite the defaults
% using explicit options in \includegraphics[width, height, ...]{}
\setkeys{Gin}{width=\maxwidth,height=\maxheight,keepaspectratio}
% Set default figure placement to htbp
\makeatletter
\def\fps@figure{htbp}
\makeatother
\setlength{\emergencystretch}{3em} % prevent overfull lines
\providecommand{\tightlist}{%
	\setlength{\itemsep}{0pt}\setlength{\parskip}{0pt}}
\setcounter{secnumdepth}{-\maxdimen} % remove section numbering
\ifLuaTeX
\usepackage{selnolig}  % disable illegal ligatures
\fi

\author{}
\date{}
\usepackage{multirow}
\usepackage[inline]{enumitem}
\usepackage[margin=1.0in]{geometry}
\usepackage[english]{babel}
\usepackage[utf8]{inputenc}
\usepackage{fancyhdr}

\pagestyle{fancy}
\fancyhf{}
\rhead{\includegraphics[width=5.21667in, height=0.38819in]{image1.png}}
\lhead{ Reasoning: Linear Arrangements }
\lfoot{www.talentsprint.com }
\rfoot{\thepage}
\begin{document}
	
 

\begin{center}
	{\Large \textbf{Linear Arrangements \\}}
\end{center}

{\large \textbf{ Part 1 - Basic \\}}

\textbf{Model 1: Simple Arrangement}

\textbf{Directions (1-5):} Study the following information to answer the given questions:\\
\includegraphics[width=0.60555in,height=0.32083in]{image2.png}
i. Five topics, A, B, C, D, E are to be discussed one topic on each day, from Monday to
Saturday.\\
ii. Topic A will be discussed before E and topic B will be discussed before D.\\
iii. Topics B and D will not be discussed on the first day.\\
iv. There will be one rest day denoted by F.\\
v. There will be a gap of two days between the days on which topics D and B will be
discussed.\\
vi. Topic C will be discussed immediately before the rest day. The rest day will not be the
second or the fourth day.\\

1. Which of the following is the correct sequence of the discussion on the topics including the
rest day ‘F’?\\
\includegraphics[width=0.60555in,height=0.32083in]{image2.png}
1) AEBFCD \hspace{2mm}2) ABECFD \hspace{2mm}3) AEBCFD
\hspace{2mm}4) Cannot be determined \hspace{2mm}5) None of these\\

2. Which of the following is a correct statement?\\
\includegraphics[width=0.60555in,height=0.32083in]{image2.png}
1) Topic A will be discussed on Tuesday.\\
2) Discussion on topic C will be immediately preceded by discussion on topic B.\\
3) discussion on topic B will take place before that on A\\
4) Thursday is the rest day.\\
5) None of these\\

3. On which of the following days will the topic C be discussed?\\
\includegraphics[width=0.60555in,height=0.32083in]{image2.png}
1) Tuesday \hspace{2mm}2) Wednesday \hspace{2mm}3) Friday
\hspace{2mm}4) Cannot be determined \hspace{2mm}5) None of these\\

4. How many days gap will be there between the days on which topic E and B will be
discussed?\\
\includegraphics[width=0.60555in,height=0.32083in]{image2.png}
1) Nil \hspace{2mm}2) One \hspace{2mm}3) Two \hspace{2mm}4) Three \hspace{2mm}5) None of these\\

5. With reference to A, the discussion on topic E will take place \_\_\_\_\_.\\
\includegraphics[width=0.60555in,height=0.32083in]{image2.png}
1) immediately on the next day \hspace{2mm}2) after a day’s gap \hspace{2mm}3) after three days
\hspace{2mm}4) Cannot be determined \hspace{2mm}5) None of these\\

\textbf{Directions (6-10):} Study the following information to answer the given questions:\\
\includegraphics[width=0.60555in,height=0.32083in]{image2.png}
Ten
executives A,B,C,D,E,F,G,H,I and J stay in flats in two rows opposite each other. One row has 5
flats facing North and the other row has 5 flats facing South. F’s flat is second to the right of J’s
flat, which is exactly opposite of C’s flat facing North. D’s flat is on the immediate left of J’s flat
and A’s flat on the immediate left of C’s flat. B’s flat is on the right of C’s flat. I and E have flats
at the two ends of the same row. Flats of E and H are opposite each other.
[June 28, 2016 @ 1h 50m 15s]\\

6. Which of the following is the correct position of G’s flat?\\
\includegraphics[width=0.60555in,height=0.32083in]{image2.png}
1) Facing North \hspace{2mm}2) Opposite C’s flat \hspace{2mm}3) To the left of F’s flat
\hspace{2mm}4) To the left of J’s flat \hspace{2mm}5) None of these\\

7. Who stays opposite B?\\
\includegraphics[width=0.60555in,height=0.32083in]{image2.png}
1) F \hspace{2mm}2) G \hspace{2mm}3) D \hspace{2mm}4) F or G \hspace{2mm}5) None of these\\

8. Which of the following statements is definitely true?\\
\includegraphics[width=0.60555in,height=0.32083in]{image2.png}
1) J’s flat is facing North \hspace{2mm}2) G and A have flats opposite each other.
\hspace{2mm}3) B’s flat is to the right of E’s flat \hspace{2mm}4) G’s flat is to the left of J’s flat
\hspace{2mm}5) None of these\\

9. Which of the following groups of persons have flats in the same row?\\
\includegraphics[width=0.60555in,height=0.32083in]{image2.png}
1) ACG \hspace{2mm}2) GIE \hspace{2mm}3) GJF \hspace{2mm}4) FAJ \hspace{2mm}5) None of these\\

10. Which of the following pairs of executives stay in the flats at the ends of a row?\\
\includegraphics[width=0.60555in,height=0.32083in]{image2.png}
1) FD \hspace{2mm}2) HD \hspace{2mm}3) HG \hspace{2mm}4) GD \hspace{2mm}5) None of these\\

\textbf{Model 2:} Complex Arrangement\\

\textbf{Directions (11-17):} Study the following information to answer the given questions:\\
\includegraphics[width=0.60555in,height=0.32083in]{image2.png}
P, Q, R, S, T, V and W are sitting in a straight line facing north. Each one of them lives on a
different floor in the same building which is numbered from one to seven.\\
Q sits fourth to the left of the person living on the 6th floor. Either Q or the person living on the
6
th floor sits at the extreme ends of the line.
Only one person sits between Q and W. W lives on the 3rd floor. The person living on the 1st
floor sits third to right of S. S is not an immediate neighbour of W. Only one person lies between
T and person who live on the 2nd floor. P and R are immediate neighbours of each other. P does
not live on the 6th floor. One who lives on the 5th floor sits third to right of the one who lives on
the7th floor.\\

11. Who amongst the following lies on the 4th floor?\\
\includegraphics[width=0.60555in,height=0.32083in]{image2.png}
1) P \hspace{2mm}2) Q \hspace{2mm}3) R \hspace{2mm}4) S \hspace{2mm}5) V\\

12. On which of the following floors does T live?\\
\includegraphics[width=0.60555in,height=0.32083in]{image2.png}
1) 1st \hspace{2mm}2) 2nd \hspace{2mm}3) 5th \hspace{2mm}4) 6th \hspace{2mm}5) 7th\\

13. How many floors are there between the floors on which V and P live?\\
\includegraphics[width=0.60555in,height=0.32083in]{image2.png}
1) One \hspace{2mm}2) Two \hspace{2mm}3) Three \hspace{2mm}4) Four \hspace{2mm}5) None\\

\textbf{Directions (14-15):} Four of the following five are alike in a certain way based on the given
arrangement \includegraphics[width=0.60555in,height=0.32083in]{image2.png}and thus form a group. Which is the one that does not belong to that group?\\

14. 1) W \hspace{2mm}2) T \hspace{2mm}3) S \hspace{2mm}4) P \hspace{2mm}5) Q\\

15. 1) T- 2nd floor \hspace{2mm}2) R-7th floor \hspace{2mm}3) V-3rd floor \hspace{2mm}4) S- 5th floor \hspace{2mm}5) Q-6th floor\\

16. If all the persons are made to live in alphabetical order from the bottom-most floor to the
topmost floor \includegraphics[width=0.60555in,height=0.32083in]{image2.png}(i.e. P lives on the 1st floor, Q lives on the 2nd floor and finally W lives on the 7th
floor), who would still live on the same floor as the original arrangement?\\
1) R \hspace{2mm}2) V \hspace{2mm}3) W \hspace{2mm}4) T \hspace{2mm}5) S\\

17. Which of the following is true with respect to the given arrangement?\\
\includegraphics[width=0.60555in,height=0.32083in]{image2.png}
1) The one who lives on the 5th floor is an immediate neighbour of S.\\
2) V lives on the 1st floor.\\
3) T sits second to the left of the person who lives on the 2nd floor.\\
4) R and V are immediate neighbours of each other.\\
5) The one who lives on the 4th floor sits at one of the extreme ends of the line.\\

\textbf{Directions (18-25):} Study the following information to answer the given questions:\\
\includegraphics[width=0.60555in,height=0.32083in]{image2.png}
The annual gathering of a school was organized on a day in the morning hours. Six difference
items, viz. drama, singing, mimicry, speech, story-telling and dance, are to be performed by six
children A, B, C, D, E and F not necessarily in the same order. The program begins with song
not sung by B and ends with dance. C performs mimicry immediately after speech. E performs
drama just before dance. D or F is not available for the last performance. Speech is not given by
A. An interval of 30 minutes is given immediately after mimicry with three more items
remaining to be performed. D performs immediately after interval.
18. Which item is performed by F?\\
1) Drama \hspace{2mm}2) Song \hspace{2mm}3) Speech \hspace{2mm}4) Story-telling \hspace{2mm}5) None of these\\

19. Who performed dance?\\
\includegraphics[width=0.60555in,height=0.32083in]{image2.png}
1) A \hspace{2mm}2) B \hspace{2mm}3) F
\hspace{2mm}4) Data inadequate \hspace{2mm}5) None of these\\

20. Who was the first performer?\\
\includegraphics[width=0.60555in,height=0.32083in]{image2.png}
1) A \hspace{2mm}2) B \hspace{2mm}3) C
\hspace{2mm}4) Data inadequate \hspace{2mm}5) None of these\\

21. Which item is performed by D?\\
\includegraphics[width=0.60555in,height=0.32083in]{image2.png}
1) Dance \hspace{2mm}2) Story-telling \hspace{2mm}3) Singing \hspace{2mm}4) Drama \hspace{2mm}5) None of these\\

22. Who was the fifth performer?\\
\includegraphics[width=0.60555in,height=0.32083in]{image2.png}
1) E \hspace{2mm}2) A \hspace{2mm}3) C \hspace{2mm}4) B \hspace{2mm}5) None of these\\

23. Who gave the speech?\\
\includegraphics[width=0.60555in,height=0.32083in]{image2.png}
1) F \hspace{2mm}2) B \hspace{2mm}3) B or F \hspace{2mm}4) D \hspace{2mm}5) None of these\\

24. Which of the following is a correct combination?\\
\includegraphics[width=0.60555in,height=0.32083in]{image2.png}
1) C–Drama \hspace{2mm}2) E-Mimicry \hspace{2mm}3) A-Speech
\hspace{2mm}4) D-Story telling \hspace{2mm}5) None of these\\

25. Which item was performed by A?\\
\includegraphics[width=0.60555in,height=0.32083in]{image2.png}
1) Song \hspace{2mm}2) Dance \hspace{2mm}3) Speech
\hspace{2mm}4) Dance or song \hspace{2mm}5) None of these\\

\textbf{Answers}
\begin{tabular}{c c c c c c c c c c c c c c}
1 - 3& 2 - 2& 3 - 5& 4 - 1& 5 -1& 6 - 3& 7 - 3& 8 - 2& 9 - 3& 10 - 5\\
11 - 4& 12 - 1& 13 - 2 &14 - 3& 15 - 5& 16 - 5& 17 - 5& 18 - 5& 19 - 4 &20 - 4\\
21 - 2& 22 - 1& 23 - 3& 24 - 4& 25 - 4\\
\end{tabular}


\textbf{Note:} The date and time mentioned against some questions refer to the doubts clarification
session on Reasoning Ability in which the question was solved.
%\includegraphics[width=0.60555in,height=0.32083in]{image2.png}
%\hspace{2mm}

\end{document}