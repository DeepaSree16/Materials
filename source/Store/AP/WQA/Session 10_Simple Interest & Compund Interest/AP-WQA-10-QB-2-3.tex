

\documentclass{article} 

\usepackage[utf8]{inputenc} 
\usepackage[english]{babel}
\usepackage{amsmath}
\usepackage{amssymb}
\usepackage{txfonts}
\usepackage{mathdots}
\usepackage[classicReIm]{kpfonts}
\usepackage{graphicx}

\usepackage{multirow}
\usepackage[margin=1.0in]{geometry}
\usepackage[english]{babel}
\usepackage[utf8]{inputenc}
\usepackage{fancyhdr}
\usepackage{tabularx}
\pagestyle{fancy}
\fancyhf{}
\rhead{\noindent \\  
	\noindent \\ \noindent \\ \includegraphics[width=1.0in, height=0.38819in]{images/logo.png}}
\lhead{ Quantitative Aptitude: Simple Interest and Compound Interest}
\lfoot{www.talentsprint.com }
\rfoot{\thepage}



\begin{document}
	
	\noindent \begin{center}
		{\Large \textbf{Simple Interest and Compound Interest \\  }}
	\end{center}
	
	\noindent 
	
	\noindent 
	
	\noindent \\   \textbf{Part 2 - Advanced \\  }
	
	\noindent 
	
	\noindent \\  
	
	1.   A man borrows Rs. 21000 at 10\% compound interest. How much he has to pay equally at the \noindent \\ \includegraphics*[width=0.60in, height=0.52in]{images/image1}end of each year, to settle his loan in two years?
	
	\noindent \begin{tabular}{p{1.7in} p{1.6in} p{1.6in}} \\ 
 1) Rs. 12000           & 2) Rs. 12100     & 3) Rs. 12200     \\
4) Rs. 12300 \\
\end{tabular}
	
	\noindent 
	
	\noindent 
	
	\noindent  \\  
	
	2.   What annual payment will discharge a debt of Rs. 6,450 due in 4 years at 5\% per annum \noindent \\ \includegraphics*[width=0.60in, height=0.52in]{images/image1}simple interest?
	
	\noindent \begin{tabular}{p{1.7in} p{1.6in} p{1.6in}} \\ 
 1) Rs. 1,400            & 2) Rs. 1,500      & 3) Rs. 1,550      \\
4) Rs. 1,600 \\
\end{tabular}
	
	\noindent  \\  
	
	3.   A sum of money at compound interest doubles itself in 15 years. It will become eight times \noindent \\ \includegraphics*[width=0.60in, height=0.52in]{images/image1}of itself in
	
	\noindent \begin{tabular}{p{1.7in} p{1.6in} p{1.6in}} \\ 
 1) 45 years             & 2) 48 years       & 3) 54 years       \\
4) 60 years \\
\end{tabular}
	
	\noindent 
	
	\noindent 
	
	\noindent  \\  
	
	4.   The time in which Rs. 80,000 amounts to Rs. 92,610 at 10\% pa at compound interest, interest \noindent \\ \includegraphics*[width=0.60in, height=0.52in]{images/image1}being compounded semi-annually is
	
	\noindent \begin{tabular}{p{1.7in} p{1.6in} p{1.6in}} \\ 
 1) 1 1/2 years         & 2) 2 years         & 3) 2 1/2 years   \\
4) 3 years \\
\end{tabular}
	
	\noindent 
	
	\noindent 
	
	\noindent 
	
	\noindent 
	
	\noindent 
	
	\noindent  \\  
	
	\noindent \noindent \\ \includegraphics*[width=0.59in, height=0.52in]{images/image1}5.   Given the difference between simple interest for 2 years and compound interest for 2 years on the same sum and at the same rate of interest compounded annually is 120. The difference between simple interest for 3 years and compound interest for 3 years on the same sum and at the same rate of interest is 366. Find the rate of interest.
	
	\noindent 
	
	\noindent  \\  
	
	\noindent 6.   The simple interest earned on a sum of Rs 3000 in 2 years is Rs 20 more than on Rs 2500 for
	
	\noindent 
	
	\noindent \noindent \\ \includegraphics*[width=0.59in, height=0.52in]{images/image1}the same period at the same rate of interest. If a person is getting an interest of Rs 2280 on the same rate of interest and time period which is thrice of the rate of interest, then what is amount he has invested?
	
	\noindent \begin{tabular}{p{1.7in} p{1.6in} p{1.6in}} \\ 
 1) 15500                  & 2) 11200           & 3) 17500           \\
4) 19000           & 5) None of these  \\
\end{tabular}
	
	\noindent  \\  
	
	7.   Reena had Rs 10,000 with her. Out of this money, she lend some money to Akshay for 2 \noindent \\ \includegraphics*[width=0.60in, height=0.52in]{images/image1}years at 15\% SI. She lent remaining money to Brijesh for an equal number of years at the rate
	
	\noindent of 18\%. After 2 years, Reena found that Akshay had given her Rs 360 more as interest
	
	\noindent 
	
	\noindent compared to Brijesh. The amount of money which Reena had lent to Brijesh must be
	
	\noindent 
	
	\noindent \begin{tabular}{p{1.7in} p{1.6in} p{1.6in}} \\ 
 1) 4000                    & 2) 2500             & 3) 3500             \\
4) 4200             & 5) None of these  \\
\end{tabular}
	
	\noindent 
	
	\noindent  \\  
	
	8.   A man invests an amount of Rs. 15,860 in the names of his three sons A, B and C in such a \noindent \\ \includegraphics*[width=0.60in, height=0.52in]{images/image1}way that they get the same amount after 2, 3 and 4 years respectively. If the rate of simple
	
	\noindent interest is 5\%, then find the ratio in which the amount was invested for A, B and C?
	
	\noindent  \\  
	
	9.   What is the interest for the third year if Rs. 26,600 is lent out at the rate of interest 11\% pa, \noindent \\ \includegraphics*[width=0.60in, height=0.52in]{images/image1}the interest being compounded annually? (in approximate value)
	
	\noindent \begin{tabular}{p{1.7in} p{1.6in} p{1.6in}} \\ 
 1) 3277.38               & 2) 3287.38        & 3) 3288.38        \\
4) 3277.83        & 5) None of these  \\
\end{tabular}
	
	\noindent  \\  
	
	10. Prem invested a certain sum in scheme A, which offers simple interest at the rate of 8\% per \noindent \\ \includegraphics*[width=0.60in, height=0.52in]{images/image1}annum for 4 years. He also invested Rs.2000 in Scheme B, which offers compound interest
	
	\noindent (compounded annually) at 20\% pa for 2 years.  If the interest     earned from scheme A is a
	
	\noindent 
	
	\noindent 7/11 of the interest earned from scheme B, what is sum invested in scheme A?
	
	\noindent 
	
	\noindent \begin{tabular}{p{1.7in} p{1.6in} p{1.6in}} \\ 
 1) Rs.4000              & 2) Rs.3000        & 3) Rs.4500        \\
4) Rs.3600        & 5) None of these  \\
\end{tabular}
	
	\noindent 
	
	\noindent  \\  
	
	\noindent 11. A certain sum of money invested at a certain rate of C.I. increases by 96\% of its initial value
	
	\noindent 
	
	\noindent \noindent \\ \includegraphics*[width=0.59in, height=0.52in]{images/image1}in 2 years. If the same sum of money invested at S.I at same rate of interest, then how many years would it triple itself?
	
	\noindent 
	
	\noindent 
	
	\noindent  \\  
	
	12. Equal sum of money were invested in scheme A and scheme B for two years. Scheme A \noindent \\ \includegraphics*[width=0.60in, height=0.52in]{images/image1}offers simple interest and scheme B offers compound interest (compounded annually) and the  rate  of interest (p.c.p.a.) for both the  schemes are  same. The  interest accrued from
	
	\noindent 
	
	\noindent scheme A after two years is Rs. 1920 and from scheme B is Rs. 2112. Had the rate of interest (p.c.p.a.) of scheme A been 4\% more, what would have been the total interest accrued from both the scheme?
	
	\noindent \begin{tabular}{p{1.7in} p{1.6in} p{1.6in}} \\ 
 1) Rs. 2252             & 2) Rs. 2336       & 3) Rs. 2480       \\
4) Rs. 2304       & 5) Rs. 2284 \\
\end{tabular}
	
	\noindent  \\  
	
	13. Two equal sums of money are lent at the same time at 8\% and 7\% p.a. SI. The former is \noindent \\ \includegraphics*[width=0.60in, height=0.52in]{images/image1}recovered 6 months earlier than the latter and the amount in each is Rs. 2560. The sum and
	
	\noindent the time for which the sums of money are lent out are
	
	\noindent  \\  
	
	14. A man gives 50\% of his savings of Rs 84,100 to his wife and divided the remaining sum \noindent \\ \includegraphics*[width=0.60in, height=0.52in]{images/image1}among his two sons A and B of 15 and 13 years of age respectively. He divided it in such a
	
	\noindent way that each of his sons when they attain the age of 18 years would receive the same
	
	\noindent 
	
	\noindent amount at 5\% compound interest per annum. The share of B was
	
	\noindent 
	
	\noindent \begin{tabular}{p{1.7in} p{1.6in} p{1.6in}} \\ 
 1) 20000                  & 2) 20050                        & 3) 22000                        \\
4) 22050 \\
\end{tabular}
	
	\noindent  \\  
	
	15. Divide Rs 7806 between anushka and kavita so that anushka's share at the end of 7 years is \noindent \\ \includegraphics*[width=0.60in, height=0.52in]{images/image1}equal to kavita's share at the end of 9 years CI being 4\%. Find the share of anushka and
	
	\noindent kavita respectively.
	
	\noindent 
	
	\noindent  \\  
	
	\noindent 16. Hari took an education loan from a nationalized bank for his 2 years course of MBA. He
	
	\noindent \noindent \\ \includegraphics*[width=0.60in, height=0.52in]{images/image1} took the loan of Rupees 5 lakh such that he would be charged at 7\% p.a at C.I during his course  and at 9\% C.I after the completion of the course. He returned half of the amount
	
	\noindent 
	
	\noindent which he had to be paid on the completion of his studies and remaining after 2 years. What is the total amount returned by Hari?
	
	\noindent 
	
	\noindent  \\  
	
	\noindent 17. A bank offers 5\% c.i calculated on half yearly basis. A customer deposits Rs.1600 each on 1st
	
	\noindent \noindent \\ \includegraphics*[width=0.61in, height=0.52in]{images/image1} January and 1st July of a year. At the end of the year, the amount he would have gained by way of interest is?
	
	\noindent 
	
	\noindent 
	
	\noindent  \\  
	
	\noindent 18. What annual payment will discharge a debt of Rs.7620 due in 3 years at 16 2/3\% per annum compound interest?
	
	\noindent 
	
	\noindent  \\  
	
	\noindent 19. A invested an amount with Bank X for two years at simple rate of interest 15 pcpa. He
	
	\noindent 
	
	\noindent \noindent \\ \includegraphics*[width=0.61in, height=0.52in]{images/image1}invested the entire amount obtained from Bank X after two years with Bank Y at compound rate of  interest 12 p.c.p.a. for two years. If he finally received Rs. 73382.4, what was the
	
	\noindent 
	
	\noindent amount invested by him with Bank X?
	
	\noindent 
	

		\begin{tabular}{p{1.7in} p{1.6in} p{1.6in}} \\ 
 1) Rs.50000  & 2) Rs.48000  & 3) Rs.45000
		\\
4) Rs.56000  & 5) Rs.42000   \\
\end{tabular}

	
	
	\noindent 
	
	\noindent  \\  
	
	\noindent 20. A sum of cash at simple interest amounts to Rs. 42480 in 9 year. If the rate of interest is
	
	\noindent 
	
	\noindent increased by 25\% the same sum amounts to Rs. 44110 in the same time. The rate of interest is
	
	\noindent 
	
	\noindent \begin{tabular}{p{1.7in} p{1.6in} p{1.6in}} \\ 
 1) 5\%                      & 2) 2\%                & 3) 3\%                \\
4) 4\%                & 5) None of these  \\
\end{tabular}
	
	\noindent 
	
	\noindent  \\  
	
	\noindent 21. Subham took an educational loan from a nationalized bank for his 4-years course of B. Tech.
	
	\noindent 
	
	\noindent He took the loan of Rs 5 lakh such that he would be charged at 3\% p.a. at CI during his course and at 5\% CI after the completion of the course, he returned half of the amount which he had to be paid on the completion of his studies and remaining after 2 years. What is the total amount which is paid by Subham?
	
	\noindent 
	
	\noindent 
	
		\begin{tabular}{p{1.7in} p{1.6in} p{1.6in}} \\ 
 1) Rs. 492585 & 2) Rs. 561585 & 3) Rs. 612575 
		\\
4) Rs. 591595  & 5) None of these  \\
\end{tabular}  \\ 

	

\newpage
	
	\noindent  \\  \textbf{Answers}
	
	\noindent 
	
	\noindent  \\  
	
	\noindent 1.   Rs. 12100
	
	\noindent 
	
	\noindent 2.   Rs. 1,500
	
	\noindent 
	
	\noindent 3.   45 years
	
	\noindent 
	
	\noindent 4.   11/2 years
	
	\noindent 
	
	\noindent 5.   5\%
	
	\noindent 
	
	\noindent 6.   19000
	
	\noindent 
	
	\noindent 7.   4000
	
	\noindent 
	
	\noindent 8.   276:264:253
	
	\noindent 
	
	\noindent 9.   None of these
	
	\noindent 
	
	\noindent 10. None of these
	
	\noindent 
	
	\noindent 11. 5 years
	
	\noindent 
	
	\noindent 12. Rs. 2304
	
	\noindent 
	
	\noindent 13. 2000, 4 years
	
	\noindent 
	
	\noindent 14. 20000
	
	\noindent 
	
	\noindent 15. 4056:3750
	
	\noindent 
	
	\noindent 16. --
	
	\noindent 
	
	\noindent 17. --
	
	\noindent 
	
	\noindent 18. 3430
	
	\noindent 
	
	\noindent 19. Rs.45000
	
	\noindent 
	
	\noindent 20. 2\%
	
	\noindent 21. 
	
\end{document}