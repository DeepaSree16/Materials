
\documentclass{article} 

\usepackage[utf8]{inputenc} 
\usepackage[english]{babel}
\usepackage{amsmath}
\usepackage{amssymb}
\usepackage{txfonts}
\usepackage{mathdots}
\usepackage[classicReIm]{kpfonts}
\usepackage{graphicx}

\usepackage{multirow}
\usepackage[margin=1.0in]{geometry}
\usepackage[english]{babel}
\usepackage[utf8]{inputenc}
\usepackage{fancyhdr}
\usepackage{tabularx}
\pagestyle{fancy}
\fancyhf{}
\rhead{\noindent \\  
	\noindent \\ \noindent \\  
	\noindent \\ \includegraphics[width=1.0in, height=0.38819in]{images/logo.png}}
\lhead{ Quantitative Aptitude: Simple Interest and Compound Interest}
\lfoot{www.talentsprint.com }
\rfoot{\thepage}





\begin{document}
	
	
	\noindent \begin{center}
		{\large \textbf{Simple Interest and Compound Interest \\}}
	\end{center}
	
	\noindent 
	
	\noindent 
	
	\noindent 
	
	\noindent 
	
	\noindent 
	
	\noindent {\large \textbf{Additional Examples \\}}
	
	\noindent \\ 
	
	1.   The simple interest on Rs. 4,000 in 3 years at the rate of x\% per annum equals the simple  
	\noindent \\ \includegraphics*[width=0.60in, height=0.52in]{images/image1}interest on Rs. 5,000 at the rate of 12\% per annum in 2 years. The value of x is
	
	\noindent \begin{tabular}{p{1.7in} p{1.6in} p{1.6in}} \\ 
 1) 10\%                    & 2) 6\%                & 3) 8\%                \\
4) 9\% \\
\end{tabular}
	
	\noindent  \\ 
	
	2.   A sum of money at compound interest doubles itself in 15 years. It will become eight times  
	\noindent \\ \includegraphics*[width=0.60in, height=0.52in]{images/image1}of itself in
	
	\noindent \begin{tabular}{p{1.7in} p{1.6in} p{1.6in}} \\ 
 1) 45 years             & 2) 48 years       & 3) 54 years       \\
4) 60 years \\
\end{tabular}
	
	\noindent  \\ 
	
	3.   Simple interest on a certain sum is 16/25 of the sum. The rate per cent if the rate per cent and  
	 
	\noindent \\ \includegraphics*[width=0.60in, height=0.52in]{images/image1}time (in years) are equal, is
	
	\noindent \begin{tabular}{p{1.7in} p{1.6in} p{1.6in}} \\ 
 1) 6\%                      & 2) 8\%                & 3) 10\%              \\
4) 12\% \\
\end{tabular}
	
\noindent \\ 	4.   A sum of money becomes eight times in 3 years, it the rate is compounded annually. In how  
	 
	\noindent \\ \includegraphics*[width=0.60in, height=0.52in]{images/image1}much time will the same amount at the same compound rate become sixteen times?
	
	\noindent \begin{tabular}{p{1.7in} p{1.6in} p{1.6in}} \\ 
 1) 6 years               & 2) 4 years         & 3) 8 years         \\
4) 5 years \\
\end{tabular}
	
	\noindent 
	
	\noindent  \\ 
	
	\noindent 5.   The CI accrued on an amount of Rs 25500 at the end of 3 years is 8440.50. What would be the
	
	\noindent 
	
	\noindent  
	 
	\noindent \\ \includegraphics*[width=0.59in, height=0.52in]{images/image1}SI on the same amount at the same rate and same period?
	
	\noindent 
	
	\noindent 
	
	\noindent 
	
\newpage
	 
	\noindent \\ \includegraphics*[width=0.59in, height=0.52in]{images/image1}6.   At the rate of 8 1/2\% p.a simple interest, a sum of Rs. 4800 will earn how much interest in 2 years 3 months? (using \% metho\\
4)
	
	\noindent 
	
	\noindent  \\ 
	
	\noindent 7.   A sum of Rs. 10,000 is invested in a scheme where interest is compounded       annually    for
	
	\noindent 
	
	\noindent  
	 
	\noindent \\ \includegraphics*[width=0.60in, height=0.52in]{images/image1}two years at the rate of 10\% per annum. Find the number of years if it is a simple interest scheme and rest of the conditions are the same.
	
	\noindent 
	
	\noindent \begin{tabular}{p{1.7in} p{1.6in} p{1.6in}} \\ 
 1) 1.5 years            & 2) 3 years         & 3) 2 years         \\
4) 2.1 years      & 5) 1.75 years \\
\end{tabular}
	
	\noindent  \\ 
	
	8.   Irfan borrows a sum of Rs. 64000 at 5\% pa compound interest. He repays a certain amount at  
	 
	\noindent \\ \includegraphics*[width=0.60in, height=0.52in]{images/image1}the end of one year and the balance amount of Rs. 35700 at the end of the second year. What
	
	\noindent amount does he repay in the first year?
	
	\noindent 
	
	\noindent \begin{tabular}{p{1.7in} p{1.6in} p{1.6in}} \\ 
 1) Rs. 34000           & 2) Rs. 37200     & 3) Rs. 36400     \\
4) Rs. 35700    & 5) Rs. 33200 \\
\end{tabular}
	
	\noindent 
	
	\noindent 
	
	\noindent  \\ 
	
	\noindent 9.   HariLal and Hari Prasad have equal amounts. HariLal invested all his amount at 10\%
	
	\noindent  
	 
	\noindent \\ \includegraphics*[width=0.60in, height=0.52in]{images/image1}compounded annually for 2 years and Hari Prasad invested 1/4 at 10\% compound interest
	
	\noindent (annually) and rest at r\% per annum at simple interest for the same 2 years period. The
	
	\noindent 
	
	\noindent amount received by both at the end of 2 year is same. What is the value of r?
	
	\noindent  \\ 
	
	10. The  simple  interest  earned  on  a  sum  of  money  in  9  years  is  2178.  If  the  principal  is  
	 
	\noindent \\ \includegraphics*[width=0.60in, height=0.52in]{images/image1}quadrupled after 5 years then what will be the total interest at the end of 9 years?
	
	\noindent \begin{tabular}{p{1.7in} p{1.6in} p{1.6in}} \\ 
 1) 5082                    & 2) 5100             & 3) 5208             \\
4) 5808             & 5) 5978 \\
\end{tabular}
	
	\noindent 
	
\newpage
	
	\noindent 11. A man borrows money at 3\% per annum interest payable yearly and lend it immediately at
	
	\noindent  
	 
	\noindent \\ \includegraphics*[width=0.60in, height=0.52in]{images/image1}5\% interest (compoun\\
4) payable half-yearly and thereby gains Rs 330 at the end of the year.The sum borrowed is
	
	\noindent 
	
	\noindent \begin{tabular}{p{1.7in} p{1.6in} p{1.6in}} \\ 
 1) 17500                  & 2) 16500           & 3) 15000           \\
4) 16000           & 5) None of these  \\
\end{tabular}
	
	\noindent 
	
	\noindent 
	
	\noindent 
	
	\noindent 
	
	\noindent 
	
	\noindent   \\ 
	 
	\noindent \\ \includegraphics*[width=0.59in, height=0.52in]{images/image1} 12. Mr. Shiva invested equal amount of money in two different firms which gives 15\% simple interest per annum for 3.5 years and 5 years respectively. If the difference between their invest is Rs 225, the amount invested by Mr. Shiva is
	
	\noindent 
	
	\noindent  \\ 
	
	\noindent 13. Arun and Gaurav have to clear their respective loans by paying 3 equal annual instalments
	
	\noindent  
	 
	\noindent \\ \includegraphics*[width=0.60in, height=0.52in]{images/image1}of Rs. 30,000 each. Arun pays @ 10\% per annum of simple interest while Gaurav pays 10\%
	
	\noindent per annum compound interest. What is the difference in their loan amounts?
	
	\noindent 
	
	\noindent 
	
	\noindent  \\ 
	
	14. A man invested P for 2 years @ 11\% simple interest in scheme A. He invested P+600 in  
	 
	\noindent \\ \includegraphics*[width=0.60in, height=0.52in]{images/image1}compound interest for 2 years @ 20\% - Annual compounding. If the amount received from scheme  A  was less than amount received by scheme B by Rs.1216, what is the principal  amount?
	
	\noindent 
	
	\noindent  \\ 
	
	\noindent 15. Raju invested equal sum of money in two schemes. Under scheme X the CI rate was 10\% per
	
	\noindent 
	
	\noindent  
	 
	\noindent \\ \includegraphics*[width=0.60in, height=0.52in]{images/image1}annum and under scheme Y the CI rate was 12\% per annum. The interest after 2 years on the sum invested in scheme X was Rs.63. How much is the interest earn under scheme Y after 2 years?
	
	\noindent 
	
	\noindent 
	
	\noindent  \\ 
	
	16. If P borrowed the same amount as Q from R at the same rate of interest for 3 years at simple  
	 
	\noindent \\ \includegraphics*[width=0.60in, height=0.52in]{images/image1} interest while Q borrowed at compound interest compounded annually and the difference between  their interests is INR 3896.2 then find the rate pcpa if P borrowed INR 25,000.
	
	\noindent 
	
	\noindent (interest)
	
	\noindent 
	
	\noindent \begin{tabular}{p{1.7in} p{1.6in} p{1.6in}} \\ 
 1) 20\%                    & 2) 20.5\%           & 3) 21\%              \\
4) 22\%              & 5) 22.5\% \\
\end{tabular}
	
	\noindent 
	
	\noindent  \\ 
	
	17. If the compound interest on a certain sum for two consecutive years is Rs.220 and Rs.242.  
	 
	\noindent \\ \includegraphics*[width=0.60in, height=0.52in]{images/image1} Then Rate of interest is?
	
	\noindent  \\ 
	
	\noindent 18. Simple interest on the sum A @ 11\% per annum and compound interest on Sum B which is
	
	\noindent  
	 
	\noindent \\ \includegraphics*[width=0.60in, height=0.52in]{images/image1} 400 more than A in 2yrs is 140\% more of simple interest of A. Find the value of A?
	
	\noindent  \\ 
	
	19. 850 are invested for 3 years at the rate of 17.5\% per year on simple interest. What will be a  
	 
	\noindent \\ \includegraphics*[width=0.60in, height=0.52in]{images/image1} total amount at the end of 3 years?
	
	\noindent \begin{tabular}{p{1.7in} p{1.6in} p{1.6in}} \\ 
 1) 1,147.50              & 2) 998.15          & 3) 1,296.25       \\
4) 1, 295, 50     & 5) None of these  \\
\end{tabular}
	
	\noindent 
	
	\noindent  \\ 
	
	\noindent 20. The difference between the SI and TD on a certain sum of money for 6 months at 6\% per
	
	\noindent  
	 
	\noindent \\ \includegraphics*[width=0.60in, height=0.52in]{images/image1} annum is Rs.27. Find the sum
	
	\noindent \begin{tabular}{p{1.7in} p{1.6in} p{1.6in}} \\ 
 1) Rs.30900              & 2) Rs.20800     & 3) Rs.30800     \\
4) Rs.30600 \\
\end{tabular}
	
	\noindent 
	
	\noindent  \\ 
	
	21. A sum of Rs.198 deposited at CI doubles itself after 4 years. After 20 years it will become  
	 
	\noindent \\ \includegraphics*[width=0.60in, height=0.52in]{images/image1}\begin{tabular}{p{1.7in} p{1.6in} p{1.6in}} \\ 
 1) Rs.6336              & 2) Rs.5894        & 3) Rs.9250        \\
4) Rs.7932       & 5) None of these  \\
\end{tabular}
	
	\noindent  \\ 
	
	\noindent 22. Divide Rs. 50752 between Rohit and Rani in such a way that the share of Rohit at the end of
	
	\noindent  
	 
	\noindent \\ \includegraphics*[width=0.60in, height=0.52in]{images/image1}33 years equals that of Rani at the end of 35 years at compound interest 24\% per annum.
	
	\noindent Find the share of Rohit.
	
	\noindent 
	
	\noindent \begin{tabular}{p{1.7in} p{1.6in} p{1.6in}} \\ 
 1) Rs.29956             & 2) Rs.30752      & 3) Rs.31272      \\
4) Rs.30560     & 5) None of these  \\
\end{tabular}
	
	\noindent 
	
	\noindent 
	
	\noindent  \\ 
	
	\noindent 23. Manish borrowed Rs. 1150 from Anil @ S.I rate 6\% for 3 years. He then added some more money to borrowed sum and lent it to Sunil for same time @ S.I rate 9\%. If Manish gains Rs.
	
	\noindent 274.95 by way of interest on borrowed sum as well as his own amount, then what is the sum lent by him to Sunil?
	
	\noindent 
	
	\noindent  \\ 
	
	\noindent 24. A father left a will of Rs. 68,000 to be divided between his two sons aged 10 years and 12 years such that they may get equal amount when each attains the age of 18 years. If the money is reckoned at 10\% p.a. then find how much each gets at the time of the will.
	
	\noindent \begin{tabular}{p{1.7in} p{1.6in} p{1.6in}} \\ 
 1) Rs. 30,000, Rs. 38,000                 & 2) Rs. 28,000, Rs. 40,000
	
	\noindent 
	
	\noindent & 3) Rs. 32,000, Rs. 36,000                 \\
4) Cannot be determined
	
	\noindent 
	
	\noindent & 5)   None of these \\
\end{tabular}
	
	\noindent 
	
	\noindent  \\ 
	
	\noindent 25. What annual payment that will discharge a debt at Rs. 10260 due in 5 years, the rate of interest being 4\% per annum?
	
	\noindent \begin{tabular}{p{1.7in} p{1.6in} p{1.6in}} \\ 
 1) Rs. 1850             & 2) Rs. 1750       & 3) Rs. 1900       \\
4) Rs. 2000      & 5) Rs. 2050 \\
\end{tabular}
	
	\noindent 
	
	\noindent  \\ 
	
	\noindent 26. What is the compound interest on Rs. 7 lakh for three years if the rate of interest is 5\% for
	
	\noindent 
	
	\noindent the first year, 8\% for the second year and 12\% for the thired year?
	
	\noindent 
	

		\begin{tabular}{p{1.7in} p{1.6in} p{1.6in}} \\ 
 1) Rs. 163600  & 2) Rs. 189056  & 3) Rs. 194064 
		\\
4) Rs. 201040  & 5) Rs. None of these  \\
\end{tabular}

	
	
	
	\noindent  \\ 
	
	\noindent 27. The difference between simple interest and compound interest for 2 years is 4500 and the
	
	\noindent  
	 
	\noindent \\ \includegraphics*[width=0.60in, height=0.52in]{images/image1} same for 3 years is 14175, find the sum?
	
	\noindent \begin{tabular}{p{1.7in} p{1.6in} p{1.6in}} \\ 
 1) 300000                & 2) 200000         & 3) 500000          \\
4) 600000 \\
\end{tabular}
	
	\noindent 
	
	\noindent 
	
	\noindent  \\ 
	
	\noindent 28. Ravi has 10000 Rupees with him. He intends to divide them between his two daughters whose ages are 14 and 10 respectively. Both the daughters will deposit the received amount in a bank which offers a interest of 8 pa. He intends to divide the amount in such a way that both the daughters have an equal amount when they turn 18. What is the approximate amount that he should give to her elder daughter?
	
	\noindent \begin{tabular}{p{1.7in} p{1.6in} p{1.6in}} \\ 
 1) 4472                    & 2) 3780             & 3) 5541              \\
4) 4753             & 5) 4258 \\
\end{tabular}
	
	\noindent 
\end{document}