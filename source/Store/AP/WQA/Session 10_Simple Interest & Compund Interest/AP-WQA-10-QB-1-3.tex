


\documentclass{article} 

\usepackage[utf8]{inputenc} 
\usepackage[english]{babel}
\usepackage{amsmath}
\usepackage{amssymb}
\usepackage{txfonts}
\usepackage{mathdots}
\usepackage[classicReIm]{kpfonts}
\usepackage{graphicx}

\usepackage{multirow}
\usepackage[margin=1.0in]{geometry}
\usepackage[english]{babel}
\usepackage[utf8]{inputenc}
\usepackage{fancyhdr}
\usepackage{tabularx}
\pagestyle{fancy}
\fancyhf{}
\rhead{\noindent \\  
\noindent \\ \includegraphics[width=1.0in, height=0.38819in]{images/logo.png}}
\lhead{ Quantitative Aptitude: Simple Interest and Compound Interest}
\lfoot{www.talentsprint.com }
\rfoot{\thepage}

\begin{document}
	
	
	
	
	\noindent \begin{center}
		{\Large \textbf{Simple Interest and Compound Interest \\}}
	\end{center}
	
	\noindent 
	
	\noindent 
	
	\noindent {\large \textbf{Part 1 - Basic \\}}
	
	\noindent 
	
	\noindent 
	
	\noindent \\ \textbf{Model 1: Simple Interest}
	
	\noindent 
	
	\noindent 
	
	\noindent \\  1.   Simple interest on an amount after 24 months at the rate of 2\% per quarter is Rs. 960. What is
	
	\noindent  
\noindent \\ \includegraphics*[width=0.60in, height=0.52in]{images/image1} the amount?
	
	\noindent \begin{tabular}{p{1.7in} p{1.6in} p{1.6in}} \\ 
 1)  Rs. 2000                &  2) Rs. 5750          &  3) Rs. 6000          \\
4) Rs. 4800          & 5) None of these  \\
\end{tabular}
	
	\noindent 
	
	\noindent 
	
	\noindent 
	
	\noindent \\  2.   The simple interest obtained on a certain amount at 7.5\% p.a. for two years is Rs. 232.50. What is the amount invested?
	
	\noindent \begin{tabular}{p{1.7in} p{1.6in} p{1.6in}} \\ 
 1) Rs. 2000                 &  2) Rs. 1575          &  3) Rs. 1659          \\
4) Rs. 1600          & 5) None of these  \\
\end{tabular}
	
	\noindent 
	
	\noindent 
	
	\noindent 
	
	\noindent \\  3.   What will be the on Rs. 10000 after 3 years at the rate of 5\% per quarter?
	
	\noindent 
	

		\begin{tabular}{p{1.7in} p{1.6in} p{1.6in}} \\ 
 1) Rs. 3000  &  2) Rs. 6000 &  3) Rs. 5000
		\\
4) Cannot be determined  & 5) None of these  \\
\end{tabular}   
	
	
	
	
	\noindent 
	
	\noindent 
	
	\noindent \\  4.   Karan took a loan on simple interest at the rate of 12\% per year, after 8 months he paid
	
	\noindent  
\noindent \\ \includegraphics*[width=0.60in, height=0.52in]{images/image1}Rs. 8100. How much loan was taken by Karan?     \textbf{[February 28, 2015 @ 1h 56m 17s]}
	
	\noindent \begin{tabular}{p{1.7in} p{1.6in} p{1.6in}} \\ 
 1) Rs. 7500                 &  2) Rs. 8000          &  3) Rs. 6,500         \\
4) Rs. 7000          & 5) None of these  \\
\end{tabular}
	
	\noindent 
	
	\noindent 
	
	\noindent 
	
	\noindent \\  5.   An amount becomes Rs. 8,800 in four years at 15\% p.a. What is that amount?
	
	\noindent 
	
	\noindent \begin{tabular}{p{1.7in} p{1.6in} p{1.6in}} \\ 
 1) Rs. 5,500                &  2) Rs. 7,500         &  3) Rs. 5,800         \\
4) Rs. 6,400         & 5) None of these  \\
\end{tabular}
	
	\noindent 
	
	\noindent 
	
	\noindent 
	
	\noindent \\  6.   Rs. 850 are invested for 3 years at the rate of 17.5 \% per year on simple interest. What will be a total amount at the end of 3 years?
	
	\noindent \begin{tabular}{p{1.7in} p{1.6in} p{1.6in}} \\ 
 1) Rs. 1,147.50           &  2) Rs. 998.15       &  3) Rs. 1,296.25    \\
4) Rs. 1, 295, 50  & 5) None of these  \\
\end{tabular}
	
	\noindent 
	
	\noindent \\  7.   Shilpa took a loan of Rs. 800 at the rate of 11\% per year for 7 months. How much does she need to pay at the end of 7 months?
	
	\noindent \begin{tabular}{p{1.7in} p{1.6in} p{1.6in}} \\ 
 1) Rs. 851                   &  2) Rs. 852            &  3) Rs. 950            \\
4) Rs. 951            & 5) None of these  \\
\end{tabular}
	
	\noindent 
	
	\noindent 
	
	\noindent 
	
	\noindent \\  8.   Mehra invests an amount of Rs. 18000 to obtain a simple interest at the rate of 15\% p.a. for
	
	\noindent 
	
	\noindent 6 years. What total amount will Mehra get at the end of 6 years?
	
	\noindent 
	
	\noindent \begin{tabular}{p{1.7in} p{1.6in} p{1.6in}} \\ 
 1) Rs. 33,500              &  2) Rs. 35,000       &  3) Rs. 34,000       \\
4) Rs. 34,200       & 5) None of these  \\
\end{tabular}
	
	\noindent 
	
	\noindent 
	
	\noindent 
	
	\noindent \\  9.   Vijay borrowed some money from Vishnu at simple interest of 8\% per eight months. If after
	
	\noindent  
\noindent \\ \includegraphics*[width=0.61in, height=0.52in]{images/image1}4 years, Vishnu got Rs. 2664 as principal and interest, how much did Vijay borrow from
	
	\noindent Vishnu?     \textbf{[September 12, 2014 @ 1h 56m 17s]}
	
	\noindent 
	

		\begin{tabular}{p{1.7in} p{1.6in} p{1.6in}} \\ 
 1) Rs. 2018  &  2) Rs. 2000  &  3) Rs. 1800  
		\\
4) Cannot be determined  & 5) None of these  \\
\end{tabular} 

	
	
	
	\noindent 
	
	\noindent 
	
	\noindent \\  10. An amount doubles itself at the end of 8 years with a certain rate of simple interest. What
	
	\noindent  
\noindent \\ \includegraphics*[width=0.61in, height=0.52in]{images/image1}will be the total simple interest on Rs. 8000 at that rate at the end of four years?
	
	\noindent \textbf{[September 12, 2014 @ 1h 58m 17s]}
	
	\noindent 
	

		\begin{tabular}{p{1.7in} p{1.6in} p{1.6in}} \\ 
 1) Rs. 4000 &  2) Rs. 2000  &  3) Rs. 8000  
		\\
4) Data Inadequate  & 5) None of these  \\
\end{tabular}
	
	
	
	
	\noindent 
	
	\noindent 
	
	\noindent 
	
	\noindent 
	
	\noindent 
	
	\noindent 
	
	\noindent \\  \textbf{Model 2: Compound Interest}
	
	\noindent 
	
	\noindent  \\ 
	
	11. What will be the compound interest on Rs. 5000 for 2 years at 12\% per annum?  
\noindent \\ \includegraphics*[width=0.61in, height=0.52in]{images/image1}\begin{tabular}{p{1.7in} p{1.6in} p{1.6in}} \\ 
 1) Rs. 1250                 &  2) Rs. 1200          &  3) Rs. 1272          \\
4) Rs. 2174          & 5) None of these  \\
\end{tabular}
	
	\noindent 
	
	\noindent \\  12. What will be the compound interest on Rs. 5000 for 2 years at 7\% per annum?
	
	\noindent 
	
	\noindent \begin{tabular}{p{1.7in} p{1.6in} p{1.6in}} \\ 
 1) Rs. 725                   &  2) Rs. 700            &  3) Rs. 724.50       \\
4) Rs. 714.50       & 5) None of these  \\
\end{tabular}
	
	\noindent 
	
	\noindent 
	
	\noindent \\  13. The compound interest on a certain amount for 2 years at the rate of 5\% is Rs. 102.5. Find the amount.
	
	\noindent \begin{tabular}{p{1.7in} p{1.6in} p{1.6in}} \\ 
 1) Rs. 500                   &  2) Rs. 725            &  3) Rs. 850            \\
4) Rs. 1000          & 5) None of these  \\
\end{tabular}
	
	\noindent \\  
	
	14. Manish deposited some money in a bank at the rate of 6\% p.a. for 2 years at Compound  
\noindent \\ \includegraphics*[width=0.61in, height=0.52in]{images/image1} interest. How much money was deposited if he gets Rs. 11236 on maturity?
	
	\noindent \begin{tabular}{p{1.7in} p{1.6in} p{1.6in}} \\ 
 1) Rs. 15000               &  2) Rs. 14000        &  3) Rs. 10000        \\
4) Rs. 16000        & 5) None of these  \\
\end{tabular}
	
\newpage
	
	15. If the annual increase in the population of a town is 4\% and the present population is 16224,  
\noindent \\ \includegraphics*[width=0.61in, height=0.52in]{images/image1}what was the population two years ago?             \textbf{[July 12, 2014 @ 1h 17m 10s]}
	
	\noindent \begin{tabular}{p{1.7in} p{1.6in} p{1.6in}} \\ 
 1) 15000                  &  2) 14000           &  3) 15500           \\
4) 16000           & 5) None of these  \\
\end{tabular}
	
	\noindent  \\ 
	
	16. Brijesh borrowed a sum of Rs. 2000 at 2\% per month simple interest on a yearly basis. It was  
\noindent \\ \includegraphics*[width=0.60in, height=0.52in]{images/image1}decided that if the sum would not be returned at the end of the year interest would be
	
	\noindent charged on the fixed interest. If the sum was returned after two years then how much did
	
	\noindent 
	
	\noindent Brijesh pay?           \textbf{[July 12, 2014 @ 33m 40s]}
	
	\noindent 
	
	\noindent \begin{tabular}{p{1.7in} p{1.6in} p{1.6in}} \\ 
 1) Rs. 3,600                &  2) Rs. 3,844         &  3) Rs. 3,700         \\
4) Rs. 3,100         & 5) None of these  \\
\end{tabular}
	
	\noindent 
	
	\noindent 
	
	\noindent \\  
	
	17. The compound interest on Rs. 800 at a certain rate for two years is Rs. 65.28. What would be the  
\noindent \\ \includegraphics*[width=0.60in, height=0.52in]{images/image1}approximate compound interest on the same amount for three years?
	
	\noindent \textbf{[July 12, 2014 @ 21m 15s]}
	
	\noindent 
	
	\noindent \begin{tabular}{p{1.7in} p{1.6in} p{1.6in}} \\ 
 1) Rs. 100                                                          &  2) Rs. 85                                        &  3) Rs. 90
	
	\noindent 
	
	\noindent \\
4) Cannot be determined                            & 5) None of these  \\
\end{tabular}
	
	\noindent 
	
	\noindent 
	
	\noindent \textbf{Model 3: Varying Rates of Interest}
	
	\noindent 
	
	\noindent 
	
	\noindent \\  18. If Rs. 20000 is given as loan for a period of 3 years with interest rates 5\%, 7\% and 9\% for the 1st,
	
	\noindent 
	
	\noindent 2nd and 3rd years respectively, what is the total amount that needs to be paid in the end?
	
	\noindent 
	
	\noindent \begin{tabular}{p{1.7in} p{1.6in} p{1.6in}} \\ 
 1) Rs. 23500               &  2) Rs. 24200        &  3) Rs. 18000        \\
4) Rs. 24000        & 5) None of these  \\
\end{tabular}
	
	\noindent 
	
	\noindent 
	
	\noindent 
	
	\noindent  \\ 19. If Rs. 10000 is given as loan for a period of 3 years with interest rates 6\%, 8\% and 10\% for the
	
	\noindent 
	
	\noindent 1st, 2nd and 3rd years, what is the total amount that needs to be paid in the end?
	
	\noindent 
	
	\noindent \begin{tabular}{p{1.7in} p{1.6in} p{1.6in}} \\ 
 1) Rs. 13000               &  2) Rs. 15000        &  3) Rs. 18000        \\
4) Rs. 12400        & 5) None of these  \\
\end{tabular}
	
	\noindent  \\ 
	
	20. Nitin borrowed some money at the rate of 6\% p.a. for the first three years, 9\% p.a. for the  
\noindent \\ \includegraphics*[width=0.60in, height=0.52in]{images/image1}next 5 years and 13\% p.a. for the period beyond eight years. If the total interest paid by him
	
	\noindent at the end of 11 years is Rs. 8160, how much money did he borrow?
	
	\noindent 
	
	\noindent \textbf{[June 20, 2015 @ 1h 44m 40s]}
	
	\noindent 
	
	\noindent \begin{tabular}{p{1.7in} p{1.6in} p{1.6in}} \\ 
 1) Rs. 8000                 &  2) Rs. 10000        &  3) Rs. 12000        \\
4) Rs. 6000          & 5) None of these  \\
\end{tabular}
	
	\noindent 
	
	\noindent 
	
	\noindent 
\newpage
	\noindent  21.  An amount of Rs. 10000 is taken as a loan by Vivek at compound interest charging 5 pcpa for
	
	\noindent  
\noindent \\ \includegraphics*[width=0.60in, height=0.52in]{images/image1}1st year,  10 pcpa for the 2nd year and 20 pcpa for the 3rd year. What is the total interest to be paid by Vivek after 3 years?
	
	\noindent 
	
	\noindent \begin{tabular}{p{1.7in} p{1.6in} p{1.6in}} \\ 
 1) Rs. 3860                 &  2) Rs. 4380          &  3) Rs. 2140          \\
4) Rs. 1780          & 5) None of these  \\
\end{tabular}
	
	\noindent 
	
	\noindent 
	
	\noindent 
	
	\noindent \\  22. An amount of Rs. 10000 is taken as a loan by Karthik at compound interest charging 8\% p.a. for 1st year and 9\% p.a. for the 2nd year. How much is the total to be paid by Karthik after 2 years?
	
	\noindent \begin{tabular}{p{1.7in} p{1.6in} p{1.6in}} \\ 
 1) Rs. 16000               &  2) Rs. 14000        &  3) Rs. 12000        \\
4) Rs. 11772        & 5) None of these  \\
\end{tabular}
	
	\noindent 
	
	\noindent \textbf{Model 4: Difference between CI and SI}
	
	\noindent 
	
	\noindent 
	
	\noindent  \\ 23. Find the difference between the simple and compound interest at 5\% pa for 2 years on a principal of Rs. 2000
	
	\noindent \begin{tabular}{p{1.7in} p{1.6in} p{1.6in}} \\ 
 1) Rs. 5                       &  2) Rs. 50              &  3) Rs. 10              \\
4) Rs. 25              & 5) None of these  \\
\end{tabular}
	
	\noindent 
	
	\noindent 
	
	\noindent 
	
	\noindent \\  24. In two years, at the rate of 5\% p.a., the difference of compound and simple interest is Rs. 25.
	
	\noindent 
	
	\noindent What is the principal?
	
	\noindent 
	
	\noindent \begin{tabular}{p{1.7in} p{1.6in} p{1.6in}} \\ 
 1) Rs. 11,000              &  2) Rs. 10,050       &  3) Rs. 10,500       \\
4) Rs. 10,000       & 5) None of these  \\
\end{tabular}
	
	\noindent 
	
	\noindent 
	
	\noindent 
	
	\noindent  \\ 25. If the difference between the simple interest and compound interest on some amount at 20\%
	
	\noindent  
\noindent \\ \includegraphics*[width=0.60in, height=0.52in]{images/image1} pa for 3 years is Rs. 48, then what must be the principal amount?
	
	\noindent \begin{tabular}{p{1.7in} p{1.6in} p{1.6in}} \\ 
 1) Rs. 240                   &  2) Rs. 375            &  3) Rs. 480            \\
4) Rs. 180            & 5) None of these  \\
\end{tabular}
	
	\noindent 
	
	\noindent 
	
	\noindent  \\ 

	26. The difference between C.I. \& S.I. on a sum of money for 3 years at 5\% is Rs. 13316. What is
	
	\noindent  
\noindent \\ \includegraphics*[width=0.60in, height=0.52in]{images/image1}the sum?   \textbf{[August 08, 2015 @ 54m 30s]}
	
	\noindent \begin{tabular}{p{1.7in} p{1.6in} p{1.6in}} \\ 
 1) Rs. 16000               &  2) Rs. 17500        &  3) Rs. 17000        \\
4) Rs. 18000        & 5) None of these  \\
\end{tabular}
	
	\noindent 
	
	\noindent 
	
	\noindent 
	
	\noindent 
	
	\noindent 
	
	\noindent 
	
	\noindent 
	
	\noindent 
	\newpage
	\noindent   \textbf{Model 5: Principal/Rate of Interest with Respect to Total Amount for Two Different}
	
	\noindent 
	
	\noindent  \\ \textbf{Periods}
	
	\noindent 
	
	\noindent 
	
	\noindent  \\ 27. On simple interest, a sum of money becomes Rs. 1120 in 4 years and Rs. 1360 in 7 years. How
	
	\noindent  
\noindent \\ \includegraphics*[width=0.60in, height=0.52in]{images/image1}much money is deposited?
	
	\noindent \begin{tabular}{p{1.7in} p{1.6in} p{1.6in}} \\ 
 1) Rs. 900                   &  2) Rs. 700            &  3) Rs. 800            \\
4) Rs. 1200          & 5) None of these  \\
\end{tabular}
	
	\noindent 
	
	\noindent 
	
	\noindent  \\ 28. On simple interest, a sum of money becomes Rs. 1,102.5 in three years and Rs. 1,237.5 in 5 years.
	
	\noindent 
	
	\noindent How much money is deposited?
	
	\noindent 
	
	\noindent \begin{tabular}{p{1.7in} p{1.6in} p{1.6in}} \\ 
 1) Rs. 900                   &  2) Rs. 700            &  3) Rs. 1100          \\
4) Rs. 1200          & 5) None of these  \\
\end{tabular}
	
	\noindent  \\ 
	
	29. A sum of money invested at compound interest amounts to Rs. 800 in 3 years and Rs. 882 in 5  
\noindent \\ \includegraphics*[width=0.60in, height=0.52in]{images/image1} years. What is the rate of interest?
	
	\noindent \begin{tabular}{p{1.7in} p{1.6in} p{1.6in}} \\ 
 1) 2.5\%                   &  2) 4\%                &  3) 5\%                \\
4) 6.66\%           & 5) None of these  \\
\end{tabular}
	
	\noindent 
	
	\noindent 
	
	\noindent 
	
	\noindent  \\ 30. A sum of money invested at compound interest amounts to Rs. 800 in 3 years and to Rs. 840 in 4 years. What is the rate of interest per annum?
	
	\noindent \begin{tabular}{p{1.7in} p{1.6in} p{1.6in}} \\ 
 1) 2.5\%                   &  2) 4\%                &  3) 5\%                \\
4) 6.66\%           & 5) None of these  \\
\end{tabular}
	
	\noindent 
	
	\noindent 
	
	\noindent 
	
	\noindent 
	
	\noindent 
	
	\noindent \textbf{Model 6: Principal Based on CI in 1st and 2nd Year}
	
	\noindent 
	
	\noindent  \\ 
	
	31. On a particular amount, the compound interest at the end of one year is Rs. 40 and in the 2nd  year is Rs. 42. How much money was deposited?
	
	\noindent 
	
	\noindent 
	
	\noindent 
	
	\noindent 
	
	\noindent 
	
	\noindent 
	
	\noindent 
	
	\noindent 
	
	\noindent  \\ 32. On a given amount the compound interest at the end of the first year was Rs. 88 and the second year was Rs. 96.80.  How much money was invested?
	
	\noindent \begin{tabular}{p{1.7in} p{1.6in} p{1.6in}} \\ 
 1) Rs. 880                                                          &  2) Rs. 996                                      &  3) Rs. 800
	
	\noindent 
	
	\noindent \\
4) Cannot be determined                            & 5) None of these  \\
\end{tabular}
	
	\noindent 
	
	\noindent \textbf{Model 7: Half - yearly Compounding}
	
	\noindent 
	
	\noindent 
	
	\noindent  \\ 33. A sum of Rs. 40000 is invested for 18 months at 20\% p.a. on compound interest. If the interest
	
	\noindent  
\noindent \\ \includegraphics*[width=0.60in, height=0.52in]{images/image1} is compounded half yearly, what will be the interest to be paid?
	
	\noindent \begin{tabular}{p{1.7in} p{1.6in} p{1.6in}} \\ 
 1) Rs. 13530               &  2) Rs. 13080        &  3) Rs. 13540        \\
4) Rs. 13240        & 5) None of these  \\
\end{tabular}
	
	\noindent 
	
	\noindent 
	
	\noindent 
	
	\noindent  \\ 34. A sum of Rs. 30000 is invested for 18 months at 12\% p.a. on compound interest. If the interest is compounded half yearly, how much does it become on maturity?
	
	\noindent \begin{tabular}{p{1.7in} p{1.6in} p{1.6in}} \\ 
 1) Rs. 35730.48          &  2) Rs. 30800        &  3) Rs. 35400.60   \\
4) Rs. 38400.60   & 5) None of these  \\
\end{tabular}
	
	\noindent 
	
	\noindent 
	
	\noindent 
	
	\noindent \\  \textbf{Model 8: Compound Interest for a Time Period in Non-integer Years}
	
	\noindent 
	
	\noindent 
	
	\noindent \\  35. An amount of Rs. 10000 was deposited in a bank for a period of 27 months at the rate of 20\%
	
	\noindent  
\noindent \\ \includegraphics*[width=0.60in, height=0.52in]{images/image1} pa on compound interest. What will be the amount received on maturity?
	
	\noindent \begin{tabular}{p{1.7in} p{1.6in} p{1.6in}} \\ 
 1) Rs. 15120               &  2) Rs. 12400        &  3) Rs. 14260        \\
4) Rs. 12500        & 5) None of these  \\
\end{tabular}
	
	\noindent 
	
	\noindent 
	
	\noindent 
	
	\noindent \\  36. Amit has given a loan to Sumit an amount of Rs. 20000 at an interest rate 8 \% p.a. for a period of 30 months. If interest charged is at compound interest, how much does Sumit need to pay in the end?
	
	\noindent \begin{tabular}{p{1.7in} p{1.6in} p{1.6in}} \\ 
 1) Rs. 23000               &  2) Rs. 24000.36   &  3) Rs. 24261.12   \\
4) Rs. 25020.54   & 5) None of these  \\
\end{tabular}
	
	\noindent 
	
	\noindent 
	
	\noindent 
	
	\noindent  \\ 
	
	\noindent  \\ \textbf{Model 9: Difference between CI and SI for Different Rates of Interest}
	
	\noindent 
	
	\noindent 
	
	\noindent  \\ 37. What is the difference between compound interest and simple interest for the sum of
	
	\noindent 
	
	\noindent  
\noindent \\ \includegraphics*[width=0.60in, height=0.52in]{images/image1} Rs. 20000 over a 2 year period, if the compound interest is calculated at 20\% p.a. and simple interest is calculated at 23\% p.a.?
	
	\noindent 
	
	\noindent \begin{tabular}{p{1.7in} p{1.6in} p{1.6in}} \\ 
 1) Rs. 200                   &  2) Rs. 125         &  3) Rs. 250       \\
4) Rs. 400     & 5) None of these  \\
\end{tabular}
	
	\noindent 
	
	\noindent \\  38. Varun borrows Rs. 1500 from two money lenders. He pays interest at the rate of 12\% per
	
	\noindent  
\noindent \\ \includegraphics*[width=0.60in, height=0.52in]{images/image1} annum for one loan and at the rate of 14\% pa for the other. How much does he borrow at
	
	\noindent 12\% pa if the total interest paid at the end of the year is Rs. 186?
	
	\noindent 
	
	\noindent \begin{tabular}{p{1.7in} p{1.6in} p{1.6in}} \\ 
 1) Rs. 1200                 &  2) Rs. 1125          &  3) Rs. 1250          \\
4) Rs. 1800          & 5) None of these  \\
\end{tabular}
	
	\noindent  \\ 
	
	39. The simple interest on a sum of money will be Rs. 300 after 5 years. In the next 5 years if the  
\noindent \\ \includegraphics*[width=0.60in, height=0.52in]{images/image1} principal is trebled, then what will be the total interest at the end of the 10th year?
	
	\noindent \begin{tabular}{p{1.7in} p{1.6in} p{1.6in}} \\ 
 1) Rs. 1200                 &  2) Rs. 1125          &  3) Rs. 1250          \\
4) Rs. 1800          & 5) None of these  \\
\end{tabular}
	
	\noindent 
	
	\noindent 
	
	\noindent  \\ 40.  Rs. 800 becomes Rs. 956 in 3 years at a certain rate of simple interest. If the rate of interest is increased by 4\%, what amount will Rs. 800 become in 3 years?
	
	\noindent 
	

		\begin{tabular}{p{1.7in} p{1.6in} p{1.6in}} \\ 
 1) Rs. 1020.80  &  2) Rs. 1025 &  3) Rs. 1052
		\\
4) Data inadequate  & 5) None of these  \\
\end{tabular}

	
	
	
	\noindent 
	
	\noindent 
	
	\noindent 
	
	\noindent 
	
	\noindent 
	
	\noindent 
	
	\noindent 
	
	\noindent 
	
	\noindent 
	
	\noindent 
	
	\noindent \\  \textbf{Answers}
	
	\noindent 
	
	\noindent 
	
	\begin{tabular}{|p{0.7in}|p{0.5in}|p{0.5in}|p{0.5in}|p{0.5in}|p{0.5in}|p{0.5in}|p{0.5in}|p{0.5in}|p{0.5in}|} \hline 
		1 - 3 & 2 - 5 & 3 - 2 & 4 - 1 & 5 - 1 & 6 - 3 & 7 - 1 & 8 - 4 & 9 - 3 & 10 - 1 \\ \hline 
		11 - 3 & 12 - 3 & 13 - 4 & 14 - 3 & 15 - 1 & 16 - 5 & 17 - 1 & 18 - 2 & 19 - 4 & 20 - 1 \\ \hline 
		21 - 1 & 22 - 4 & 23 - 1 & 24 - 4 & 25 - 2 & 26 - 2 & 27 - 3 & 28 - 1 & 29 - 3 & 30 - 3 \\ \hline 
		31 - 3 & 32 - 1 & 33 - 4 & 34 - 1 & 35 - 1 & 36 - 3 & 37 - 4 & 38 - 1 & 39 - 1 & 40 - 3 \\ \hline 
	\end{tabular}
	
	
	
	\noindent 
	
	\noindent  \\ 
	
	\noindent \\  \textbf{Note: }The date and time mentioned against some questions refer to the doubts clarification session on Quantitative Aptitude in which the question was solved.
	
\end{document}