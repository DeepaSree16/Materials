% Generated by GrindEQ Word-to-LaTeX 
\documentclass{article} % use \documentstyle for old LaTeX compilers

\usepackage[utf8]{inputenc} % 'cp1252'-Western, 'cp1251'-Cyrillic, etc.
\usepackage[english]{babel} % 'french', 'german', 'spanish', 'danish', etc.
\usepackage{amsmath}
\usepackage{amssymb}
\usepackage{txfonts}
\usepackage{mathdots}
\usepackage[classicReIm]{kpfonts}
\usepackage{graphicx}
 
\usepackage{multirow}
\usepackage[margin=1.0in]{geometry}
\usepackage[english]{babel}
\usepackage[utf8]{inputenc}
\usepackage{fancyhdr}
\usepackage{tabularx}
\pagestyle{fancy}
\fancyhf{}
\rhead{ 
\noindent \\ \includegraphics[width=1.0in, height=0.38819in]{images/logo.png}}
\lhead{ Quantitative Aptitude: }
\lfoot{www.talentsprint.com }
\rfoot{\thepage}
% You can include more LaTeX packages here 



\begin{document}
	




\noindent \textbf{Profit and Loss}

\noindent 

\noindent \textbf{Additional Examples}

\noindent 

\noindent 

1.   A man sells two articles for Rs.5000 each neither losing nor gaining in the deal. If he sold one  
\noindent \\ \includegraphics*[width=0.60in, height=0.52in]{images/image1}of them at a gain of 25\%, the other article is sold at a loss of

\noindent a) 15 2/3\%              b) 16 2/3\%       c) 17 1/3\%        d) 18 1/3\%

\noindent 

\noindent 

\noindent 

\noindent  
\noindent \\ \includegraphics*[width=0.59in, height=0.52in]{images/image1}2.   By selling an article for Rs.144, a person gained such that the percentage gain equals the cost price of the article. The cost price of the article is

\noindent a) Rs.90                   b) Rs.80                         c) Rs.75                          d) Rs.60

\noindent 

\noindent 

\noindent 

3.   A dealer offered a machine for sale for Rs.27,500 but even if he had charged 10\% less, he  
\noindent \\ \includegraphics*[width=0.60in, height=0.52in]{images/image1}would have made a profit of 10\%. The actual cost of the machine is\_

\noindent a) Rs.22,000                         b) Rs.24,250     c) Rs.22,500     d) Rs.22,275

\noindent 

4.   A shopkeeper marks his good 20\% above his cost price and gives 15\% discount on the  
\noindent \\ \includegraphics*[width=0.60in, height=0.52in]{images/image1}marked price. His gain per cent is --

\noindent a) 5\%                      b) 4\%                c) 2\%                d) 1\%

\noindent 

\noindent 

5.   A man bought an article listed at Rs.1500 with a discount of 20\% offered on the list price.  
\noindent \\ \includegraphics*[width=0.60in, height=0.52in]{images/image1}What additional discount must be offered to the man to bring the net price to Rs.1104?

\noindent a) 8\%                      b) 10\%              c) 12\%              d) 15\%

\noindent 

\noindent 

\noindent  
\noindent \\ \includegraphics*[width=0.59in, height=0.52in]{images/image1}6.   A man sold 20 apples for Rs.100 and gained 20\%. How many apples did he but for Rs.100?

\noindent 

\noindent a) 20                        b) 22                 c) 24                  d) 25

\noindent 

\noindent 7.   The single discount equal to three consecutive discounts of 10\%, 12\% and 15\% is

\noindent  
\noindent \\ \includegraphics*[width=0.60in, height=0.52in]{images/image1}a) 37\%                    b) 24.76\%         c) 32.68\%         d) 41.68\%        e) None of these

\noindent 

\noindent 8.   A person bought an article at Rs.380 and spent 20\% of selling price on repairing and got 20\%

\noindent  
\noindent \\ \includegraphics*[width=0.60in, height=0.52in]{images/image1}profit. The cost price of the article would be

\noindent a) Rs.500                b) Rs.600          c) Rs.120          d) Rs.480          e) None of these

\noindent 

\noindent 

9.   Two partners.A and B invest in the ratio of 3:4. Their monthly expenditure is 36000. What  
\noindent \\ \includegraphics*[width=0.60in, height=0.52in]{images/image1}should be their income so that A gets 6000 as profit?

\noindent 10. The profit earned by selling an article for Rs. 590 is double what was earned when the same

\noindent  
\noindent \\ \includegraphics*[width=0.60in, height=0.52in]{images/image1}article was sold for Rs. 475.  What would be the selling price of the article, if it is sold at 20\%

\noindent profit?

\noindent 

\noindent 
\begin{tabular}{p{1.7in} p{1.6in} p{1.6in}} \\ 
	1) 432                      &  2) 436               &  3) 424               \\
4) 415               & 5) 445 \\
\end{tabular}

\noindent 

11. Sachin Karki bought 963 articles and sold 900 of them for the     price he paid for 963 articles  
\noindent \\ \includegraphics*[width=0.60in, height=0.52in]{images/image1}to Rohan Bagchi and sold the remaining articles at the same price per article as the other 900

\noindent to his brother. The percentage gain on the entire transaction is:

\noindent 

\noindent 
\begin{tabular}{p{1.7in} p{1.6in} p{1.6in}} \\ 
	1) 7.5\%                   &  2) 8\%                &  3) 9\%                \\
4) 7\%                & 5) None of these  \\
\end{tabular}

\noindent 

12. An article is sold at 24\% profit. If the cost price and selling price are less by Rs.12 and Rs. 8  
\noindent \\ \includegraphics*[width=0.60in, height=0.52in]{images/image1}respectively the profit percentage increases by 16\%. Find the cost price.

\noindent 
\begin{tabular}{p{1.7in} p{1.6in} p{1.6in}} \\ 
	1) 48                        &  2) 55                 &  3) 58                 \\
4) 56                 & 5) 50 \\
\end{tabular}

\noindent 

13. A shopkeeper buys 300 pens for Rs. 2400 out of which 220 are from distributor m and the  
\noindent \\ \includegraphics*[width=0.60in, height=0.52in]{images/image1}remaining from distributor n. At what price per pen must he sell the pens from distributor

\noindent m, so that if he sells the pens from distributor n at one quarter of that price, he makes profit

\noindent 

\noindent of 50\% on entire transaction.

\noindent 

\noindent 
\begin{tabular}{p{1.7in} p{1.6in} p{1.6in}} \\ 
	1) Rs. 16                 &  2) Rs. 14           &  3) Rs. 8             \\
4) Rs. 12           & 5) Rs. 15 \\
\end{tabular}

\noindent 

14. What will be  the  marked  price if the  difference between the  selling price  of an article  
\noindent \\ \includegraphics*[width=0.60in, height=0.52in]{images/image1}discounted at 30\% and selling price of another article discounted at 10\% and again 20\% is

\noindent Rs.144?

\noindent 

\noindent 

15. When  a  bicycle  manufacturer  reduced  its  selling  price  by  50\%,  the  no.  of  bicycle  sold  
\noindent \\ \includegraphics*[width=0.60in, height=0.52in]{images/image1} radically increased by 600\%. Initially the manufacturer was getting only 140\% profit. What

\noindent is the percentage increase of his profit?

\noindent 

\noindent 

\noindent 16. If percentage of profit made, when an article is sold for Rs.78, is twice as when it is sold for

\noindent  
\noindent \\ \includegraphics*[width=0.60in, height=0.52in]{images/image1}Rs.69, the cost price of the article is

\noindent 
\begin{tabular}{p{1.7in} p{1.6in} p{1.6in}} \\ 
	1) Rs.49                   &  2) Rs.51                         &  3) Rs.57                         \\
4) Rs.60 \\
\end{tabular}

\noindent 

\noindent 

\noindent 

\noindent  
\noindent \\ \includegraphics*[width=0.59in, height=0.52in]{images/image1}17. A man bought 25 crates of oranges for Rs.10, 000. He lost 5 crates. In order to earn a total profit of 25\% of the total cost, he would have to sell each of the remaining crates at

\noindent 
\begin{tabular}{p{1.7in} p{1.6in} p{1.6in}} \\ 
	1) Rs.650                 &  2) Rs.625          &  3) Rs.600          \\
4) Rs.575 \\
\end{tabular}

\noindent 

\noindent 

\noindent 

\noindent 

\noindent 

\noindent 

\noindent 

\noindent 

\noindent 18. A dishonest trader marks his goods up by 80\% and gives discount of 25\%. Besides he gets

\noindent  
\noindent \\ \includegraphics*[width=0.60in, height=0.52in]{images/image1}20\% more amount per kg from wholesaler and sells 10\% less per kg to customer. What is the overall profit percentage?

\noindent 

\noindent 
\begin{tabular}{p{1.7in} p{1.6in} p{1.6in}} \\ 
	1) 80\%                    &  2) 60\%              &  3) 70\%              \\
4) 50\% \\
\end{tabular}

\noindent 

\noindent 

\noindent 

\noindent 19. There is 210\% profit on the cost price of an item. If the cost price is increased by 40\%

\noindent  
\noindent \\ \includegraphics*[width=0.60in, height=0.52in]{images/image1}keeping the selling price constant, then the profit will be what percent of the selling price?

\noindent 

\noindent  
\noindent \\ \includegraphics*[width=0.59in, height=0.52in]{images/image1}20. Each of the two car is sold at same price. A profit of 10\% on first and a loss of 7\% made on second. What is the combined loss or gain percentage?

\noindent 

\noindent 

\noindent 

\noindent 21. Cost of an article varies linearly with the weight. The price of the article in its entire weight

\noindent 

\noindent  
\noindent \\ \includegraphics*[width=0.60in, height=0.52in]{images/image1} is Rs.8100. If the article was sold by breaking it into 3 pieces in the ratio of 1:3:5, find the loss incurred to the merchant

\noindent 

\noindent 
\begin{tabular}{p{1.7in} p{1.6in} p{1.6in}} \\ 
	1) 3600                    &  2) 5000             &  3) 2000             \\
4) 4600             & 5) 2700 \\
\end{tabular}

\noindent 

22. A person buys some articles. He sold 40\% articles at 20\% profit and remaining at 33(1/  3)\%. If  
\noindent \\ \includegraphics*[width=0.60in, height=0.52in]{images/image1} percent profit is calculated on selling price, then what is the ratio of selling price of articles

\noindent sold at 20\% profit to selling price of articles sold at 33(1/ 3)\% profit?

\noindent 

\noindent 

\noindent 

23. At what percentage above the cost price must an article be marked so as to gain 33\% after  
\noindent \\ \includegraphics*[width=0.60in, height=0.52in]{images/image1} allowing a customer a discount of 5\%?

\noindent 
\begin{tabular}{p{1.7in} p{1.6in} p{1.6in}} \\ 
	1) 48\%                    &  2) 43\%              &  3) 40\%              \\
4) 38\%              & 5) None of these  \\
\end{tabular}

24. A balance of a trader weighs 10 \% less than it should be. Still the trade marks-up his goods  
\noindent \\ \includegraphics*[width=0.60in, height=0.52in]{images/image1} to get the overall profit of 20\%. What is the markup on the cost price?

\noindent 
\begin{tabular}{p{1.7in} p{1.6in} p{1.6in}} \\ 
	1) 40\%                    &  2) 8\%                &  3) 25\%              \\
4) 16.66\%         & 5) None of these  \\
\end{tabular}

\noindent 

\noindent 

25. A man bought 500 m of electronic wire at 50 paise/m. He sold 50\% of it at a profit of 5\%. At  
\noindent \\ \includegraphics*[width=0.60in, height=0.52in]{images/image1} what percent should he sell the remainder so as to gain 10\% on the whole transaction

\noindent 

26. Sudhira sold a watch at a profit of 20\%. Had she bought it at 10\% less and sold it at Rs. 30  
\noindent \\ \includegraphics*[width=0.60in, height=0.52in]{images/image1} less, she would still have gained 20\%. Find the cost price of the watch.

\noindent 

27. A dealer sold three-fourth of his articles at a gain of 20\% and the remaining at cost price.  
\noindent \\ \includegraphics*[width=0.60in, height=0.52in]{images/image1}Find the gain earned by him in the whole transaction

\noindent 

\noindent 

\noindent 

\noindent 28. The ratio of c.p \& m.r.p=2:3 \& ratio of discount \& profit= 3:2 Then find profit percentage?

\noindent  
\noindent \\ \includegraphics*[width=0.60in, height=0.52in]{images/image1}

\noindent 

\noindent  
\noindent \\ \includegraphics*[width=0.59in, height=0.52in]{images/image1}29. Sumant wants to sell a mobile at a profit of 20\%. He bought it at 10\% less and sold it at Rs.60

\noindent 

\noindent less, still he gained 20\%. The cost price of mobile is

\noindent 

30. The price of an article is cut by 3\%. To restore to its original value, the new price must be  
\noindent \\ \includegraphics*[width=0.60in, height=0.52in]{images/image1}increased by

\noindent 
\begin{tabular}{p{1.7in} p{1.6in} p{1.6in}} \\ 
	1) 3\%                      &  2) 7.11\%           &  3) 3.09\%           \\
4) 2.69\% \\
\end{tabular}

\noindent 

\noindent 31. A seller mark the price of 50\% above the cost price and gives 10\% discount on the item.

\noindent 

\noindent While selling the cheats customer by giving 20\% less in weight. Find the overall profit percent(approximate)?

\noindent 
\begin{tabular}{p{1.7in} p{1.6in} p{1.6in}} \\ 
	1) 26\%                    &  2) 65\%              &  3) 68\%              \\
4) 72\%              & 5)76\%  \\
\end{tabular}

\noindent 

\noindent 

\noindent 32. A man bought mangoes at the rate of 9 for Rs.35 and sold them at the rate of 10 for Rs.55. how many mangoes should he sell to earn a net profit of Rs.58?

\noindent 
\begin{tabular}{p{1.7in} p{1.6in} p{1.6in}} \\ 
	1) 36                        &  2) 29                 &  3) 30                 \\
4) 72                 & 5) None of these  \\
\end{tabular}

\noindent 

\noindent 33. The marked price of an article is Rs.5000. But due to a special festival offer a certain percent  
\noindent \\ \includegraphics*[width=0.60in, height=0.52in]{images/image1} of discount is declared. Mr. X availed this opportunity and bought the article at reduced price. He  then sold it at Rs.5000 and thereby made a profit of 11 1/9\%. The percentage of discount allowed was?

\noindent 

\noindent a)  10                       b) 3 1/3             c) 7 1/2             d) 11 1/9

\noindent 

\noindent 34. If the profit of two articles A \& B are equal, then what will be ratio of discounts on both of 

 
\noindent \\ \includegraphics*[width=0.60in, height=0.52in]{images/image1} these items

\noindent a) 4:7                       b) 15:28            c) 7:4                 d) 28:15            e) Can't determined

\noindent 

\noindent 

\noindent 
 
\noindent \\ \includegraphics*[width=0.59in, height=0.52in]{images/image1}35. 10\%  discount  is  offered  on  an  item.  By  applying  a  promo  code  a  customer  wins  4\%

\noindent 

\noindent cashback. What is the effective discount

\noindent 

\noindent 

\noindent 36. Monika purchased a pressure cooker at 9/10th its selling price and sold it at 8\% more than its

\noindent 
 
\noindent \\ \includegraphics*[width=0.60in, height=0.52in]{images/image1} SP. Find her gain per cent

\noindent a) 10                        b) 20                 c) 30                  d) 40

37. A tradesman gives 4\% discount on the marked price and gives 1 article free for buying  
\noindent \\ \includegraphics*[width=0.60in, height=0.52in]{images/image1}every 15 articles and thus gain 35\%. The marked price is above the cost price by

\noindent a) 20\%                    b) 39\%              c) 40\%              d) 50\%

38. Mr. Rajbir buys two articles for Rs. 800. He sells first article at a profit of 15\% and second  
\noindent \\ \includegraphics*[width=0.60in, height=0.52in]{images/image1}article at a loss of 5\%. After doing both the transactions Mr. Rajbir realizes that he has

\noindent neither earned profit nor did he incur any loss. At what cost (in Rs.) did Rajbir sell the loss

\noindent 

\noindent making article?

\noindent 

\noindent a) 550                      b) 500               c) 650                d) 600               e) 570

\noindent 

\noindent 

\noindent 

39. Cheap and Best, a Kirana shop bought some apples at 4 per rupee and an equal number at 5  
\noindent \\ \includegraphics*[width=0.60in, height=0.52in]{images/image1} per rupee. He then sold the entire quantity at 9 for 2 rupees. What is his percentage profit or

\noindent 

\noindent 

\noindent 

\noindent 

\noindent 

40. The profit earned when an article is sold for 800 is 20 times the loss incurred when it is sold  
\noindent \\ \includegraphics*[width=0.60in, height=0.52in]{images/image1} for 275. At what price should the article be sold if it is desired to make a profit of 25\%?

\noindent 
\begin{tabular}{p{1.7in} p{1.6in} p{1.6in}} \\ 
	1) 300                      &  2) 350               &  3) 375               \\
4) 400               & 5) None of these  \\
\end{tabular}

\noindent 

\noindent 

41. 2 kg of tea and 3 kg of sugar together costs Rs.39. The price of tea has risen by 25\% and that  
\noindent \\ \includegraphics*[width=0.60in, height=0.52in]{images/image1} of sugar by 20\%. Hence the same quantities of tea and sugar now cost Rs.48.30. Find the

\noindent original price of tea per kg

\noindent 

\noindent a) Rs.14.90/kg        b) Rs.15.00/kg c) Rs.16.00/kg  d) Rs.14.40/kg e) None of these

\noindent 

\noindent 42. A dishonest vendor professes to sell fruits at the cost price but he uses a weight of 800 grams

\noindent  
\noindent \\ \includegraphics*[width=0.60in, height=0.52in]{images/image1} in lieu of 1 kg weight. Find his percentage gain

\noindent a) 22\%                    b) 24\%              c) 25\%              d) 30\%              e) 20\%

\noindent 

\noindent 

\noindent 

\noindent 

43. A trader sells an article at a loss of 8\%, but when he increases the selling price of the article  
\noindent \\ \includegraphics*[width=0.60in, height=0.52in]{images/image1} by 164 he earns a profit of 2.25\% on the cost price. If he sells the same article for Rs.1760,

\noindent what is his profit percentage?

\noindent 

\noindent a) 12\%                    b) 10\%              c) 14\%              d) 18\%              e) None of these

\noindent 

\noindent 

\noindent  \\ 

44. 65\% of goods are sold at 2\% loss while the remaining are sold at 16\% profit. If there is a total  
\noindent \\ \includegraphics*[width=0.60in, height=0.52in]{images/image1} profit of Rs.430, the worth of goods sold is

\noindent a) Rs.11650             b) Rs.12850      c) Rs.10000      d) Rs.9658        e) None of these

\noindent 

\noindent 

\noindent 

\noindent \\  45. A man bought a tractor and a trolley for Rs.450000. He sold the trolley and the tractor at

\noindent 

\noindent  
\noindent \\ \includegraphics*[width=0.60in, height=0.52in]{images/image1}10\% and 25\% profit respectively, thereby making 15\% profit on the whole. Find the cost price of the tractor

\noindent 

\noindent a) Rs.300000           b) Rs.325000    c) Rs.296500    d) Rs.306000    e) None of these

\noindent 

\noindent  \\ 

46. \textbf{Directions (I - V): }In a sale at hotel XYZ, Manav bought a clothing item, some of his friends 
 
\noindent \\ \includegraphics*[width=0.60in, height=0.52in]{images/image1} bought dresses as well as other items and all were happy to have saved some money. The

\noindent following information is available.

\noindent 

\noindent \\  1.   A bicycle was bought at a discount of 50\% by a person whose name starts with H.

\noindent 

\noindent \\  2.   Chandra bought the 1500 item for 80\% of its value.

\noindent 

\noindent \\  3.   The tires were sold at Rs.100 less than their original price.

\noindent 

\noindent \\  4.   A clothing item was sold for Rs.50

\noindent 

\noindent \\  5.   Pradeep spent Rs.400 less than Chandra.

\noindent 

\noindent \\  6.   Harry paid for his dress with a 1000 Rupee note and received Rs.925 in change


\noindent \\  7.  Manav spent less for his item than Sanjay, who spent less than Pradeep. \\
\noindent \\	 8.  The highest price item did not sell at the highest price, nor did the lowest price item sell at the lowest price. \\
\noindent \\	 9. A dress, a sweater, dresser, a mobile phone, tires and a bicycle were sold not necessarily respectively to Manav, Pradeep, Sanjay, Chandra, Harry and Harendra. The prices paid for the six items were Rs.800, 600, 1200, 1000, 75 and 50 and their original prices were Rs.200, 1500, 300, 2000, 900 and 1200 (again not necessarily respectively). \\  

\noindent \\ I.  Who has been looking for a dresser for a long time? \\  


a) Harry                                     b) Manav                                   c) Pradeep d) Harendra                              e) Someone other than these   three

\noindent 

\noindent 

\noindent  \\ II.         The discount received by Pradeep amounted to:

\noindent 

\noindent a) Rs.150          b) Rs.125          c) Rs.100           d) Rs.50            e) Rs.200

\noindent 

\noindent 

\noindent 

\noindent \\  III.        What item did Harendra buy?

\noindent 

	a) Dress  b) Sweater  c) Tires 
	d) Mobile phone e) None of these    




\noindent 

\noindent 

\noindent \\  IV.       How much did Chandra spend?

\noindent 

\noindent a) Rs.800          b) Rs.600          c) Rs.1200         d) Rs.500      e) None of these







\noindent \\  V.         Which of the following discount was not got by any of the six?

\noindent 

\noindent a) Rs.125          b) Rs.600          c) Rs.250           d) Rs.500          e) Rs.100

\noindent 

\noindent 

\noindent 

\noindent \\  47. Ram purchased a toy and sold it at a loss of 15\%. Had he bought it for 25\% less and sold it

\noindent 

\noindent for Rs. 117 more, he would have earned a profit of 35\%. What is the cost price of the toy?

\noindent 

\noindent a) Rs.720                b) Rs.745          c) Rs.825          d) Rs.840          e) Rs.790

\noindent 

\noindent 

\noindent 

\noindent \\  48. A businessman marked the price of his goods 40\% more than his cost price. He then sells

\noindent 

\noindent 1/4th of his stock at a discount of 20\% and 1/2th of the stock at the marked price, and the rest at a discount of 60\%. Find his gain percentage.

\noindent a) 10\%                    b) 20\%              c) 15\%              d) 12\%              e) 18\%

\noindent 

\noindent 

\noindent  \\ 49. Tinkawala purchased the article for Rs.123684. He sold 60\% of those at a profit of 16.66\%

\noindent 

\noindent and rest at a loss. Find the loss percentage of the remaining if the overall loss is 14\%? A) 20\%                   B) 30\%              C) 60\%             D) 66.66\%

\noindent 

\noindent 

\noindent \\  50. A shopkeeper sold a TV at a profit of 35\%. Had he sold it for Rs 528 less, he would have gained 24\%. For what price should he sell it in order to gain 40\% ?

\noindent a) Rs. 6040             b) Rs. 6480       c) Rs. 6720       d) Rs. 7210      e) None of these

\noindent 

\noindent 

\noindent \\  51. A person sells two fans for Rs.3200. The cost price of first fan is equal to the selling price of the second fan. If the first is sold at 40\% loss and the second at 100\% profit, what is the total profit or loss?

\noindent 
\begin{tabular}{p{1.7in} p{1.6in} p{1.6in}} \\ 
	1) 160         &  2) 200               &  3) 240               \\
4) 280               & 5) None of these  \\
\end{tabular}

\noindent 

\noindent 

\noindent \\  52. Mukesh buys oranges at the price of Rs. 15 a dozen. 25\% of the oranges get spoilt during the transportation process. At what price must he sell an orange if he wants to realise a profit percentage of 80\%?

\noindent a) 2             b) 2.5                c) 3                    d) 3.5                e) 4

\noindent 

\noindent 

\noindent 

\noindent 

\noindent  \\ 53. Sonu sold an article to Monu at 35\% profit. Monu spent 3780 rupee in repairing and sold it

\noindent 

\noindent to Ravi at 18.18\% loss. Now cp of article for Ravi is 20\% more than the S.P of article for Sonu. Had Sonu sold the article at 30\% loss, then what will be the S.P of article?

\noindent 

\noindent 

\noindent \\ 54. A grain dealer cheats to the extent of 10\% while buying as well as selling, by using false weights. His total gain is?



\noindent 

\end{document}