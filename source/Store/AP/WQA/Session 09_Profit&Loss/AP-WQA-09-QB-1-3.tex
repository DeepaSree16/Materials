


\documentclass{article} 

\usepackage[utf8]{inputenc} 
\usepackage[english]{babel}
\usepackage{amsmath}
\usepackage{amssymb}
\usepackage{txfonts}
\usepackage{mathdots}
\usepackage[classicReIm]{kpfonts}
\usepackage{graphicx}

\usepackage{multirow}
\usepackage[margin=1.0in]{geometry}
\usepackage[english]{babel}
\usepackage[utf8]{inputenc}
\usepackage{fancyhdr}
\usepackage{tabularx}
\pagestyle{fancy}
\fancyhf{}
\rhead{\includegraphics[width=1.0in, height=0.38819in]{images/logo.png}}
\lhead{ Quantitative Aptitude: Profit and Loss}
\lfoot{www.talentsprint.com }
\rfoot{\thepage}
\begin{document}
	
	\noindent 
	
	\noindent 
	
	\noindent 
	
	\noindent 

	
	\noindent \begin{center}
		{\Large \textbf{Profit and Loss \\}}
	\end{center}
	
	\noindent 
	
	\noindent 
	
	\noindent 
	
	\noindent 
	
	\noindent {\large \textbf{Part 1 - Basic \\}}
	
	\noindent 
	
	\noindent \\ \textbf{Model 1: Basic Profit/Loss}
	
	\noindent 
	
	\noindent \\ 1.   An Umbrella was sold at a profit of 20\%. What is the selling price of the Umbrella, if its cost price is Rs. 180?
	
	
	\noindent  \\ 
	\begin{tabular}{p{1.7in} p{1.6in} p{1.6in}} \\ 
 1) Rs. 216                   & 2) Rs. 200            & 3) Rs. 160            \\
4) Rs. 36              & 5) None of these \\
\end{tabular}
	
	\noindent 
	
	\noindent 
	
	\noindent 
	
	\noindent  \\ 2.   A person sold an article for Rs. 20 and earned a profit of 25\%. What is the cost price of the article?
	
	\noindent  \\ 
	\begin{tabular}{p{1.7in} p{1.6in} p{1.6in}} \\ 
 1) Rs. 16                     & 2) Rs. 14              & 3) Rs. 12              \\
4) Rs. 18              & 5) None of these \\
\end{tabular}
	
	\noindent 


	\noindent 
	
	\noindent 
	
	\noindent  \\ 3.   An article was sold for Rs. 13,000 at a loss of 35\%. What is the cost price of the article?
	
	\noindent 
	


	\noindent  \\ 
	\begin{tabular}{p{1.7in} p{1.6in} p{1.6in}} \\ 
 1) Rs. 16,000             & 2) Rs. 13,700       & 3) Rs. 15,000       \\
4) Rs. 20,000       & 5) None of these \\
\end{tabular} 
	
	\noindent 
	
	\noindent 
	
	\noindent 
	
	\noindent \\  4.   Harshad bought 15 DVD players at Rs. 4,500 each and sold all of them at the total price of
	
	\noindent 
	
	\noindent Rs. 81,000. What is the percent profit earned in the deal?
	
	\noindent 
	
	\noindent  \\ 
	\begin{tabular}{p{1.7in} p{1.6in} p{1.6in}} \\ 
 1) 16                        & 2) 20                 & 3) 25                 \\
4) 20.5              & 5) None of these \\
\end{tabular}
	
	\noindent 
	
	\noindent 
	
	\noindent 
	
	\noindent  \\ 5.   A shopkeeper sells 200 m of cloth for Rs. 9,000 at a profit of Rs. 5 per m. What is the cost price of 1 m of cloth?
	
	\noindent 
	
	\noindent  \\ 
	\begin{tabular}{p{1.7in} p{1.6in} p{1.6in}} \\ 
 1) Rs. 45                     & 2) Rs. 40              & 3) Rs. 35              \\
4) Rs. 30              & 5) None of these \\
\end{tabular}
	
	\noindent \\  

\noindent 6.   Naresh purchased a TV set for Rs. 11,250 after getting a discount of 10\% on the labeled price. He spent Rs. 150 on transport and Rs. 800 on installation. At what price should it be sold so that the profit earned would be 15\%?              \textbf{[October 18, 2014 @ 29m 15s]}
	
	\noindent 
	
	\noindent \includegraphics*[width=0.60in, height=0.52in]{images/image1}
 \\ 	\begin{tabular}{p{1.7in} p{1.6in} p{1.6in}} \\ 
 1) Rs. 12,937.50         & 2) Rs. 14,030       & 3) Rs. 13,450       \\
4) Rs. 15,467.50  & 5) None of these \\
\end{tabular}
	
	\noindent 
	
	\noindent 
	
	\noindent 
	
	\noindent \\  7.   Manoj sold an article for Rs. 15,000. Had he offered a discount of 10\% on the selling price, he would have earned a profit of 8\%? What is the cost price?
	
	\noindent \includegraphics*[width=0.60in, height=0.52in]{images/image1}
	\begin{tabular}{p{1.7in} p{1.6in} p{1.6in}} \\ 
 1) Rs. 12,500              & 2) Rs. 13,500       & 3) Rs. 12,250       \\
4) Rs. 13,250       & 5) None of these \\
\end{tabular}
	
	\noindent 
	
\noindent \\ 	8.   If Ramu buys books at 11 books for Rs. 10 and sells at 10 books for Rs. 12, then what will be his gain percent?        \textbf{[August 02, 2014 @ 36m 15s]}
	
	\noindent \includegraphics*[width=0.60in, height=0.52in]{images/image1}
	\begin{tabular}{p{1.7in} p{1.6in} p{1.6in}} \\ 
 1) 11\%                    & 2) 22\%              & 3) 32\%              \\
4) 15\%              & 5) None of these \\
\end{tabular}
	
	\noindent 
	
	\noindent 
	
	\noindent 
	
	\noindent \textbf{Model 2: SP/CP with Respect to Two Different Profit/Loss Percentages}
	
	\noindent 
	
	\noindent  \\ 9.   A watch was sold at a loss of 9\%. It was observed that if the selling price was Rs. 420 more, the profit made would have been 5\%. What is the actual selling price of the watch?
	
	\noindent \includegraphics*[width=0.60in, height=0.52in]{images/image1}
	\begin{tabular}{p{1.7in} p{1.6in} p{1.6in}} \\ 
	1)2700                 & 2) Rs. 2730          & 3) Rs. 3270          \\
4) Rs. 3000                       & 5) None of these \\
\end{tabular}
	
	\noindent 
	
	\noindent 
	
	\noindent  \\ 10. After selling a book, Rohan found that he had made a loss of 12\%. He also found that had he sold it for Rs. 36 more, he would have made a profit of 6\%. What was the initial loss?
	
	\noindent 
	
	\begin{tabular}{|p{1.9in}|p{1.5in}|p{0.6in}|} \hline 

 1) Rs. 12  & 2) Rs. 18 & 3) Rs. 20 \\ \hline 
		\\
4) Data inadequate  & 5) None of these &  \\ \hline 
	\end{tabular}
	
	
	
	\noindent 
	
	\noindent  \\ 11. Praveen sold an article for Rs. 1170 at a profit of 30\%. What should be the selling price if the desired profit is 40\%?
	
	\noindent 
	\begin{tabular}{p{1.7in} p{1.6in} p{1.6in}} \\ 
 1) Rs. 1330                 & 2) Rs. 990            & 3) Rs. 1287          \\
4) Rs. 1260          & 5) None of these \\
\end{tabular}
	
	\noindent 
	
	\noindent 
	
	\noindent 
	
	\noindent  \\ 12. Sameer sold an article Rs. 460 and earned a profit of 15\%. At what price should it have been sold so as to earn a profit of 20\%?
	
	\noindent 
	\begin{tabular}{p{1.7in} p{1.6in} p{1.6in}} \\ 
 1) Rs. 465                   & 2) Rs. 480            & 3) Rs. 498            \\
4) Rs. 485            & 5) None of these \\
\end{tabular}
	
	\noindent 
	
	\noindent 
	
	\noindent 
	
	\noindent \textbf{Model 3: CP Based on Profit and Loss Relationship}
	
	\noindent 
	
	\noindent  \\ 13. The profit earned by selling a phone for Rs. 18,000 is the same as the loss incurred by selling it for Rs. 16,800. What is the cost price of the phone?
	
	\noindent 
	\begin{tabular}{p{1.7in} p{1.6in} p{1.6in}} \\ 
 1) Rs. 17,400              & 2) Rs. 17,000       & 3) Rs. 17,500       \\
4) Rs. 17,600       & 5) None of these \\
\end{tabular}
	
	\noindent 
	
	\noindent 
	
	\noindent 
	
	\noindent \\  14. The profit earned by selling an article for Rs. 625 is the same as the loss incurred by selling the article for Rs. 435. What is the cost price of the article?
	
	\noindent 
	\begin{tabular}{p{1.7in} p{1.6in} p{1.6in}} \\ 
 1) Rs. 530                   & 2) Rs. 520            & 3) Rs. 540            \\
4) Rs. 550            & 5) None of these \\
\end{tabular}
	
	\noindent 
	
	\noindent 
	
	\noindent \\  15. The profit earned by selling an article for Rs. 536 is the same as the loss incurred after the article for Rs. 426. What is the cost price of the article?
	
	\noindent 
	\begin{tabular}{p{1.7in} p{1.6in} p{1.6in}} \\ 
 1) Rs. 448                   & 2) Rs. 470            & 3) Rs. 481            \\
4) Rs. 500            & 5) None of these \\
\end{tabular}
\newpage
\noindent \\ 	16. The profit earned by selling a shirt for Rs. 1200 is twice the loss incurred when the shirt is sold for Rs. 600. What is the cost price of the shirt?
	
	\noindent \includegraphics*[width=0.61in, height=0.52in]{images/image1}
	\begin{tabular}{p{1.7in} p{1.6in} p{1.6in}} \\ 
 1) Rs. 800                   & 2) Rs. 1000          & 3) Rs. 900            \\
4) Rs. 750            & 5) None of these \\
\end{tabular}
	
	\noindent 
	
	\noindent 
	
	\noindent 
	
	\noindent \\  17. The profit earned by selling a wrist watch for Rs. 5800 is twice the loss incurred after selling the wrist watch for Rs. 4300. What is the cost price of the wrist watch?
	
	\noindent \begin{tabular}{p{1.7in} p{1.6in} p{1.6in}} \\ 
 1) Rs. 5300                 & 2) Rs. 5100          & 3) Rs. 4900          \\
4) Rs. 4800          & 5) None of these \\
\end{tabular}
	
	\noindent 
	
	18. The profit earned by selling an article for Rs. 4080 is half the loss incurred after selling the same article for Rs. 3660. What is the cost price of the article?       \textbf{[August 02, 2014 @ 41m 22s]}
	
	\noindent \includegraphics*[width=0.60in, height=0.52in]{images/image1}
	\begin{tabular}{p{1.7in} p{1.6in} p{1.6in}} \\ 
 1) Rs. 3785                 & 2) Rs. 3800          & 3) Rs. 3775          \\
4) Rs. 3940          & 5) None of these \\
\end{tabular}
	
	\noindent 
	
	\noindent 
	
	\noindent \textbf{Model 4: Profit/Loss Percentage Based on Quantity Sold}
	
	\noindent 
	
	\noindent \\  19. Sridhar sold 16 pens at the cost of 20 pens. What is the profit or loss percentage made by him?
	
	\noindent \begin{tabular}{p{1.7in} p{1.6in} p{1.6in}} \\ 
 1) 4\% profit            & 2) 4\% loss        & 3) 25\% profit   \\
4) 25\% loss      & 5) Cannot be determined  \\
\end{tabular}
	
	\noindent 
	
	\noindent 
	
	\noindent 
	
	\noindent \\  20. Prem sold 10 pens at the cost of 12 similar pens. What \% profit or loss does he make in this transaction?
	
	\noindent \begin{tabular}{p{1.7in} p{1.6in} p{1.6in}} \\ 
 1) 20\% profit                                                 & 2) 25\% profit                             & 3) 16.66\% loss
	
	\noindent 
	
	\noindent \\
4) 20\% loss                                                    & 5) None of these \\
\end{tabular}
	
	\noindent 
	
	\noindent  \\ 21. Aniruddh sold 21 books at the cost price of 18 books. Find the percentage profit or loss in
	
	\noindent 
	
	\begin{tabular}{|p{1.8in}|p{1.6in}|p{1.0in}|} \hline 

	
 1) 14.28\% profit & 2) 14.28\% loss  & 3) 16.66\% profit \\
4) 16.66\% loss & 5) None of these   \\ \hline 
	\end{tabular}
	
	
	
	\noindent 
	
	\noindent 
	
	\noindent \\  22. Ajit sold 20 apples at the cost price of 16 apples. What profit/loss does he make?
	
	\noindent 
	
	\noindent \begin{tabular}{p{1.7in} p{1.6in} p{1.6in}} \\ 
 1) Rs. 10                                                            & 2) Cannot be determined        & 3) Rs. 5
	
	\noindent 
	
	\noindent \\
4) Rs. 6                                                              & 5) None of these \\
\end{tabular}
	
	\noindent 
	
	\noindent 
	
	\noindent 
	
	\noindent 
	
	\noindent 
	
	\noindent \textbf{Model 5: Overall Profit/Loss Percentage When S1 = S2 and \%P = \%L}
	
	\noindent 
	
	\noindent 
	
	\noindent \\  23.  A shopkeeper bought two ceiling fans for Rs. 800 each. He sold one fan at a profit of 12\% and
	
	\noindent \includegraphics*[width=0.60in, height=0.52in]{images/image1}the other at a loss of 12\%. What would be his overall profit or loss in the transaction?
	
	\noindent 
	\begin{tabular}{p{1.7in} p{1.6in} p{1.6in}} \\ 
 1) No Profit No loss                                    & 2) Loss 1.44\%                            & 3) Profit 1.44\%
	
	\noindent 
	
	\noindent \\
4) Loss 4\%                                                     & 5) None of these \\
\end{tabular}
	
	\noindent 
	
	\noindent 
	
	\noindent 
	\newpage
	\noindent \\  24. A person bought two articles for the same price and sold them at a profit of 10\% on one and
	
	\noindent 
	
	\noindent a loss of 10\% on the other. What is the overall profit or loss \% made by him?
	
	\noindent 
	
	\begin{tabular}{p{1.9in} p{1.5in} p{0.9in}|}  

 1) 1\% Loss & 2) 1\% Profit & 3) Profit 4\% \\  
		
4) No Profit No loss & 5) None of these &  \\  
	\end{tabular}
	
	
	
	\noindent  \\ 	25. A person sold two articles for Rs. 1200 each; he made a profit of 20\% and a loss of 20\% on the  other. What will be the overall profit or loss in percentage?
	
	\noindent 
	
	\noindent 
	
	\noindent 
	
	\noindent 
	
	\noindent 
	
	\noindent  \\ 26. If a person sold two articles at the same price and realized 10\% profit on one and 10\% loss on the other article. What net profit/loss \% does he make?
	
	\noindent 
	
	\begin{tabular}{p{1.9in} p{1.5in} p{0.9in}|}  

 1) No Profit No loss & 2) Loss 1\% & 3) Profit 1\% \\  
		
4) Loss 10\% & 5) None of these &  \\  
	\end{tabular}
	
	
	
	\noindent 
	
	\noindent 
	
	\noindent \textbf{Model 6: Effective Percentage Problem}
	
	\noindent 
	
	\noindent \\  27. A sold an article to B at a profit of 20\% and B sold the same article to C at a profit of 10\%. If
	
	\noindent \includegraphics*[width=0.60in, height=0.52in]{images/image1} C bought it for Rs. 2640, how much did A pay for it?
	
	\noindent \begin{tabular}{p{1.7in} p{1.6in} p{1.6in}} \\ 
 1) Rs. 2000                 & 2) Rs. 1500          & 3) Rs. 1600          \\
4) Rs. 1540          & 5) None of these \\
\end{tabular}
	
	\noindent 
	
	\noindent 
	
	\noindent 
	
	\noindent  \\ 28. Prathik sold a music system to Karthik at 20\% gain and Karthik sold it to Swasthik at 40\% gain. If Swasthik paid Rs. 10,500 for the music system, what amount did Prathik pay for the same?
	
	\noindent \begin{tabular}{p{1.7in} p{1.6in} p{1.6in}} \\ 
 1) Rs. 8,240                                                       & 2) Rs. 7,500                                    & 3) Rs. 6,250
	
	\noindent 
	
	\noindent \\
4) Cannot be determined                            & 5) None of these \\
\end{tabular}
	
	\noindent 
	
	\noindent 
	
	\noindent 
	
	\noindent \\  29. Prakash sold an article to Prem at 10\% profit. Prem sold the same to Raj at 10\% loss. If
	
	\noindent 
	
	\noindent Prakash bought the article at Rs. 2000, at what price did Raj buy the same?
	
	\noindent 
	
	\noindent \begin{tabular}{p{1.7in} p{1.6in} p{1.6in}} \\ 
 1) Rs. 2,000                & 2) Rs. 1,980         & 3) Rs. 2,500         \\
4) Rs. 1,800         & 5) None of these \\
\end{tabular}
	
	\noindent 
	
	\noindent 
	
	\noindent 
	
	\noindent 
	
	\noindent \textbf{Model 7: Dishonest Dealer Problem}
	
	\noindent 
	
	\noindent \\  30. A dishonest dealer claims to sell his goods at the cost price but uses a weight of 800 gm instead of 1 kg. What will be the profit percentage in this transaction?
	
	\noindent \begin{tabular}{p{1.7in} p{1.6in} p{1.6in}} \\ 
 1) 15\%                    & 2) 20\%              & 3) 25\%              \\
4) 32\%              & 5) None of these \\
\end{tabular}
	
	\noindent 
	
	\noindent 
	
	\noindent 
	
	\noindent  \\ 31. A milkman sells milk at the cost price but delivers only 1800 ml instead of 2 liters. What will be the profit percentage made by the milkman?
	
	\noindent \begin{tabular}{p{1.7in} p{1.6in} p{1.6in}} \\ 
 1) 15\%                    & 2) 11.11\%         & 3) 20\%              \\
4) 25\%              & 5) None of these \\
\end{tabular}
	
	\noindent 
	
	\noindent 
	
	\noindent 
	
	\noindent 
	\newpage
	\noindent \textbf{Model 8: Marked Price and Discount}
	
	\noindent 
	
	\noindent \\  \includegraphics*[width=0.59in, height=0.52in]{images/image1}
	32. An article was sold for Rs. 480 after a discount of 20\%. What is the marked price?
	
	\noindent 
	
	\begin{tabular}{p{2.0in} p{1.4in} p{0.7in}} 

 1) Rs. 400 & 2) Rs. 576 & 3) Rs. 600 \\  
		
4) Cannot be determined & 5) None of these &  \\  
	\end{tabular}
	
	
	
	\noindent 
	
	\noindent \\  33. If the selling price of Rs. 24 results in a 20\% discount on the list price, then what selling price would result in a 30\% discount on list price?
	
	\noindent \begin{tabular}{p{1.7in} p{1.6in} p{1.6in}} \\ 
 1) Rs. 21                     & 2) Rs. 24              & 3) Rs. 18              \\
4) Rs. 15              & 5) None of these \\
\end{tabular}
	
	\noindent 
	
	\noindent 
	
	\noindent 
	
	\noindent  \\ 34. In a sale, a perfume is available at a discount of 15\% on the selling price. If the perfume's discounted selling price is Rs. 3740, what was the original selling price of the perfume?
	
	\noindent \begin{tabular}{p{1.7in} p{1.6in} p{1.6in}} \\ 
 1) Rs. 4,324                & 2) Rs. 4,386         & 3) Rs. 4,400         \\
4) Rs. 4,294         & 5) None of these \\
\end{tabular}
	
	\noindent 
	
	\noindent 
	
	\noindent \\  35. A shopkeeper marks his goods in such a way that even after allowing a discount of 20\%, he
	
	\noindent \includegraphics*[width=0.60in, height=0.52in]{images/image1} makes a profit of 12\%. How much percent above the cost price is the marked price?
	
	\noindent \begin{tabular}{p{1.7in} p{1.6in} p{1.6in}} \\ 
 1) 32\%                    & 2) 8\%                & 3) 12\%              \\
4) 40\%              & 5) None of these \\
\end{tabular}
	
	\noindent 
	
	\noindent 
	
	\noindent \\  36. A shopkeeper marks his goods in such a way that after allowing a discount of 10\%, he gains
	
	\noindent 
	
	\noindent 17\%. How much percent above C.P. is the marked price?
	
	\noindent 
	
	\noindent \begin{tabular}{p{1.7in} p{1.6in} p{1.6in}} \\ 
 1) 50\%                    & 2) 30\%              & 3) 27\%              \\
4) 7\%                & 5) None of these \\
\end{tabular}
	
	\noindent  \\ 
	
	37. At what price should a shopkeeper mark a radio that cost him Rs. 1200 in order that he may \includegraphics*[width=0.60in, height=0.52in]{images/image1}offer a discount of 20\% on the marked price and still make a profit of 25\%?
	
	\noindent \textbf{[August 02, 2014 @ 43m 00s]}
	
	\noindent 
	
	\noindent \begin{tabular}{p{1.7in} p{1.6in} p{1.6in}} \\ 
 1) Rs. 1675                 & 2) Rs. 1875          & 3) Rs. 1900          \\
4) Rs. 2025          & 5) None of these \\
\end{tabular}
	
	\noindent 
	
	\noindent 
	
	\noindent 
	
	\noindent  \\ 38. Sanjay bought a microwave oven and paid 10\% less than the original price. He sold it with
	
	\noindent 
	
	\noindent \includegraphics*[width=0.60in, height=0.52in]{images/image1}30\% profit on the price he had paid. What percentage of profit did Sanjay earn on the original price?       \textbf{[September 12, 2014 @ 59m 52s]}
	
	\noindent 
	
	\noindent \begin{tabular}{p{1.7in} p{1.6in} p{1.6in}} \\ 
 1) 17\%                    & 2) 20\%              & 3) 27 \%             \\
4) 32\%              & 5) None of these \\
\end{tabular}
	
	\noindent 
	
	\noindent 
	\newpage
	\noindent   \textbf{Answers \\ }
	
	\noindent 
	
	\noindent  
	
	\begin{tabular}{|p{0.7in}|p{0.5in}|p{0.4in}|p{0.5in}|p{0.5in}|p{0.4in}|p{0.5in}|p{0.4in}|p{0.5in}|p{0.4in}|} \hline 
		1 - 1 & 2 - 1 & 3 - 4 & 4 - 2 & 5 - 2 & 6 - 2 & 7 - 1 & 8 - 3 & 9 - 2 & 10 - 5 \\ \hline 
		11 - 4 & 12 - 2 & 13 - 1 & 14 - 1 & 15 - 3 & 16 - 1 & 17 - 4 & 18 - 4 & 19 - 3 & 20 - 1 \\ \hline 
		21 - 2 & 22 - 2 & 23 - 1 & 24 - 4 & 25 - 4 & 26 - 2 & 27 - 1 & 28 - 3 & 29 - 2 & 30 - 3 \\ \hline 
		31 - 2 & 32 - 3 & 33 - 1 & 34 - 3 & 35 - 4 & 36 - 2 & 37 - 2 & 38 - 1 &  &  \\ \hline 
	\end{tabular}
	
	
	
	\noindent 
	
	\noindent  \\ 
	
	\noindent \textbf{Note: }The date and time mentioned against some questions refer to the doubts clarification
	
	\noindent session on Quantitative Aptitude in which the question was solved.
\end{document}