

\documentclass{article} 

\usepackage[utf8]{inputenc} 
\usepackage[english]{babel}
\usepackage{amsmath}
\usepackage{amssymb}
\usepackage{txfonts}
\usepackage{mathdots}
\usepackage[classicReIm]{kpfonts}
\usepackage{graphicx}

\usepackage{multirow}
\usepackage[margin=1.0in]{geometry}
\usepackage[english]{babel}
\usepackage[utf8]{inputenc}
\usepackage{fancyhdr}
\usepackage{tabularx}
\pagestyle{fancy}
\fancyhf{}
\rhead{\includegraphics[width=1.0in, height=0.38819in]{images/logo.png}}
\lhead{ Quantitative Aptitude: Profit and Loss}
\lfoot{www.talentsprint.com }
\rfoot{\thepage}
\begin{document}

\noindent \begin{center}
	{\Large \textbf{Profit and Loss \\}}
\end{center}

\noindent 

\noindent 

\noindent 

\noindent {\large \textbf{Terminology: \\}}

\noindent 

\noindent 

\noindent 

\noindent \\ \textbf{Cost Price: }The amount for which the shopkeeper buys the article is known as the Cost Price. It is usually written as CP or C.

\noindent 

\noindent 

\noindent \\ \textbf{Selling Price: }The amount for which he sells the article is known as the Selling Price. It is usually written as SP or S.

\noindent 

\noindent 

\noindent \\ \textbf{Profit: }The excess of the selling price over the cost price of an article is called the profit or the gain.

\noindent 

\noindent 

\noindent \\ \textbf{Loss: }Sometimes, the shopkeeper has to sell the article for a price which is less than its cost price. In this case, the excess of the cost price over the selling price is called the loss.

\noindent \textbf{Note: }Profit and loss are always calculated with CP, as the base.

\noindent 

\noindent 

\noindent 

\noindent \\ \textbf{Marked Price: }This is denoted as MP and is the price at which the article is usually marked.  It is also known as Labelled Price or Printed Price or Tag Price or List Price.

\noindent 

\noindent 

\noindent \\ \textbf{Discount:  }The waiver of cost from the Marked Price that the trader allows a customer is called Discount.

\noindent 

\noindent \\ \textbf{Note:}

\noindent 

\noindent If no discount is offered on the marked price, then the selling price is equal to marked price. If some discount is offered on the marked price, then the selling price is equal to marked price less discount.

\noindent 

\noindent \\ \textbf{Basic Formulae:}

\noindent 

\noindent 

\noindent 

\noindent \\ 1.   Gain or Profit = Selling Price - Cost Price = SP - CP

\noindent 

\noindent 2.   Loss = Cost Price - Selling Price = CP - SP

\noindent 

\noindent 3.   Discount = MP - SP

\noindent 4.  $ Gain\% =  \frac{Gain}{CP} \mathrm{\times} 100 = \frac{SP - CP}{CP} \mathrm{\times} = 100$

\noindent 

\noindent 5.  $ Loss\% =  \frac{Loss}{CP} \mathrm{\times} 100 = \frac{CP - SP}{CP} \mathrm{\times} = 100$

\noindent 6.   When SP and gain\% are given, then $  CP = SP \mathrm{\times} \frac{100}{100+Profit} $

\noindent 7.   When CP and gain\% are given, then $  SP = CP \mathrm{\times} \frac{100+Profit}{100} $

\noindent 8.   When SP and loss\% are given, then $  CP = SP \mathrm{\times} \frac{100}{100-Loss} $

\noindent 9.   When CP and loss\% are given, then $  SP = CP \mathrm{\times} \frac{100-Loss}{100} $

\noindent 10. $  Discount\% = \frac{Discount}{MP} \mathrm{\times} 100 = \frac{MP -SP}{MP} \mathrm{\times} 100$

\noindent 11. When MP and discount \% are given, then $ SP = MP  \mathrm{\times} \frac{100}{100 - Discount\%}$



\noindent \\ \textbf{Important Points: \\}

\noindent 

\noindent 

\noindent 

\noindent $\mathrm{\centerdot}$   SP = 115\% CP =$\mathrm{>}$ Profit = 15\% (since SP is 15\% more than CP)

\noindent 

\noindent $\mathrm{\centerdot}$   SP = 120\% CP =$\mathrm{>}$ Profit  = 20\%

\noindent 

\noindent $\mathrm{\centerdot}$   SP = 108\% CP =$\mathrm{>}$ Profit = 8\%

\noindent $\mathrm{\centerdot}$   SP = 144.8\% CP =$\mathrm{>}$ Profit = 44.8\% and so on. 


\noindent \\ Similarly, 

\noindent \\ $\mathrm{\centerdot}$   SP = 90\% CP =$\mathrm{>}$ Loss = 10\% (Since SP is 10\% less than CP)

\noindent 

\noindent $\mathrm{\centerdot}$   SP = 80\% CP =$\mathrm{>}$ Loss = 20\%

\noindent 

\noindent $\mathrm{\centerdot}$   SP = 76\% CP =$\mathrm{>}$ Loss = 24\% and so on.

\noindent 

\noindent 
\newpage

\noindent \\ \textbf{Shortcut Formulae:}

\noindent 

\noindent 

\noindent 

\noindent \\ 1.   If a man buys \textbf{a }items for Rs.\textbf{b }and sells \textbf{c }items for Rs.\textbf{d}, then the Gain or loss \% = $ \frac{ad}{bc} \mathrm{\times} 100$

\noindent 

\noindent \\ \textbf{Note: }
a) In case of gain percent, the result will be positive. 
b) In case of loss percent, the result will be negative.

\noindent 

\noindent 

\noindent \\  2.   If the CP of x articles = SP of y articles, then 
Gain or loss \% = $ \frac{x-y}{y} \mathrm{\times} 100\% $

\noindent 

\noindent \\ \textbf{Note:}
a) If x $\mathrm{>}$ y, it is \% gain.
b) If x $\mathrm{<}$ y, it is \% loss.

\noindent 

\noindent 

\noindent 

\noindent \\ 3.   A sells an article to B at a gain or loss of x\%, and B sells it to C at a gain or loss of y\%. If C pays Rs.z for it to B, Then  CP  for $ A = Z \mathrm{\times} (\frac{100}{100 \mathrm{\pm} x} \mathrm{\times} \frac{100}{100 \mathrm{\pm} y} )$



\noindent \\ \textbf{Note: }
a) If x or y is negative it indicates a loss.
b)  If x or y is positive it indicates a gain.

\noindent 

\noindent 

\noindent 

\noindent \\ 4.   If two items are \textbf{bought }each at same price , then sold one item at Profit of x\% and the other at loss of x\%, Then overall profit or loss\%  is \textbf{Zero }(No Profit \& No Loss)

\noindent 

\noindent 

\noindent 

\noindent 

\noindent \\ 5.   If two items are \textbf{sold}, each at rupees P, one at a gain of X \%and other at a loss of X\%, Then overall profit or loss\% is always a \textbf{loss }as $\mathrm{\Rightarrow }$ $ Loss = \frac{x^{2}}{100} $

\noindent 

\noindent 

\noindent 

\noindent \\ 6.   A dishonest dealer professes to sell his goods at cost price, but he uses a weight of lesser

\noindent weight. Then Gain\% = $ \frac{True Weight - False Weight}{False Weight} \mathrm{\times} 100 $

\noindent 

\noindent 

\noindent 

\noindent \\ 7.   A dishonest shopkeeper sells an item at a profit of x\% and uses a weight which is y\% less. $ \frac{x\% + y\% less weight}{100 - y\%less weight} \mathrm{\times} 100 $

\noindent 

\noindent \\ \textbf{Ex: - }A dishonest shopkeeper sells an item at a profit of 10\% and uses a weight which is 10\% less. $ Profit\% = \frac{(10 +10)\%}{100\% - 10\%} \mathrm{\times} 100 = \frac{20}{90} \mathrm{\times} 100 = 22.22\%$


\noindent \\ \textbf{Ex: - }A dishonest shopkeeper sells an item at a loss of 5\% and uses a weight which is 24\% less.$ Profit\% = \frac{(-5 + 24)\%}{100\% - 24\%} \mathrm{\times} 100 = \frac{19}{76} \mathrm{\times} 100 = 25\%$



\noindent 

\noindent 

\noindent 

\noindent \\ 8.   A person sells two articles at the same price, one at a profit of  P\% and another at a loss of L\% Then, net Profit or $ Loss \% = \frac{100(P-L) - 2 \mathrm{\times} P \mathrm{\times} L}{(200 + P - L)}$



\noindent 

\noindent 

\noindent 

\noindent \\ 9.   A person sells two articles at the same price, one at a loss of  L\% and another at a loss of  N\% 
Then, net Profit or $ Loss \% = \frac{100(-L-N) + 2 \mathrm{\times} L \mathrm{\times} N}{(200 - L - N)}$


\noindent \\ 10. If two successive discounts on an article at x\% and y\% respectively, then the a. Overall Discount = $ [x + y - \frac{xy}{100}]\% $

\noindent 

\noindent or

\noindent 

\noindent \\ b. $ 100 - (\frac{100 - x}{100} \mathrm{\times} \frac{100 - y}{100} \mathrm{\times} 100) $(This method applicable for two or more values)

\end{document}