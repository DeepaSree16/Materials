


\documentclass{article} 

\usepackage[utf8]{inputenc} 
\usepackage[english]{babel}
\usepackage{amsmath}
\usepackage{amssymb}
\usepackage{txfonts}
\usepackage{mathdots}
\usepackage[classicReIm]{kpfonts}
\usepackage{graphicx}

\usepackage{multirow}
\usepackage[margin=1.0in]{geometry}
\usepackage[english]{babel}
\usepackage[utf8]{inputenc}
\usepackage{fancyhdr}
\usepackage{tabularx}
\pagestyle{fancy}
\fancyhf{}
\rhead{\noindent \\ \includegraphics[width=1.0in, height=0.38819in]{images/logo.png}}
\lhead{ Quantitative Aptitude: Profit and Loss}
\lfoot{www.talentsprint.com }
\rfoot{\thepage}

\begin{document}
	
	
	
	\noindent 
	
	\noindent 
	
	\noindent 
	
	\noindent 
	
	\noindent \begin{center}
		{\Large \textbf{Profit and Loss \\}}
	\end{center}
	
	\noindent 
	
	\noindent 
	
	\noindent 
	
	\noindent 
	
	\noindent {\large \textbf{Part 1 - Basic \\}}
	
	\noindent \\
	
	\noindent \textbf{Model 1: Basic Profit/Loss \\}
	
	\noindent 
	
	\noindent \\ 1.   An Umbrella was sold at a profit of 20\%. What is the selling price of the Umbrella, if its cost price is Rs.180?
	
	\noindent  
	\begin{tabular}{p{1.7in} p{1.6in} p{1.6in}} \\ 
 1) Rs.216                 & 2) Rs.200          & 3) Rs.160      \\
     4) Rs.36            & 5) None of these  \\
\end{tabular}
	
	\noindent 
	
	\noindent 
	
	\noindent 
	
	\noindent 2.   A person sold an article for Rs.20 and earned a profit of 25\%. What is the cost price of the article?
	
	\noindent  
	\begin{tabular}{p{1.7in} p{1.6in} p{1.6in}} \\ 
 1) Rs.16                   & 2) Rs.14            & 3) Rs.12      \\
      4) Rs.18            & 5) None of these  \\
\end{tabular}
	
	\noindent 
	
	\noindent 
	
	\noindent 
	
	\noindent 3.   An article was sold for Rs.13,000 at a loss of 35\%. What is the cost price of the article?
	
	\noindent 
	
	\noindent  
	\begin{tabular}{p{1.7in} p{1.6in} p{1.6in}} \\ 
 1) Rs.16,000                         & 2) Rs.13,700     & 3) Rs.15,000    \\
  4) Rs.20,000     & 5) None of these  \\
\end{tabular}
	
	\noindent 
	
	\noindent 
	
	\noindent 
	
	\noindent 4.   Harshad bought 15 DVD players at Rs.4,500 each and sold all of them at the total price of
	
	\noindent 
	
	\noindent Rs.81,000. What is the percent profit earned in the deal?
	
	\noindent 


	
	\noindent  
	\begin{tabular}{p{1.7in} p{1.6in} p{1.6in}} \\ 
 1) 16                        & 2) 20                 & 3) 25                 \\
4) 20.5              & 5) None of these  \\
\end{tabular}
	
	\noindent 
	
	\noindent 
	
	\noindent 
	
	\noindent 5.   A shopkeeper sells 200 m of cloth for Rs.9,000 at a profit of Rs.5 per m. What is the cost price of 1 m of cloth?
	
	\noindent  
	\begin{tabular}{p{1.7in} p{1.6in} p{1.6in}} \\ 
 1) Rs.45                   & 2) Rs.40            & 3) Rs.35            \\
4) Rs.30            & 5) None of these  \\
\end{tabular}
	
	\noindent 
	
	\noindent 
	
	\noindent 6.   Naresh purchased a TV set for Rs.11,250 after getting a discount of 10\% on the labeled price.
	
	\noindent \noindent \\ \includegraphics*[width=0.60in, height=0.52in]{images/image1}He spent Rs.150 on transport and Rs.800 on installation. At what price should it be sold so
	
	\noindent that the profit earned would be 15\%?                   \textbf{[October 18, 2014 @ 29m 15s]}
	
	\noindent 
	
	\noindent  
	\begin{tabular}{p{1.7in} p{1.6in} p{1.6in}} \\ 
 1) Rs.12,937.50       & 2) Rs.14,030     & 3) Rs.13,450     \\
4) Rs.15,467.50 & 5) None of these  \\
\end{tabular}
	
	\noindent 
	
	\noindent 
	
	\noindent 
	
	\noindent 7.   Manoj sold an article for Rs.15,000. Had he offered a discount of 10\% on the selling price, he would have earned a profit of 8\%? What is the cost price?
	
	\noindent  
	\begin{tabular}{p{1.7in} p{1.6in} p{1.6in}} \\ 
 1) Rs.12,500                         & 2) Rs.13,500     & 3) Rs.12,250     \\
4) Rs.13,250     & 5) None of these  \\
\end{tabular}
	
	\noindent 
	
	\noindent 
	
	\noindent 
	
	8.   If Ramu buys books at 11 books for Rs.10 and sells at 10 books for Rs.12, then what will be \noindent \\ \includegraphics*[width=0.60in, height=0.52in]{images/image1}his  gain percent? \textbf{[August 02, 2014 @ 36m 15s]}
	
	\noindent  
	\begin{tabular}{p{1.7in} p{1.6in} p{1.6in}} \\ 
 1) 11\%                    & 2) 22\%              & 3) 32\%              \\
4) 15\%              & 5) None of these  \\
\end{tabular}
	
	\noindent 
	
	\noindent 
	
	\noindent 
	\newpage
	\noindent \textbf{Model 2: SP/CP with Respect to Two Different Profit/Loss Percentages}
	
	\noindent 
	
	\noindent 9.   A watch was sold at a loss of 9\%. It was observed that if the selling price was Rs.420 more, the profit made would have been 5\%. What is the actual selling price of the watch?
	
	\noindent  
	\begin{tabular}{p{1.7in} p{1.6in} p{1.6in}} \\ 
 1) Rs.2700               & 2) Rs.2730        & 3) Rs.3270        \\
4) Rs.3000                     & 5) None of these  \\
\end{tabular}
	
	\noindent 
	
	\noindent 
	
	\noindent 10. After selling a book, Rohan found that he had made a loss of 12\%. He also found that had he sold it for Rs.36 more, he would have made a profit of 6\%. What was the initial loss?
	
	\noindent 
	

		 
	\begin{tabular}{p{1.7in} p{1.6in} p{1.6in}} \\ 
 1) Rs.12  & 2) Rs.18  & 3) Rs.20 
		\\
4) Data inadequate  & 5) None of these  \\
\end{tabular} 
	
	
	
	
	\noindent 
	
	\noindent 
	
	\noindent 11. Praveen sold an article for Rs.1170 at a profit of 30\%. What should be the selling price if the
	
	\noindent 
	
	\noindent desired profit is 40\%?
	
	\noindent 
	
	\noindent  
	\begin{tabular}{p{1.7in} p{1.6in} p{1.6in}} \\ 
 1) Rs.1330               & 2) Rs.990          & 3) Rs.1287        \\
4) Rs.1260        & 5) None of these  \\
\end{tabular}
	
	\noindent 
	
	\noindent 
	
	\noindent 
	
	\noindent 12. Sameer sold an article Rs.460 and earned a profit of 15\%. At what price should it have been sold so as to earn a profit of 20\%?
	
	\noindent  
	\begin{tabular}{p{1.7in} p{1.6in} p{1.6in}} \\ 
 1) Rs.465                 & 2) Rs.480          & 3) Rs.498          \\
4) Rs.485          & 5) None of these  \\
\end{tabular}
	
	\noindent 
	
	\noindent 
	
	\noindent 
	
	\noindent 
	
	\noindent 
	
	\noindent \textbf{Model 3: CP Based on Profit and Loss Relationship}
	
	\noindent 
	
	\noindent 13. The profit earned by selling a phone for Rs.18,000 is the same as the loss incurred by selling it for Rs.16,800. What is the cost price of the phone?
	
	\noindent  
	\begin{tabular}{p{1.7in} p{1.6in} p{1.6in}} \\ 
 1) Rs.17,400                         & 2) Rs.17,000     & 3) Rs.17,500     \\
4) Rs.17,600     & 5) None of these  \\
\end{tabular}
	
	\noindent 
	
	\noindent 
	
	\noindent 
	
	\noindent 14. The profit earned by selling an article for Rs.625 is the same as the loss incurred by selling the article for Rs.435. What is the cost price of the article?
	
	\noindent  
	\begin{tabular}{p{1.7in} p{1.6in} p{1.6in}} \\ 
 1) Rs.530                 & 2) Rs.520          & 3) Rs.540          \\
4) Rs.550          & 5) None of these  \\
\end{tabular}
	
	\noindent 
	
	\noindent 
	
	\noindent 
	
	\noindent 15. The profit earned by selling an article for Rs.536 is the same as the loss incurred after the article for Rs.426. What is the cost price of the article?
	
	\noindent  
	\begin{tabular}{p{1.7in} p{1.6in} p{1.6in}} \\ 
 1) Rs.448                 & 2) Rs.470          & 3) Rs.481          \\
4) Rs.500          & 5) None of these  \\
\end{tabular}
	
	\noindent 
	
	\noindent 
	
	\noindent 
	
	\noindent 
	
	16. The profit earned by selling a shirt for Rs.1200 is twice the loss incurred when the shirt is sold for Rs.600. What is the cost price of the shirt?
	
	\noindent \noindent \\ \includegraphics*[width=0.61in, height=0.52in]{images/image1} 
	\begin{tabular}{p{1.7in} p{1.6in} p{1.6in}} \\ 
 1) Rs.800                 & 2) Rs.1000        & 3) Rs.900          \\
4) Rs.750          & 5) None of these  \\
\end{tabular}
	
	\noindent 
	
	\noindent 17. The profit earned by selling a wrist watch for Rs.5800 is twice the loss incurred after selling
	
	\noindent 
	
	\noindent the wrist watch for Rs.4300. What is the cost price of the wrist watch?
	
	\noindent 
	
	\noindent  
	\begin{tabular}{p{1.7in} p{1.6in} p{1.6in}} \\ 
 1) Rs.5300               & 2) Rs.5100        & 3) Rs.4900        \\
4) Rs.4800        & 5) None of these  \\
\end{tabular}
	
	\noindent 
	
	\noindent 
	
	\noindent 
	
	\noindent 18. The profit earned by selling an article for Rs.4080 is half the loss incurred after selling the \noindent \\ \includegraphics*[width=0.60in, height=0.52in]{images/image1}same article for Rs.3660. What is the cost price of the article?     \textbf{[August 02, 2014 @ 41m 22s]}
	
	\noindent  
	\begin{tabular}{p{1.7in} p{1.6in} p{1.6in}} \\ 
 1) Rs.3785               & 2) Rs.3800        & 3) Rs.3775        \\
4) Rs.3940        & 5) None of these  \\
\end{tabular}
	
	\noindent 
	
	\noindent 
	
	\noindent 
	
	\noindent 
	
	\noindent \textbf{Model 4: Profit/Loss Percentage Based on Quantity Sold}
	
	\noindent 
	
	\noindent 19. Sridhar sold 16 pens at the cost of 20 pens. What is the profit or loss percentage made by him?
	
	\noindent  
	\begin{tabular}{p{1.7in} p{1.6in} p{1.6in}} \\ 
 1) 4\% profit            & 2) 4\% loss        & 3) 25\% profit   \\
4) 25\% loss      & 5) Cannot be determined \\
\end{tabular}
	
	\noindent 
	
	\noindent 
	
	\noindent 
	
	\noindent 20. Prem sold 10 pens at the cost of 12 similar pens. What \% profit or loss does he make in this transaction?
	
	\noindent  
	\begin{tabular}{p{1.7in} p{1.6in} p{1.6in}} \\ 
 1) 20\% profit                                                 & 2) 25\% profit                             & 3) 16.66\% loss
	
	\noindent 
	
	\noindent \\
4) 20\% loss                                                    & 5) None of these  \\
\end{tabular}
	
	\noindent 
	
	\noindent 
	
	\noindent 
	
	\noindent 
	
	\noindent 
	
	\noindent 21. Aniruddh sold 21 books at the cost price of 18 books. Find the percentage profit or loss in this transaction?
	
	\noindent 
	


		 
	\begin{tabular}{p{1.7in} p{1.6in} p{1.6in}} \\ 
 1) 14.28\% profit   & 2) 14.28\% loss\  & 3) 16.66\% profit  \\
4) 16.66\% loss
		 & 5) None of these  \\

	\end{tabular}
	
	
	
	\noindent 
	
	\noindent 
	
	\noindent 22. Ajit sold 20 apples at the cost price of 16 apples. What profit/loss does he make?
	
	\noindent 
	
	\noindent  
	\begin{tabular}{p{1.7in} p{1.6in} p{1.6in}} \\ 
 1) Rs.10                                                          & 2) Cannot be determined        & 3) Rs.5
	
	\noindent 
	
	\noindent \\
4) Rs.6                                                            & 5) None of these  \\
\end{tabular}
	
	\noindent 
	
	\noindent 
	
	\noindent \textbf{Model 5: Overall Profit/Loss Percentage When S1 = S2 and \%P = \%L}
	
	\noindent 
	
	\noindent 23.  A shopkeeper bought two ceiling fans for Rs.800 each. He sold one fan at a profit of 12\%
	
	\noindent \noindent \\ \includegraphics*[width=0.60in, height=0.52in]{images/image1}and the other at a loss of 12\%. What would be his overall profit or loss in the transaction?
	
	\noindent 
	
	\noindent 
	
	\noindent 
	
	\noindent 
	
	\noindent 
	
	\noindent 
	
	\noindent 24. A person bought two articles for the same price and sold them at a profit of 10\% on one and a loss of 10\% on the other. What is the overall profit or loss \% made by him?
	
	\noindent 
	
	
		 
	\begin{tabular}{p{1.7in} p{1.6in} p{1.6in}} \\ 
 1) 1\% Loss & 2) 1\% Profit & 3) Profit 4\% 
		\\
4) No Profit No loss & 5) None of these  \\
\end{tabular} 

	
	
	
	\noindent 
	
	\noindent 
	
	\noindent 25. A person sold two articles for Rs.1200 each; he made a profit of 20\% and a loss of 20\% on the
	
	\noindent \noindent \\ \includegraphics*[width=0.60in, height=0.52in]{images/image1} other. What will be the overall profit or loss in percentage?
	
	\noindent 
	
	\noindent 
	
	\noindent 
	
	\noindent 
	
	\noindent 
	
	\noindent 
	
	\noindent 26. If a person sold two articles at the same price and realized 10\% profit on one and 10\% loss on the other article. What net profit/loss \% does he make?
	
	\noindent 
	

		 
	\begin{tabular}{p{1.7in} p{1.6in} p{1.6in}} \\ 
 1) No Profit No loss  & 2) Loss 1\% & 3) Profit 1\% \\  
		
4) Loss 10\%  & 5) None of these  \\
\end{tabular}   \\  
	
	
	
	
	\noindent 
	
	\noindent 
	
	\noindent \textbf{Model 6: Effective Percentage Problem}
	
	\noindent 
	
	\noindent 27. A sold an article to B at a profit of 20\% and B sold the same article to C at a profit of 10\%. If
	
	\noindent \noindent \\ \includegraphics*[width=0.60in, height=0.52in]{images/image1} C bought it for Rs.2640, how much did A pay for it?
	
	\noindent  
	\begin{tabular}{p{1.7in} p{1.6in} p{1.6in}} \\ 
 1) Rs.2000               & 2) Rs.1500        & 3) Rs.1600        \\
4) Rs.1540        & 5) None of these  \\
\end{tabular}
	
	\noindent 
	
	\noindent 
\newpage
	\noindent 28. Prathik sold a music system to Karthik at 20\% gain and Karthik sold it to Swasthik at 40\%
	
	\noindent 
	
	\noindent gain. If Swasthik paid Rs.10,500 for the music system, what amount did Prathik pay for the same? 
	
		 
	\begin{tabular}{p{1.7in} p{1.6in} p{1.6in}} \\ 
 1) Rs.8,240 & 2) Rs.7,500  & 3) Rs.6,250 \\
4) Cannot be determined & 5) None of these  \\
\end{tabular}
	
	
	
	
	\noindent 
	
	\noindent 
	
	\noindent 29. Prakash sold an article to Prem at 10\% profit. Prem sold the same to Raj at 10\% loss. If
	
	\noindent 
	
	\noindent Prakash bought the article at Rs.2000, at what price did Raj buy the same?
	
	\noindent 
	
	\noindent  
	\begin{tabular}{p{1.7in} p{1.6in} p{1.6in}} \\ 
 1) Rs.2,000              & 2) Rs.1,980       & 3) Rs.2,500       \\
4) Rs.1,800       & 5) None of these  \\
\end{tabular}
	
	\noindent 
	
	\noindent 
	
	\noindent 
	
	\noindent 
	
	\noindent \textbf{Model 7: Dishonest Dealer Problem}
	
	\noindent 
	
	\noindent 30. A dishonest dealer claims to sell his goods at the cost price but uses a weight of 800 gm instead of 1 kg. What will be the profit percentage in this transaction?
	
	\noindent  
	\begin{tabular}{p{1.7in} p{1.6in} p{1.6in}} \\ 
 1) 15\%                    & 2) 20\%              & 3) 25\%              \\
4) 32\%              & 5) None of these  \\
\end{tabular}
	
	\noindent 
	
	\noindent 
	
	\noindent 
	
	\noindent 31. A milkman sells milk at the cost price but delivers only 1800 ml instead of 2 liters. What will be the profit percentage made by the milkman?
	
	\noindent  
	\begin{tabular}{p{1.7in} p{1.6in} p{1.6in}} \\ 
 1) 15\%                    & 2) 11.11\%         & 3) 20\%              \\
4) 25\%              & 5) None of these  \\
\end{tabular}
	
	\noindent 
	
	\noindent 
	
	\noindent 
	
	\noindent 
	
	\noindent \textbf{Model 8: Marked Price and Discount}
	
	\noindent 
	
	32. An article was sold for Rs.480 after a discount of 20\%. What is the marked price? \noindent \\ \includegraphics*[width=0.60in, height=0.52in]{images/image1}  
	\begin{tabular}{p{1.7in} p{1.6in} p{1.6in}} \\ 
 1) Rs.400                                                        & 2) Rs.576                                    & 3) Rs.600
	
	\noindent 
	
	\noindent \\
4) Cannot be determined                            & 5) None of these  \\
\end{tabular}
	
	\noindent 
	
	\noindent 
	
	\noindent 33. If the selling price of Rs.24 results in a 20\% discount on the list price, then what selling price
	
	\noindent 
	
	\noindent would result in a 30\% discount on list price?
	
	\noindent 
	
	\noindent  
	\begin{tabular}{p{1.7in} p{1.6in} p{1.6in}} \\ 
 1) Rs.21                   & 2) Rs.24                         & 3) Rs.18                         \\
4) Rs.15                         & 5) None of these  \\
\end{tabular}
	
	\noindent 
	
	\noindent 
	
	\noindent 34. In a sale, a perfume is available at a discount of 15\% on the selling price. If the perfume's discounted selling price is Rs.3740, what was the original selling price of the perfume?
	
	\noindent  
	\begin{tabular}{p{1.7in} p{1.6in} p{1.6in}} \\ 
 1) Rs.4,324              & 2) Rs.4,386       & 3) Rs.4,400       \\
4) Rs.4,294       & 5) None of these  \\
\end{tabular}
	
	\noindent 
	
	\noindent 
	
	\noindent 
	
	\noindent 
	
	35. A shopkeeper marks his goods in such a way that even after allowing a discount of 20\%, he \noindent \\ \includegraphics*[width=0.60in, height=0.52in]{images/image1} makes a profit of 12\%. How much percent above the cost price is the marked price?
	
	\noindent  
	\begin{tabular}{p{1.7in} p{1.6in} p{1.6in}} \\ 
 1) 32\%                    & 2) 8\%                & 3) 12\%              \\
4) 40\%              & 5) None of these  \\
\end{tabular}s
	
	\noindent 
	
	\noindent 
	
	\noindent 
	
	\noindent 
	
	\noindent 36. A shopkeeper marks his goods in such a way that after allowing a discount of 10\%, he gains
	
	\noindent 
	
	\noindent 17\%. How much percent above C.P. is the marked price?
	
	\noindent 
	
	\noindent  
	\begin{tabular}{p{1.7in} p{1.6in} p{1.6in}} \\ 
 1) 50\%                    & 2) 30\%              & 3) 27\%              \\
4) 7\%                & 5) None of these  \\
\end{tabular}s
	
	\noindent 
	
	\noindent 
	
	\noindent 
	
	37. At what price should a shopkeeper mark a radio that cost him Rs.1200 in order that he may \noindent \\ \includegraphics*[width=0.60in, height=0.52in]{images/image1}offer a discount of 20\% on the marked price and still make a profit of 25\%?
	
	\noindent \textbf{[August 02, 2014 @ 43m 00s]}
	
	\noindent 
	
	\noindent  
	\begin{tabular}{p{1.7in} p{1.6in} p{1.6in}} \\ 
 1) Rs.1675               & 2) Rs.1875        & 3) Rs.1900        \\
4) Rs.2025        & 5) None of these  \\
\end{tabular}
	
	\noindent 
	
	\noindent 
	
	\noindent 
	
	\noindent 38. Sanjay bought a microwave oven and paid 10\% less than the original price. He sold it with
	
	\noindent 
	
	\noindent \noindent \\ \includegraphics*[width=0.60in, height=0.52in]{images/image1}30\% profit on the price he had paid. What percentage of profit did Sanjay earn on the original price?       \textbf{[September 12, 2014 @ 59m 52s]}
	
	\noindent 
	
	\noindent  
	\begin{tabular}{p{1.7in} p{1.6in} p{1.6in}} \\ 
 1) 17\%                    & 2) 20\%              & 3) 27 \%             \\
4) 32\%              & 5) None of these  \\
\end{tabular}
	
	\noindent 
	
	\noindent 
	
	\noindent 
	
	\noindent 
	
	\noindent 
	
	\noindent 
	
	\noindent 
	
	\noindent 
	
	\noindent 
	
	\noindent 
	
	\noindent 
	
	\noindent 
	
	\noindent 
	
	\noindent 
	
	\noindent 
	
	\noindent 
	
	\noindent 
	
	\noindent 
	
	\noindent 
	
	\noindent 
	
	\noindent 
	
	\noindent 
	
	\noindent 
	
	\noindent \textbf{Answers}
	
	\noindent 
	
	\noindent 
	
	\begin{tabular}{|p{0.7in}|p{0.5in}|p{0.4in}|p{0.5in}|p{0.5in}|p{0.4in}|p{0.5in}|p{0.4in}|p{0.5in}|p{0.4in}|} \hline 
		1 - 1 & 2 - 1 & 3 - 4 & 4 - 2 & 5 - 2 & 6 - 2 & 7 - 1 & 8 - 3 & 9 - 2 & 10 - 5 \\ \hline 
		11 - 4 & 12 - 2 & 13 - 1 & 14 - 1 & 15 - 3 & 16 - 1 & 17 - 4 & 18 - 4 & 19 - 3 & 20 - 1 \\ \hline 
		21 - 2 & 22 - 2 & 23 - 1 & 24 - 4 & 25 - 4 & 26 - 2 & 27 - 1 & 28 - 3 & 29 - 2 & 30 - 3 \\ \hline 
		31 - 2 & 32 - 3 & 33 - 1 & 34 - 3 & 35 - 4 & 36 - 2 & 37 - 2 & 38 - 1 &  &  \\ \hline 
	\end{tabular}
	
	
	
	\noindent 
	
	\noindent 
	
	\noindent \textbf{Note: }The date and time mentioned against some questions refer to the doubts clarification
	
	\noindent session on Quantitative Aptitude in which the question was solved.
	
	\noindent 
	
	\noindent 
	
	\noindent 
	
	\noindent 
	
	\noindent 
	
	\noindent 
	
	\noindent 
	
	\noindent 
	
	\noindent 
	
	\noindent 
	
	\noindent 
	
	\noindent 
	
	\noindent 
	
	\noindent 
	
	\noindent 
	
	\noindent 
	
	\noindent 
	
	\noindent 
	
	\noindent 
	
	\noindent 
	
	\noindent 
	
	\noindent 
	
	\noindent 
	
	\noindent 
	
	\noindent 
	
	\noindent 
	
	\noindent 
	
	\noindent 
	
	\noindent 
	
	\noindent 
	
	\noindent 
	
	\noindent 
	
	\noindent 
	
	\noindent {\large \textbf{\\ Part 2 - Advanced \\}}
	
	\noindent 
	
	\noindent 
	
	\noindent 
	
	1.   A man bought oranges at the rate of 8 for Rs.34 and sold them at the rate of 12 for Rs.57. \noindent \\ \includegraphics*[width=0.60in, height=0.52in]{images/image1}How  many oranges should be sold to earn a net profit of Rs.45?
	
	\noindent 
	
	\noindent a) 90                        b) 100               c) 135                d) 150
	
	\noindent 
	
	2.   Which of the following successive discounts is better to a customer \noindent \\ \includegraphics*[width=0.60in, height=0.52in]{images/image1}a) 20\%, 15\%, 10\% or          b) 25\%, 12\%, 8\%, ?
	
	\noindent a) a is better                                                  b) b is better
	
	\noindent 
	
	\noindent c) a or b (both are same)                             d) None of these
	
	\noindent 
	
	\noindent 
	
	\noindent 
	
	\noindent \noindent \\ \includegraphics*[width=0.59in, height=0.52in]{images/image1}3.   By selling an article, a man makes a profit of 25\% of its selling price. His profit per cent is a) 20                        b) 25                 c) 16 2/3           d) 33 1/3
	
	\noindent 
	
	\noindent 
	
	\noindent 4.   The list price of an article is Rs.160 and a customer buys it for Rs.122.40 after two successive
	
	\noindent \noindent \\ \includegraphics*[width=0.60in, height=0.52in]{images/image1}discounts. If the first discount is 10\%, then second discount is --
	
	\noindent a) 12\%                    b) 10\%              c) 14\%              d) 15\%
	
	\noindent 
	
	5.   The cost price of an article is 64\% of the marked price. The gain percentage after allowing a \noindent \\ \includegraphics*[width=0.60in, height=0.52in]{images/image1}discount of 12\% on the marked price is
	
	\noindent a) 37.5\%                 b) 48\%              c) 50.5\%           d) 52\%
	
	\noindent 
	
	\noindent 
	
	\noindent 
	
	6.   A dishonest seller sells the goods at 6 1/4\% loss on the cost price but uses 12 1/2 \% less \noindent \\ \includegraphics*[width=0.60in, height=0.52in]{images/image1}weight. What is his percentage profit or loss?
	
	\noindent 
	
	\noindent 7.   A merchant makes a profit of p\% by selling an article for Rs.24. Had the CP and SP been
	
	\noindent 
	
	\noindent \noindent \\ \includegraphics*[width=0.60in, height=0.52in]{images/image1}interchanged, it would have led to a loss of 62.5\% of p. With the latter cost price, what should be the new selling price to get a profit of p\%?
	
	\noindent 
	
	\noindent  
	\begin{tabular}{p{1.7in} p{1.6in} p{1.6in}} \\ 
 1) 28.20                   & 2) 38.40            & 3) 48.60            \\
4) 58.45 \\
\end{tabular}
	\noindent 
	
	\noindent 
	
	\noindent 8.   An article is sold at 20\% profit. If its CP is increased by Rs.50 and at the same time, if it's SP
	
	\noindent \noindent \\ \includegraphics*[width=0.60in, height=0.52in]{images/image1}is also increased by Rs.30, the percentage of profit is decreased by 3 1/3 percentage. Find the cost price of the article.
	
	\noindent 
	
	\noindent 
	
	\noindent 
	
	9.   A person sold an electronic watch at Rs.96 in such a way that his percentage profit is same \noindent \\ \includegraphics*[width=0.60in, height=0.52in]{images/image1}as the cost price of the watch. If he sells it at twice the profit of its previous profit, then new
	
	\noindent selling price will be:
	
	\noindent 
	
	\noindent a) Rs.132                 b) Rs.150          c) Rs.192           d) Rs.180
	
	\noindent 
	
	10. A horse is sold at a profit of 25\%. If both the cost price and selling price are 200 less, the \noindent \\ \includegraphics*[width=0.60in, height=0.52in]{images/image1}profit will be 5\% more. The cost price is
	
	\noindent a) 1100                    b) 1200             c) 1000              d) 900
	
	\noindent 
	
	\noindent 11. The price of a land passing through three hands rises on the whole by  65\%. If the first and
	
	\noindent 
	
	\noindent \noindent \\ \includegraphics*[width=0.60in, height=0.52in]{images/image1}the second sellers earned 20\% and 25\% profit, respectively. Find the profit earned by  third seller.
	
	\noindent 
	
	\noindent a) 20\%                    b) 55\%              c) 10\%              d) 25\%
	
	\noindent 
	
	\noindent 
	
	\noindent 12. A shopkeeper allows 23\% commission on his advertised price and still makes a profit of
	
	\noindent \noindent \\ \includegraphics*[width=0.60in, height=0.52in]{images/image1}10\%. If he gains 56 on one item, advertised price of the item is
	
	\noindent a) 820                      b) 780               c) 790                d) 800               e) None of these
	
	\noindent 
	
	\noindent 
	
	\noindent 
	
	\noindent 
	
	13. Ram bought two articles A and B at a total cost of Rs.8000. He sold article A at 20\% profit \noindent \\ \includegraphics*[width=0.60in, height=0.52in]{images/image1}and article B at 12\% loss. In the whole deal he made no gain and no loss. At what price
	
	\noindent should Ram have sold article B to make an overall profit of 25\%?
	
	\noindent 
	
	\noindent  
	\begin{tabular}{p{1.7in} p{1.6in} p{1.6in}} \\ 
 1) 6400                    & 2) 4800             & 3) 4400             \\
4) 6250             & 5) None of these  \\
\end{tabular}
	
	\noindent 
	
	14. A dealer makes a profit of 20\% by selling an article. What would be the percentage change \noindent \\ \includegraphics*[width=0.60in, height=0.52in]{images/image1}in the profit percent, if he paid 20\% less for it and the customer paid 10\% more for it?
	
	\noindent  
	\begin{tabular}{p{1.7in} p{1.6in} p{1.6in}} \\ 
 1) 225\%                  & 2) 125\%            & 3) 180\%            \\
4) 260\%            & 5) None of these  \\
\end{tabular}
	
	\noindent 
	
	\noindent 
	
	\noindent 15. A 20\% discount is given on the marked price of an e book on cash purchase and 10\%
	
	\noindent 
	
	\noindent \noindent \\ \includegraphics*[width=0.60in, height=0.52in]{images/image1}discount is given on credit purchase. If a person buys 70\% of the e book by cash purchase and 30\% of the e book by credit purchase, what is the effective percentage discount that he
	
	\noindent 
	
	\noindent gets?
	
	\noindent 
	
	\noindent a) 12\%                    b) 17\%              c) 23\%              d) 35\%              e) None of the above
	
	\noindent 
	
	\noindent 
	
	16. Total of the costs of a ceiling fan and a table fan is Rs.980/-. If the cost of the ceiling fan is \noindent \\ \includegraphics*[width=0.60in, height=0.52in]{images/image1}increased by 16 2/3\% and that of table fan is decreased by 12 1/2\%, their costs will be the
	
	\noindent same. The cost of the table fan exceeds the cost of a ceiling fan by?
	
	\noindent 
	
	\noindent  
	\begin{tabular}{p{1.7in} p{1.6in} p{1.6in}} \\ 
 1) 70                        & 2) 98                 & 3) 105                   \\
4) 126           & 5) 140 \\
\end{tabular}
	
	\noindent 
	
	\noindent 
	
	\noindent 
	
	\noindent 
	
	\noindent 
	
	\noindent 
	
	\noindent 
	
	\noindent 17. Arunima sells an item to Bhusan at 25\% profit. Bhusan sells it to Charu at 18\% profit and
	
	\noindent \noindent \\ \includegraphics*[width=0.60in, height=0.52in]{images/image1}Charu sells it to Divya at Rs.432 profit. The difference between the cost price of Divya and the cost price of Arunima is Rs.1287. How much did Bhusan pay to Arunima for the item?
	
	\noindent 
	

		 
	\begin{tabular}{p{1.7in} p{1.6in} p{1.6in}} \\ 
 1) Rs.2250 & 2) Other than given options  & 3)  Rs.2150 
		\\
4) Rs.2350  & 5) Rs.2200  \\
\end{tabular}

	
	
	
	\noindent 
	
	\noindent 
	
	\noindent 18. A person sells a table at a gain of 14 2/7\% and a chair at a loss of 20\% but on the whole he
	
	\noindent 
	
	\noindent \noindent \\ \includegraphics*[width=0.60in, height=0.52in]{images/image1}gains Rs.100. But if he sells the chair at a gain of 20\% and the table at a loss of 8 4/7\% then there is no loss or gain. Find the cost price of chair.
	
	\noindent 
	
	\noindent  
	\begin{tabular}{p{1.7in} p{1.6in} p{1.6in}} \\ 
 1) 700                      & 2) 820               & 3) 650               \\
4) 750               & 5) None of these  \\
\end{tabular}
	
	19. P sold an article to Q at 30\% profit. Q sold it to R at 20\% profit. Q's profit was Rs. 20    less \noindent \\ \includegraphics*[width=0.60in, height=0.52in]{images/image1}than P's profit. Find P's cost price.
	
	\noindent  
	\begin{tabular}{p{1.7in} p{1.6in} p{1.6in}} \\ 
 1) 400                      & 2) 1000             & 3) 500               \\
4) 800               & 5) None of these  \\
\end{tabular}
	
	\noindent 
	
	20. A pen producing company knows that on an average 10\% of the produced pens are always \noindent \\ \includegraphics*[width=0.60in, height=0.52in]{images/image1}defective so are rejected before picking. Company promises to deliver 7200 to its wholesaler
	
	\noindent at Rs. 10 each. It estimates the overall profit on the manufactured pens to be 25\%. What is
	
	\noindent 
	
	\noindent the manufacturing cost of each pen?
	
	\noindent 
	
	\noindent  
	\begin{tabular}{p{1.7in} p{1.6in} p{1.6in}} \\ 
 1) 6.5                       & 2) 7.2                & 3) 8.5                \\
4) 9                   & 5) None of these  \\
\end{tabular}
	
	\noindent 
	
	\noindent 
	
	\noindent 
	
	\noindent 21. A man purchases 40 watches and decides the MRP at 25\% more than c.p. and he offers 10\%
	
	\noindent 
	
	\noindent \noindent \\ \includegraphics*[width=0.60in, height=0.52in]{images/image1}discount on cash payments and 5\% discount on credit. He sells 3/4 of the stock on cash and rest of the stock on credit. If he gains Rs. 2250 then find c.p. of the watch?
	
	\noindent 
	
	\noindent 
	
	\noindent 
	
	\noindent 22. A vendor sells apples at a certain price in order to make a profit of 30\%. If he charges Rs 1.5
	
	\noindent 
	
	\noindent \noindent \\ \includegraphics*[width=0.60in, height=0.52in]{images/image1}higher per apple he would get a profit of 60\%. Find the original price at which he sold an apple?
	
	\noindent 
	
	\noindent  
	\begin{tabular}{p{1.7in} p{1.6in} p{1.6in}} \\ 
 1) 3.75                     & 2) 4.25              & 3) 4.85              \\
4) 3.25              & 5) None of these  \\
\end{tabular}
	
	\noindent 
	
	\noindent 
	
	\noindent 
	
	\noindent 23. A dishonest trader marks up his goods by 80\% and gives a discount of 25\%. Besides, he gets
	
	\noindent \noindent \\ \includegraphics*[width=0.60in, height=0.52in]{images/image1}20\% more amount per kg from wholesaler and sell 10\% less per kg to the customer. What is the overall profit percentage?
	
	24. A dealer marks an article at a price that gives him a profit of 30\%. 6\% of consignment of \noindent \\ \includegraphics*[width=0.60in, height=0.52in]{images/image1}goods was lost in fire in his premises, 24\% was soiled and had to be sold at half the cost
	
	\noindent price. If the remainder was sold at the marked price, what percentage profit or loss did the
	
	\noindent 
	
	\noindent dealer make on that consignment?
	
	\noindent 
	
	\noindent  
	\begin{tabular}{p{1.7in} p{1.6in} p{1.6in}} \\ 
 1) 2\%                      & 2) 2.5\%             & 3) 3\%                \\
4) 6.2\%             & 5) None of these  \\
\end{tabular}
	
	\noindent 
	
	\noindent 
	
	\noindent 
	
	25. The Maximum Retail Price (MRP) of a product is 55\% above its manufacturing cost. The \noindent \\ \includegraphics*[width=0.60in, height=0.52in]{images/image1}product is sold through a retailer, who earns 23\% profit on his purchase price. What is the profit percentage (expressed in nearest integer) for the manufacturer who sells his product
	
	\noindent 
	
	\noindent to the retailer? The retailer gives 10\% discount on MRP
	
	\noindent 
	
	\noindent  
	\begin{tabular}{p{1.7in} p{1.6in} p{1.6in}} \\ 
 1) 31\%                    & 2) 22\%              & 3) 15\%              \\
4) 13\%              & 5) 11\% \\
\end{tabular}
	
	\noindent 
	
	\noindent 
	
	\noindent 
	
	\noindent 
	
	26. The profit made when a trader increases his selling price by 10\% and sells eight articles \noindent \\ \includegraphics*[width=0.60in, height=0.52in]{images/image1}equals the loss made when the trader decreases his selling price by 10\% and sells nine
	
	\noindent articles. Find the ratio of the original selling price and cost price of each article?
	
	\noindent 
	
	\noindent 
	
	\noindent 
	
	27. An  article  costing  Rs.  20  was  marked  25\%  above  the  cost  price.  After  two  successive \noindent \\ \includegraphics*[width=0.60in, height=0.52in]{images/image1} discounts of the same percentage, the customer now pays Rs. 20.25. What would be the percentage change in profit had the price been increased by the same percentage twice
	
	\noindent 
	
	\noindent successively instead of reducing it?
	
	\noindent 
	
	\noindent 28. Harris has two weights of one kilogram at his shop. One of the weights is accurate while the
	
	\noindent \noindent \\ \includegraphics*[width=0.60in, height=0.52in]{images/image1}other actually weighs 1100 grams. While measuring the goods, he uses one of the two weights. The probability that he uses a wrong weight is 3/5. If he sells the goods at the cost
	
	\noindent 
	
	\noindent price, how much profit or loss will he make?
	
	\noindent 
	

		 
	\begin{tabular}{p{1.7in} p{1.6in} p{1.6in}} \\ 
 1) 5.66\% profit  & 2) 5.66\% loss  & 3) 11.32\% profit 
		\\
4) 11.32\% loss  & 5) Cannot be determined  \\
\end{tabular}
	
	
	
	
	\noindent 
	
	\noindent 29. The cost price of an article increases by Rs. 100. The selling price increases by 10\%. If the
	
	\noindent \noindent \\ \includegraphics*[width=0.60in, height=0.52in]{images/image1}new profit decreases from 15\% to 10\%, what is the original cost price?
	
	\noindent  
	\begin{tabular}{p{1.7in} p{1.6in} p{1.6in}} \\ 
 1) Rs 765                & 2) Rs 666.66     & 3) Rs 700          \\
4) Rs 540          & 5) Rs 766.67 \\
\end{tabular}
	
	\noindent 
	
	\noindent 
	
	\noindent 
	
	30. Rohan went to buy an Android mobile, the shopkeeper told him to pay 20\% tax on list price \noindent \\ \includegraphics*[width=0.60in, height=0.52in]{images/image1} if he asks for the bill. Rohan manages to get the discount of 5\% on purchasing the mobile with a bill and he paid the shopkeeper Rs. 8550. Besides he manages to avoid paying 20\%
	
	\noindent 
	
	\noindent tax on the listed price but on discounted price. What is the amount of discount he received in the overall transaction?
	
	\noindent  
	\begin{tabular}{p{1.7in} p{1.6in} p{1.6in}} \\ 
 1) 260                      & 2) 750               & 3) 450               \\
4) 900               & 5) 500 \\
\end{tabular}
	
	\noindent 
	
	\noindent 
	
	\noindent 
	
	\noindent 31. Amit wants to purchase a carbon mobile handset. The shopkeepers told him to pay 21\% tax if he asked for the bill. He managed to get a discount of 7\% on the actual sale price (without tax) of the mobile and paid to shopkeeper Rs. 6789. In doing so, he managed to avoid to pay the 21\% tax. What is the amount of discount that Amit has received on the selling price (inclusive of tax)?
	
	\noindent a) Rs. 2420             b) Rs. 2044       c) Rs. 1896       d) Rs. 2276      e) Rs. 2400
	
	\noindent 
	
	\noindent 
	
	\noindent 32. A person buys some articles. He sold 40\% of articles at 20\% profit and remaining at 33 1/3\% profit. If percent profit is calculated on selling price then what is the ratio of selling price of articles sold at 20\% profit to the articles sold at 33 1/3\% profit.
	
	\noindent 
	
	\noindent 
	
	\noindent 33. A Fruit seller buys a certain dozen of apples at Rs.324. he sold a 1/3rd of apples at 33(1/ 3) \%
	
	\noindent 
	
	\noindent profit on each apple and sold 25\% of the remaining apples at 66(2/ 3)\% less profit on each apple than in previous sale. He sold the rest of the apples such that he gets Rs. 2(13/30) profit on each apple and 40\% profit overall sale. Find the CP of one dozen of apples.
	
	\noindent 
	
	\noindent 
	
	\noindent 34. A product is sold at 10\% discount on marked price and he earns a profit of 50\%. He wants to earn 16 23\% more profit than the value of earlier profit after allowing a discount of 20\%. Find how much \% more he should mark the price of goods from initial marked price.
	
	\noindent 
	
	\noindent 
	
	\noindent 35. An oil refinery buys oil at Rs.3600 per barrel. There is 10\% wastage. If the refinery wants to earn 5\% profit then at what price should it sell including 8\% tax on selling price (in Rs. Per barrel)?
	
	\noindent 
	
	\noindent 
	
	\noindent 36. The A shopkeeper marked-up the price of A and B by 25\% and 20\% respectively and offered
	
	\noindent 
	
	\noindent 10\% and 20\% discounts respectively on their marked price. If marked price of B is Rs.150 more than that of A and the selling price of B is Rs.45 more than A, what is the profit earned by the shopkeeper after selling A.
	
	\noindent 
	
	\noindent 
	
	\noindent 
	
	\noindent 
	
	\noindent \textbf{Answers}
	
	\noindent 
	
	\noindent 
	
	\noindent 1.   90
	
	\noindent 
	
	\noindent 2.   b is better
	
	\noindent 
	
	\noindent 3.   33 1/3
	
	\noindent 
	
	\noindent 4.   15\%
	
	\noindent 
	
	\noindent 5.   37.5\%
	
	\noindent 
	
	\noindent 6.   50/7\%
	
	\noindent 
	
	\noindent 7.   38.40
	
	\noindent 
	
	\noindent 8.   Rs.850
	
	\noindent 
	
	\noindent 9.   Rs.132
	
	\noindent 
	
	\noindent 10. 1200
	
	\noindent 
	
	\noindent 11. 10\%
	
	\noindent 
	
	\noindent 12. 800
	
	\noindent 
	
	\noindent 13. 6400
	
	\noindent 
	
	\noindent 14. 225\%
	
	\noindent 
	
	\noindent 15. 17\%
	
	\noindent 
	
	\noindent 16. 140
	
	\noindent 
	
	\noindent 17. Rs.2250
	
	\noindent 
	
	\noindent 18. 750
	
	\noindent 
	
	\noindent 19. 500
	
	\noindent 
	
	\noindent 20. 7.2
	
	\noindent 
	
	\noindent 21. 400
	
	\noindent 
	
	\noindent 22. None of these
	
	\noindent 
	
	\noindent 23. 80\%
	
	\noindent 
	
	\noindent 24. 3\%
	
	\noindent 
	
	\noindent 25. 13\%
	
	\noindent 
	
	\noindent 26. 4000\%
	
	\noindent 
	
	\noindent 27. 5.66\% loss
	
	\noindent 
	
	\noindent 28. Rs 666.66
	
	\noindent 
	
	\noindent 29. 450
	
	\noindent 
	
	\noindent 30. --
	
	\noindent 
	
	\noindent 31. Rs. 2044
	
	\noindent 
	
	\noindent 32. 5 : 9 (SP)
	
	\noindent 
	
	\noindent 33. 54 (CP of one dozen apple)
	
	\noindent 
	
	\noindent 34. 18.75 \%
	
	\noindent 
	
	\noindent 35. --
	
	\noindent 
	
	\noindent 36. --
	
	\noindent 
	
	\noindent 
	
	\noindent 
\end{document}