


\documentclass{article} 

\usepackage[utf8]{inputenc} 
\usepackage[english]{babel} 
\usepackage{amsmath}
\usepackage{amssymb}
\usepackage{txfonts}
\usepackage{mathdots}
\usepackage[classicReIm]{kpfonts}
\usepackage{graphicx}

 
\usepackage{multirow}
\usepackage[margin=1.0in]{geometry}
\usepackage[english]{babel}
\usepackage[utf8]{inputenc}
\usepackage{fancyhdr}
\usepackage{tabularx}
\pagestyle{fancy}
\fancyhf{}
\rhead{\includegraphics[width=1.0in, height=0.38819in]{images/logo.png}}
\lhead{ Quantitative Aptitude: }
\lfoot{www.talentsprint.com }
\rfoot{\thepage}

\begin{document}




\noindent \begin{center}
	{\Large \textbf{Percentages \\}}
\end{center}


\noindent {\large \textbf{Additional Examples \\}}

\noindent 

\noindent \\


1.   The value of a machine depreciates every year by 10\%. If its present value is Rs. 50,000 thenthe value of the machine after 2 years is

\noindent  \includegraphics*[width=0.60in, height=0.52in]{images/image1} 


\begin{tabular}{p{1.7in} p{1.6in} p{1.6in}} \\ 
 a) Rs. 40,050             &    b) Rs. 45,000      &   c) Rs. 40,005        \\
d) Rs. 40,500  \\
\end{tabular}
                 

\noindent 

\noindent 

\noindent \\
 

2.   If 90\% of A = 30\% of B and B = 2x\% of A, then the value of x is  \\

\noindent \includegraphics*[width=0.60in, height=0.52in]{images/image1}  
\begin{tabular}{p{1.7in} p{1.6in} p{1.6in}} \\ 
	a) 450              &     b) 400         &     c) 300            \\
	 d) 150  \\
\end{tabular}
                                              

\noindent 

\noindent \\
3.   The Price of rice decreases by 6.25\% and because of this reduction, Vandana is able to buy 1kg more for 120. Find the reduced price per kg.

\noindent 

\noindent   \includegraphics*[width=0.60in, height=0.52in]{images/image1} 
\begin{tabular}{p{1.7in} p{1.6in} p{1.6in}} \\ 
	 a) 7.5               &      b) 6.5        &     c) 5.5             \\
	4) 4.5        &    5) None of these 	\\
\end{tabular}
                                                            

\noindent 

\noindent \\
 

4.   Ram saves 35\% of his salary and Shyam saves 45\% of his salary. The difference of their  savings  is  Rs  5000.  Ram's  salary  is  double  than  that  of  Shyam.  Find  their  salaries.

\noindent   \includegraphics*[width=0.60in, height=0.52in]{images/image1}   

\noindent 

\noindent 

\noindent 

\noindent 

\noindent \\


5.   Due to lack of labourers the output of a company decrease by 25\%. What percentage of labourers should be increased to bring the production back, to earlier one?

\noindent  \includegraphics*[width=0.60in, height=0.52in]{images/image1}  
\begin{tabular}{p{1.7in} p{1.6in} p{1.6in}} \\ 
 a) 23.33\%              &     b) 33.33\%       &    c) 43.33\%             \\
d) 53.33\%          &  e) none of these 	\\
\end{tabular}  
                                

\noindent \\
6.   The population of a city was 8000. In a year, the population of males and females increases by 10\% and 8\% respectively. The total population increases to 8720. What is the present population of males in the city?

\noindent 

\noindent   \includegraphics*[width=0.60in, height=0.52in]{images/image1}
\begin{tabular}{p{1.7in} p{1.6in} p{1.6in}} \\ 
 a) 4500               &    b) 4400       &     c) 4200              \\
	d) 3500        &   e) None of these
	 \\
\end{tabular}  
                                            
\noindent 
\newpage
\noindent \\
7.   The present population of a city is 24 lakhs. If annual birth and death rates are 6.8\% and 1.8\%, then what will be the population after 3 years?

\noindent   \includegraphics*[width=0.60in, height=0.52in]{images/image1} 
\begin{tabular}{p{1.7in} p{1.6in} p{1.6in}} \\ 
a) 32.8 lakhs         &   b) 27.7 lakhs        &     c) 29 lakhs           \\
d) 26.5 lakhs     &  5) 35.2 lakhs \\
\end{tabular}
                     

\noindent 

\noindent 

\noindent \\
 8.   In the previous government, Party Q was in the opposition. Now increasing the seats by 33.33\% Q is the ruling party and thus party Q enjoys twice the majority than that of party P in the previous government. If there were only two parties P and Q and the fix no. of seats be 500 in the parliament of Hum-tum, then the no. of seats of the Q in the new government is?

\noindent   \includegraphics*[width=0.60in, height=0.52in]{images/image1} 
\begin{tabular}{p{1.7in} p{1.6in} p{1.6in}} \\ 
 a) 250          &    b) 300           &    c) 450           \\
	d) 200    &  5) None of these \\
\end{tabular}
                                                             
\noindent \\


9.   There is a 4\% increase in volume when a liquid freezes to its solid state. The percentage  decrease when solid melts to liquid again, is

\noindent   \includegraphics*[width=0.60in, height=0.52in]{images/image1} 

\begin{tabular}{p{1.7in} p{1.6in} p{1.6in}} \\ 
1) 3 3/13\%           &   2) 4\%          &    3) 4 1/13\%           \\
 4) 3 11/13\%    &  5) None of these\\
\end{tabular}
                                      

\noindent \\
10. In a motor of 120 machine parts, 5\% parts were defective. In another motor of 80 machine parts, 10\% parts were defective. For the two motors considered together, the percentage of defective machine parts were

\noindent 

\noindent   \includegraphics*[width=0.60in, height=0.52in]{images/image1}   
\begin{tabular}{p{1.7in} p{1.6in} p{1.6in}} \\ 
	1) 7           &  2) 6.5        &    3) 7.5             \\
	  4) 8    & 5) None of these\\
\end{tabular}
                                                                     

\noindent 

\noindent \\


11. When 15\% is lost in grinding wheat, a country can export 30 lakh tons of wheat.On the other  hand, if 10\% is lost in grinding, it can export 40 lakh tons of wheat. The production of wheat in the country is:

\noindent 

\noindent   \includegraphics*[width=0.60in, height=0.52in]{images/image1} 
\begin{tabular}{p{1.7in} p{1.6in} p{1.6in}} \\ 
	1) 20 lakh tons          &  2) 80 lakh tons       &    3) 200 lakh tons              \\
	 4) 800 lakh tons  \\
\end{tabular}

                                              

\noindent 

\noindent                                        

\noindent 

\noindent \\
12. In mathematics paper, a student scored 30\% in the first paper out of a total 180. How much should he score in the second paper (out of 150), if he is to get at least 50\% marks of the total.

\noindent 

\noindent   \includegraphics*[width=0.60in, height=0.52in]{images/image1}   
\begin{tabular}{p{1.7in} p{1.6in} p{1.6in}} \\ 
1) 111           &   2) 150         &     3) 125           \\
 4) 110  &       5) None of these \\
\end{tabular}
                                                    

\noindent 

\noindent 

\noindent 

\newpage
13. In an examination, it is necessary to get 35\% of the total to pass there are 3 papers, a boy gets 62 out of 120 and 35 out of 150 in the first two, how much he get out of 180 in the third paper to pass

\noindent 

\noindent   \includegraphics*[width=0.60in, height=0.52in]{images/image1} 
\begin{tabular}{p{1.7in} p{1.6in} p{1.6in}} \\ 
 1) 50        &   2) 60       &   3) 60.5        \\
4) 75.3   &       5) None of these\\
\end{tabular}
                                                                

\noindent 

\noindent 

\noindent \\


14. Terms of a salesman were changed from a flat commission of 8\% on all his sales to a fixed salary  of  20000  per  month  plus  6\%  commission  on  all  the  sales  exceeding  1.2l.  If  the remuneration as per new scheme was 4000 less than that by the previous scheme per month, his sales were worth?

\noindent 

\noindent   \includegraphics*[width=0.60in, height=0.52in]{images/image1}  
\begin{tabular}{p{1.7in} p{1.6in} p{1.6in}} \\ 
a) 5.2l         &   b) 6.5l      &   c) 7.6l       \\
	4) d) 8.4l    &     5) None\\
\end{tabular}
                                                          

\noindent \\
 

15. The sale of Company N is 40\% less than that of Company T. Then by what per cent is the sale of company T more than that of N?

\noindent   \includegraphics*[width=0.60in, height=0.52in]{images/image1}  
\begin{tabular}{p{1.7in} p{1.6in} p{1.6in}} \\ 
	1) 66\%        &   2) 20\%     &  3) 40\%     \\
	4) 10\%   &      5) None of these\\
\end{tabular}
                                                         

\noindent 

\noindent 

\noindent \\


16. Each side of a cube is decreased by 25\%. Find the ratio of the volumes of the original cube  and the resulting cube

\noindent  \includegraphics*[width=0.60in, height=0.52in]{images/image1} 
\begin{tabular}{p{1.7in} p{1.6in} p{1.6in}} \\ 
	1) 8:1       &  2) 64:27       &  3) 64:1     \\
	4) 27:64   &       5) None of these \\
\end{tabular}
                                                            

\noindent \\
17. In an election between two students, One gets 72\% of total votes. If the total votes are 8200, by how many votes did the winner win the election?

\noindent   \includegraphics*[width=0.60in, height=0.52in]{images/image1}  
\begin{tabular}{p{1.7in} p{1.6in} p{1.6in}} \\ 
	 1) 4405      &  2) 2200       &  3) 3608      \\
	4) 4206   &       5) None of these \\
\end{tabular}

                                                    

\noindent 

\noindent 

\noindent 

\noindent \\
18. The weight of two persons Anil and Anand are in the ratio of 4:5. Anand's weight increased by 20\% and the total weight of Anil and Anand together became 135 kg with an increased of 25\%. By what percent did the weight of Anil increase?

\noindent 

\noindent   \includegraphics*[width=0.60in, height=0.52in]{images/image1}  
\begin{tabular}{p{1.7in} p{1.6in} p{1.6in}} \\ 
a) 28.25\%       &  b) 30.75\%     &   c) 31.25\%   \\
 d) 25.5\%   &       e) None of these \\
\end{tabular}
                                       

\noindent 

\noindent 

\noindent 

\noindent \\
19. Mr. Ankit invests 14\% of his monthly income every month i.e. Rs.1,750 in shares, 8\% in Insurance policies and 7\% in fixed deposits. What is the total annual amount invested by him?

\noindent 

\noindent   \includegraphics*[width=0.60in, height=0.52in]{images/image1}
\begin{tabular}{p{1.7in} p{1.6in} p{1.6in}} \\ 
a) Rs.3275      &  b) Rs.3450     &   c) Rs.3625     \\
d) Rs.3800   &       e) None of these\\
\end{tabular}

                                  

\noindent 

\noindent 

\newpage
20. A  company  gives  12\%  commission  to  his  salesman  upto  the  sale  of  Rs.10000  and commission of 9.5\% on the sales above Rs.10000. If salesman deposited Rs.17850 in the company after deducting his commission. Find the total sales

\noindent   \includegraphics*[width=0.60in, height=0.52in]{images/image1}
\begin{tabular}{p{1.7in} p{1.6in} p{1.6in}} \\ 
 a) 22000      &  b) 24000     &  c) 20000     \\
 d) 28000    &      e) None of these \\
\end{tabular}
                                             

\noindent 

\noindent 

\noindent \\
21. A person loses 75\% of his money in the first bet, 75\% of the remaining in the second and 75\% of the remaining in the third bet and returns home with Rs.2 only. His initial money was

\noindent 

\noindent   \includegraphics*[width=0.60in, height=0.52in]{images/image1}  
\begin{tabular}{p{1.7in} p{1.6in} p{1.6in}} \\ 
	 1) Rs.64   &   2) Rs.128    & 3) Rs.256    \\
	 4) Rs.512      &     5) None of these \\
\end{tabular}
                                           

\noindent 

\noindent 

\noindent \\
 22. In  an  examination  the  percentage  of  students  qualified  with  respect  to  the  number  of students appeared from DP School is 80\%. The number of students appeared from St Joseph School is 20\% more than the number of students appeared from DP School, and the number  of students qualified from St Joseph's School is 40\% more than the number of students qualified from DP School. What is the percentage of students qualified with respect to those who appeared from St Joseph's School?

\noindent   \includegraphics*[width=0.60in, height=0.52in]{images/image1} 
\begin{tabular}{p{1.7in} p{1.6in} p{1.6in}} \\ 
	a) 191/2\%   &  b) 577/6\%    & c) 280/3\%  \\
	d) 680/7\%     &     e) None of these \\
\end{tabular}
     

\noindent 

\noindent \\
23. A merchant claims a loss of 4\% on tea, but uses a weight equal to 840 grams instead of 1 kg. The merchant actually makes

\noindent   \includegraphics*[width=0.60in, height=0.52in]{images/image1} 

\begin{tabular}{p{1.7in} p{1.6in} p{1.6in}} \\ 
	a) 45/4\% profit & b) 100/7\% profit   & c) 10\% profit5 \\
	d) 100/7\% loss    &     e) 99/7\% profit \\
\end{tabular}
                                                        

\noindent 

\noindent                               

\noindent 

\noindent 

\noindent 

\noindent \\
24. The red blood cells in a blood sample grows by 10\% per hour in first two hours, decreases by 10\% in the next one hour, remains constant in the next one hour and again increases by5\% per hour in next two hours. If the original count of the red blood cells in the sample is 40000, find the approximate red blood cell count at the end of 6 hours.
 
\noindent   \includegraphics*[width=0.60in, height=0.52in]{images/image1} 
\begin{tabular}{p{1.7in} p{1.6in} p{1.6in}} \\ 
	a) 40000 & b) 45025  & c) 48025 \\
	d) 50025 &     e) 50050  \\
\end{tabular}



\noindent 

\noindent 

\noindent \\
25. Richa's science test consist of 85 questions from three sections i.e. A, B and C. 10 questions from section A, 30 questions from section B and 45 question from section C. Although, she answered 70\% of section A, 50\% of section B and 60\% of section C correctly. She did not pass the test because she got less than 60\% of the total marks. How many more questions she would have to answer correctly to earn 60\% of the marks which is passing grade?



\begin{tabular}{p{1.7in} p{1.6in} p{1.6in}} \\ 
	a) 4 & b) 2  & c) 5 \\
	d) 6 &     e) 8  \\
\end{tabular}

\noindent 

\noindent 

\noindent \\
 26. Gaurika went to the shop and bought things worth Rs. 30, out of which 40 paise went on sales tax on taxable purchases. If the tax rate was 8\% then what was the cost of tax free items?

\noindent 

\begin{tabular}{p{1.7in} p{1.6in} p{1.6in}} \\ 
a) Rs. 24.6 & b) Rs. 23.8  & c) Rs. 22.6 \\
d) None of these &     e) Cannot be determined  \\
\end{tabular}



\noindent 

\noindent \\
27. Forty percent of the employees of a certain company are men, and 75 percent of the men earn more than Rs. 25,000 per year. If 45 percent of the company's employees earn more than Rs. 25,000 per year, what fraction of the women employed be the company earn Rs.25,000 per year or loss?

\noindent 
\begin{tabular}{p{1.7in} p{1.6in} p{1.6in}} \\ 
	a) 2/ 11  & b) 1/4   &  c) 1/3 \\
 d) 3/4  &    e) None of thesed  \\
\end{tabular}
\noindent                                                            

\noindent 

\noindent 

\noindent \\
 28. MDTV is a very popular TV channel. It releases programs from 8 : 30 am to 12 : 00 pm. It telecasts 60 advertisements each of eight seconds and 16 advertisements each of 30 seconds. What is the percentage of time devoted in a day for the advertisements?


\begin{tabular}{p{1.7in} p{1.6in} p{1.6in}} \\ 
	 1) 1.5\%  &  2) 1.66\%     &  3) 2\%  \\
   4) 2.5\%   \\
\end{tabular}

                                     

\noindent 

\noindent 

\noindent \\
 29. Income tax is raised from 4 paise to 5 paise in a rupee but the revenue is increased by 10\% only. Find the decrease percent in the amount taxed?

\noindent 

\noindent 

\noindent \\
30. Seats for Mathematics, Science and arts in a school are in the ratio 5 : 7 : 8. There is a proposal to increase these seats by X\%, Y\% and Z\% respectively. And the ratio of increased seats is 2 : 3 : 4, which of the following is true?

\noindent   \includegraphics*[width=0.60in, height=0.52in]{images/image1}   
\begin{tabular}{p{1.7in} p{1.6in} p{1.6in}} \\ 
	1) a) X = 50; Z = 40  &  2) 1.66\%     &  c) X = 40; Z = 75    \\
	d) X = 50; Z = 40 &  e) Y = 50; X = 75 \\
\end{tabular}
                                                                               

\noindent 

\end{document}