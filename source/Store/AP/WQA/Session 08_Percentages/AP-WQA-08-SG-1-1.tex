
\documentclass{article} 

\usepackage[utf8]{inputenc} 
\usepackage[english]{babel} 
\usepackage{amsmath}
\usepackage{amssymb}
\usepackage{txfonts}
\usepackage{mathdots}
\usepackage[classicReIm]{kpfonts}
\usepackage{graphicx}

 
\usepackage{multirow}
\usepackage[margin=1.0in]{geometry}
\usepackage[english]{babel}
\usepackage[utf8]{inputenc}
\usepackage{fancyhdr}
\usepackage{tabularx}
\pagestyle{fancy}
\fancyhf{}
\rhead{\includegraphics[width=1.21667in, height=0.38819in]{images/logo.png}}
\lhead{ Quantitative Aptitude: }
\lfoot{www.talentsprint.com }
\rfoot{\thepage}

\begin{document}


\noindent 

\noindent \begin{center}
	{\LARGE \textbf{Percentages \\}}
\end{center}



\noindent {\large \textbf{Terminology: \\}}

\noindent 

\noindent 

\noindent 

\noindent \\ The term percent means parts per 100 or "for every hundred''. Thus, when we say a man made a profit of 20 percent we mean to say that he gained Rs.20 for every hundred rupees he invested in the business, i.e., 20/100 rupees for each Rupee. \\

\noindent 

\noindent 

\noindent \textbf{The abbreviation of percent is p.c. and it is generally denoted by \%. \\}

\noindent 

\noindent 

\noindent 

\noindent 1.   \textbf{A Percentage can be expressed as a Fraction.}

10\% can be expressed as 10/100 or 1/10.

To express a percentage as a fraction divide it by 100 $\mathrm{\Rightarrow } a\% = \frac{a}{100}$. \\

\noindent 

\noindent 

\noindent 

\noindent 2.   \textbf{To express a fraction as a percent multiply it by 100}



 $\mathrm{\Rightarrow } \frac{2}{3} = [\frac{2}{3} \mathrm{\times} 100 ]\% = 66.66\%$ \\


\noindent 3.   \textbf{A Percentage can be expressed as a Decimal. To express percentage as a decimal we remove the symbol \% and shift the decimal point by two places to the left. For example, 10\% can be expressed as 0.1}

\noindent \\ \textbf{Example. }Express $ 6 \frac{1}{2}$\% as a decimal.

\noindent \\ \textbf{Sol. }  $ 6 \frac{1}{2}$\% = 13/2\% = 6.5\% = = 0.065 \\

\noindent 

\noindent 

\noindent 4. \textbf{To express decimal as a percentage we shift the decimal point by two places to the right and write the number obtained with the symbol \% or simply we multiply the decimal with 100}

\noindent \\ \textbf{Example: }Express 0.7 as a percentage.

\noindent \\ \textbf{Sol: } 0.7 = 0.7 $\mathrm{\times}$ 100\% =  $ \frac{7}{10} \mathrm{\times} 100 =70\% $

\noindent 

\noindent 

\noindent 

\noindent \textbf{Fraction to Percentage Conversion Values: \\}

\noindent 


\begin{tabular}{|p{0.9in}|p{0.2in}|p{0.3in}|p{0.2in}|p{0.2in}|p{0.3in}|p{0.3in}|p{0.3in}|p{0.3in}|p{0.3in}|p{0.3in}|p{0.3in}|} \hline 
Fraction & 1/2 & 1/3 & 1/4 & 1/5 & 1/6 & 1/7 & 1/8 & 1/9 & 1/10 & 1/11 & 1/12 \\ \hline 
Percentage & 50 & 33.33 & 25 & 20 & 16.66 & 14.28 & 12.5 & 11.11 & 10 & 9.09 & 8.33 \\ \hline 
\end{tabular}



\noindent 

\noindent \\ {\large \textbf{\\ Basic Formulae: \\}}

\noindent 

\noindent 

\noindent 

\noindent \\ 1.   If the present population of a city is P and rate of increase or decrease is r\% then Population after n years is = $ P(1 \mathrm{\pm} \frac{r}{100})n $ 

\noindent \\ 2.   If the present population of a city is P and rate of increase is r\% then Population before  n years is = $ P(1 \mathrm{\pm} \frac{r}{100})n $

\noindent \\  \textbf{Note:}


a)   +ve sign indicates Increase 
b)  -ve sign indicates Decrease



\noindent \\ 3.   Total Polled Votes = Valid votes + Invalid Votes

\noindent 

\noindent \\  4.  Valid votes = Winner Votes + Looser votes

\noindent 

\noindent \\  5.   Majority Votes =Winner Votes - Looser votes



\noindent \\  6.   If X is R\% of a given number N, then N = $ \frac{x \mathrm{\times} 100}{R} $



\noindent \\ 7.   If the price of commodity increases by R \% then reduction in consumption so as not change the expenditure is $ \frac{r}{100 + r} \mathrm{\times} 100  $


\noindent \\ 8.   If the price of commodity increases by R \% then reduction in consumption so as not change the expenditure is $ \frac{r}{100 - r} \mathrm{\times} 100  $

\noindent 

\noindent \\ 9.   If a value changed (decrease/increase) successively by x\% and y\%, then net change in value

= $[\mathrm{\pm}x \mathrm{\pm}y \mathrm{\pm} \frac{xy}{100} ]$



\noindent \\ \textbf{Note: \\}

a)   +ve sign indicates Increase

b)  -ve sign indicates Decrease

\noindent \\ 10. If A's income is r\% more than that of B, then B's income is $\frac{r}{100 + r} \mathrm{\times} 100$ less than that of A.

\noindent \\ 11. If A's income is r\% less than that of B, then B's income is $\frac{r}{100 - r} \mathrm{\times} 100$ more than that of A

\noindent 

\noindent \\ 12. A person spends x\% of his monthly income on item A and y\% of the remaining on item B. If he saves Rs. P. Then the monthly Income = $ [\frac{p \mathrm{\times} 100 \mathrm{\times} 100 }{(100 - x) \mathrm{\times} (100 - y)}] $



\noindent \\ \textbf{Important Note:}

\noindent 

\noindent \\ \textbf{Consumption Based Questions}

\noindent 

\noindent For fixed total expenditure

\noindent \\ 

\begin{tabular}{p{2.0in} p{2.0in}}  
	Price goes up by {\dots}\%  &  Consumption comes down by {\dots}\% \\ 
	$ \frac{1}{6} $ = 16.66     & $ \frac{1}{7} $ = 14.28 \\ 
	$ \frac{1}{5} $ = 20        & $ \frac{1}{6} $ = 16.66 \\ 
	$ \frac{1}{4} $ = 25        & $ \frac{1}{5} $ = 20 \\ 
	$ \frac{1}{3} $ = 33.33     & $ \frac{1}{4} $ = 25 \\ 
	$ \frac{1}{2} $ = 50        & $ \frac{1}{3} $ = 33.33 \\ 
    $ \frac{1}{1} $= 100        & $ \frac{1}{2} $ = 50 \\ 
\end{tabular}



\noindent \\  For fixed total expenditure

\begin{tabular}{p{2.0in} p{2.0in}}  
	Price come down  by {\dots}\%  &  Consumption goes up by {\dots}\% \\ 
	$ \frac{1}{5}  = 20$     & $ \frac{1}{4} = 25$ \\ 
	20        & 33.33 \\ 
	33.33        & 50 \\ 
	50     & 100 \\ 
	75        & 300 \\ 
	
\end{tabular}



\noindent \\ \textbf{Concept of effective percentage:}

\noindent \\ The  effective  percentage  is  calculated  as  $ a + b + \frac{ab}{100} $ where a and b are percentage change in 2 variable. 

\noindent If the change is negative/there is a decrease in value then put a negative symbol in front of the percentage in the formula

\noindent 

\noindent This  concept  is  very  useful  in  profit  and  loss,  simple  and  compound  interest  etc. Concept of effective percentage can be extended to more than two variables too.













\end{document}