
\documentclass{article} 

\usepackage[utf8]{inputenc} 
\usepackage[english]{babel} 
\usepackage{amsmath}
\usepackage{amssymb}
\usepackage{txfonts}
\usepackage{mathdots}
\usepackage[classicReIm]{kpfonts}
\usepackage{graphicx}

 
\usepackage{multirow}
\usepackage[margin=1.0in]{geometry}
\usepackage[english]{babel}
\usepackage[utf8]{inputenc}
\usepackage{fancyhdr}
\usepackage{tabularx}
\pagestyle{fancy}
\fancyhf{}
\rhead{\includegraphics[width=1.21667in, height=0.38819in]{images/logo.png}}
\lhead{ Quantitative Aptitude: }
\lfoot{www.talentsprint.com }
\rfoot{\thepage}

\begin{document}



\noindent \begin{center}
	{\LARGE \textbf{Percentages \\}}
\end{center}



\noindent {\large \textbf{Part 1 - Basic: \\}}

\noindent 

\noindent \\ \textbf{Model 1: Basic Percentage}

\noindent 

\noindent \\ 1.   Sameer spends 20\% of his monthly salary on house-rent, 25\% on food, 10\% on transportation, 15\% on education of his children and 18\% on household expenses. He saves the remaining amount of Rs. 4800. What is his monthly salary?

\noindent 

\noindent \includegraphics*[width=0.61in, height=0.52in]{images/image1} 1) Rs. 40000               2) Rs. 28000        3) Rs. 32000        4) Rs. 36000        5) None of these

\noindent 

\noindent 

\noindent 

\noindent \\ 2.   Ishan spent Rs. 35,645 on buying a bike, Rs. 24,355 on buying a television, and the remaining 20\% of the total amount he had as cash with him. What was the total amount?



\noindent\includegraphics*[width=0.61in, height=0.52in]{images/image1} 1) Rs. 60,000                  2) Rs. 72,000       3) Rs. 75,000       4) Rs. 80,000       5) None of these

\noindent 

\noindent 

\noindent \\ 3.   Sujatha invests 7\% i.e. Rs. 2170, of her monthly salary in mutual funds. Later she invests 18\% of her monthly salary in recurring deposits. Also she invests 6\% of her salary on NSC's. What is the total annual income invested by Sujata?
\noindent \textbf{[February 28, 2015 @ 2h 05m 15s]}

\noindent 

\noindent \includegraphics*[width=0.61in, height=0.52in]{images/image1}  1) Rs. 125320             2) Rs. 113520      3) Rs. 135120      4) Rs. 115320      5)None of these

\noindent 

\noindent 

\noindent 

\noindent 

\noindent 

\noindent 

\noindent 

\noindent \\ 4.   Sonali invests 15\% of her monthly salary in insurance policies. She spends 55\% of her monthly salary in shopping and on household expenses. She saves the remaining amount of Rs. 12750. What is Sonali's monthly income?

\noindent 

\noindent \includegraphics*[width=0.61in, height=0.52in]{images/image1} 1) Rs. 42,500              2) Rs. 38,800       3) Rs. 40,000       4) Rs. 35,500       5) None of these

\noindent 

\noindent 

\noindent 

\noindent \\5.   Gaurav spends 40\% of the amount he received from father on hostel expenses, 20\% on books and stationery and 50\% of the remaining on transport. He saves Rs. 450 which is half the  remaining  amount  after  spending  on  hostel  expenses,  books  and  stationery  and transport. How much money did he get from his father?

\noindent 

\noindent \includegraphics*[width=0.61in, height=0.52in]{images/image1} 1) Rs. 3000                 2) Rs. 6000          3) Rs. 4500          4) Rs. 5000          5) None of these

\noindent 

\noindent 

\noindent 

\noindent 

\noindent \textbf{\\ Model 2: Percentage of Percentage \\}

\noindent 

\noindent \\ 6.   Prerna decided to donate 30\% of her monthly income to an orphanage. But on the day of donation, she changed her mind and donated Rs. 4200, which is 70\% of what she had decided earlier. What should be the actual donation as per her earlier decision?

\noindent 

\begin{tabular}{p{2.0in}p{1.4in}p{1.7in}} 
1) Rs. 8000 & 2) Rs. 5000 & 3) Rs. 6,000 \\ 
4) Cannot be determined & 5) None of these &  \\  
\end{tabular}



\noindent 

\noindent 
\newpage
\noindent \\ 7.   Praveen decided to donate 15\% of his salary to a charity. On the day of donation he changed his mind and donated Rs. 1,896, which was 80\% of what he had decided earlier. How much is Praveen's salary?

\noindent 

\begin{tabular}{p{2.0in} p{1.4in} p{0.8in}} 
1) Rs. 18,500 & 2) Rs. 10,250 & 3) Rs. 15,800 \\ 
4) Cannot be determined & 5) None of these &  \\
\end{tabular}



\noindent 

\noindent 

\noindent 

\noindent \\ 8.   The price of a car is Rs. 5, 00,000. It was insured to 90\% of its price. The car got completely damaged in an accident and the insurance company paid 80\% of the   insured amount. That is the difference between the price of the car and the amount of insurance received?

\noindent 

\noindent \includegraphics*[width=0.61in, height=0.52in]{images/image1} 1) Rs. 1, 40,000          2) Rs. 40,000       3) Rs. 70,000       4) Rs. 80,000       5) None of these

\noindent 

\noindent 

\noindent \\ 9.   The price of a car is Rs. 325000. It was insured to 85\% of its price. The car was    damaged completely in an accident and the insurance company paid 90\% of the insurance. What was the difference between the price of the car and the amount received?

\noindent \includegraphics*[width=0.61in, height=0.52in]{images/image1} 1) Rs. 32500               2) Rs. 48750        3) Rs. 76375        4) Rs. 81250        5) None of these

\noindent 

\noindent 

\noindent \\ 10. 15\% of 45\% of a number is 105.3. What is 24\% of that number?

\noindent \textbf{[February 28, 2015 @ 24m 17s]}

\noindent \includegraphics*[width=0.61in, height=0.52in]{images/image1} 1) 385.5                   2) 374.4            3) 390               4) 375               5) None of these

\noindent 

\noindent 

\noindent \\  11. 20\% of 65\% of a number is 130. What is 54\% of that number?

\noindent 

\noindent 1) 250                      2) 540               3) 260               4) 275               5) None of these

\noindent 

\noindent 

\noindent 

\noindent 

\noindent \textbf{\\ Model 3: Election Problem \\}

\noindent 

\noindent \\ 12. In an election between two candidates, the winner secured 58\% of the total votes cast and wins by a majority of 2400 votes. How many votes did the loser get?

\noindent \includegraphics*[width=0.61in, height=0.52in]{images/image1} 1) 6300                    2) 7200             3) 3700             4) 4400             5) None of these

\noindent 

\noindent 

\noindent 

\noindent \\ 13. In an election between two candidates, one gets 72\% of the total votes. If the total votes are 8200, by how many votes did the winner win the election?

\noindent 

\noindent \includegraphics*[width=0.61in, height=0.52in]{images/image1} 1) 1835                    2) 1722             3) 3608             4) 4428             5) None of these

\noindent \\ 14.  In a college election between two students, 10\% of the votes cast are invalid. The winner gets 70\% of the valid votes and defeats the loser by 1800 votes. How many votes were totally cast?

\noindent 

\noindent \includegraphics*[width=0.61in, height=0.52in]{images/image1} 1) 1800                    2) 7200             3) 5000             4) 3600             5) None of these

\noindent 

\noindent 

\noindent 

\noindent \\ 15. In a college election fought between two candidates, one candidate got 55\% of the total valid votes. 15\% of the votes were invalid. If the total votes were 15,200, what is the number of valid votes the other candidate got?

\noindent 1) 7106                    2) 6840             3) 8360             4) 5814             5) None of these

\noindent 

\noindent 

\noindent 

\noindent 

\noindent \textbf{\\ Model 4: A Is What Percent of B \\}

\noindent 

\noindent \\ 16. In an examination, Ramesh scored 30\% less than Suresh and Mahesh scored 20\% less than Suresh. Ramesh's score is what per cent of Mahesh's score?

\noindent \includegraphics*[width=0.61in, height=0.52in]{images/image1} 1) 80\%                    2) 70\%              3) 40\%              4) 87.5\%           5) None of these

\noindent 

\noindent \\ 17. 65 is what \% of 50?

\noindent 

\noindent 1) 250                      2) 240               3) 160               4) 130               5) None of these

\noindent 

\noindent 

\noindent 

\noindent 

\noindent \textbf{\\ Model 5: A Is What Percent More/Less than B \\}

\noindent 

\noindent \\ 18. If A's salary is 25\% more than B's salary, then by what per cent is B's salary less than A's salary?

\noindent 

\noindent 1) 25\%                                                            2) 20\%                           3) 16.66\%

\noindent 

\noindent 4) Cannot be determined                            5)None of these

\noindent 

\noindent 

\noindent 

\noindent 

\noindent \\  19. If Arjun's salary is 20\% more than that of Bheem, then how much percent is Bheem's salary less than that of Arjun?

\noindent 

\noindent 1) 16.66\%               2) 20\%              3) 40\%              4) 10\%              5) None of these

\noindent 

\noindent 

\noindent 

\noindent \\ 20. The sale of Company N is 40\% less than that of Company T. Then by what per cent is the sale of company T more than that of N?

\noindent \includegraphics*[width=0.61in, height=0.52in]{images/image1} 1) 66\%                    2) 20\%              3) 40\%              4) 10\%              5) None of these

\noindent 

\noindent 

\noindent 

\noindent \\ 21. If A's salary is 25\% less than B's salary, then B's salary is what \% more than A's salary?

\noindent 

\begin{tabular}{p{2.0in} p{1.1in} p{0.9in} }  
1) Cannot be determined & 2) 21 & 3) 33.33 \\ 
4) 25 & 5) 20 &  \\ 
\end{tabular}



\noindent 

\noindent \textbf{\\ Model 6: Percentage Change \\}

\noindent \\ 22. The profit made by a company in the present year is Rs. 15, 00,000. Two years ago, the profit  made by the same company was Rs. 24, 00,000. What is the percentage change in the profit made by the company?

\noindent 

\noindent 1) 66.66\%  inc       2) 66.66\% dec  3) 37.5\% inc     4) 37.5\% dec    5) None of these

\noindent 

\noindent 

\noindent 

\noindent \textbf{\\ Model 7: Effective Percentage Change \\}

\noindent 

\noindent \\ 23. The length of a rectangle increased by 40\% and its breadth increased by 20\%. What will be the percentage change in the area of the rectangle?

\noindent \includegraphics*[width=0.61in, height=0.52in]{images/image1} 1) 50\% inc              2)10\% inc         3) 44\% inc        4) 68\% inc        5) None of these

\noindent 

\noindent 

\noindent 
\newpage
\noindent \\ 24. The population of a town was 48600. It increased by 25\% in the first year and decreased by 8\% in the second year. What will be the population of the town at the end of two years?

\noindent \textbf{[July 12, 2014 @ 44m 49s]}

\noindent 

\noindent \includegraphics*[width=0.61in, height=0.52in]{images/image1} 1) 65610                  2) 55580           3) 60750           4) 64850           5) None of these

\noindent 

\noindent 

\noindent 

\noindent \\ 25. The revenue of a shop in the month of March was Rs. 40,000. In the month of April, the shopkeeper announced a discount of 20\% and hence his sales went up by 20\%. What will be the revenue in the month of April?

\noindent 1) Rs. 40,000              2) Rs. 50,000       3) Rs. 38,400       4) Rs. 64,800       5) None of these

\noindent 

\noindent 

\noindent 

\noindent \\ 26. When the cost of petroleum increases by 40\%, a man reduces his annual consumption by 20\%. Find the percentage change in his annual expenditure on petroleum.

\noindent 

\noindent 1)  20\%                   2) 16\%              3) 12\%              4) 40\%              5) None of these

\noindent \\ 27. The price of sugar increased by 25\%. What should be the percentage decrease in the   consumption of sugar by a family, such that their expenditure on sugar remains the same?

\noindent 1) 20\%                    2) 16\%              3) 12\%              4) 40\%              5) None of these

\noindent 

\noindent 

\noindent 

\noindent \\ 28. If the cost of coriander sold is increased by 33.33\%. What should be the \% decrease in consumption to keep expenditure same?

\noindent 1) 29\%                    2) 25\%              3) 30\%              4) 25\%              5) None of these

\noindent 

\noindent \textbf{Model 8: Miscellaneous}

\noindent 

\noindent \\ 29. A student has to secure 40\% marks to pass an examination. He secured 75 marks and was

\noindent \includegraphics*[width=0.60in, height=0.52in]{images/image1} declared FAIL 45 marks. Find the maximum marks set for the examination.

\noindent 1) 600                      2) 200               3) 100               4) 300               5) None of these

\noindent \\ 30. In an examination, the cut off mark is 40\% and a student is declared to be failed by 4 markswhen he scored 38\%. What is the maximum score a student can get in this exam?

\noindent \textbf{[January 31, 2015 @ 49m 50s]}

\noindent 

\noindent  \includegraphics*[width=0.60in, height=0.52in]{images/image1} 1) 100                      2) 300               3) 400               4) 200               5) None of these

\noindent 

\noindent 

\noindent 

\noindent  \\ 31. A student has to secure 30\% marks to get through. If he gets 40 marks and fails by 20 marks, find the maximum marks set for the examination.

\noindent 1) 600                      2) 200               3) 100               4) 300               5) None of these

\noindent 

\noindent 

\noindent \\ 32. 75\% of a number when added to 75 becomes the number itself. The number is: 

\noindent \includegraphics*[width=0.60in, height=0.52in]{images/image1} 1) 150                      2) 200               3) 225               4) 300               5) None of these

\noindent 

\noindent \\ 33. Difference between 54\% of a number and 63\% of the same number is 72. What is 80\% of that number?

\noindent 1) 720                      2) 600               3) 640               4) 900               5) None of these

\noindent 

\noindent 

\noindent 

\noindent \\ 34. If 25\% of a number is subtracted from a second number, the second number reduces to its five - sixth. What is the ratio of the first number to the second number?

\noindent 1) 1:3                                                              2) 2:3                                           3) 3:2

\noindent 

\noindent 4) Data inadequate                                      5) None of these

\noindent 

\noindent 

\noindent \\  35. 75\% of a number is equal to four-fifths of another number. What is the ratio between the

\noindent 

\noindent first number and the second number?

\noindent 

\noindent 1) 5:3                       2) 15:16            3) 3:5                4) 16:15            5) None of these

\noindent 

\noindent 

\noindent 

\noindent \\ 36. Two-fifth of one-third of three-seventh of a number is 15. What is 40\% of that number?

\noindent 

\noindent 1) 72                        2) 84                 3) 136               4) 140               5) None of these

\noindent 

\noindent 

\noindent 

\noindent \\ 37. In a test consisting of 300 questions, Deepika answered 40\% of the first 100 questions correctly. What per cent of the remaining 200 questions does she need to answer correctly

\noindent for her grade on the entire exam to be 50\%?

\noindent 

\begin{tabular}{p{2.0in} p{1.4in} p{0.6in}|}  
1) 75\% & 2) 55\% & 3) 60\% \\ 
4) Cannot be determined & 5) None of these &  \\ 
\end{tabular}



\noindent 

\noindent 

\noindent \\ 38. In a class of 50 students and 5 teachers, each student got sweets that are 12\% of the total number of students and each teacher got sweets that are 20\% of the total number of students. How many sweets were there?

\noindent 1) 345                      2) 365               3) 330               4) 350               5) None of these

\noindent \\ 39. The income of A is 150\% of the income of B and the income of C is 120\% of the income of A. If the total income of A, B and C together is Rs. 86,000, what is C's income?

\noindent \textbf{[April 18, 2015 @ 30m 40s]}

\noindent 

\noindent \includegraphics*[width=0.60in, height=0.52in]{images/image1} 1) Rs. 30,000              2) Rs. 32,000       3) Rs. 20,000       4) Rs. 36,000       5) None of these

\noindent \\ 40. The price of sugar falls by 12\%. How many quintals can be bought for the same money which was sufficient to buy 44 quintals at the higher price? \textbf{[August 02, 2014 @ 1h 49m 15s]}

\noindent \includegraphics*[width=0.60in, height=0.52in]{images/image1} 1) 54                        2) 50                 3) 38                 4) 48                 5) None of these

\noindent 

\noindent 

\noindent 

\noindent 

\noindent \textbf{Answers}

\noindent 

\noindent 

\begin{tabular}{|p{0.8in}|p{0.5in}|p{0.5in}|p{0.5in}|p{0.5in}|p{0.5in}|p{0.5in}|p{0.5in}|p{0.5in}|p{0.5in}|} \hline 
1 - 1 & 2 - 3 & 3 - 4 & 4 - 1 & 5 - 3 & 6 - 3 & 7 - 3 & 8 - 1 & 9 - 3 & 10 - 2 \\ \hline 
11 - 2 & 12 - 1 & 13 - 3 & 14 - 3 & 15 - 4 & 16 - 4 & 17 - 4 & 18 - 2 & 19 - 1 & 20 - 1 \\ \hline 
21 - 3 & 22 - 4 & 23 - 4 & 24 - 5 & 25 - 3 & 26 - 3 & 27 - 1 & 28 - 4 & 29 - 4 & 30 - 4 \\ \hline 
31 - 2 & 32 - 4 & 33 - 3 & 34 - 2 & 35 - 4 & 36 - 5 & 37 - 2 & 38 - 4 & 39 - 4 & 40 - 2 \\ \hline 
\end{tabular}



\noindent 

\noindent \\

\noindent \textbf{\\ Note: }The date and time mentioned against some questions refer to the doubts clarification session on Quantitative Aptitude in which the question was solved.

\noindent 

\noindent 

\noindent 

\noindent 



\end{document}