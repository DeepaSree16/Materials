\documentclass{article} 

\usepackage[utf8]{inputenc} 
\usepackage[english]{babel}
\usepackage{amsmath}
\usepackage{amssymb}
\usepackage{txfonts}
\usepackage{mathdots}
\usepackage[classicReIm]{kpfonts}
\usepackage{graphicx}

\usepackage{multirow}
\usepackage[margin=1.0in]{geometry}
\usepackage[english]{babel}
\usepackage[utf8]{inputenc}
\usepackage{fancyhdr}
\usepackage{tabularx}
\pagestyle{fancy}
\fancyhf{}
\rhead{\includegraphics[width=1.0in, height=0.38819in]{images/logo.png}}
\lhead{ Quantitative Aptitude: Ratio and Proportion}
\lfoot{www.talentsprint.com }
\rfoot{\thepage}



\begin{document}
	\noindent \begin{center}
		{\Large \textbf{Ratio and Proportion \\}}
	\end{center}
	
	
	\noindent {\large \textbf{\\Part 2 - Advanced}}



\noindent \\  

\noindent 

\noindent \\   1.   An employer reduces the number of employees in the ratio 8:5 and increases their wages in

\noindent  
	\noindent \\ \includegraphics*[width=0.60in, height=0.52in]{images/image1}the ratio 7:9. As a result, the overall wages bill is\_

\noindent \begin{tabular}{p{1.7in} p{1.6in} p{1.6in}} \\ 
 1) Increased in the ratio 56:69                    &  2) Decreased in the ratio 56:45

\noindent 

\noindent &  3) Increased in the ratio 13:17                    \\
4) Decreased in the ratio 17:13 \\
\end{tabular}

\noindent 

\noindent 

\noindent 

\noindent \\   2.   If x:y = 5:6, then (3x² - 2y²) : (y² - x²) is

\noindent  
	\noindent \\ \includegraphics*[width=0.60in, height=0.52in]{images/image1}\begin{tabular}{p{1.7in} p{1.6in} p{1.6in}} \\ 
 1) 7:6                       &  2) 11:3              &  3) 3:11               \\
4) 6:7 \\
\end{tabular}

\noindent 

\noindent \\   3.   What number should be added to or subtracted from each term of the ratio 17:24 so that it

\noindent  
	\noindent \\ \includegraphics*[width=0.60in, height=0.52in]{images/image1}becomes equal to 1:2?

\noindent \begin{tabular}{p{1.7in} p{1.6in} p{1.6in}} \\ 
 1) 5 is subtracted                                          &  2) 10 is added

\noindent 

\noindent &  3) 7 is added                                                 \\
4) 10 is subtracted \\
\end{tabular}

\noindent \\  4.   Monthly incomes of A and B are in the ratio of 4:3 and their expenses bear the ratio 3:2. Each  
	\noindent \\ \includegraphics*[width=0.60in, height=0.52in]{images/image1}of them saves Rs. 6,000 at the end of the month, then the monthly income of A is

\noindent \begin{tabular}{p{1.7in} p{1.6in} p{1.6in}} \\ 
 1) Rs. 12,000              &  2) Rs. 24,000       &  3) Rs. 30,000        \\
4) Rs. 60,000 \\
\end{tabular}

\noindent 

\noindent 

\noindent  \\  5.   One year ago, the ratio between A's and B's salary was 3:4. The ratio of their individual

\noindent  
	\noindent \\ \includegraphics*[width=0.60in, height=0.52in]{images/image1}salaries between last and this year was 4:5 and 2:3 respectively. Now, the total of their salaries is Rs. 41600. A's present salary is

\noindent 

\noindent \begin{tabular}{p{1.7in} p{1.6in} p{1.6in}} \\ 
 1) Rs. 10400               &  2) Rs. 16000        &  3) Rs. 25600         \\
4) Rs. 31200 \\
\end{tabular}

\noindent 

\noindent 

\noindent 
\newpage
\noindent   6.   The incomes of A, B and C are in the ratio 7:9:12 and their spending are in the ratio 8:9:15. If

\noindent 

\noindent  
	\noindent \\ \includegraphics*[width=0.60in, height=0.52in]{images/image1}A saves 1/4th of his income, then the savings of A, B and C are in the ratio of
\\	
\begin{tabular}{p{1.7in} p{1.6in} p{1.6in}} \\ 
 1) 69:56:48              &  2) 47:74:99       &  3) 37:72:49        \\
4) 56:99:69 \\
\end{tabular}

\noindent 

\noindent \\   7.   The third proportional to $ ( \frac{x}{u} + \frac{y}{x}) $ and $ \sqrt{x^{2} + y^{2}} $

\noindent \begin{tabular}{p{1.7in} p{1.6in} p{1.6in}} \\ 
 1) xy &  2) $\mathrm{\sqrt{xy}}$ &  3) ${}^{3}$$\mathrm{\sqrt{xy}}$  \\
4) ${}^{4}$$\mathrm{\sqrt{xy}}$ \\
\end{tabular}

\noindent 

\noindent \\   8.Brothers A and B had some savings in the ratio 2:5. They decided to buy a gift for their

\noindent 

\noindent  
	\noindent \\ \includegraphics*[width=0.60in, height=0.52in]{images/image1}sister, sharing the cost in the ratio 3:4. After they bought, A spent two--third of his amount while B is left with 145. Then the value of the gift is?

\noindent 

\noindent \begin{tabular}{p{1.7in} p{1.6in} p{1.6in}} \\ 
 1) Rs. 70                     &  2) Rs. 105            &  3) Rs. 140             \\
4) Rs. 175 \\
\end{tabular}

\noindent  \\  9.   A, B, C and D started a business and their investment were in the ratio 3:4:5:6. B and C are  
	\noindent \\ \includegraphics*[width=0.60in, height=0.52in]{images/image1}working partners and they get equal salaries. The ratio of the total annual income of B and C

\noindent was 9:10. If the total annual profit was 84000, find B's salary.

\noindent 


	\begin{tabular}{p{1.7in} p{1.6in} p{1.6in}} \\ 
 1) Rs. 8,000  &  2) Rs. 18,000  &  3) Rs. 12,000
	\\
4) Rs. 15,000  & 5) Rs. 20,000  \\
\end{tabular}




\noindent 

\noindent  \\  10. Rabinder spends 60\% of his monthly salary on rent, EMI and miscellaneous expenses in the

\noindent  
	\noindent \\ \includegraphics*[width=0.60in, height=0.52in]{images/image1}ratio of 2:1:3.  If he spends a total of Rs. 16050 on rent and EMI together, how much is his monthly salary?

\noindent 

\noindent \begin{tabular}{p{1.7in} p{1.6in} p{1.6in}} \\ 
 1) 50300                  &  2) 49500           &  3) 46750           \\
4) 53500           & 5) 54500 \\
\end{tabular}

\noindent 

\noindent 

\noindent \\   

11. Rs. 78 is divided among 7 men, 11 women and 5 boys so that each women may have thrice  
	\noindent \\ \includegraphics*[width=0.60in, height=0.52in]{images/image1}as much as a boy and as much as a man and a boy together. Find the share of a man.

\noindent \begin{tabular}{p{1.7in} p{1.6in} p{1.6in}} \\ 
 1) 1.50                     &  2) 3                   &  3) 4.50              \\
4) 6 \\
\end{tabular}
\newpage
\noindent   12. In a vegetable shop, there are three types of vegetables namely, cabbage, potato \& carrot in  
	\noindent \\ \includegraphics*[width=0.60in, height=0.52in]{images/image1}the ratio 1 : 57 : 3. The shop owner proposes a new scheme to increase his sales. With every 15 potatoes, 1 carrot is given as free \& when a customer receives 11 free carrots, he gets 1 cabbage as bonus. Cabbages \& carrots are only given as free products. A man buys some potatoes and receives 12 cabbages along with some carrots. After this the shop owner is left with no carrot. What is the number of potatoes left in the shop?

\noindent 

\noindent \begin{tabular}{p{1.7in} p{1.6in} p{1.6in}} \\ 
 1) 555                      &  2) 528               &  3) 654               \\
4) 344               & 5) 599 \\
\end{tabular}

\noindent 

\noindent 

\noindent  \\  

13. A vendor sells 2 kinds of sugar. The per kg rate of the dearer sugar is 25\% more than the  
	\noindent \\ \includegraphics*[width=0.60in, height=0.52in]{images/image1}cheaper one. In what ratio should the cheaper and dearer be mixed and sold at the per kg

\noindent rate of dearer sugar to gain 10\% per kg?

\noindent 

\noindent \begin{tabular}{p{1.7in} p{1.6in} p{1.6in}} \\ 
 1) 5 : 6                     &  2) 6 : 5              &  3) 5 : 3              \\
4) 2 : 5              & 5) None of these  \\
\end{tabular}

\noindent  \\  14. Iron ores from 3 regions are mixed together to extract iron. Total 4 elements are present in  
	\noindent \\ \includegraphics*[width=0.60in, height=0.52in]{images/image1} iron ore -carbon, iron, nickel, oxygen. In 1st iron ore batch contains the 4 elements in ratio 3:5:7:11 respectively. 2nd iron ore batch contains elements in ratio 4:5:9:3 resp and 3rd iron ore batch contains elements in the ratio 2:6:5:1 resp. Iron ores are mixed in ratio 2:3:4. All ratios are by weight . Find the ratio of iron to carbon in total mixture.

\noindent \begin{tabular}{p{1.7in} p{1.6in} p{1.6in}} \\ 
 1) 49:26                   &  2) 26:49            &  3) 23:43             \\
4) 43:23            & 5) None of these  \\
\end{tabular}



\noindent  \\  

\noindent \\   \textbf{Answers}

\noindent 

\noindent 

\noindent \\   1.   Decreased in the ratio 56:45

\noindent 

\noindent 2.   3:11

\noindent 

\noindent 3.   10 is subtracted

\noindent 

\noindent 4.   Rs. 24,000

\noindent 

\noindent 5.   Rs. 16000

\noindent 

\noindent 6.   56:99:69

\noindent 

\noindent 7.   xy

\noindent 

\noindent 8.   Rs. 140

\noindent 

\noindent 9.   Rs. 15,000

\noindent 

\noindent 10. 53500

\noindent 

\noindent 11. 3

\noindent 

\noindent 12. 528

\noindent 

\noindent 13. 5 : 6

\noindent 

\noindent 14. - 


\noindent 
\end{document}