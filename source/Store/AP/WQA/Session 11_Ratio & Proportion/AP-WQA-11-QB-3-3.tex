\documentclass{article} 

\usepackage[utf8]{inputenc} 
\usepackage[english]{babel}
\usepackage{amsmath}
\usepackage{amssymb}
\usepackage{txfonts}
\usepackage{mathdots}
\usepackage[classicReIm]{kpfonts}
\usepackage{graphicx}

\usepackage{multirow}
\usepackage[margin=1.0in]{geometry}
\usepackage[english]{babel}
\usepackage[utf8]{inputenc}
\usepackage{fancyhdr}
\usepackage{tabularx}
\pagestyle{fancy}
\fancyhf{}
\rhead{\includegraphics[width=1.0in, height=0.38819in]{images/logo.png}}
\lhead{ Quantitative Aptitude: Ratio and Proportion}
\lfoot{www.talentsprint.com }
\rfoot{\thepage}



\begin{document}
	\noindent \begin{center}
		{\Large \textbf{Ratio and Proportion \\}}
	\end{center}
	
	
	\noindent {\large \textbf{\\Additional Examples}}



\noindent \\  

\noindent  \\   1.   The ratio of 252.5 : 53 is same as --

\noindent  
	\noindent \\ \includegraphics*[width=0.60in, height=0.52in]{images/image1}
	
\begin{tabular}{p{1.7in} p{1.6in} p{1.6in}} \\ 
 1) 5 : 3                     &  2) 5 : 6              &  3) 1 : 25             \\
4) 25 : 1 \\
\end{tabular}

\noindent  \\  

2.   If  78 is divided into three parts which are in the ratio 1 : 1/3 : 1/6, the middle part is  
	\noindent \\ \includegraphics*[width=0.60in, height=0.52in]{images/image1}\begin{tabular}{p{1.7in} p{1.6in} p{1.6in}} \\ 
 1) 9 1/3                    &  2) 13                 &  3) 17 1/3           \\
4) 18 1/3 \\
\end{tabular}

\noindent  \\  

\noindent 3.   A drum of kerosene is 3/4 full. When 30 litres of kerosene is drawn from it, it remains 7/12

\noindent  
	\noindent \\ \includegraphics*[width=0.60in, height=0.52in]{images/image1}full. The capacity of the drum is

\noindent \begin{tabular}{p{1.7in} p{1.6in} p{1.6in}} \\ 
 1) 120 litres            &  2) 135 litres     &  3) 150 litres      \\
4) 180 litres \\
\end{tabular}

\noindent  \\  

4.   Monthly incomes of A and B are in the ratio of 4:3 and their expenses bear the ratio 3:2. Each  
	\noindent \\ \includegraphics*[width=0.60in, height=0.52in]{images/image1}of them saves Rs. 6,000 at the end of the month, then the monthly income of A is

\noindent \begin{tabular}{p{1.7in} p{1.6in} p{1.6in}} \\ 
 1) Rs. 12,000          &  2) Rs. 24,000    &  3) Rs. 30,000    \\
4)  60,000 \\
\end{tabular}

\noindent \\  
5.   Hans and Bhaskar have salaries that jointly amount to Rs. 10,000 per month. They spend the  
	\noindent \\ \includegraphics*[width=0.60in, height=0.52in]{images/image1}same  amount  monthly and  then it  is found ratio  of  their  savings  is 6:1.  Which  of the  following can be Hans's salary?

\noindent 

\noindent \begin{tabular}{p{1.7in} p{1.6in} p{1.6in}} \\ 
 1) 6000                    &  2) 5000             &  3) 4000             \\
4) 3000 \\
\end{tabular}

\noindent 

\noindent  \\  

6.   Three numbers are in the ratio 1 : 2 : 3. By adding 5 to each of them, the new numbers are in  
	\noindent \\ \includegraphics*[width=0.60in, height=0.52in]{images/image1}the ratio 2 : 3 : 4. The numbers are

\noindent  \\  

7.   Number of rings in boxes are in the ratio, 2 : 3 : 5. If 20 extra rings are added to each of the  
	\noindent \\ \includegraphics*[width=0.60in, height=0.52in]{images/image1}boxes, the new ratio of the rings are 4 : 5 : 7. What is the total number of rings in the three

\noindent boxes before the increase?

\noindent 

\noindent  \\  

\noindent 8.   Rs. 78 is divided among 7 men, 11 women and 5 boys so that each women may have thrice

\noindent  
	\noindent \\ \includegraphics*[width=0.60in, height=0.52in]{images/image1}as much as a boy and as much as a man and a boy together. Find the share of a man.

\noindent \begin{tabular}{p{1.7in} p{1.6in} p{1.6in}} \\ 
 1) 1.50                     &  2) 3                   &  3) 4.50              \\
4) 6 \\
\end{tabular}

\noindent  \\  

9.   The total number of men, women and children working in a factory is 18. They earn Rs. 4000  
	\noindent \\ \includegraphics*[width=0.61in, height=0.52in]{images/image1}in day. If the sum of the wages of all the men, women and children is in the ratio of 18:10:12

\noindent and if the wages of an individual man, woman and child is in the ratio 6 : 5 : 3, then how

\noindent 

\noindent much a woman earn in a day?

\noindent 

\noindent \begin{tabular}{p{1.7in} p{1.6in} p{1.6in}} \\ 
 1) 400                      &  2) 250               &  3) 150               \\
4) 120               & 5) None of these  \\
\end{tabular} 

\noindent 

\noindent  \\  

10. One year ago the ratio between Laxman's and Gopal's salary was 3:4. The ratio of their  
	\noindent \\ \includegraphics*[width=0.61in, height=0.52in]{images/image1}individual salaries between last year and this year salaries are 4:5 and 2:3 respectively. At

\noindent present the total of their salary is 4290. The salary of Laxman now is?

\noindent 


\noindent \\   11. Answer the questions based on the following information.

\noindent  
	\noindent \\ \includegraphics*[width=0.61in, height=0.52in]{images/image1}Alphonso, on his deathbed, keeps half his property for his wife and divides the rest equally among  his  three sons; Ben, Carl and Dave. Some years later, Ben dies leaving half his property to his widow and half to his brothers Carl and Dave together, sharing equally. When Carl makes his will, he keeps half his property for his widow the rest he bequeaths to his younger brother Dave. When Dave dies some years later, he keeps half his property for his  widow  and  the  remaining  for  his  mother.  The  mother  now  has  Rs.  15,75,000. Q. What was the worth of the total property?

\noindent \begin{tabular}{p{1.7in} p{1.6in} p{1.6in}} \\ 
 1) Rs. 30 lakhs                                             &  2) Rs. 8 lakhs                           &  3) Rs. 18 lakhs

\noindent 

\noindent \\
4) Rs. 24 lakhs                                             & 5) Rs. 25 lakhs \\
\end{tabular}

\noindent 

\noindent  \\  

\noindent  \\  

\noindent \\   Q. What was Carl's original share?

\noindent 

\noindent \begin{tabular}{p{1.7in} p{1.6in} p{1.6in}} \\ 
 1) Rs. 4 lakhs                                               &  2) Rs. 12 lakhs                         &  3) Rs. 6 lakhs

\noindent 

\noindent \\
4) Rs. 5 lakhs                                               & 5) Rs. 2 lakhs \\
\end{tabular}

\noindent 

\noindent 

\noindent \\   Q. What was the ratio of the property owned by the widows of the three sons in the end?

\noindent 

\noindent \begin{tabular}{p{1.7in} p{1.6in} p{1.6in}} \\ 
 1) 7 : 9 : 13             &  2) 8 : 10 : 15    &  3) 5 : 7 : 9        \\
4) 9 : 12 : 13    & 5) 8 : 9 : 13 \\
\end{tabular}

\noindent  \\  12. The cost of an uncut diamond varies directly as the square of its weight. A diamond was cut  
	\noindent \\ \includegraphics*[width=0.60in, height=0.52in]{images/image1}into four pieces with their weights in the ratio of 1: 2: 3: 4. If the loss in the total value of the

\noindent diamond was Rs. 49000, then price of the original diamond was?

\noindent 

\noindent \begin{tabular}{p{1.7in} p{1.6in} p{1.6in}} \\ 
 1) 40000                  &  2) 10000           &  3) 56000           \\
4) 70000           & 5) None of these  \\
\end{tabular}

\noindent 

\noindent  \\  

13. X, Y, Z were sharing profits in the ratio of 4:3:2. Y retired from the firm and X \& Z decided  
	\noindent \\ \includegraphics*[width=0.60in, height=0.52in]{images/image1}to share profits in the ratio of 3:2. Calculate the gaining ratio.





\noindent  \\  

\noindent 14. The railway fares of air conditioned sleeper and ordinary sleeper class are in the ratio 4:1.

\noindent  
	\noindent \\ \includegraphics*[width=0.60in, height=0.52in]{images/image1}The number of passengers travelled by air conditioned sleeper and ordinary sleeper classes were in the ratio 3:25. If the total collection was Rs. 37,000, how much did air conditioner

\noindent 

\noindent sleeper passengers pay?

\noindent 

\noindent \begin{tabular}{p{1.7in} p{1.6in} p{1.6in}} \\ 
 1) Rs. 15,000          &  2) Rs. 10,000    &  3) Rs. 12,000    \\
4) Rs. 16,000 \\
\end{tabular}

\noindent 

\noindent 

\noindent  \\  

15. A milkman makes 80\% profit by selling milk mixed with water at 2/- per liter. Compute the  
	\noindent \\ \includegraphics*[width=0.60in, height=0.52in]{images/image1} ratio of milk and water in the sold mixture if the cost price of 1-liter pure milk is 100/9.

\noindent \begin{tabular}{p{1.7in} p{1.6in} p{1.6in}} \\ 
 1) 1 : 9                     &  2) 9 : 1              &  3) 9 : 2              \\
4) 2 : 9              & 5) None \\
\end{tabular}

\noindent 
\newpage
\noindent 16. A jar contains a mixture of two liquids A and B in the ration 4 is to 1. When 10 lt of the  
	\noindent \\ \includegraphics*[width=0.60in, height=0.52in]{images/image1} mixture is replaced with liq B, the ratio becomes 2 is to 3. The volume of liq A present in the

\noindent jar earlier was

\noindent 

\noindent \begin{tabular}{p{1.7in} p{1.6in} p{1.6in}} \\ 
 1) 10lt                     &  2) 16lt               &  3) 15lt               \\
4) 20lt \\
\end{tabular}

\noindent  \\  

\noindent 17. Amit and Manju are best friends. Both secured a job during the campus placements. Ratio of

\noindent  
	\noindent \\ \includegraphics*[width=0.60in, height=0.52in]{images/image1}their salary is in the ratio of 7 : 8. After getting their joining location, Amit got a job in his

\noindent home  town  whereas  Manju  got  a  job  outside  her  hometown  due  to  which  Manju's expenditure becomes more than Amit's expenditure is in the ratio of 6 : 7. If both of them saved 8000, then, find the salary of Manju?

\noindent \begin{tabular}{p{1.7in} p{1.6in} p{1.6in}} \\ 
 1) 40000                  &  2) 48000           &  3) 49000            \\
4) 56000           & 5) 64000 \\
\end{tabular}

\noindent \\  
18. A jar was full with honey. A person replaced 25\% of the honey with sugar solution. He  
	\noindent \\ \includegraphics*[width=0.60in, height=0.52in]{images/image1} repeated the process 5 times and as a result there was only 972 grams of honey left in the jar,

\noindent the remaining part of the jar being filled with sugar solution. The initial quantity of honey in the jar was

\noindent 

\noindent \begin{tabular}{p{1.7in} p{1.6in} p{1.6in}} \\ 
	
 1) 4.096 kg             &  2) 5.385 kg       &  3) 5.86 kg         \\
4) 3.42 kg        & 5) None of these  \\
\end{tabular}
\noindent \\  
19. A certain amount of money has been divided between two persons P and Q in the ratio 3 : 5,  
	\noindent \\ \includegraphics*[width=0.60in, height=0.52in]{images/image1}but it was divided in the ratio of 2 : 3 and thereby Q loses Rs. 10. What was the amount?

\noindent 
\begin{tabular}{p{1.7in} p{1.6in} p{1.6in}} \\ 
 1) Rs. 250               &  2) Rs. 300         &  3) Rs. 350         \\
4) Rs. 400        & 5) None of these  \\
\end{tabular}



\noindent  \\   \\  

\noindent \textbf{Answers}

\noindent 

\noindent  \\  

\begin{tabular}{|p{0.7in}|p{0.5in}|p{0.5in}|p{0.5in}|p{0.5in}|p{0.5in}|p{0.5in}|p{0.5in}|p{0.5in}|p{0.5in}|} \hline 
	1 - b & 2 - d & 3 - c & 4 - c & 5 - d & 6 - d & 7 - b & 8 - b & 9 - d & 10 - a \\ \hline 
\end{tabular}

 
\end{document}