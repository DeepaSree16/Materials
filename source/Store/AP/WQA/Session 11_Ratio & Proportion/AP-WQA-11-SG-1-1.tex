\documentclass{article} 

\usepackage[utf8]{inputenc} 
\usepackage[english]{babel}
\usepackage{amsmath}
\usepackage{amssymb}
\usepackage{txfonts}
\usepackage{mathdots}
\usepackage[classicReIm]{kpfonts}
\usepackage{graphicx}

\usepackage{multirow}
\usepackage[margin=1.0in]{geometry}
\usepackage[english]{babel}
\usepackage[utf8]{inputenc}
\usepackage{fancyhdr}
\usepackage{tabularx}
\pagestyle{fancy}
\fancyhf{}
\rhead{\includegraphics[width=1.0in, height=0.38819in]{images/logo.png}}
\lhead{ Quantitative Aptitude: Ratio and Proportion}
\lfoot{www.talentsprint.com }
\rfoot{\thepage}



\begin{document}
	\noindent \begin{center}
		{\Large \textbf{Ratio and Proportion \\}}
	\end{center}




\noindent 

\noindent 

\noindent 

\noindent 

\noindent  \\ When two numbers are represented in the form of another, this is done by expressing one number as a fraction of another

\noindent 

\noindent \\  a : b if there are two numbers ax, bx then ax/bx = a/b a : b

\noindent 

\noindent 

\noindent Comparison of two quantities by division

\noindent 

\noindent 

\noindent Relation of one quantity with another

\noindent 

\noindent 

\noindent 

\noindent 

\noindent \\  

\noindent \\  
\textbf{Duplicate ratio:}

\noindent 

\noindent 

\noindent \\  Ratio of squares of two numbers is called duplicate ratio of the two numbers

\noindent 

\noindent 

\noindent E.g.:$  3^{2}/4^{2} $ = 9/16 is duplicate ratio of 3/4

\noindent 

\noindent 

\noindent 

\noindent 

\noindent 

\noindent  \\ \textbf{Triplicate ratio:}

\noindent 

\noindent 

\noindent \\   Ratio of cubes $ 3^{3}/4^{3} $ = 27/64 is triplicate ratio of ¾

\noindent 

\noindent 

\noindent Sub duplicate ratio of square root $ \frac{3}{4} $is sub duplicate ratio of 9/16

\noindent 

\noindent 

\noindent  \\ \textbf{Proportion:}

\noindent 

\noindent 

\noindent \\  Equality of two ratios is called proportions

\noindent 

\noindent 

\noindent if a/b = c/d then a, b, c, d are in proportions we write this as 
a : b  : : c : d $\mathrm{\to}$ a, d extremes

\noindent 

\noindent c, b. means

\noindent 

\noindent 

\noindent \\   -$\mathrm{>}$ product of extremes = Product of means a x d = b x c

\noindent \\  \textbf{Fourth Proportional:}

\noindent 

\noindent 

\noindent \\   If a : b : : c  : x then x is called fourth proportional of a, b, c and $ x = \frac{b \mathrm{\times} c}{a} $

\noindent 

\noindent \\   -$\mathrm{>}$ Mean proportional

\noindent 

\noindent 

\noindent \\   If a : x : : x : b , x is called mean or second proportional of a, b 


\noindent \\  $ x^{2} $ = ab, x = $\mathrm{\sqrt{}}$ab

\noindent \\   If a/b = c/d then

\noindent \\   $ \frac{a+b}{b} = \frac{c+d}{d} $ coponendo)


\noindent \\   $ \frac{a-b}{b} = \frac{c-d}{d} $     (dividendo)


\noindent \\   $ \frac{a+b}{a-b} = \frac{c+d}{c-d} $ (componendo / dividend)

\noindent \\   $ \frac{a}{b} = \frac{a+c}{b+d} = \frac{a-c}{b-d}$

\noindent  \\  Then a/b = c/d = e/f - - = K = $ \frac{a+c+e--}{b+d+f--} $

\noindent 

\end{document}