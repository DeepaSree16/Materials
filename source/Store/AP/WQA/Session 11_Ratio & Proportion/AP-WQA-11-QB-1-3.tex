
\documentclass{article} 

\usepackage[utf8]{inputenc} 
\usepackage[english]{babel}
\usepackage{amsmath}
\usepackage{amssymb}
\usepackage{txfonts}
\usepackage{mathdots}
\usepackage[classicReIm]{kpfonts}
\usepackage{graphicx}

\usepackage{multirow}
\usepackage[margin=1.0in]{geometry}
\usepackage[english]{babel}
\usepackage[utf8]{inputenc}
\usepackage{fancyhdr}
\usepackage{tabularx}
\pagestyle{fancy}
\fancyhf{}
\rhead{\includegraphics[width=1.0in, height=0.38819in]{images/logo.png}}
\lhead{ Quantitative Aptitude: Ratio and Proportion}
\lfoot{www.talentsprint.com }
\rfoot{\thepage}



\begin{document}
	\noindent \begin{center}
		{\Large \textbf{Ratio and Proportion \\}}
	\end{center}
	

	\noindent {\large \textbf{\\ Part 1 - Basic}}
	
	\noindent 
	
	\noindent \\  
	
	\noindent \\   \textbf{Model 1: Basic Ratio and Proportion}
	
	\noindent 
	
	\noindent 
	
	\noindent \\   1.   A company 'A' sells 53 cm model TV at the price of Rs. 7,000 whereas another company 'S' sells the same model at the price of Rs. 16,800. What is the ratio of their respective selling prices?
	
	\noindent \begin{tabular}{p{1.7in} p{1.6in} p{1.6in}} \\ 
 1) 12:5                     &  2) 5:12              &  3) 1:2                \\
4) 2:1                & 5) None of these  \\
\end{tabular}
	
	\noindent 
	
	\noindent 
	
	\noindent 
	
	\noindent \\   2.   There are 2304 workers in company A and 2816 in company B. What is the ratio of workers in company A to that of B?
	
	\noindent \begin{tabular}{p{1.7in} p{1.6in} p{1.6in}} \\ 
 1) 7:13                     &  2) 13:7              &  3) 9:11              \\
4) 11:9              & 5) None of these  \\
\end{tabular}
	
	\noindent 
	
	\noindent 
	
	\noindent 
	
	\noindent \\   3.   In entrance test, the ratio of applicants to successful students was 21:11. If 1176 students appeared in the test, how many got through it?
	
	\noindent \begin{tabular}{p{1.7in} p{1.6in} p{1.6in}} \\ 
 1) 715                      &  2) 616               &  3) 605               \\
4) 563               & 5) None of these  \\
\end{tabular}
	
	\noindent 
	
	\noindent 
	
	\noindent 
	
	\noindent \\   4.   207 students appeared in an examination out of 276 registered students. Find out the ratio between registered to the appeared students.
	
	\noindent \begin{tabular}{p{1.7in} p{1.6in} p{1.6in}} \\ 
 1) 12:23                   &  2) 23:12            &  3) 9:23              \\
4) 23:9              & 5) None of these  \\
\end{tabular}
	
	\noindent  \\  
	
	5.   The ratio of boys and girls studying in a school is 17:18. If the number of boys is 150 less  
	\noindent \\ \includegraphics*[width=0.61in, height=0.52in]{images/image1}than the number of girls, then what is the total number of girls?
	
	\noindent \begin{tabular}{p{1.7in} p{1.6in} p{1.6in}} \\ 
 1) 2700                    &  2) 2800             &  3) 2900             \\
4) 2100             & 5) None of these  \\
\end{tabular}
	
	\noindent 
	
	\noindent 
	
	\noindent 
	
	\noindent \\   6.   The ratio of boys to girls in a school is 6:5. The number of boys is more than the number of
	
	\noindent 
	
	\noindent girls by 200. What is the total number of girls in the school?
	
	\noindent 
	
	\noindent \begin{tabular}{p{1.7in} p{1.6in} p{1.6in}} \\ 
 1) 800                      &  2) 1000             &  3) 1200             \\
4) 2200             & 5) None of these  \\
\end{tabular}
	
	\noindent  \\  
	
	7.   The cost of 3 digital cameras and 5 cell phones is Rs. 35,290. What is the cost of 9 digital  
	\noindent \\ \includegraphics*[width=0.61in, height=0.52in]{images/image1}camera and 15 cell phones?          \textbf{[May 09, 2015 @ 12m 40s]}
	
	\noindent \begin{tabular}{p{1.7in} p{1.6in} p{1.6in}} \\ 
 1) Rs. 1,05,870                                                  &  2) Rs. 6,46,960                              &  3) Rs. 1,76,450
	
	\noindent 
	
	\noindent \\
4) Cannot be determined                           & 5) None of these  \\
\end{tabular}
	
	\noindent 
	
	\noindent 
	
	\noindent 
	
	\noindent \\   8.   Twice the square of a number is six times the other number. What is the ratio of the first number to the second? 
	
		\begin{tabular}{p{1.7in} p{1.6in} p{1.6in}} \\ 
 1) 1:4 &  2) 2:5 &  3) 1:3  \\
4) cannot be determined   & 5) None of these  \\
\end{tabular} 
	
	
	
	
	\noindent 
	
	\noindent 
	
	\noindent  \\  \textbf{Model 2: Distribution of Amount among the given Persons}
	
	\noindent 
	
	\noindent 
	
	\noindent \\   9.   A profit of Rs. 8000 is to be distributed among A, B and C in the proportions of 5:2:3
	
	\noindent  
	\noindent \\ \includegraphics*[width=0.60in, height=0.52in]{images/image1}respectively. What is the difference between the shares of A and B?
	
	\noindent \begin{tabular}{p{1.7in} p{1.6in} p{1.6in}} \\ 
 1) Rs. 1800                 &  2) Rs. 2400          &  3) Rs. 3600          \\
4) Rs. 900            & 5) None of these  \\
\end{tabular}
	
	\noindent 
	
	\noindent 
	
	\noindent 
	
	\noindent \\   10. A certain amount is to be distributed among Samiksha, Purva and Neha in the proportion of  5:3:4. If the difference between Purva's and Samiksha's share is Rs. 1200. How much did Neha get?
	
	\noindent \begin{tabular}{p{1.7in} p{1.6in} p{1.6in}} \\ 
 1) Rs. 2400                                                        &  2) Rs. 1600                                     &  3) Rs. 2200
	
	\noindent 
	
	\noindent \\
4) Cannot be determined                            & 5) None of these  \\
\end{tabular}
	
	\noindent 
	
	\noindent 
	
	\noindent 
	
	\noindent 
	
	\noindent \\   \textbf{Model 3: Calculation of Total Amount Based on the given Ratio and Difference}
	
	\noindent 
	
	\noindent 
	
	\noindent  \\  11. A sum of money is divided among A, B, C and D ratio 3:5:8:9 respectively. If the share of D
	
	\noindent  
	\noindent \\ \includegraphics*[width=0.60in, height=0.52in]{images/image1}is Rs. 1872 more than the share of A, then what is the total amount of money of B \& C together?
	
	\noindent 
	
	\noindent \begin{tabular}{p{1.7in} p{1.6in} p{1.6in}} \\ 
 1) Rs. 4156                 &  2) Rs. 4165          &  3) Rs. 4056          \\
4) Rs. 4065          & 5) None of these  \\
\end{tabular}
	
	\noindent 
	
	\noindent 
	
	\noindent 
	
	\noindent \\   12. A sum of money is divided among A, B, C and D in the ratio 5:8:9:11. If the share of B is Rs. 2475 more than the share of A then what is the total amount of money of A \& C together?
	
	\noindent 
	
	\noindent \begin{tabular}{p{1.7in} p{1.6in} p{1.6in}} \\ 
 1) Rs. 9900                 &  2) Rs. 11550        &  3) Rs. 10725        \\
4) Rs. 9075          & 5) None of these  \\
\end{tabular}
	
	\noindent 
	
	\noindent 
	
	\noindent 
	
	\noindent 
	
	\noindent \\   \textbf{Model 4: New Ratio When There Is Absolute Change in the Values}
	
	\noindent 
	
	\noindent 
	
	\noindent \\   13. Amit, Sumit and Vinit divide an amount of Rs. 2800 amongst themselves in the ratio of
	
	\noindent 
	
	\noindent  
	\noindent \\ \includegraphics*[width=0.60in, height=0.52in]{images/image1} 5:6:3 respectively. If an amount of Rs. 200 is added to each of their shares, what will be the new ratio of their shares of the amount?
	
	\noindent 
	
	\noindent \begin{tabular}{p{1.7in} p{1.6in} p{1.6in}} \\ 
 1) 8:9:6                    &  2) 6:7:4             &  3) 7:8:5             \\
4) 4:5:2             & 5) None of these  \\
\end{tabular}
	
	\noindent 
	
	\noindent 
	
	\noindent 
	\newpage
	\noindent   14. A, B and C divide an amount of Rs. 4000 amongst themselves in the ratio of 2:5:1 respectively. If an amount of Rs. 800 is added to each of their shares, what will be the new ratio of their shares of the amount?
	
	\noindent \begin{tabular}{p{1.7in} p{1.6in} p{1.6in}} \\ 
 1) 18:33:13              &  2) 6:7:4             &  3) 7:8:9             \\
4) 3:6:2             & 5) None of these  \\
\end{tabular}
	
	\noindent 
	
	\noindent 
	
	\noindent 
	
	\noindent 
	
	\noindent \\   \textbf{Model 5: New Ratio When There Is a Percentage Change in the Values}
	
	\noindent 
	
	\noindent 
	
	\noindent \\   15. The ratio of the number of students studying in a school A, B and C is 5:6:8. If the
	
	\noindent  
	\noindent \\ \includegraphics*[width=0.60in, height=0.52in]{images/image1}number of students studying in each of the schools is increased by 30\%, 25\% and 25\% respectively, what will be the new ratio?
	
	\noindent 
	
	\noindent \begin{tabular}{p{1.7in} p{1.6in} p{1.6in}} \\ 
 1) 14:15:20              &  2) 13:15:20       &  3) 13:14:15       \\
4) 15:17:19       & 5) None of these  \\
\end{tabular}
	
	\noindent 
	
	\noindent 
	
	\noindent 
	
	\noindent  \\  16. The ratio of length and breadth of a rectangle is 3:2. If the breadth is increased by 20\% and the length is increased by 10\%, then what will be the new ratio of breadth and length?
	
	\noindent \begin{tabular}{p{1.7in} p{1.6in} p{1.6in}} \\ 
 1) 8:11                     &  2) 11:8              &  3) 7:10              \\
4) 22:33            & 5) None of these  \\
\end{tabular}
	
	\noindent 
	
	\noindent 
	
	\noindent 
	
	\noindent 
	
	\noindent \\   \textbf{Model 6: Combined Ratio}
	
	\noindent 
	
	\noindent 
	
	\noindent \\   17. Sita and Gita's ages are in the ratio of 3:4, Gita and Lata's ages are in the ratio of 4:7
	
	\noindent  
	\noindent \\ \includegraphics*[width=0.60in, height=0.52in]{images/image1} and Lata and Ram's ages are in the ratio of 7:9. What is the ratio of Sita's and Ram's ages?
	
	\noindent 
	
	\noindent \begin{tabular}{p{1.7in} p{1.6in} p{1.6in}} \\ 
 1) 3:5                       &  2) 3:7                &  3) 1:3                \\
4) 4:9                & 5) None of these  \\
\end{tabular}
	
	\noindent 
	
	\noindent 
	
	\noindent  \\  
	
	18. The ratio of income of A and B is 3:5 respectively and C and B is 9:7 respectively. If the  difference between the income of A and C is Rs. 4800, what is C's income?
	
	\noindent 
	
	\noindent 
	
	\noindent  \\  
	
	
	
	19. The ages of A and B are in the ratio of 3:5 and ages of C and D are in the ratio of 7:9. If the  
	\noindent \\ \includegraphics*[width=0.60in, height=0.52in]{images/image1}age difference between A and C is 24 years, what is the age of A?
	
	\noindent \textbf{[October 04, 2014 @ 28m 30s]}
	
	\noindent 
	
	\noindent \begin{tabular}{p{1.7in} p{1.6in} p{1.6in}} \\ 
 1) 20 years             &  2) 35 years       &  3) 17 years       \\
4) 24 years       & 5) None of these  \\
\end{tabular}
	
	\noindent 
	\newpage
	\noindent    20. A sum of Rs. 817 is divided among A, B and C such that A receives 25\% more than B and B
	
	\noindent  
	\noindent \\ \includegraphics*[width=0.60in, height=0.52in]{images/image1} receives 25\% less than C. What is A's share in the amount?
	
	\noindent \begin{tabular}{p{1.7in} p{1.6in} p{1.6in}} \\ 
 1) Rs. 228                   &  2) Rs. 247            &  3) Rs. 285            \\
4) Rs. 304            & 5) None of these  \\
\end{tabular}
	
	\noindent 
	
	\noindent 
	
	\noindent \\   

	21. If $ \frac{1}{4} $th area of a rectangular plot is 2700 sq m and the width of that plot is 90 m, what is the
	
	\noindent 
	
	\noindent ratio between the width and the length of the plot?
	
	\noindent 
	
	\noindent \begin{tabular}{p{1.7in} p{1.6in} p{1.6in}} \\ 
 1) 3:4                       &  2) 4:3                &  3) 3:1                \\
4) 1:3                & 5) None of these  \\
\end{tabular}
	
	\noindent 
	
	\noindent 
	
	\noindent \\   
	
	22. The ratio of length and breadth of a rectangular plot is 5:3. If the perimeter of the plot is 96m,
	\noindent \\ \includegraphics*[width=0.60in, height=0.52in]{images/image1} what is its area? \textbf{[May 09, 2015 @ 14m 40s]}
	
\noindent  \\  
	

	\noindent \\  {\large  \textbf{Answers}}
	
	\noindent 
	
	\noindent  \\  
	
	\begin{tabular}{|p{0.7in}|p{0.5in}|p{0.5in}|p{0.5in}|p{0.5in}|p{0.5in}|p{0.5in}|p{0.5in}|p{0.5in}|p{0.5in}|} \hline 
		1 - 2 & 2 - 3 & 3 - 2 & 4 - 5 & 5 - 1 & 6 - 2 & 7 - 1 & 8 - 4 & 9 - 2 & 10 - 1 \\ \hline 
		11 - 3 & 12 - 2 & 13 - 2 & 14 - 1 & 15 - 2 & 16 - 1 & 17 - 3 & 18 - 2 & 19 - 5 & 20 - 3 \\ \hline 
		21 - 1 & 22 - 3 & \multicolumn{8}{|p{3.6in}|}{} \\ \hline 
	\end{tabular}
	
	
	
	\noindent 
	
	\noindent  \\  
	
	\noindent \\   \textbf{Note: }The date and time mentioned against some questions refer to the doubts clarification session on Quantitative Aptitude in which the question was solved.
	
	\noindent 
\end{document}