% Options for packages loaded elsewhere
\PassOptionsToPackage{unicode}{hyperref}
\PassOptionsToPackage{hyphens}{url}
%
\documentclass[
]{article}
\usepackage{amsmath,amssymb}
\usepackage{lmodern}
\usepackage{iftex}
\ifPDFTeX
  \usepackage[T1]{fontenc}
  \usepackage[utf8]{inputenc}
  \usepackage{textcomp} % provide euro and other symbols
\else % if luatex or xetex
  \usepackage{unicode-math}
  \defaultfontfeatures{Scale=MatchLowercase}
  \defaultfontfeatures[\rmfamily]{Ligatures=TeX,Scale=1}
\fi
% Use upquote if available, for straight quotes in verbatim environments
\IfFileExists{upquote.sty}{\usepackage{upquote}}{}
\IfFileExists{microtype.sty}{% use microtype if available
  \usepackage[]{microtype}
  \UseMicrotypeSet[protrusion]{basicmath} % disable protrusion for tt fonts
}{}
\makeatletter
\@ifundefined{KOMAClassName}{% if non-KOMA class
  \IfFileExists{parskip.sty}{%
    \usepackage{parskip}
  }{% else
    \setlength{\parindent}{0pt}
    \setlength{\parskip}{6pt plus 2pt minus 1pt}}
}{% if KOMA class
  \KOMAoptions{parskip=half}}
\makeatother
\usepackage{xcolor}
\IfFileExists{xurl.sty}{\usepackage{xurl}}{} % add URL line breaks if available
\IfFileExists{bookmark.sty}{\usepackage{bookmark}}{\usepackage{hyperref}}
\hypersetup{
  hidelinks,
  pdfcreator={LaTeX via pandoc}}
\urlstyle{same} % disable monospaced font for URLs
\usepackage{longtable,booktabs,array}
\usepackage{calc} % for calculating minipage widths
% Correct order of tables after \paragraph or \subparagraph
\usepackage{etoolbox}
\makeatletter
\patchcmd\longtable{\par}{\if@noskipsec\mbox{}\fi\par}{}{}
\makeatother
% Allow footnotes in longtable head/foot
\IfFileExists{footnotehyper.sty}{\usepackage{footnotehyper}}{\usepackage{footnote}}
\makesavenoteenv{longtable}
\usepackage{graphicx}
\makeatletter
\def\maxwidth{\ifdim\Gin@nat@width>\linewidth\linewidth\else\Gin@nat@width\fi}
\def\maxheight{\ifdim\Gin@nat@height>\textheight\textheight\else\Gin@nat@height\fi}
\makeatother
% Scale images if necessary, so that they will not overflow the page
% margins by default, and it is still possible to overwrite the defaults
% using explicit options in \includegraphics[width, height, ...]{}
\setkeys{Gin}{width=\maxwidth,height=\maxheight,keepaspectratio}
% Set default figure placement to htbp
\makeatletter
\def\fps@figure{htbp}
\makeatother
\setlength{\emergencystretch}{3em} % prevent overfull lines
\providecommand{\tightlist}{%
  \setlength{\itemsep}{0pt}\setlength{\parskip}{0pt}}
\setcounter{secnumdepth}{-\maxdimen} % remove section numbering
\ifLuaTeX
  \usepackage{selnolig}  % disable illegal ligatures
\fi

\author{}
\date{}
\usepackage{multirow}
\usepackage[margin=1.0in]{geometry}
\usepackage[english]{babel}
\usepackage[utf8]{inputenc}
\usepackage{fancyhdr}

\pagestyle{fancy}
\fancyhf{}
\rhead{\includegraphics[width=5.21667in, height=0.38819in]{images/image1.png}}
\lhead{ Quantitative Aptitude: Number System }
\lfoot{www.talentsprint.com }
\rfoot{\thepage}
\begin{document}




\begin{center}
	{\Large \textbf{Number System}}
\end{center}


\textbf{Part 2 - Advanced \\}

\begin{enumerate}
	\item The fourth root of 24010000 is --\\
	\includegraphics[width=0.60555in,height=0.32083in]{images/image2.png} a) 7 b) 49 c) 490 d) 70
	
	\item A number x when divided by 289 leaves 18 as the remainder. The same number when divided by 17 leaves y as a remainder. The value of y is\\
	\includegraphics[width=0.60555in,height=0.32083in]{images/image2.png} a) 5 b) 2 c) 3 d) 1
	
	\item If the sum of two numbers be multiplied by each number separately, the products so obtained are 247 and 114. The sum of the numbers is\\
	\includegraphics[width=0.60555in,height=0.32083in]{images/image2.png} a) 19 b) 20 c) 21 d) 23
	
	\item The number 0.121212 \_\_\_ in the form p/q is equal to\\
	\includegraphics[width=0.60555in,height=0.32083in]{images/image2.png} a) 4/11 b) 2/11 c) 4/33 d) 2/33
	
	\item By what least number should 675 be multiplied so as to obtain a perfect cube number?\\
	\includegraphics[width=0.60555in,height=0.32083in]{images/image2.png} a) 3 b) 5 c) 24 d) 40
	
	\item I multiplied a natural number by 18 and another by 21 and added the products. Which of the following could be the sum? \\
	\includegraphics[width=0.60555in,height=0.32083in]{images/image2.png} a) 2007 b) 2008 c) 2006 d) 2002
	
	\item If a and b are two odd positive integers, by which of the following integers (a4 -- b4) always divisible? \\
	\includegraphics[width=0.60555in,height=0.32083in]{images/image2.png} a) 3 b) 6 c) 8 d) 12
	
	\item When 231 is divided by 5 the remainder is \\
	\includegraphics[width=0.60555in,height=0.32083in]{images/image2.png} a) 4 b) 3 c) 2 d) 1
	
	\item From each of two given numbers, half the smaller number is subtracted. After such subtraction, the larger number is 4 times as large as the smaller number. What is the ratio of the numbers? \\
	\includegraphics[width=0.60555in,height=0.32083in]{images/image2.png} a) 4:1 b) 4:5 c) 5:2 d) 1:4 5) None of these
	
	\item Twice the square of a number is more than eleven times the number by 21. The number can have which of the following values? \\
	\includegraphics[width=0.60555in,height=0.32083in]{images/image2.png} a) 4 or -7/2 b) 7 or -3/2 c) 3 or -7/2 d) 9/2 or -4 e) None of these
	
	\item If a two-digit number is three times the sum of it's digits, the number formed by interchanging the digits is how many times their sum? \\
	\includegraphics[width=0.60555in,height=0.32083in]{images/image2.png} a) 7 b) 8 c) 9 d) 11
	
	\item If we write 45 as the sum of 4 numbers so that when 2 is added to 1st number, 2  subtracted from 2nd number, 3rd multiplied by 2 and 4th divided by 2, we get the same result. Then the 4 numbers are \\
	\includegraphics[width=0.60555in,height=0.32083in]{images/image2.png} 1) 1,8,15,21 2) 8,12,5,20 3) 8,12,10,15 4) 2,12,5,26 5) None of these
	
	\item If we write 45 as the sum of 4 numbers so that when 2 is added to 1st number, 2  subtracted from 2nd number, 3rd multiplied by 2 and 4th divided by 2, we get the same result. Then the 4 numbers are \\
	\includegraphics[width=0.60555in,height=0.32083in]{images/image2.png} 1) 1,8,15,21  2) 8,12,5,20  3) 8,12,10,15  4) 2,12,5,26 5) None of these
	
	


\item The unit digit in the product (2467)153 x (341)72 is \\
\includegraphics[width=0.60555in,height=0.32083in]{images/image2.png} 1) 3 2) 9 3) 7 4) 1

\item The greatest number among 350, 440, 530 and 620 is \\
\includegraphics[width=0.60555in,height=0.32083in]{images/image2.png} 1) $ 6^{20} $ 2) $ 3^{50} $ 3) $4^{40}  $ 4)$ 5^{30} $ 

\item If we divide 1010 + 10 100 + 101000 by 6, then what will be the remainder? \\
\includegraphics[width=0.60555in,height=0.32083in]{images/image2.png}




\item The sum of 4 numbers is 196. If 6 is added to the first number, 6 is subtracted from the second number, third number is multiplied by 6 and 4th number is divided by 6, then all the resultant numbers are equal. What is the difference between largest and smallest of the original numbers? \\
\includegraphics[width=0.60555in,height=0.32083in]{images/image2.png} 1) 96 2) 104 3) 120 4) 140 5) 126


\item In dividing a number by 585, a student employed the method of short division. He divided the number successively 5, 9 and 13 (factors of 585) and got the remainders 4, 8 and 12 respectively. If he had divided the number by 585, the remainder would have been \\
\includegraphics[width=0.60555in,height=0.32083in]{images/image2.png} 1) 24 2) 144 3) 292 4) 584


\item A single reservoir supplies the petrol to the whole city, while the reservoir is fed by a single pipeline filling the reservoir with the stream of uniform volume. When the reservoir is full and if 40,000 litres of petrol is used daily, the supply fails in 90 days. If 32,000 litres of petrol is used daily, it fails in 60 days. How much petrol can be used daily without the supply ever failing? \\
\includegraphics[width=0.60555in,height=0.32083in]{images/image2.png} 1) 64000 litres 2) 56000 litres  3) 78000 litres 4) 60000 litres




\item A Mahabalipuram Temple has some magical bells which tolls 18 times in a day simultaneously. But every bell tolls at different intervals of time, but not in fraction of minutes. The maximum number of bells in the temple can be \\
\includegraphics[width=0.60555in,height=0.32083in]{images/image2.png} 1) 18 2) 10 3) 24 4) 6 5) None of these


\item Arun was paid Rs. 1552 during a period of 31 days. During this period he was absent for 3 days and was fined Rs. 29 on his first day of absence and Rs. 2 more than the previous day for each remaining day of absence. He was paid the full salary only for 21 days as he came late on the other days. Those who came late were given only 1/3rd of the salary for that day. What was the total salary paid for that month to a worker who came on time every day and was never absent? \\
\includegraphics[width=0.60555in,height=0.32083in]{images/image2.png} 1) 2185.5 2) 2180 3) 2100 4) 2300 5) None of these

\item If a perfect square, not divisible by 6, be divided by 6, the remainder will be \\
\includegraphics[width=0.60555in,height=0.32083in]{images/image2.png} 1) 1, 3 or 5 2) 1, 2 or 5 3) 1, 3 or 4 4) 1, 2 or 4

\item ($ 2^{51} $ + $ 2^{52} $ + $2^{53}  $ +$ 2^{54} $ + $ 2^{55} $) is divisible by \\
\includegraphics[width=0.60555in,height=0.32083in]{images/image2.png} 1) 23 2) 58 3) 124 4) 127


\item X and Y together started a business. X invests 2000 at the beginning of every quarter and Y withdraws 1000 at the end of every quarter. If their initial investment were 1000 and 4000 respectively and they make 39000 profit for the year, find the respective shares of X and Y? \\
\includegraphics[width=0.60555in,height=0.32083in]{images/image2.png}


\item A man lost half of his initial amount in the gambling after playing 3 rounds. The rule of gambling is that if he wins he will receive Rs 100, but he has to give 50\% of the total amount after each round. Luckily he won all the three rounds. The initial amount with which he had started the gambling was? \\
\includegraphics[width=0.60555in,height=0.32083in]{images/image2.png} 1) 500/3 2) 700/3 3) 300 4) 600




\item Deepak stores x kg of sugar. A buys one-third this amount plus 1 kg of sugar. B buys one- fourth of the remaining amount plus half a kg of sugar. Then C also buys 60\% the remaining amount plus 1.3 kg of sugar. Thereafter, no sugar is left. What is the value of x? \\
\includegraphics[width=0.60555in,height=0.32083in]{images/image2.png} 1) 7 kg 2) 7.5 kg 3) 8 kg 4) 9 kg 5) 10 kg

\item Begining from - 23, N consecutive integers are taken. The sum of them is 130. Find the value of N. \\
\includegraphics[width=0.60555in,height=0.32083in]{images/image2.png} 1) 47 2) 5 3) 23 4) 29 5) 52

\item A five - digit number is formed using 1, 3, 5, 7 and 9 without repeating any one of them. What is the sum of all such possible numbers? \\
\includegraphics[width=0.60555in,height=0.32083in]{images/image2.png} 1) 6666600 2) 6666660 3) 6666666 4) None of these

\end{enumerate}
\newpage
{\large \textbf{Answers} \\}

1. 70

2. 1

3. 19

4. 4/33

5. 5

6. 2007

7. 8

8. 3

9. 5:2

10. 7 or -3/2

11. 8

12. 8,12,5,20

13. 52

14. 7

15. 440

16. 0

17. 140

18. 584

19. 56000 litres

20. 10

21. 2185.5

22. 1, 3 or 4

23. 124

24. 700/3

25. 8850

26. 9kg


27. 52

28. 6666600



\end{document}
