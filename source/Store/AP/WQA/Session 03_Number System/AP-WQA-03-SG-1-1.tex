% Options for packages loaded elsewhere
\PassOptionsToPackage{unicode}{hyperref}
\PassOptionsToPackage{hyphens}{url}
%
\documentclass[
]{article}
\usepackage{amsmath,amssymb}
\usepackage{lmodern}
\usepackage{iftex}
\ifPDFTeX
  \usepackage[T1]{fontenc}
  \usepackage[utf8]{inputenc}
  \usepackage{textcomp} % provide euro and other symbols
\else % if luatex or xetex
  \usepackage{unicode-math}
  \defaultfontfeatures{Scale=MatchLowercase}
  \defaultfontfeatures[\rmfamily]{Ligatures=TeX,Scale=1}
\fi
% Use upquote if available, for straight quotes in verbatim environments
\IfFileExists{upquote.sty}{\usepackage{upquote}}{}
\IfFileExists{microtype.sty}{% use microtype if available
  \usepackage[]{microtype}
  \UseMicrotypeSet[protrusion]{basicmath} % disable protrusion for tt fonts
}{}
\makeatletter
\@ifundefined{KOMAClassName}{% if non-KOMA class
  \IfFileExists{parskip.sty}{%
    \usepackage{parskip}
  }{% else
    \setlength{\parindent}{0pt}
    \setlength{\parskip}{6pt plus 2pt minus 1pt}}
}{% if KOMA class
  \KOMAoptions{parskip=half}}
\makeatother
\usepackage{xcolor}
\IfFileExists{xurl.sty}{\usepackage{xurl}}{} % add URL line breaks if available
\IfFileExists{bookmark.sty}{\usepackage{bookmark}}{\usepackage{hyperref}}
\hypersetup{
  hidelinks,
  pdfcreator={LaTeX via pandoc}}
\urlstyle{same} % disable monospaced font for URLs
\usepackage{longtable,booktabs,array}
\usepackage{calc} % for calculating minipage widths
% Correct order of tables after \paragraph or \subparagraph
\usepackage{etoolbox}
\makeatletter
\patchcmd\longtable{\par}{\if@noskipsec\mbox{}\fi\par}{}{}
\makeatother
% Allow footnotes in longtable head/foot
\IfFileExists{footnotehyper.sty}{\usepackage{footnotehyper}}{\usepackage{footnote}}
\makesavenoteenv{longtable}
\usepackage{graphicx}
\makeatletter
\def\maxwidth{\ifdim\Gin@nat@width>\linewidth\linewidth\else\Gin@nat@width\fi}
\def\maxheight{\ifdim\Gin@nat@height>\textheight\textheight\else\Gin@nat@height\fi}
\makeatother
% Scale images if necessary, so that they will not overflow the page
% margins by default, and it is still possible to overwrite the defaults
% using explicit options in \includegraphics[width, height, ...]{}
\setkeys{Gin}{width=\maxwidth,height=\maxheight,keepaspectratio}
% Set default figure placement to htbp
\makeatletter
\def\fps@figure{htbp}
\makeatother
\setlength{\emergencystretch}{3em} % prevent overfull lines
\providecommand{\tightlist}{%
  \setlength{\itemsep}{0pt}\setlength{\parskip}{0pt}}
\setcounter{secnumdepth}{-\maxdimen} % remove section numbering
\ifLuaTeX
  \usepackage{selnolig}  % disable illegal ligatures
\fi

\author{}
\date{}
\usepackage{multirow}
\usepackage[margin=1.0in]{geometry}
\usepackage[english]{babel}
\usepackage[utf8]{inputenc}
\usepackage{fancyhdr}

\pagestyle{fancy}
\fancyhf{}
\rhead{\includegraphics[width=5.21667in, height=0.38819in]{images/image1.png}}
\lhead{ Study Note - Number Systems}
\lfoot{www.talentsprint.com/examprep }
\rfoot{\thepage}
\begin{document}



\begin{center}
	{\Large \textbf{\\ Number Systems}}
\end{center}

\textbf{\\ \\ Place value of numbers:}

123 -\textgreater{} 1 hundred + 2 tens + 3 ones

In general a number ab is represented in terms of its place value as
-\textgreater{} 10a+b
\\
\textbf{\\ \\ Classification of numbers:}


Numbers are broadly classified as real numbers and complex numbers. We
will be looking into only real numbers.

Real numbers can be classified as follows

\begin{itemize}
	\item Rational Numbers- Numbers, which can be represented in the form p/q
	where q is not equal to zero
	
	eg: 2/3, 8/1
	
	\item Irrational- Numbers which cannot be represented in the form p/q where q not equal to zero 
	
	eg: $ \sqrt{3}, \sqrt{2} $
	
	\item Natural Numbers- > {1, 2, 3, …..$ \infty $} -> +ve counting number
	
	\item Whole numbers -> {0, 1, 2, 3 …..$ \infty $} – natural no: including zero
	
	\item Integers -> set of whole number along with -ve numbers {-$ \infty $.........-3, -2, ….0, 1, 2, ….$ \infty $}
	
	\item Even Numbers-> divisible by 2 {2, 4, 6…}
	
	\item odd Numbers-> not divisible by 2 {1, 3, 5, …}
	
	\item Prime Numbers-> no factors other than 1 \& itself {2, 3, 5, 7, 11….}
	
	\item Composite Numbers -> numbers having other factors besides 1\& itself {4, 6, 8, 9, 10, 
		12….}
	
	\item Coprimes - numbers which are primes with respect to each other. They have no common 
	factors other than 1
	Eg:- 22,9. Co primes necessarily need not be prime numbers
	
\end{itemize}

\textbf{\\  Factors:} To factor a number means to break it up into integers
that can be multiplied together to


get the original number

eg: 12 -\textgreater{} 2 x 6, 4 x 3 where 1, 2, 6, 4, 3, 12 are factors
of 12

\textbf{\\  Note:} For every number 1 and itself are always factors.

\textbf{\\  Finding factors of a number:}

If a number is expressed in terms of its prime factors as Ax× By× Cz
then the number of factors


of the given number is (x + 1) × (y + 1) x (z + 1)


The number of prime factors of this number is given by (x+y+z) and the
number of unique

prime factors of this number is no. of base values (a, b, c) (in above
example it is 3 i.e. the three


unique prime factors here are a, b and c)

\textbf{\\ Division Method:} Dividend = (Divisor × Quotient) + Remainder



\textbf{\\ Divisibility by 2} - all even numbers

\textbf{\\ Divisibility by 3} - sum of digits of numbers is divisible by 3

Ex: 729 -\textgreater{} 7 + 2 + 9 = 18 is divisible by 3 so the number
is divisible by 3

\textbf{\\ Divisibility by 4} -- last 2 digits of numbers is divisible by 4

Ex: 724 - last 2 digits is 24 which is divisible by 4. Therefore, the
number is divisible by 4

\textbf{\\ Divisibility by 5} -- last digit 5/0

\textbf{\\ Divisibility by 6} -- number should be divisible by 3 \& 2 for
it to be divisible by 6


\textbf{\\ Divisibility by 7} -- multiply unit digit by 2, subtract that
value from remaining digits. Keep


continuing this process

Ex: 11347

Last digit of 11347 is 7 -\textgreater{} 7 x 2 = 14

1134 - 14 = 1120

last digit of 1120 is 0 -\textgreater{} 0 x 2 = 0

112 - 0 = 112

last digit of 112 is 2 -\textgreater{} 2 x 2 = 4

11 - 4 = 7 =\textgreater{} div by 7

\textbf{\\ Divisibility by 8} - last 3 digits divisible by 8

2512 -\textgreater{} 512 divisible by 8 so the number is divisible by 8

\textbf{\\ Divisibility by 9 -} sum of digits is divisible by 9

Ex: 729 -\textgreater{} 7 + 2 + 9 = 18 is divisible by 9. So the number
is divisible by 9

\textbf{\\ Divisibility by 10} -- last digit `0'







find sum of alternatives
digits \& subtracts thus \& equal to zero or multiple


of 11.

Ex: 1 2 3 2 1

1 + 3 + 1 = 5

2 + 2 = 4

5 - 4 = 1 so not divisible by 11

Ex: 1 4 6 4 1

1 + 6 + 1 = 8

4 + 4 = 8

8 - 8 = 0 so divisible by 11

\textbf{\\ Divisibility by 12} - \textgreater{} Number should be divisible
by 4 \& 3


\textbf{\\ Divisibility by 13} -\textgreater{} Multiply last digit by 4.
Add it to remaining digit. Keep continuing this


process

Ex: 50661 -\textgreater{} 1 x 4 = 4

5066 + 4 = 5070 -\textgreater{} 0 x 4 = 0

507 + 0 = 507 -\textgreater{} 7 x 4 = 28

50 + 28 = 78.

78 is divisible by 13. So the number is divisible by 13


\textbf{\\ Divisibility by 17} -\textbf{\\ \textgreater{}} multiply last digit
with 5. Subtract the answer from the remaining digits.


Keep continuing the process

Ex: 3978 -\textgreater{} 8 x 5 = 40

397 - 40 = 357 -\textgreater{} 7 x 5 = 35

35- 35 = 0

So divisible by 17.

\textbf{\\ Finding the unit digit of the given number with certain power
value.}

\textbf{\\ Pattern method:}

1. If the unit digit of the given number is 0, 1, 5 or 6, then the same
number will be

the unit digit of the given number (i.e., 0, 1, 5 or 6).

2. If the unit digit of the given number is 4, then we have to check the
power value

ie., whether the given power is odd or even power. If it is odd power
then, the

unit digit is 4, if the power is even number, then the unit will be 6.

3. The same condition is applicable for 9. If it odd power then, the
unit digit is 9, if

it is even power, then the unit digit is 1.

4. If the unit digit of the given number is 2, 3, 7, or 8, then


\begin{center}
	\begin{tabular}{ |c|c|c|c| } 
	
		\hline
		\multirow {4} {15em}{\center{Unit 2}} & 	$2^{1} $ = 2 \\ 
		& $ 2^{2} $ = 4  \\ 
		& $ 2^{3} $ = 8  \\ 
		& $ 2^{4} $ = 6  \\ 
		\hline
		
		\multirow {4} {*}{		Unit 3		} & $ 3^{1} $ = 3  \\ 
		& $ 3^{2} $ = 9  \\ 
		& $ 3^{3} $ = 7  \\ 
		& $ 3^{4} $ = 1  \\ 
		\hline
		\multirow {4} {*}{Unit 3} & $ 7^{1} $ = 7  \\ 
		& $ 7^{2} $ = 9  \\ 
		& $ 7^{3} $ = 3  \\ 
		& $ 7^{4} $ = 1  \\ 
		\hline
		\multirow {4} {*}{Unit 3} & $ 8^{1} $ = 8  \\ 
		& $ 8^{2} $ = 4  \\ 
		& $ 8^{3} $ = 2  \\ 
		& $ 8^{4} $ = 6  \\ 
		\hline
	\end{tabular}
\end{center}
\begin{quote}
\textbf{\\ Shortcut formulae:}

1.A number being divided by d1 and d2 successively leaves remainders r1
and r2

respectively. Then remainder when the same number is divided by d1 × d2

Remainder = d1 × r2 +\_r1
\end{quote}









\end{document}
